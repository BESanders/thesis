\section{Evaluation}\label{sec:experiments}


\def\benchToy{\texttt{Micro}}            
\def\benchFold{\texttt{Folds}}          
\def\benchVec{\texttt{Vector}}         
\def\benchSort{\texttt{ListSort}}       
\def\benchGsort{\texttt{Data.List.sort}} 
\def\benchSplay{\texttt{Data.Set.Splay}} 
\def\benchMap{\texttt{Data.Map.Base}}  


\def\locToy{32}  
\def\specToy{19}    
\def\cspecToy{948}    
\def\nspecToy{9}    
\def\annToy{4}    
\def\cannToy{115}    
\def\nannToy{1}    
\def\modToy{TBD}    
\def\timeToy{2}    

\def\locFold{0}  
\def\specFold{0}  
\def\annFold{0}  
\def\modFold{TBD}  
\def\timeFold{0}  

\def\locVec{56}   
\def\specVec{56}   
\def\nspecVec{14}   
\def\cspecVec{2203}   
\def\annVec{2}  
\def\cannVec{0}  
\def\nannVec{0}  
\def\modVec{TBD}  
\def\timeVec{14}

%ListSort.hs
\def\locSort{29}  
\def\specSort{4}  
\def\cspecSort{193}  
\def\nspecSort{4}  
\def\annSort{1}  
\def\cannSort{47}  
\def\nannSort{1}  
\def\modSort{TBD}  
\def\timeSort{3}  

%GhcListSort.hs
\def\locGsort{71} 
\def\specGsort{3}%lines
\def\cspecGsort{142}%characters
\def\nspecGsort{3}%number
\def\annGsort{1}
\def\cannGsort{47}
\def\nannGsort{1}
\def\modGsort{TBD}
\def\timeGsort{8} 

%Splay.hs
\def\locSplay{136} 
\def\specSplay{15} 
\def\nspecSplay{15} 
\def\cspecSplay{617} 
\def\annSplay{11} 
\def\nannSplay{6} 
\def\cannSplay{590} 
\def\modSplay{TBD} 
\def\timeSplay{15} 

\def\locMap{1399}   
\def\specMap{119}   
\def\cspecMap{10830}   
\def\nspecMap{86}   
\def\annMap{31}   
\def\cannMap{761}   
\def\nannMap{7}   
\def\modMap{TBD}   
\def\timeMap{235}

%total
\edef\locTot{\number\numexpr
  \locToy + \locFold + \locVec + 
  \locGsort + \locSort + \locSplay + \locMap 
  \relax}   

\edef\specTot{\number\numexpr
  \specToy + \specFold + \specVec + 
  \specGsort + \specSort + \specSplay + \specMap 
  \relax}   

\edef\annTot{\number\numexpr
  \annToy + \annFold + \annVec + 
  \annGsort + \annSort + \annSplay + \annMap 
  \relax}   

\def\modTot{0}   

\edef\timeTot{\number\numexpr
  \timeToy + \timeFold + \timeVec + 
  \timeGsort + \timeSort + \timeSplay + \timeMap 
  \relax}   


\begin{table}[t!]
\caption[A quantitative evaluation of Abstract Refinements.]{ 
\textbf{(LOC)} is the number of non-comment Haskell source code lines as reported by \textit{sloccount},
%
\textbf{(Specs)} is the number of lines of type specifications,
%
\textbf{(Annot)} is the number of lines of other annotations, including refined
datatype definitions, type aliases and measures, required for verification,
%
\textbf{(Time)} is the time in seconds taken for verification.
}
\label{tab:eval}
\centering
% \captionsetup{justification=centering}
\begin{tabular}{lrrrrr}
\hline
\textbf{Program}& \textbf{LOC} & \textbf{Specs} & \textbf{Annot} &  \textbf{Time (s)} \\ 
\hline \hline
\benchToy       & \locToy   & \specToy   & \annToy   &  \timeToy     \\
\benchVec       & \locVec   & \specVec   & \annVec   &  \timeVec     \\
\benchSort      & \locSort  & \specSort  & \annSort  &  \timeSort    \\
\benchGsort     & \locGsort & \specGsort & \annGsort &  \timeGsort   \\
\benchSplay     & \locSplay & \specSplay & \annSplay &  \timeSplay   \\
\benchMap       & \locMap   & \specMap   & \annMap   &  \timeMap     \\
\hline
\textbf{Total}  & \locTot   & \specTot   & \annTot   &  \timeTot     \\
\hline
\end{tabular}
\end{table}

In this section, we empirically evaluate the expressiveness and
usability of abstract refinement types by 
implementing abstract refinement in \toolname 
as uninterpreted functions.
%
We use \toolname to 
typecheck a set of challenging benchmark programs.
%
(We defer the task of extending the metatheory to a call-by-name calculus to future work).

\begin{comment}
\mypara{\toolname} We have implemented abstract refinement 
in \toolname, a refinement type checker for Haskell.
\toolname verifies Haskell source one module (.hs file) at a time. 
It takes as input:
\begin{itemize}
\item A \emph{target} Haskell source file, with the desired refinement
types specified as a special form of comment annotation, 
\item An (optional) set of type specifications for imported definitions; these 
can either be put directly in the source for the corresponding modules,
if available, or in special \verb+.spec+ files otherwise. For imported 
functions for which no signature is given, \toolname conservatively uses 
the non-refined  Haskell type. %with the trivial refinement @True@ everywhere.
\item An (optional) set of logical qualifiers, which are predicate 
templates from which refinements are automatically synthesized
\cite{LiquidPLDI08}.
%
Formally, a logical qualifier is a predicate whose variables range
over the program variables, the special value variable $\valu$, and
\emph{wildcards} $\placev$, which \toolname instantiates with the
names of program variables.
%
Aside from the qualifiers given by the user, \toolname also uses
qualifiers mined from the refinement type annotations present in the
program.
\end{itemize}
After analyzing the program, \toolname returns as output:
\begin{itemize}
\item Either \textsc{Safe}, indicating that all the specifications indeed
verify, or \textsc{Unsafe}, indicating there are refinement type errors,
together with the positions in the source code where type checking fails
(\eg functions that do not satisfy their signatures, or callsites where 
the inputs don't conform to the specifications).
%
\item
%
  An HTML file containing the program source code annotated with
  inferred refinement types for all sub-expressions in the program.
%
  The inferred refinement type for each program expression is the
  strongest possible type over the given set of logical qualifiers.
%
  When a type error is reported, the programmer can use the inferred
  types to determine why their program does not typecheck:
  they can examine what properties \toolname \emph{can} deduce about 
  various program expressions and add more qualifiers or alter 
  the program as necessary so that it typechecks.
%
\end{itemize}

\mypara{Implementation}
%
\toolname verifies the contents of a single file (module) at a time as follows. 
%
First, the Haskell source is fed into GHC, which desugars the program
to GHC's ``core" intermediate representation~\cite{VytiniotisJM12}.
%
Second, the desugared program,
the type signatures for the module functions (which are to be verified) and 
the type signatures for externally imported functions (which are assumed to hold)
are sent to the constraint generator, which traverses the core bindings in a
syntax-directed manner to generate subtyping constraints.
%
The resulting constraints are simplified via our subtyping rules
(\S~\ref{sec:check}) into simple logical implication
constraints.
%
Finally, the implication constraints, together with the logical
qualifiers provided by the user and harvested from the type
signatures, are sent into an SMT- and abstract interpretation-based
fixpoint computation procedure that determines if the constraints are
satisfiable \cite{GrafSaidi97,Houdini}.
%
If so, the program is reported to be \emph{safe}.
%
Otherwise, each unsatisfiable constraint is mapped back to the
corresponding program source location that generated it and a
potential error is reported at that line in the program.
\end{comment}

\mypara{Benchmarks}
We have evaluated \toolname over the following list of benchmarks
which, in total, represent the different kinds of reasoning described in
\S~\ref{sec:overview}.
%
While we can prove, and previously have proved~\cite{LiquidPLDI09},
many so-called ``functional correctness" properties of these data
structures using refinement types, in this work we focus on the key
invariants which are captured by abstract refinements.

\begin{itemize}
\item \benchToy, which includes several functions demonstrating 
parametric reasoning with base values, type classes, and higher-order 
loop invariants for traversals and folds, as described in
\S~\ref{sec:overview:parametric} and \S~\ref{sec:overview:induction};

\item \benchVec, which includes the domain- and range-generic @Vec@
  functions and several ``clients"
  that use the generic @Vec@ to implement incremental initialization,
  null-terminated strings, and memoization, as described in
  \S~\ref{sec:overview:index};

\item \benchSort, which includes various textbook sorting algorithms
including insert-, merge- and quick-sort. We verify that the functions
actually produce sorted lists, \ie are of type @IncrList a@, as described in
\S~\ref{sec:overview:rec};

\item \benchGsort, which includes three non-standard, optimized list 
sorting algorithms, as found in the \verb+base+ package. 
These employ lists that are increasing and decreasing, as well as 
lists of (sorted) lists, but we can verify that they also finally 
produce values of type @IncrList a@;

\item \benchSplay, which is a purely functional, top-down splay set 
library from the \verb+llrbtree+ package. We verify that all the 
interface functions take and return binary search trees;

\item \benchMap, which is the widely-used implementation of functional
maps from the \verb+containers+ package. We verify that all the interface
functions preserve the crucial binary search ordering property and various
related invariants.
\end{itemize}
%
Table~\ref{tab:eval} quantitatively summarizes the results of our
evaluation.
%
We now give a qualitative account of our experience using \toolname
by discussing what the specifications and other annotations look like.

\mypara{Specifications are usually simple}
In our experience, abstract refinements greatly simplify writing
specifications for the \emph{majority} of interface or public functions.
For example, for \benchMap, we defined the refined version of the
@Tree@  ADT (actually called @Map@ in the source, we reuse the type from
\S~\ref{sec:overview:rec} for brevity), and then instantiated
it with the concrete refinements for binary-search ordering with the alias
@BST k v@  as described in \S~\ref{sec:overview:rec}.
%%$${\centering\begin{tabular}{c}\begin{code}
%%type BST k v = Tree <{\x y -> x > y}, {\x y-> x < y}> k v
%%\end{code}\end{tabular}}$$
Most refined specifications were just the Haskell types
with the @Tree@ type constructor replaced with the 
alias @BST@. For example, the type of 
@fromList@ is refined from @(Ord k) => [(k, a)] -> Tree k a@ to 
@(Ord k) => [(k, a)] -> BST k a@.
%%$${\centering\begin{tabular}{c}\begin{code}
%%insert   :: (Ord k) => k -> v   -> Tree k v -> Tree k v
%%union    :: (Ord k) => k -> v   -> Tree k v -> Tree k v
%%fromList :: (Ord k) => [(k, a)] -> Tree k a 
%%\end{code}\end{tabular}}$$
%%simply get transformed to
%%$${\centering\begin{tabular}{c}\begin{code}
%%insert   :: (Ord k) => k -> v   -> BST k v -> BST k v  
%%union    :: (Ord k) => k -> v   -> BST k v -> BST k v  
%%fromList :: (Ord k) => [(k, a)] -> BST k a              
%%\end{code}\end{tabular}}$$
Furthermore, intra-module Liquid type inference permits 
the automatic synthesis of necessary stronger types for
private functions.
%%$$
%%{\centering
%%\begin{tabular}{c}\begin{code}
%%balanceL :: kcut:k -> a                -- root key, value
%%            -> BST {v:k | v < kcut} a  -- left  tree
%%            -> BST {v:k | v > kcut} a  -- right tree
%%            -> BST k a                 -- output tree
%%\end{code}
%%\end{tabular}}
%%$$
%but of course, one may explicitly specify these types too.

\mypara{Auxiliary Invariants are sometimes Difficult}
However, there are often rather thorny \emph{internal} functions with tricky
invariants, whose specification can take a bit of work. For example, the
function @trim@ in {\benchMap} has the following behavior (copied verbatim
from the documentation):
``\verb-trim blo bhi t- trims away all subtrees that surely
  contain no values between the range \verb-blo- to \verb-bhi-. 
   The returned tree is either empty or the key of the 
   root is between \verb-blo- and \verb-bhi-."
Furthermore @blo@ (resp. @bhi@) are specified as option 
(\ie @Maybe@) values with @Nothing@ denoting $-\infty$ (resp. $+\infty$). 
%
Fortunately, refinements suffice to encode such properties. 
First, we define measures
%
\begin{code}
  measure isJust     :: Maybe a -> Bool 
  isJust (Just x)    = true
  isJust (Nothing)   = false

  measure fromJust   :: Maybe a -> a 
  fromJustS (Just x) = x 

  measure isBin       :: Tree k v -> Bool
  isBin (Bin _ _ _ _) = true
  isBin (Tip)         = false

  measure key :: Tree k v -> k 
  key (Bin k _ _ _)   = k 
\end{code}
%
which respectively embed the @Maybe@ and @Tree@ root value into the
refinement logic, after which we can type the @trim@ function as
\begin{code}
  trim :: (Ord k) => blo:Maybe k 
                  -> bhi:Maybe k 
                  -> BST k a 
                  -> {v:BST k a | bound(blo, v, bhi)}
\end{code}
where @bound@ is simply a refinement alias
\begin{code}
  refinement bound(lo, v, hi) 
    =  isBin(v) => isJust(lo) => fromJust(lo) < key(v) 
    &&  isBin(v) => isJust(hi) => fromJust(hi) > key(v)
\end{code}
That is, the output refinement states that the root is appropriately 
lower- and upper- bounded if the relevant terms are defined. 
Thus, refinement types allow one to formalize the crucial behavior as
machine-checkable documentation.

\mypara{Code Modifications} On a few occasions we also have to change the
code slightly, typically to make explicit values on which various
invariants depend. Often, this is for a trivial reason; a simple
re-ordering of binders so that refinements for \emph{later} binders 
can depend on earlier ones. Sometimes we need to introduce
``ghost" values so we can write the specifications (\eg the @foldr@ in
\S~\ref{sec:overview:induction}). Another example is illustrated by the use
of list @append@ in @quickSort@. Here, the @append@ only produces a sorted
list if the two input lists are sorted and such that each element in
the first is less than each element in the second. 
We address this with a special @append@ parameterized on @pivot@ 
%append :: k:a -> IncrList {v:a|v<k} -> IncrList {v:a|v>k} -> IncrList a
%append :: k:a -> IncList {v:a|v < k} -> IncList {v:a|v > k} 
%       -> IncList a
\begin{code}
  append :: pivot:a                    
         -> IncrList {v:a | v < pivot}  
         -> IncrList {v:a | v > pivot}  
         -> IncrList a              
  append pivot [] ys     = pivot : ys
  append pivot (x:xs) ys = x : append pivot xs ys
\end{code}
%%after which \toolname infers that
%%$$\centering\begin{tabular}{c}\begin{code}
%%\end{code}\end{tabular}$$
%thereby proving @quickSort@ returns a sorted list.


%\mypara{Challenges for Future Work}
%\subsection{Challenges}
%\begin{verbatim}
%- Error messages
%- Incremental checking
%\end{verbatim}
