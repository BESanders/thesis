\section{Evaluation}\label{sec:evaluation}

% We implemented our technique by extending \toolname~\cite{vazou13}. 
% Next, we describe the tool, the benchmarks, 
% and a quantitative summary of our results.
% We then present a qualitative discussion 
% of how \toolname was used to verify safety, 
% termination, and functional correctness 
% properties of a large library, and 
% discuss the strengths and limitations 
% unearthed by the study.

% \subsection{Benchmarks and Results}

% Our goal was to use \toolname to verify a suite of real-world Haskell 
% programs, to evaluate whether our approach is
% %
% %\begin{itemize}
% %  \item 
%     \emph{efficient}  enough for large programs,
% %  \item 
%     \emph{expressive} enough to  specify key correctness properties, and
% %  \item 
%     \emph{precise}    enough to  verify idiomatic Haskell codes.
% %\end{itemize}
% %
% %\mypara{Benchmarks}
% Thus, we used these libraries as benchmarks:
% %
% \begin{itemize}
% \item \texttt{GHC.List} and \texttt{Data.List}, which together implement many standard
%       list operations; we verify various
%       size related properties,
% \item \texttt{Data.Set.Splay}, which implements a splay-tree
%       based functional set data type; we verify that all interface 
%       functions terminate and return well ordered trees,
% \item \texttt{Data.Map.Base}, which implements a functional 
%       map data type; we verify that all interface functions 
%       terminate and return binary-search ordered trees~\cite{vazou13}, 
% \item \libvectoralgos, which includes a suite of 
%       ``imperative'' (\ie monadic) array-based sorting algorithms; 
%       we verify the correctness of vector accessing, indexing, and slicing.
% \item \bytestring, a library for manipulating byte arrays, we
%       verify termination, low-level memory safety, and high-level
%       functional correctness properties\includeProof{(\ref{sec:bytestring})},
% \item \libtext, a library for high-performance unicode text 
%       processing; we verify various pointer safety and 
%       functional correctness properties (\ref{sec:text}),
%       during which we find a subtle bug.
% \end{itemize}
% %
% We chose these benchmarks as they represent a wide spectrum of idiomatic
% Haskell codes: the first three are widely used libraries based on recursive 
% data structures, the fourth and fifth perform subtle, low-level arithmetic 
% manipulation of array indices and pointers, and the last is a rich, high-level
% library with sophisticated application-specific invariants. 
% These last three libraries are especially representative as they pervasively 
% intermingle high level abstractions like higher-order loops, folds, and fusion, 
% with low-level pointer manipulations in order to deliver high-performance. 
% They are an appealing target for \toolname, as refinement types are an ideal 
% way to statically enforce critical invariants that are outside the scope of
% run-time checking as even Haskell's highly expressive type system.

We now turn to a quantitative evaluation of our experiments with \toolname.

\begin{table*}[ht!]
\begin{scriptsize}
\caption[A quantitative evaluation of \toolname]{A quantitative evaluation of our experiments. 
  \textbf{Version} is version of the checked library.
  \textbf{LOC} is the number of non-comment lines of source code as reported by \texttt{sloccount}.
  \textbf{Mod}    is the number of modules in the benchmark and \textbf{Fun} is the number
                  of functions.
  \textbf{Specs}  is the number (/ line-count) of type specifications and aliases,
                  data declarations, and measures provided.
  \textbf{Annot}  is the number (/ line-count) of other annotations provided,
                  these include invariants and hints for
                  the termination checker.
  \textbf{Qualif} is the number (/ line-count) of provided qualifiers.
  \textbf{Time (s)}   is the time, in seconds, required to run \toolname. }
\label{table:realworldhaskell:results}
\centering
\begin{tabular}{|l|r|rrr|rrr|r|}
\hline
\textbf{Module} &\textbf{Version}           & \textbf{LOC} & \textbf{Mod} & \textbf{Fun} & \textbf{Specs} & \textbf{Annot} & \textbf{Qualif} & \textbf{Time (s)}\\
\hline\hline
\textsc{GHC.List} &      {7.4.1}    & 309          & 1            & 66           & 29 / 38        & 6 / 6          & 0 / 0           & 15 \\
\textsc{Data.List} &    {4.5.1.0}    & 504          & 1            & 97           & 15 / 26        & 6 / 6          & 3 / 3           & 11 \\
\hline
\textsc{Data.Map.Base} &   {0.5.0.0} & 1396         & 1            & 180          & 125 / 173      & 13 / 13        & 0 / 0           & 174 \\
\hline
\textsc{Data.Set.Splay} &  {0.1.1} & 149          & 1            & 35           & 27 / 37        & 5 / 5          & 0 / 0           & 27 \\
\hline
\textsc{HsColour} &{1.20.0.0}          & 1047         & 16           & 234          & 19 / 40        & 5 / 5          & 1 / 1           & 196 \\
\hline
\textsc{XMonad.StackSet} & {0.11}  & 256          & 1            & 106          & 74 / 213       & 3 / 3          & 4 / 4           & 27 \\
\hline
\textsc{ByteString} &    {0.9.2.1}   & 3505         & 8            & 569          & 307 / 465      & 55 / 55        & 47 / 124        & 294 \\
\hline
\textsc{Text}        &    {0.11.2.3}  & 3128         & 17           & 493          & 305 / 717      & 52 / 54        & 49 / 97         & 499 \\
\hline
\textsc{Vector-Algorithms}& {0.5.4.2} & 1218         & 10           & 99           & 76 / 266       & 9 / 9          & 13 / 13         & 89 \\
\hline
\textbf{Total}            & & 11512        & 56           & 1879         & 977 / 1975     & 154 / 156      & 117 / 242       & 1336 \\
\hline
\end{tabular}
\end{scriptsize}
\end{table*}


\subsection{Results}

We have used the following libraries as benchmarks:
%
\begin{itemize}
\item \texttt{GHC.List} and \texttt{Data.List}, which together implement many standard
      list operations; we verify various
      size related properties,
\item \texttt{Data.Set.Splay}, which implements a splay-tree
      based functional set data type; we verify that all interface 
      functions terminate and return well ordered trees,
\item \texttt{Data.Map.Base}, which implements a functional 
      map data type; we verify that all interface functions 
      terminate and return binary-search ordered trees~\cite{vazou13}, 
\item \texttt{HsColour}, a syntax highlighting program for Haskell code, we
      verify totality of all functions (\S~\ref{sec:totality}),
\item \texttt{XMonad}, a tiling window manager for X11, we verify the uniqueness
      invariant of the core datatype, as well as some of the QuickCheck
      properties (\S~\ref{sec:xmonad}),
\item \bytestring, a library for manipulating byte arrays, we
      verify termination, low-level memory safety, and high-level
      functional correctness properties (\S~\ref{sec:bytestring}),
\item \libtext, a library for high-performance unicode text 
      processing; we verify various pointer safety and 
      functional correctness properties (\S~\ref{sec:text}),
      during which we find a subtle bug,
\item \libvectoralgos, which includes a suite of 
      ``imperative'' (\ie monadic) array-based sorting algorithms; 
      we verify the correctness of vector accessing, indexing, and slicing \etc.
\end{itemize}

Table~\ref{table:results} summarizes our experiments, which covered 56 modules
totaling 11,512 non-comment lines of source code and 1,975 lines of specifications.
%
The results are on a machine with an Intel Xeon X5660 and 32GB of RAM~(no benchmark required more than 1GB.)
%
The upshot is that \toolname is very effective on real-world code bases.
%
The total overhead due to hints, \ie the sum of \bfAnnot and \bfQualif, is 3.5\% of \bfLOC.
%
The specifications themselves are machine checkable versions of the comments 
placed around functions describing safe usage and behavior, and required around
two lines to express on average.
%
% Our default metric, namely the first parameter with an associated size measure,
% suffices to prove \todonum\% of (recursive) functions terminating.
%
% \todonum\% require a hint (\ie the position of the decreasing argument) or a
% witness (\todonum\% required both), and the remaining \todonum\% were marked as potentially diverging.
%
% Of the \todonum functions marked as potentially diverging, we suspect \todonum
% actually terminate but were unable to prove so.
%
While there is much room for improving the running times, the tool is fast enough 
to be used interactively, verify a handful of API functions and associated helpers 
in isolation.

\subsection{Limitations}\label{sec:discussion}
%% \NV{
%% * When applying LiquidHaskell to a given library were any improvements to the tool necessary to complete the analysis, or was the tool sufficiently complete such that other users (not the maintainers) could have performed the same analysis?
%% }\NV{
%% * How practical is organizing and keeping source copies of all imported modules?
%% }\NV{
%% * When writing code, how can a developer know what constructs will present verification difficulties of the sort in HsColour?
%% }

Our case studies also highlighted several limitations
of \toolname that we will address in future work. 
In most cases, we could alter the code slightly to 
facilitate verification. 

\mypara{Ghost parameters} are sometimes needed in 
order to materialize values that are not needed 
for the computation, but are necessary to prove 
various specifications. For example, the @piv@ 
parameter in the @append@ function for red-black trees
(\S~\ref{sec:redblack}).

\mypara{Fixed-width integer and floating-point numbers}
\toolname uses the theories of linear arithmetic and
real numbers to reason about numeric operations. In some
cases this causes us to lose precision, \eg when we have to
approximate the behavior of bitwise operations. We could
address this shortcoming by using the theory of bit-vectors
to model fixed-width integers, but we are unsure of the
effect this would have on \toolname's performance.

\mypara{Higher-order functions} 
must sometimes be \emph{specialized} because the 
original type is not precise enough. 
%
For example, the \texttt{concat} function that
concatenates a list of input @ByteString@s %@xs@,
pre-allocates the output region by computing the 
total size of the input.
%
\begin{code}
  len = sum . map length $ xs
\end{code}
%$
Unfortunately, the type for @map@ is not sufficiently
precise to conclude that the value @len@ equals 
@bLens xs@, se we must manually specialize
the above into a single recursive traversal that 
computes the lengths.
%
Rather than complicating the type system with a very
general higher-order type for @map@ we suspect the 
best way forward will be to allow the user to specify
inlining in a clean fashion.

\mypara{Functions as Data} Several libraries like \lbtext
encode data structures like (finite) streams using functions,
in order to facilitate fusion. Currently, it is not possible
to describe sizes of these structures using measures, as this
requires describing the sizes of input-output chains starting
at a given seed input for the function. In future work, it will 
be interesting to extend the measure mechanism to support 
multiple parameters (\eg a stream and a seed) in order to reason
about such structures. 

\mypara{Lazy binders} sometimes get in the way of verification. 
A common pattern in Haskell code is to define \emph{all} 
local variables in a single @where@ clause and use them 
only in a subset of all branches. 
%
\toolname flags a few such definitions as \emph{unsafe},
not realizing that the values will only be demanded in a
specific branch. 
%
Currently, we manually transform the code by pushing 
binders inwards to the usage site. % (via @let@).
This transformation could be easily automated.

\mypara{Assumes} 
which can be thought of as ``hybrid" run-time checks,
had to be placed in a couple of cases where the verifier 
loses information. 
%
One source is the introduction of assumptions about
mathematical operators that are currently conservatively 
modeled in the refinement logic (\eg that multiplication 
is commutative and associative). 
These may be removed by using more advanced non-linear 
arithmetic decision procedures.

\mypara{Error messages} are a crucial part of any type-checker.
Currently, we report error locations in the provided source
file and output the failed constraint(s). Unfortunately, the
constraints often refer to intermediate values that
have been introduced during the ANF-transformation, which
obscures their relation to the program source. In future work,
we may attempt to map these intermediate values back to their
source expressions, which should increase the comprehensibility
of our error messages. Another interesting possibility would be
to search for concrete counterexamples when \toolname detects an
invalid constraint.


%% TECHREP
%% \RJ{CUT: No context} Another is the introduction 
%% of axioms about the semantics of uninterpreted 
%% measures, like @numchars@ used to model the 
%% number of (unicode) characters in a packed 
%% array~\cite{techreport}.
%% %
%% \RJ{CUT: Not enough context} Another source is current 
%% limitations in handling Haskell features like 
%% rank-2 polymorphism \eg like the @liquidAssume@ 
%% in @mapAccumL@ for @Text@.



%%% Local Variables: 
%%% mode: latex
%%% TeX-master: "main"
%%% End: 
