\begin{itemize}
\item Reading the input.
@HsColour.hs@ reads the input arguments and checks their well-formedness
using the when statement
%
\begin{code}
when (length outFile > 1) \$ errorOut "Can only have one output file"
\end{code}
%
Currently \toolname can not take advantage of such specifications, 
thus explicit @assume@ is used.

\item Lists with structure:
@ACSS.splitSrcAndAnnos@ handles a list of Strings and makes the following assumption.
when the specific @breakS@ appears then the name and annotations follow (ie., at least two elements).
So if a list @ls@ starts with @breakS@ then it uses an refutable pattern  @(_:mname:annots) = ls@
to match the list.
This invariant about the inputs list cannot be proven statically.
Worse, \toolname has no way to express such invariant, 
as refinement of list elements \textit{relies} on polymorphism: 
\toolname naturally describes invariants that recursively hold for every list element and 
reaches it edge when reasoning about non-recursive properties.

\item The instance @Enum@ of @Highlight@ does not define the @toEnum@ method, 
which as discussed reduces to a @noMethodBinding@ error that makes verification unsafe.

\item trimContext initially checks whether there exists an element that satisfies @p@ in the list @xs@.
If it is it creates @ys = dropWhile (not p) xs@, and then calls the tail of @ys@.
By the check we know that @ys@ has at least one element, the one that satisfies @p@, 
a property that is not trivially expressed via refinement types.


\item filter data constructors: @foo (filter (`elem` [A, B, C])@ and then partially define 
@foo@ on only the data constructors @A@, @B@, and @C@.
\end{itemize}
