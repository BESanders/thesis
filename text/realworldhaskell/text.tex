Next %, to give a qualitative sense of the kinds of properties analyzed 
% during the course of our evaluation, 
we present a brief overview of the verification of \libtext, which 
is the standard library used for serious unicode text processing. 
\libtext uses byte arrays and stream fusion to guarantee 
performance while providing a high-level API.
In our evaluation of \toolname on \libtext,%~\cite{text},
we focused on two types of properties: 
(1) the safety of array index and write operations, and 
(2) the functional correctness of the top-level API.
%
These are both made more interesting by the fact that 
\libtext internally encodes characters using UTF-16, 
in which characters are stored in either two or four bytes.
%
\libtext is a vast library spanning 39 modules and 5,700 lines of
code, however we focus on the 17 modules that are relevant
to the above properties.
%
While we have verified exact functional correctness size properties
for the top-level API, we focus here on the low-level functions 
and interaction with unicode.

\spara{Arrays and Texts}
A @Text@ consists of an (immutable) @Array@ of 16-bit words,
an offset into the @Array@, and a length describing the
number of @Word16@s in the @Text@.  
The @Array@ is created and filled using a
\emph{mutable} @MArray@. 
All write operations in \libtext are performed on @MArray@s 
in the @ST@ monad, but they are \emph{frozen} into @Array@s
before being used by the @Text@ constructor.
%
We write a measure for the size of an @MArray@ and use
it to type the write and freeze operations.
%
\begin{code}
  measure malen       :: MArray s -> Int
  predicate EqLen A MA = alen A = malen MA
  predicate Ok I A     = 0 <= I < malen A
  type VO A            = {v:Int| Ok v A} 
  
  unsafeWrite  :: m:MArray s
               -> VO m -> Word16 -> ST s ()
  unsafeFreeze :: m:MArray s
               -> ST s {v:Array | EqLen v m}
\end{code}

\spara{Reasoning about Unicode}
The function @writeChar@ (abbreviating \texttt{unsafeWrite} from \texttt{UnsafeChar})
writes a @Char@ into an @MArray@.
\libtext uses UTF-16 to represent characters internally,
meaning that every @Char@ will be encoded using two or 
four bytes (one or two @Word16@s).
%
\begin{code}
  writeChar marr i c
      | n < 0x10000 = do
          unsafeWrite marr i (fromIntegral n)
          return 1
      | otherwise = do
          unsafeWrite marr i lo
          unsafeWrite marr (i+1) hi
          return 2
      where n = ord c
            m = n - 0x10000
            lo = fromIntegral
               $ (m `shiftR` 10) + 0xD800
            hi = fromIntegral
               $ (m .&. 0x3FF) + 0xDC00
\end{code}
%
The UTF-16 encoding complicates the specification of the function
as we cannot simply require @i@ to be less than the length of 
@marr@; if @i@ were @malen marr - 1@ and @c@ required two 
@Word16@s, we would perform an out-of-bounds write. 
%
We account for this subtlety with a predicate that states 
there is enough @Room@ to encode @c@.
%
% measure ord         :: Char -> Int
\begin{code}
  predicate OkN I A N  = Ok (I+N-1) A
  predicate Room I A C = if ord C < 0x10000
                         then OkN I A 1
                         else OkN I A 2
  
  type OkSiz I A = {v:Nat  | OkN  I A v}
  type OkChr I A = {v:Char | Room I A v}
\end{code}
%
@Room i marr c@ says 
``if @c@ is encoded using one @Word16@, 
  then @i@ must be less than @malen marr@,
  otherwise @i@ must be less than @malen marr - 1@.''
%
@OkSiz I A@ is an alias for a valid number of @Word16@s 
remaining after the index @I@ of array @A@. 
@OkChr@ specifies the @Char@s for which there is room (to write)
at index @I@ in array @A@.
%
The specification for @writeChar@ states that given an array \hbox{@marr@,}
an index @i@, and a valid @Char@ for which there is room at index \hbox{@i@,}
the output is a monadic action returning the number of @Word16@ occupied
by the @char@.
%
\begin{code}
  writeChar :: marr:MArray s
            -> i:Nat
            -> OkChr i marr
            -> ST s (OkSiz i marr)
\end{code}
%
\spara{Bug}
Thus, clients of @writeChar@ should only call it with suitable indices
and characters.
%
Using \toolname we found an error in one client, @mapAccumL@, 
which combines a map and a fold over a @Stream@, and stores 
the result of the map in a @Text@. Consider the inner loop of @mapAccumL@.
%
% \begin{code}
% mapAccumL f z0 (Stream next0 s0 len) =
%   (nz, Text na 0 nl)
%  where
%   mlen = upperBound 4 len
%   (na,(nz,nl)) = runST $ do
%        (marr,x) <- (new mlen >>= \arr ->
%                     outer arr mlen z0 s0 0)
%        arr      <- unsafeFreeze marr
%        return (arr,x)
%   outer arr top = loop
%    where
%     loop !z !s !i =
%       case next0 s of
%         Done          -> return (arr, (z,i))
%         Skip s'       -> loop z s' i
%         Yield x s'
%           | j >= top  -> do
%             let top' = (top + 1) `shiftL` 1
%             arr' <- new top'
%             copyM arr' 0 arr 0 top
%             outer arr' top' z s i
%           | otherwise -> do
%             let (z',c) = f z x
%             d <- writeChar arr i c
%             loop z' s' (i+d)
%           where j | ord x < 0x10000 = i
%                   | otherwise       = i + 1
% \end{code}
\begin{code}
  outer arr top = loop
   where
    loop !z !s !i =
      case next0 s of
        Done          -> return (arr, (z,i))
        Skip s'       -> loop z s' i
        Yield x s'
          | j >= top  -> do
            let top' = (top + 1) `shiftL` 1
            arr' <- new top'
            copyM arr' 0 arr 0 top
            outer arr' top' z s i
          | otherwise -> do
            let (z',c) = f z x
            d <- writeChar arr i c
            loop z' s' (i+d)
          where j | ord x < 0x10000 = i
                  | otherwise       = i + 1
\end{code}
%
Let's focus on the @Yield x s'@ case.
%
We first compute the maximum index @j@ to 
which we will write and determine the safety of a write. 
%
If it is safe to write to @j@ we call the provided 
function @f@ on the accumulator @z@ and the character 
@x@, and write the \emph{resulting} character @c@ into the array. 
%
However, we know nothing about @c@, in particular, 
whether @c@ will be stored as one or two @Word16@s! 
Thus, \toolname flags the call to @writeChar@ as \emph{unsafe}.
The error can be fixed by lifting @f z x@ into the @where@ clause and defining the
write index @j@ by comparing @ord c@ (not @ord x@). \toolname (and the authors)
readily accepted our fix.

%% INCLUDEPROOF To illustrate why the call is in fact buggy, 
%% INCLUDEPROOF consider a sample iteration of @loop@ 
%% INCLUDEPROOF where @i = malen arr - 1@ and
%% INCLUDEPROOF @ord x < 0x10000@. 
%% INCLUDEPROOF %
%% INCLUDEPROOF In this case @j@ will equal @i@ and we will enter
%% INCLUDEPROOF the @otherwise@ branch. 
%% INCLUDEPROOF %
%% INCLUDEPROOF Next, suppose @f z x@ returns a
%% INCLUDEPROOF @c@ such that  @ord c >= 0x10000@. 
%% INCLUDEPROOF %
%% INCLUDEPROOF The action @writeChar arr i c@ will write to
%% INCLUDEPROOF indices @i@ \emph{and} @i+1@ of @arr@, but 
%% INCLUDEPROOF @i+1 = malen arr@ and is not a valid index 
%% INCLUDEPROOF for writing! 
%% INCLUDEPROOF %
%% INCLUDEPROOF The error lies dormant till the next loop 
%% INCLUDEPROOF iteration, when @i = malen arr + 1@ and we 
%% INCLUDEPROOF trigger the @j >= top@ branch. 
%% INCLUDEPROOF %
%% INCLUDEPROOF Here, we allocate a larger array and copy 
%% INCLUDEPROOF the contents of the previous array into the 
%% INCLUDEPROOF new array. 
%% INCLUDEPROOF %
%% INCLUDEPROOF The @copyM arr' 0 arr 0 top@ call
%% INCLUDEPROOF only copies @top@ elements, \ie it 
%% INCLUDEPROOF \emph{does not}
%% INCLUDEPROOF copy the element \emph{at} \texttt{top},
%% INCLUDEPROOF \emph{losing} a @Word16@ and so 
%% INCLUDEPROOF yielding the wrong  output.
%% INCLUDEPROOF The fix is to replace...
%% INCLUDEPROOF \begin{code}
%% INCLUDEPROOF    | j >= top  -> do ...
%% INCLUDEPROOF    | otherwise -> do
%% INCLUDEPROOF      d <- writeChar arr i c
%% INCLUDEPROOF      loop z' s' (i+d)
%% INCLUDEPROOF    where (z',c) = f z x
%% INCLUDEPROOF          j | ord c < 0x10000 = i
%% INCLUDEPROOF            | otherwise       = i + 1
%% INCLUDEPROOF \end{code}

%%% Local Variables: 
%%% mode: latex
%%% TeX-master: "main"
%%% End: 
