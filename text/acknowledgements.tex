% \todonum{acks}
I want to thank Ranjit Jhala for being such a great supervisor. 
%
It is his unique way to interpret, simplify, and beautify my 
work
and his strong willingness and enthusiasm to do useful research 
that directed my research and influenced the way I am thinking. 

A big thanks to my committee members 
Samuel R. Buss, 
Cormac Flanagan,
Sorin Lerner, 
and Daniele Micciancio for their interest and insights in my work. 

Next, I thank all my lab-mates of the UCSD programming languages group and specifically my 
collaborators
Patric Rondon,
Alexander Bakst, and
Eric Seidel 
for providing a supportive, friendly, and inspiring working environment.

I also want to thank all these people who hosted me in my internships. 
%
%The Opa group in Paris who foresaw that types can help industrial development.
%
Dimitrios Vytiniotis and Simon Peyton-Jones
at Microsoft Research Cambridge
who were the first ghc people who believed in Liquid Haskell
and have been still supporting both the project and me personally.
%
I want to thank Jeff Polakow and all the people at Awake Networks 
for giving me the opportunity to spend an awesome summer in Mountain View 
and use Liquid Haskell in real world development code. 
%
Last but not least, 
I a huge thanks to Daan Leijen 
for hosting me at Microsoft Research Redmond 
and mentoring me since then, giving me all his insightful
knowledge about research and life in general.

Finally, I want to thank my family and friends, around the world,
who supported me throughout this journey.
%
All my San Diego friends who were next to me 
during times of both joy and sorrow and 
who turned this beautiful place into 
my second home.
%
All the people I met in my numerous trips in these five years, 
that made me realize how knowledge unites people from different parts of the globe.
%
All my friends and family in Greece that despite the distance, 
always welcome me during my escapes in my home country. 
%
Specifically, my parents Voula and Nikos and my sister Marianna
who staid up all these Sunday nights for our online calls. 

It is great that the destination is reached, 
but after all, it is the journey that matters. 

\newpage

Chapter~\ref{chapter:tool} contains material adapted from the following publication:
\noindent N. Vazou, E. Seidel, and R. Jhala,
``LiquidHaskell: Experience with Refinement Types in the Real World'', 
Haskell, 2014.

Chapter~\ref{chapter:refinedhaskell} contains material adapted from the following publication:
\noindent N. Vazou, E. Seidel, R. Jhala, D. Vytiniotis, and S. Peyton-Jones,
``Refinement Types for Haskell'', 
ICFP, 2014.

Chapter~\ref{chapter:abstractrefinements} contains material adapted from the following publication:
\noindent N. Vazou, P. Rondon, and R. Jhala,
``Abstract Refinement Types'', 
ESOP, 2013.

Chapter~\ref{boundedrefinements} contains material adapted from the following publication:
\noindent N. Vazou, A. Bakst, and R. Jhala,
``Bounded Refinement Types'',
ICFP, 2015.


Chapter~\ref{refinementrflection} has been submitted for publication of the material as it may appear in PLDI 2017:
\noindent Vazou, Niki; Choudhury, Vikraman; Scott, Ryan G.; Newton, Ryan R.; Jhala, Ranjit.
``Refinement Reflection: Parallel Legacy Languages as Theorem Provers''.

Chapter~\ref{stringmatcher} has been submitted for publication of the material as it may appear in ESOP 2017:
\noindent Vazou, Niki; Polakow, Jeff.
``Verified Parallel String Matching in Haskell''.

The dissertation author was the primary investigator and author of these papers.