\section{Capability Safe Scripting via RIO}
\label{sec:files}\label{subsec:state:files}

\begin{figure}[!ht]
\centering
\captionsetup{justification=centering}
\begin{mcode}
  pread, pwrite, plookup, pcontents,
  pcreateD, pcreateF, pcreateFP :: Priv -> Bool

  active   :: World -> Set FH 
  caps     :: World -> Map FH Priv

  pset p h = \w -> p (select (caps w) h) && h $\in$ active w
\end{mcode}
\caption{Privilege Specification.}
\label{fig:fstypes} 
\end{figure}

% \section{Script Permissions}\label{sec:files}\label{subsec:state:files}
Next, we describe how we use the \RIO monad to reason about shell
scripting, inspired by the \shill~\citep{shill} programming language.
%

\mypara{Shill} is a scripting language that restricts the
privileges with which a script may execute by using
\emph{capabilities} and \emph{dynamic contract checking}~\citep{shill} .
%
Capabilities are \emph{run-time values} 
that witness the right to use a particular resource 
(\eg a file).
%
A capability is associated with a set of privileges, 
each denoting the permission to use the capability 
in a particular way (such as the permission to write 
to a file).
%
A contract for a \shill procedure describes the 
required input capabilities and any output values.
%
The \shill runtime guarantees that system resources are accessed in
the manner described by its contract.

In this section, we turn to the problem of
preventing \shill runtime failures.
%
(In general, the verification of file system resource usage is a rich
topic outside the scope of this paper).
%
That is, assuming the \shill runtime and an API as described in
\cite{shill}, how can we use Bounded Refinement Types to encode
scripting privileges and reason about them \emph{statically?}

\mypara{We use RIO types} to specify \shill 's API operations
%
thereby providing \emph{compile-time} guarantees 
about privilege and resource usage.
%
To achieve this, we:
%
connect the state (@World@) of the \RIO monad with a
\emph{privilege specification} denoting the set of 
privileges that a program may use~(\S~\ref{sec:privilege-spec});
%
specify the \emph{file system API} in terms of this
abstraction~(\S~\ref{sec:fs-api});
%
and use the above to specify and verify the particular 
privileges that a \emph{client} of the API uses~(\S~\ref{sec:fs-client}).

\subsection{Privilege Specification}
\label{sec:privilege-spec}
%

Figure~\ref{fig:fstypes} summarizes how we specify privileges 
inside @RIO@. 
%
We use the type @FH@ to denote a file handles, analogous to \shill's
capabilities. An abstract type @Priv@ denotes the sets of privileges
that may be associated with a particular @FH@.

\mypara{To connect {World}s with Privileges} we assume 
a set of uninterpreted functions of type @Priv ->  Bool@ 
that act as predicates on values of type @Priv@, each 
denoting a particular privilege.
%
For example, given a value @p :: Priv@, the proposition 
@pread p@ denotes that @p@ includes the ``read'' privilege.
%
The function @caps@ associates each @World@ with a @Map FH Priv@,
a table that associates each @FH@ with its privileges.
% 
The function @active@ maps each @World@ to the @Set@ of
allocated @FH@s.
%
Given @x:FH@ and @w:World@, @pwrite (select (caps w) x)@
denotes that in the state @w@, the file @x@ 
may be written.
%
This pattern is generalized by the predicate @pset pwrite x w@.

\subsection{File System API Specification}
\label{sec:fs-api}
%
A privilege tracking file system API can be partitioned into the
privilege \emph{preserving} operations and the privilege \emph{extending}
operations.

\mypara{To type the privilege preserving} operations, we define a predicate
@eqP w w'@ that says that the set of privileges and active handles
in worlds @w@ and @w'@ are \emph{equivalent}.
%
\begin{code}
  eqP = \w _ w' -> caps w == caps w' && active w == active w'
\end{code}
%
We can now specify the privilege preserving operations that @read@ and @write@ files, 
and list the @contents@ of a directory, all of which require the 
capabilities to do so in their pre-conditions:
%
\begin{code}
  read :: {- Read the contents of h -}
    h:FH -> RIO <pset pread h, eqp> String
  
  write :: {- Write to the file h -}
    h:FH -> String -> RIO <pset pwrite h, eqp> ()
  
  contents :: {- List the children of h -}
    h:FH -> RIO <pset pcontents h, eqp> [Path]
\end{code} 

\mypara{To type the privilege extending} operations, we define 
predicates that say that the output world is suitably 
extended. First, each such operation \emph{allocates} 
a new handle, which is formalized as:
%
\begin{mcode}
  alloc w' w x = (x $\not \in$ active w) && active w' == {x} $\cup$ active w
\end{mcode}
%
which says that the active handles in (the new @World@) 
@w'@ are those of (the old @World@) @w@ extended with the
hitherto \emph{inactive} handle @x@.
%
Typically, after allocating a new handle, a script will
want to add privileges to the handle that are obtained
from existing privileges.

\mypara{To {create} a new file} in a directory with handle @h@ we
want the new file to have the privileges \emph{derived} from
@pcreateFP (select (caps w) h)@ (\ie the create privileges of @h@). We
formalize this by defining the post-condition of @create@ as the predicate @derivP@:
%
\begin{code}
  derivP h  = \w x w' -> 
    alloc w' w x && 
    caps w' == store (caps w) x (pcreateFP (select (caps w)) h)

  create :: {- Create a file -}
    h:FH -> Path -> RIO <pset pcreateF h, derivP h> FH
\end{code}
%
Thus, if @h@ is writable in the old @World w@ 
(@pwrite (pcreateFP (select (caps w) h))@) and
@x@ is derived from @h@ (@derivP w' w x h@ both hold),
then we know that @x@ is writable in the new @World w'@
(@pwrite (select (caps w') x)@).

\mypara{To {lookup} existing files} or create sub-directories,
we want to directly \emph{copy} the privileges of the parent handle. 
We do this by using a predicate @copyP@ as the post-condition for 
the two functions:
%
\begin{code}
  copyP h = \w x w' ->
    alloc w' w x && 
    caps w' == store (caps w) x 
                     (select (caps w) y)

  lookup :: {- Open a child of h -}
    h:FH->Path->RIO<pset plookup h, copyP h> FH

  createDir :: {- Create a directory -}
    h:FH->Path->RIO<pset pcreateD h, copyP h> FH
\end{code}
  
%%\rj{HEREHEREHERE}
%%
\subsection{Client Script Verification}
\label{sec:fs-client}
%
We now turn to a client script,
the program @copyRec@ % in Figure~\ref{fig:copysrc}
that copies the contents of the directory @f@ to the
directory @d@.
%
\begin{code}
  copyRec recur s d = 
    do cs <- contents s
       forM_ cs $ \ p -> do
         x <- flookup s p
         when (isFile x) $ do
           y <- create d p
           s <- fread x
           write y s
         when (recur && (isDir x)) $ do
           y <- createDir d p
           copyRec recur x y
\end{code}%$
%
@copyRec@ executes by first listing the contents of @f@, 
and then opening each child path @p@ in @f@. 
%
If the result is a file, it is copied to the directory @d@.
%
Otherwise, @copyRec@ recurses on @p@, if @recur@ is true.

In a first attempt to type @copyRec@ we give it the following type:
\begin{code}
  copyRec :: Bool -> s:FH -> d:FH ->
             RIO<copySpec s d, \_ _ w -> copySpec s d w> () 

  copySpec h d = \w ->
   pset pcontents h w && pset plookup h     w &&
   pset pread h     w && pset pcreateFile d w &&
   pset pwrite d    w && pset pcreateF d    w &&
   pwrite (pcreateFP (select (caps w) d)))
\end{code}              
%
The above specification gives @copyRec@ a minimal set of privileges. 
%
Given a source directory handle @s@ and destination handle @d@, the
@copyRec@ must at least:
%
\begin{inparaenum}[(1)]
  \item list the contents of @s@ (@pcontents@),
  \item open children of @s@ (@plookup@),
  \item read from children of @s@ (@pread@),
  \item create directories in @d@ (@pcreateD@),
  \item create files in @d@ (@pcreateF@), an
  \item write to (created) files in @d@ (@pwrite@).
\end{inparaenum}
%
Furthermore, we want to restrict the privileges on newly created files
to the write privilege, since @copyRec@ does not need to read from or
otherwise modify these files.

Even though the above type is sufficient to verify
the various clients of @copySpec@ it
is insufficient to verify @copySpec@'s implementation, 
as the postcondition merely states that @copySpec s d w@ holds.
%
Looking at the recursive call in the last line of @copySpec@'s implementation,
the output world @w@ is only known to satisfy @copySpec x y w@ (having
substituted the formal parameters @s@ and @d@ with the actual @x@ and
@y@), with no mention of @s@ or @d@!
%
Thus, it is impossible to satisfy the postcondition of @copyRec@, as
information about @s@ and @d@ has been lost.

\mypara{Framing} is introduced to address the above problem.
Intuitively, because no privileges are ever
\emph{revoked}, if a privilege for a file existed \emph{before} the
recursive call, then it exists \emph{after} as well.
%
We thus introduce a notion of \emph{framing} -- assertions about
unmodified state that hold before calling @copyRec@ must hold after
@copyRec@ returns.
%
Solidifying this intuition, we define a predicate @i@ to be @Stable@
when assuming that the predicate @i@
holds on @w@, if @i@ only depends on the allocated set of 
privileges, then @i@ will hold on a world @w'@ so long as
the set of priviliges in @w'@ contains those in @w@.
%
The definition of @Stable@ is derived precisely from the ways in which
the file system API may modify the current set of privileges:
%
\begin{code}
  bound Stable i = \x y w w' -> 
   i w => ( eqP w () w' || copyP y w x w' || derivP y w x w') 
       => i w'
\end{code}
%
We thus parameterize @copyRec@ by a predicate @i@, bounded by @Stable i@, 
which precisely describes the possible world transformations under which 
@i@ should be stable:
%
\begin{code}
  copyFrame i s d = \w -> i w && copySpec s d w

  copyRec :: (Stable i)
          => Bool -> s:FH -> d:FH
          -> RIO<copyFrame i s d, \_ _ w -> copyFrame i s d w> () 
\end{code}              
%
Now, we can verify @copyRec@'s body, as
at the recursive call that appears in the last line of the implementation,
@i@ is instantiated with 
%
@\w -> copySpec s d w@.

