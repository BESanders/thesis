\section{Introduction} \label{sec:intro}

Must program verifiers always choose between expressiveness
and automation?
%
On the one hand, tools based on higher order logics
and full dependent types impose no limits on expressiveness,
but require user-provided (perhaps, tactic-based) proofs.
%
On the other hand, tools based on Refinement Types~\cite{Rushby98,pfenningxi98}
trade expressiveness for automation. For example, the refinement types
%
\begin{code}
  type Pos     = {v:Int | 0 < v}
  type IntGE x = {v:Int | x <= v}
\end{code}
%
specify subsets of @Int@ corresponding to values
that are positive or larger than some other value @x@
respectively. By limiting the refinement predicates to
SMT-decidable logics~\cite{NelsonOppen}, refinement type
based verifiers eliminate the need for explicit proof terms,
and thus automate verification.

% We can specify contracts like pre- and post-conditions by
% suitably refining the input and output types of functions.

This high degree of automation has enabled the
use of refinement types for a variety of verification
tasks, ranging from array bounds checking~\cite{LiquidPLDI08},
termination and totality checking~\cite{LiquidICFP14},
protocol validation~\cite{GordonTOPLAS2011,FournetCCS11},
and securing web applications~\cite{SwamyOAKLAND11}.
%
Unfortunately, this automation comes at a price.
To ensure predictable and decidable type checking, we must
limit the logical formulas appearing in specification types
to decidable (typically quantifier free) first order theories,
thereby precluding \emph{higher-order} specifications that
are essential for \emph{modular} verification.

In this paper, we introduce \emph{Bounded Refinement Types}
which reconcile expressive higher order specifications
with automatic SMT based verification. Our approach
comprises two key ingredients.
%
Our first ingredient is a mechanism, developed by \cite{vazou13},
for \emph{abstracting} refinements over type signatures.
This mechanism is the analogue of parametric polymorphism
in the refinement setting: it increases expressiveness by
permitting generic signatures that are universally quantified
over the (concrete) refinements that hold at different
call-sites.
%
However, we observe that for modular verification, we
additionally need to \emph{constrain} the abstract refinement
parameters, typically to specify fine grained dependencies
between the parameters.
%
Our second ingredient provides a technique
for enriching function signatures with \emph{subtyping constraints}
(or \emph{bounds}) between abstract refinements that must be
satisfied by the concrete refinements at instantiation.
Thus, constrained abstract refinements are the analogue of bounded
quantification in the refinement setting and in this paper, we
show that this simple technique proves to be remarkably effective.

%% We preserved decidability of checking and
%% inference by encoding abstractly refined
%% types with uninterpreted functions obeying the
%% decidable axioms of congruence~\cite{NelsonOppen}.

\begin{itemize}
\item
First, we demonstrate via a series of short examples how bounded refinements
enable the specification and verification of diverse textbook higher order
abstractions that were hitherto beyond the scope of decidable refinement
typing~(\S~\ref{sec:overview}).

\item
Second, we formalize bounded types and show how bounds are translated
into ``ghost'' functions, reducing type checking and inference to the
``unbounded'' setting of~\cite{vazou13}, thereby ensuring that checking
remains decidable. Furthermore, as the bounds are Horn constraints, we
can directly reuse the abstract interpretation of Liquid Typing~\citep{LiquidPLDI08}
to automatically infer concrete refinements at instantiation
sites~(\S~\ref{sec:check}).

\item
Third, to demonstrate the expressiveness of bounded refinements, we
use them to build a typed library for extensible dictionaries, to
then implement a relational algebra library on top of those
dictionaries, and to finally build a library for type-safe
database access~(\S~\ref{sec:database}).

\item
Finally, we use bounded refinements to develop a \emph{Refined State Transformer}
monad for stateful functional programming, based upon Filli\^atre's method
for indexing the monad with pre- and post-conditions~\citep{Filliatre98}.
%
We use bounds to develop branching and looping combinators whose types
signatures capture the derivation rules of Floyd-Hoare logic, thereby
obtaining a library for writing verified stateful computations~(\S~\ref{sec:state}).
%
We use this library to develop a refined IO monad that tracks capabilities
at a fine-granularity, ensuring that functions only access specified
resources~(\S~\ref{sec:files}).
\end{itemize}

We have implemented Bounded Refinement Types in \toolname~\citep{LiquidICFP14}.
The source code of the examples (with slightly more verbose concrete syntax)
is at \cite{liquidhaskellgithub}.
%
While the construction of these verified abstractions is possible with full
dependent types, bounded refinements
%
keep checking automatic and decidable,
%
use abstract interpretation to automatically synthesize
refinements (\ie pre- and post-conditions and loop invariants),
and most importantly
%
enable retroactive or \emph{gradual} verification as when
erase the refinements, we get valid programs in the
host language~(\S~\ref{sec:related}).
%
%In~(\S~\ref{sec:related}) we further compare this work with dependent type systems.
%
Thus, bounded refinements point a way towards keeping our automation, and
perhaps having expressiveness too.
%
%%% Local Variables:
%%% mode: latex
%%% TeX-master: "main"
%%% End:
