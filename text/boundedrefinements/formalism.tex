\section{Formalism}\label{sec:check}

Next we formalize Bounded Refinement Types by defining
a core calculus \boundedcorelan and showing how it can
be reduced to \corelan, the core language of Abstract
Refinement Types~\ref{chapter:abstractrefinements}.
%
We start by defining the syntax
and semantics of \corelan~(\S~\ref{sec:syntax-corelan})
and the syntax of \boundedcorelan~(\S~\ref{sec:syntax-boundedcorelan}).
%
Next, we provide a translation from \boundedcorelan to
\corelan ~(\S~\ref{sec:translation}).
%
Then, we prove soundness by showing that our translation
is semantics preserving~(\S~\ref{sec:soundness}).
%
Finally, we describe how type inference remains
decidable in the presence of bounded refinements~(\S~\ref{sec:infer}).


\subsection{Syntax of \corelan}\label{sec:syntax-corelan}

\newcommand{\ra}[1]{\renewcommand{\arraystretch}{#1}}
\ra{0.9}
\begin{figure}[t!]
\centering
\captionsetup{justification=centering}
$$
\begin{array}{rrcl}
\emphbf{Expressions} \quad 
  & e & ::=     & x 
                  \spmid c 
                  \spmid \efunt{x}{\rtyp}{e} 
                  \spmid \eapp{e}{x}      \spmid
                  \elet{x}{e}{e}{\rtyp}   \\
  &   &  \spmid & \etabs{\tvar}{e}  
                  \spmid \etapp{e}{\rtyp} 
                  \epabs{\rvar}{\rtyp}{e}
                  \spmid \epapp{e}{\constraint}     \\[0.03in] 
  
\emphbf{Constants} \quad
  & c 
  & ::= 
  & \etrue \spmid \efalse \spmid \ecrash \spmid
   0 \spmid 1 \spmid -1 \spmid \dots
  \\[0.05in] 
  
\emphbf{Parametric Refinements} \quad 
& \constraint & ::= & \reft 
                        \spmid \efunt{x}{b}{\constraint}  \\[0.03in]


\emphbf{Predicates} \quad 
  & \creft & ::= & c \spmid \lnot \creft 
                   \spmid \creft = \creft 
                   \spmid % \creft < \creft 
                   \dots  \\[0.05in] 

\emphbf{Atomic Refinements} \quad 
  & \areft & ::= & \creft 
                   \spmid \rvapp{\rvar}{x} \\[0.03in] 

\emphbf{Refinements} \quad 
  & \reft & ::= & \areft 
                  \spmid \areft \wedge \reft 
                  \spmid \areft \Rightarrow \reft \\[0.03in] 

\emphbf{Basic Types} \quad 
  & b 
  & ::= & \texttt{Int}
          \spmid \texttt{Bool}
          \spmid \texttt{a}    \\[0.03in]

%% \emphbf{Refined Basic Types} \quad 
%%   & r 
%%   & ::= 
%%   % &      \tpref{b}{\areft}{\reft} 
%%   \\[0.05in]

\emphbf{Types} \quad 
  & \rtyp
  & ::=      & \tref{b}{\reft} \spmid
  \trfun{x}{\rtyp}{\rtyp}{\reft} \\[0.03in]

\emphbf{Bounded Types} \quad 
  & \bt
  & ::= & \rtyp \\[0.05in]

\emphbf{Schemata} \quad 
  & \sigma
  & ::= & \bt
          \spmid \ttabs{\tvar}{\sigma}
          \spmid \tpabs{\rvar}{\rtyp}{\sigma} \\[0.03in]
\end{array}
$$
\caption{Stratified Syntax of \texorpdfstring{\corelan}{LamB}.}
\label{fig:boundedrefinements:syntax}
\end{figure}


%%%% \begin{figure}[t!]
%%%% $$
%%%% \begin{array}{rrcl}
%%%% \centering
%%%% \emphbf{Expressions} \quad 
%%%%   & e & ::=
%%%%   &  \dots \spmid  \elet{x}{e}{e}{\rtyp}
%%%%   \\[0.05in]
%%%% 
%%%% \end{array}
%%%% $$
%%%% \caption{\textbf{Syntactic extension from \corelan to \letcorelan}}
%%%% \label{fig:letsyntax}
%%%% \end{figure}




We build our core language on top of \corelan, the language
of Abstract Refinement Types\ref{chapter:abstractrefinements}.
%
Figure~\ref{fig:boundedrefinements:syntax} summarizes the syntax of \corelan,
a polymorphic $\lambda$-calculus extended with abstract
refinements.
%
For an easier transition to the syntax of Bounded Refinement Types, 
we rewrite the syntax of \corelan as initially presented in Figure~\ref{fig:abstractrefinements:syntax}
by stratifying type schemata to include bounded types, that for now, 
are plain types.


\mypara{The Expressions} of \corelan include the usual variables $x$,
primitive constants $c$, $\lambda$-abstraction $\efunt{x}{\rtyp}{e}$,
application $\eapp{e}{e}$,
let bindings $\elet{x}{e}{e}{\rtyp}$,
type abstraction $\etabs{\alpha}{e}$,
and type application $\etapp{e}{\rtyp}$.
(We add let-binders to \corelan
from Figure~\ref{fig:abstractrefinements:syntax} as they can be reduced to $\lambda$-abstractions
in the usual way).
%
The parameter $\rtyp$ in the type application is a \emph{refinement
type}, as described shortly.  Finally, \corelan includes refinement
abstraction $\epabs{\rvar}{\rtyp}{e}$, which introduces a refinement
variable $\rvar$ (with its type $\rtyp$), which
can appear in refinements inside $e$, and the corresponding refinement
application $\epapp{e}{\constraint}$ that substitutes an abstract refinement
with the parametric refinement $\constraint$, \ie
refinements $\reft$ closed under lambda abstractions.
%%%\nv{Suggestion: replace $\epapp{e}{e}$ with $\epapp{e}{\phi}$ in the syntax}
%%%\nv{If abstract refinements can be instantiated with arbitrary expressions,}
%%%\nv{then all the careful syntax that ensures that refinements r come from}
%%%\nv{decidable theory is useless.}
%%%\nv{In Abstract refinement paper we had}
%%%\nv{arbitrary expressions as refinements thus we did not have this problem}
%%%\nv{It seems that bounds $\phi$, ie x-parametric refinements, is a good}
%%%\nv{candidate to substitute abstract refinements, ie replace $\epapp{e}{e}$}
%%%\nv{with $\epapp{e}{\phi}$}
%%%\nv{Question: is this a coincidence?}

\mypara{The Primitive Constants} of \corelan include
\etrue, \efalse, @0@, @1@, @-1@, \etc. In addition, we include a
special untypable constant \ecrash that models ``going wrong''.
Primitive operations return a crash when invoked with inputs
outside their domain, \eg when @/@ is invoked with @0@ as the
divisor or when an @assert@ is applied to \efalse.

\mypara{Atomic Refinements} $\areft$ are either concrete or abstract refinements.
%
A \emph{concrete refinement} \creft is a boolean valued expression
(such as a constant, negation, equality, \etc)
drawn from a \emph{strict subset} of the language of expressions
which includes only terms that
%
(a)~neither diverge nor crash and
%
(b)~can be embedded into an SMT decidable logic including
%
the quantifier free theory of linear arithmetic and uninterpreted
functions~\cite{LiquidICFP14}.
%
An \emph{abstract refinement} $\rvapp{\pi}{x}$ is an application of
a refinement variable $\pi$ to a sequence of program variables.
%
A \emph{refinement} \reft is either a conjunction or
implication of atomic refinements.
%
To enable inference, we only allow implications to appear within
bounds $\constraint$ (\S~\ref{sec:infer}).


\mypara{The Types of} \corelan, written $\rtyp$, include basic types,
dependent functions and schemata quantified over type and refinement
variables $\tvar$ and $\rvar$ respectively.
%
A basic type is one of $\tbint$, $\tbbool$, or a type
variable $\alpha$.
%
A refined type $\rtyp$ is either a refined basic type $\tref{b}{\reft}$,
or a dependent function type $\trfun{x}{\rtyp}{\rtyp}{\reft}$ where
the parameter $x$ can appear in the refinements of the output type.
%
(We include refinements for functions, as refined type variables can be
replaced by function types. However, typechecking ensures these refinements
are trivially true).
%
In \corelan bounded types $\bt$ are just a synonym for types $\rtyp$.
%
Finally, schemata quantify bounded types over type
and refinement variables.

\subsection{Syntax of \boundedcorelan}\label{sec:syntax-boundedcorelan}

\begin{figure}[t!]
\centering
\captionsetup{justification=centering}
$$
\begin{array}{rrcl}
\centering

\emphbf{Bounded Types} \quad 
  & \bt         & ::= & \rtyp 
                        \spmid \tconstraint{\constraint}{\bt} \\[0.03in]

\emphbf{Expressions} \quad 
  & e           & ::= & \dots 
                        \spmid \econstraint{\constraint}{e} 
                        \spmid \econstantconstraint{e}{\constraint} \\
\end{array}
$$
\caption{Extending Syntax of \corelan to \boundedcorelan.} 
\label{fig:boundedsyntax}
\end{figure}
\ra{1.0}

Figure~\ref{fig:boundedsyntax} shows how we obtain the syntax for
\boundedcorelan by extending the syntax of \corelan with
\emph{bounded} types.

\mypara{The Types} of \boundedcorelan extend those of \corelan with
bounded types $\bt$, which are the types $\rtyp$ guarded by bounds
$\constraint$.

%% {HERE: Colin:How does $\phi$ influence or  interact with $\rho$?  It’s not clear here.}
\mypara{The Expressions} of \boundedcorelan extend those of \corelan
with \emph{abstraction} over bounds $\econstraint{\constraint}{e}$ and
\emph{application} of bounds $\econstantconstraint{e}{\constraint}$.
%
Intuitively, if an expression $e$ has some type $\bt$
then $\econstraint{\constraint}{e}$ has the type
$\tconstraint{\constraint}{\bt}$.
%
We include an explicit bound application form
$\econstantconstraint{e}{\constraint}$ to simplify
the formalization; these applied bounds are automatically
synthesized from the type of $e$, and are of the form
$\overline{\efunt{x}{\bt}{}}{\etrue}$.

\mypara{Notation.}
We write
$b$,
%$\tref{b}{\reft}$ and
$\tpp{b}{\rvapp{\pi}{x}}$,
$\tpref{b}{\rvapp{\pi}{x}}{\reft}$
to abbreviate
$\tref{b}{\etrue}$,
$\tref{b}{\rvapp{\pi}{x}\ \vref}$,
$\tref{b}{\reft \wedge \rvapp{\pi}{x}\ \vref}$
%% $\tpref{b}{\true}{\true}$,
%% $\tpref{b}{\true}{\reft}$, and
%% $\tpref{b}{\areft}{\true}$
respectively.
We say a type or schema is \emph{non-refined} if all the
refinements in it are $\true$.
%
We get the \textit{shape} of a type $\rtyp$ (\ie the System-F type)
by the function $\toshape{\rtyp}$ defined:
%
\begin{align*}
\toshape{\tref{b}{\reft}} \defeq &\ b \\
\toshape{\trfun{x}{\rtyp_1}{\rtyp_2}{\reft}} \defeq &\ \toshape{\rtyp_1} \rightarrow \toshape{\rtyp_2}
\end{align*}

\subsection{Translation from \boundedcorelan to \corelan}
\label{sec:translation}

Next, we show how to translate a term from \boundedcorelan to
one in \corelan. We assume, without loss of generality that the
terms in \boundedcorelan are in Administrative Normal Form
(\ie all applications are to variables).

\mypara{Bounds Correspond To Functions} that explicitly
``witness'' the fact that the bound constraint holds at a
given set of ``input'' values.
%
That is we can think of each bound as a universally quantified
relationship between various (abstract) refinements; by ``calling''
the function on a set of input values $x_1,\ldots,x_n$, we get
to \emph{instantiate} the constraint for that particular set
of values.


\mypara{Bound Environments} \benv are used by our translation
to track the set of
bound-functions (names) that are in scope at each program point.
%
These names are distinct from the regular program variables that
will be stored in Variable Environments \cenv.
%
We give bound functions distinct names so that they cannot appear
in the regular source, only in the places where calls are inserted
by our translation.
%
The translation ignores refinements entirely; both environments
map their names to their non-refined types.

\mypara{The Translation is formalized} in
Figure~\ref{fig:translation} via a
relation $\txexpr{\cenv}{\benv}{e}{e'}$
that translates the expression
$e$ in $\boundedcorelan$ into
$e'$ in $\corelan$.
%
Most of the rules in figure~\ref{fig:translation}
recursively translate the sub-expressions.
%
Types that appear inside expressions are syntactically restricted to
not contain bounds,
thus types inside expressions do not require translation.
%
Here we focus on the three interesting rules:

\begin{enumerate}
%
\item \emphbf{At bound abstractions} $\econstraint{\constraint}{e}$
 we convert the bound $\constraint$ into a bound-function
 parameter of a suitable type,
%
\item \emphbf{At variable binding sites} \ie $\lambda$- or let-bindings,
 we \emph{use} the bound functions to \emph{materialize} the
 bound constraints for all the variables in scope after the binding,
%
\item \emphbf{At bound applications} $\econstantconstraint{e}{\constraint}$
 we \emph{provide} regular functions that witness that the bound constraints hold.
%
\end{enumerate}

\begin{figure}[t!]
$$
\begin{array}{rrcl}
\centering
 \emphbf{Variable Environment} \quad 
   & \cenv & ::=
   & \emptyset \spmid  \EXT{\cenv}{x}{\utyp}
   \\[0.05in]
 
 \emphbf{Bound Environment} \quad 
   & \benv & ::=
   & \emptyset \spmid  \EXT{\benv}{x}{\utyp}
\end{array}
$$

\judgementHead{Translation}{\txexpr{\cenv}{\benv}{e}{e}}
$$
\inference{
}{
	\txexpr{\cenv}{\benv}{x}{x}
}[\txVar]
\qquad
\inference{
}{
	\txexpr{\cenv}{\benv}{c}{c}
}[\txCon]
$$

$$
\inference{
	\cenv' = \EXT{\cenv}{x}{\toshape{\rtyp}} && \txexpr{\cenv'}{\benv}{e}{e'} 
}{
	\txexpr{\cenv}{\benv}{\efunt{x}{\rtyp}{e}}{\efunt{x}{\rtyp}{\closure{\cenv'}{\benv}{e'}{x\colon\rtyp}}}
}[\txFun]
$$
 
$$
\inference{
	\txexpr{\cenv}{\benv}{e_x}{e_x'} && \cenv' = \EXT{\cenv}{x}{\toshape{\rtyp}} &&
	\txexpr{\cenv'}{\benv}{e}{e'}
}{
	\txexpr{\cenv}{\benv}{\elet{x}{e_x}{e}{\rtyp}}
	{\elet{x}{e_x'}{\closure{\cenv'}{\benv}{e'}}{\tau}{x\colon\rtyp}}
}[\txLet]
$$

$$
\inference{
	\txexpr{\cenv}{\benv}{e_1}{e_1'} &&
	\txexpr{\cenv}{\benv}{e_2}{e_2'}
}{
	\txexpr{\cenv}{\benv}{\eapp{e_1}{e_2}}{\eapp{e_1'}{e_2'}}
}[\txApp]
$$

$$
\inference{
	\txexpr{\cenv}{\benv}{e}{e'}
}{
	\txexpr{\cenv}{\benv}{\etabs{\alpha}{e}}{\etabs{\alpha}{e'}}
}[\txTAbs]
\
\inference{
	\txexpr{\cenv}{\benv}{e}{e'}
}{
	\txexpr{\cenv}{\benv}{\etapp{e}{\rtyp}}{\etapp{e'}{\rtyp}}
}[\txTApp]
$$

$$
\inference{
	\txexpr{\cenv}{\benv}{e}{e'}
}{
	\txexpr{\cenv}{\benv}{\epabs{\rvar}{\rtyp}{e}}{\epabs{\rvar}{\rtyp}{e'}}
}[\txPAbs]
$$
 
$$
\inference{
	\txexpr{\cenv}{\benv}{e_1}{e_2'} &&
	\txexpr{\cenv}{\benv}{e_1}{e_2'}
}{
	\txexpr{\cenv}{\benv}{\epapp{e_1}{e_2}}{\epapp{e_1'}{e_2'}}
}[\txPApp]
$$
 
$$
\inference{
	\text{fresh}\ f &&
	\txexpr{\cenv}{\benv, f\colon\toshape{\txbound{\constraint}}}{e}{e'}
}{
	\txexpr{\cenv}{\benv}{\econstraint{\constraint}{e} }{\efunt{f}{\txbound{\constraint}}{e'}}
}[\txCAbs]
$$

$$
\inference{
	\txexpr{\cenv}{\benv}{e}{e'}
}{
	\txexpr{\cenv}{\benv}{\econstantconstraint{e}{\constraint}}{\eapp{e'}{\ctofun{\constraint}}}
}[\txCApp]
$$
%%\nv{Note that basic and function types (\ie $\tau$) cannot have}
%%\nv{constraints,thus they do not require translation!}
%%\rj{I don't understand the above comment}
%%\nv{the above comment says that if type application had}
%%\nv{type schema instead of t (basic or function types only)}
%%\nv{then bounds could appear in the types inside expressions}
%%\nv{so, our translation would have to translate types too}
%%\nv{ie, e [t] translates to e' [t']}
\caption{Translation Rules from \boundedcorelan to  \corelan.}
\label{fig:translation}
\end{figure}



\mypara{1.  Rule~\txCAbs} translates bound abstractions
$\econstraint{\phi}{e}$ into a plain $\lambda$-abstraction.
%
In the translated expression $\efunt{f}{\txbound{\constraint}}{e'}$
the bound becomes a function named $f$ with type
$\txbound{\constraint}$ defined:
%
\begin{align*}
\txbound{\efunt{x}{b}{\constraint}} \defeq & \tfun{x}{b}{\txbound{\constraint}}\\
\txbound{r} \defeq & \tref{\tbbool}{r}
\end{align*}
%
That is, $\txbound{\constraint}$ is a function type whose
final output carries the refinement corresponding to
the constraint in $\constraint$.
%
Note that the translation generates a fresh name $f$ for
the bound function (ensuring that it cannot be used in
the regular code) and saves it in the bound environment
$\benv$ to let us materialize the bound constraint when
translating the body $e$ of the abstraction.

\mypara{2. Rules~\txFun and~\txLet} materialize bound
constraints at variable binding sites ($\lambda$-abstractions
and let-bindings respectively).
%
%Even though application  of the axiomatic functions is required at specific places
%
If we view the bounds as universally quantified constraints
over the (abstract) refinements, then our translation exhaustively
and eagerly \emph{instantiates} the constraints at each point that
a new binder is introduced into the variable environment, over all
the possible candidate sets of variables in scope at that point.
%
The instantiation is performed by $\closure{\cenv}{\benv}{e}{x:\rtyp}$
%
%% \begin{figure}[t]
$$\begin{array}{rcl}
\closure{\cenv}{\benv}{e}{\tbind{x}{\utyp}}
  & \defeq & {\wraplet{e}{\cands{\cenv}{\benv}}} \\[0.05in]
\wraplet{e}{\set{e_1,\ldots,e_n}}
  & \defeq & {\elett{t_1}{e_1}{\ldots} \elett{t_n}{e_n}{e}} \\
  &        & {\mbox{where $t_i$ are fresh \tbbool binders}} \\[0.05in]
\cands{\cenv}{\benv}
  & \defeq & \{ \  f\ \overline{x}  \ | \ \tbind{f}{\utyp} \leftarrow \benv
                  , \ \overline{\tbind{x}{\_}} \leftarrow \cenv,\
    \EXT{\cenv}{f}{\utyp} \vdash_B \tbind{ f\ \overline{x}}{\tbbool}\ \} \
\end{array}$$
%% \caption{\textbf{Materializing Bound Functions}}
%% \label{fig:closure}
%% \end{figure}
% described in Figure~\ref{fig:closure} that
The function takes the environments
$\cenv$ and $\benv$, an expression $e$ and a variable $x$ of type
$\rtyp$ and uses let-bindings to materialize all the bound
functions in $\benv$ that accept the variable $x$.
%
Here, $\cenv \vdash_B \tbind{e}{\utyp}$ is the standard typing
derivation judgment for the non-refined System F and so
we elide it for brevity.
%%%\nv{Note that with the above definitions,}
%%%\nv{$\closure{\cenv}{\benv}{e}{\tbind{x}{\utyp}}$}
%%%\nv{will instantiate the bounds to all combinations}
%%%\nv{of variables, not only the ones that contain $x$}
%%%\nv{so you get much much repetition of instantiations}
%%%\nv{which is not unsound, but redundant}

\mypara{3. Rule~\txCApp} translates bound applications
$\econstantconstraint{e}{\constraint}$ into plain $\lambda$
abstractions that witness that the bound constraints
hold.
%
That is, as within $e$, bounds are translated to a bound
function (parameter) of type $\txbound{\constraint}$, we
translate $\constraint$ into a $\lambda$-term that, via
subtyping must have the required type $\txbound{\constraint}$.
%
We construct such a function via $\ctofun{\constraint}$
that depends only on the \emph{shape} of the bound,
\ie the non-refined types of its parameters (and not
the actual constraint itself).
%%
\begin{align*}
\ctofun{\reft} \defeq & \true \\
\ctofun{\efunt{x}{b}{\constraint}} \defeq &  \efunt{x}{b}{\ctofun{\constraint}}
\end{align*}
%
This seems odd: it is simply a constant function, how
can it possibly serve as a bound? The answer is that
subtyping in the translated \corelan term will verify
that in the context in which the above constant function
is created, the singleton $\true$ will indeed carry
the refinement corresponding to the bound constraint,
making this synthesized constant function a valid
realization of the bound function.
%
Recall that in the example @ex2@ of the overview (\S~\ref{sec:overview:implementation})
the subtyping constraint that decides is the constant $\true$
is a valid bound reduces to the equation \ref{vc:ex2}
that is a tautology.
%% \rj{ex2 EXAMPLE HERE?} \ref{vc:ex2}

\subsection{Soundness}\label{sec:soundness}

\begin{comment}
Definition of \erase{\dot}:

\begin{align*}
\erase{x}                      &= x \\
\erase{c}                      &= c\\
\erase{\eapp{e_1}{e_2}}        &= \eapp{\erase{e_1}}{\erase{e_2}}\\
\erase{\etabs{\alpha}{e}}      &= \etabs{\alpha}{\erase{e}}\\
\erase{\etapp{e}{\tau}}        &= \etapp{\erase{e}}{\erase{\tau}}\\
\erase{\epabs{\rvar}{\tau}{e}} &= \epabs{\rvar}{\erase{\tau}}{\erase{e}}\\
\erase{\epapp{e_1}{e_2}}           &= \epapp{\erase{e_1}}{\erase{e_2}}\\
%
\erase{\elet{x}{e_x}{e}{\tau}} &= \elet{x}{\erase{e_x}}{\erase{e}}{\erase{\tau}}\\
%
\erase{\econstraint{\phi}{e}}  &= \erase{e}\\
\erase{\econstantconstraint{e}{\phi}} &= \erase{e}
\end{align*}

\begin{definition}% [Operational]
$ e\boundedgoestostar{c} \Leftrightarrow {\erase{e}}\goestostar{c} $.
\end{definition}

The below does not hold:
\begin{lemma*}[Operational]%[Equivalence of Operational]
\label{theorem:operational}
If
   $\vdash_{B} e : \tau$ and
   $\txexpr{\emptyset}{\emptyset}{e}{e'}$
then
   $\erase{e} \goestostar{c} \Leftrightarrow {e'} \goestostar{c} $.
\end{lemma*}

Counterexample e = \{φ} =>  0

\end{comment}

\mypara{The Small-Step Operational Semantics} of \boundedcorelan
are defined by extending a similar semantics for \corelan
which is a standard call-by-value calculus where abstract
refinements are boolean valued functions.
%
Let $\stepcore$ denote the transition relation defining
the operational semantics of \corelan and \tclos{\stepcore}
denote the reflexive transitive closure of $\stepcore$.
%
We obtain the transition relation $\stepboundedcore$:
%
\begin{align*}
\econstantconstraint{(\econstraint{\constraint}{e})}{\constraint} &\boundedgoesto{e} &
e  & \stepboundedcore e', \text{if}\ {e \stepcore e'}
\end{align*}
%%%$$
%%%\inference{
%%%}{
%%%\econstantconstraint{(\econstraint{\constraint}{e})}{\constraint} \boundedgoesto{e}
%%%}[\rtobound]
%%%\qquad \qquad
%%%\inference{e \stepcore e'}
%%%          {e \stepboundedcore e'}[\rulename{O-Oth}]
%%%$$
%
Let $\boundedgoestostar{}$ denote the reflexive transitive
closure of $\stepboundedcore$.

\mypara{The Translation is Semantics Preserving} in the sense that
if a source term $e$ of $\boundedcorelan$ reduces to a constant
then the translated variant of $e'$ also reduces to the same
constant (as show in~\citep{vazou15techrep}):

\begin{lemma*}[Semantics Preservation]\ifextended{[Semantics Preservation]}{}
\label{theorem:operational}
If $\txexpr{\emptyset}{\emptyset}{e}{e'}$ and
   $e \boundedgoestostar{c}$
then $e' \goestostar{c}$.
\end{lemma*}

%%\rj{what about the extra let-bindings? techrep.}
%%
\ifextended{
\begin{proof}
By assumption, there exists a sequence
$e \equiv e_1 \boundedgoesto{e_2} \boundedgoesto{} \dots
\boundedgoesto{e_n\equiv c} $.
%
Let $i$ be the largest index in which rule \rtobound was applied.
%
Then, for some $\phi$ and $\phi'$,
$e_i$ contains a sub-expression of the form
$\econstantconstraint{(\econstraint{\phi}{e_i^0})}{\phi'}$.
%
Let $e_i^1$ be the expression we get if we replace
$\econstantconstraint{(\econstraint{\phi}{e_i^0})}{\phi'}$
with $e_i^0$ in $e_i$.
%
By the way we choose $i$, there exist a sequence
$e_i^1 \goestostar{c}$.

Let $e_i^2$
be the expression we get if we replace
$\econstantconstraint{(\econstraint{\phi}{e_i^0})}{\phi'}$
with $\eapp{(\efunt{f}{\txbound{\phi}}{e_i^0})}{(\ctofun{\phi'})}$ in $e_i$.
%
Then, since $f$ does not appear in $e_i^0$,
$e_i^2 \goestostar{c}$.
%
Finally,
let $g \defeq \ctofun{\phi'}$, then
by the definition of
$\ctofun{\cdot}$
we have that  $\forall e_1 \dots e_n$
if
there exists a type $\tau$ such that
$\emptyset \vdash g \ e_1 \dots e_n : \tau $,
then $g \ e_1 \dots e_n \goestostar{true}$.
%
Thus, for any expression,
if $e \goestostar{c}$, then $\elett{t}{f \ e_1 \dots e_n}{e}\goestostar{c}$

From the above, by the way we choose $i$ we have that
there exists a sequence
%
$\txex{e_i} \hookrightarrow \dots \hookrightarrow {c}$.

Since $n$ is finite, we iteratively apply the above procedure to
$e \equiv e_1\boundedgoesto{} \dots \boundedgoesto{} \txex{e_i}\hookrightarrow \dots \hookrightarrow {c}$.
until we get the sequence $ {\txex{e}}\goestostar{c}$.
\end{proof}}{}

\mypara{The Soundness of \boundedcorelan} follows by combining
the above Semantics Preservation
Lemma % ~\ref{theorem:operational}
with the soundness of \corelan.

\mypara{To Typecheck a \boundedcorelan program} $e$ we translate it
into a \corelan program $e'$ and then type check $e'$ using the rules of Figure~\ref{fig:rules}; 
if the latter check is safe, then we are guaranteed that the source term $e$ will
not crash:

\begin{theorem*}[Soundness]
\label{theorem:bounded}
If $\txexpr{\emptyset}{\emptyset}{e}{e'}$ and
   $\hastype{\emptyset}{e'}{\sigma}$
then $e \not \boundedgoestostar{\ecrash}$.
\end{theorem*}

\subsection{Inference}\label{sec:infer}

A critical feature of bounded refinements is that we can
automatically synthesize instantiations of the abstract
refinements by:
%
(1)~generating templates corresponding to the unknown types
    where fresh variables $\kvar$ denote the unknown refinements
    that an abstract refinement parameter $\rvar$ is instantiated
    with,
(2)~generating subtyping constraints over the resulting templates,
    and
(3)~solving the constraints via abstract interpretation.

\mypara{Inference Requires Monotonic Constraints.}
Abstract interpretation only works if the constraints
are \emph{monotonic}~\citep{cousotcousot77}, which in this case
means that the $\kvar$ variables, and correspondingly,
the abstract refinements $\rvar$ must only appear in
\emph{positive} positions within refinements (\ie not
under logical negations).
%
The syntax of refinements shown in Figure~\ref{fig:boundedrefinements:syntax}
violates this requirement via refinements of the
form $\rvapp{\rvar}{x} \Rightarrow \reft$.
%
%% (Note that the syntax otherwise only allows for positive
%% occurrences as $\neg (\rvapp{\rvar}{x})$ is disallowed.)

\mypara{We restrict implications to bounds} \ie prohibit
them from appearing elsewhere in type specifications.
%
Consequently, the implications only appear in the
\emph{output} type of the (first order) ``ghost''
functions that bounds are translated to.
%
The resulting subtyping constraints only have
\emph{implications inside super-types}, \ie as:
$$
\isSubType{\Gamma}{\stref{b}{\areft}}{\stref{b}{\areft_1 \Rightarrow \dots \Rightarrow \areft_{n} \Rightarrow\areft_q}}
$$
%
By taking into account the semantics of subtyping, we can
push the antecedents into the environment, \ie transform
the above into an equivalent constraint in the form:
$$
\isSubType{
\EXTT{
 \EXTT{\Gamma}{x_1}{\sxref{b_1}{\areft_1'}{x_1}},\dots
}{x_n}{\sxref{b_n}{\areft_n'}{x_n}}
}
{\stref{b}{\areft'}}
{\stref{b}{\areft_q'}}
$$
where all the abstract refinements variables $\rvar$
(and hence instance variables $\kvar$) appear positively,
ensuring that the constraints are monotonic, hence permitting
inference via Liquid Typing~\citep{LiquidPLDI08}.

\begin{comment}

\subsection{Semantics of \corelan}\label{sec:semantics-corelan}

\begin{figure*}[ht!]
\judgementHead{Well-Formedness}{\isWellFormed{\Gamma}{\sigma}}

$$\inference
    {\hastype{\Gamma, \vref:b}{\reft}{\tbbool}}
    {\isWellFormed{\Gamma}{\tref{b}{\reft}}}
    [\wtBase]
\qquad
\inference
    {
	%\hastype{\Gamma, v:\tfun{x}{\rtyp_x}{\rtyp}}{e}{\tbbool} &&
	\hastype{\Gamma}{\reft}{\tbbool} &&
    \isWellFormed{\Gamma}{\rtyp_x} &&
	\isWellFormed{\Gamma, x:\rtyp_x}{\rtyp}
    }
    {\isWellFormed{\Gamma}{\trfun{x}{\rtyp_x}{\rtyp}{\reft}}}
    [\wtFun]
$$

$$
\inference
  {\isWellFormed{\Gamma, \rvar:\rtyp}{\sigma}}
  {\isWellFormed{\Gamma}{\tpabs{\rvar}{\rtyp}{\sigma}}}
  [\wtPred]
\quad
\inference
    {\isWellFormed{\Gamma}{\sigma}}
    {\isWellFormed{\Gamma}{\ttabs{\alpha}{\sigma}}}
    [\wtPoly]
$$

\medskip \judgementHead{Subtyping}{\isSubType{\Gamma}{\sigma_1}{\sigma_2}}

$$
\inference
   {(\inter{\Gamma} \Rightarrow \inter{\reft_1} 
                 \Rightarrow  \inter{\reft_2})
                 \ \text{is valid}}
   {\isSubType{\Gamma}{\tref{b}{\reft_1}}{\tref{b}{\reft_2}}}
   [\tsubBase]
\qquad
\inference{
	\isSubType{\Gamma}{\rtyp_2}{\rtyp_1} &
	\isSubType{\Gamma, x_2:{\rtyp_2}}{\SUBST{\rtyp_1'}{x_1}{x_2}}{\rtyp_2'}	
   }
   {\isSubType{\Gamma}
	  {\trfun{x_1}{\rtyp_1}{\rtyp_1'}{\reft_1}}
	  {\trfun{x_2}{\rtyp_2}{\rtyp_2'}{\true}}
}[\tsubFun]
$$


$$
\begin{array}{c}
\inference
   {\isSubType{\Gamma, \rvar:\rtyp}{\sigma_1}{\sigma_2}}
   {\isSubType{\Gamma}{\tpabs{\rvar}{\rtyp}{\sigma_1}}{\tpabs{\rvar}{\rtyp}{\sigma_2}}}
   [\tsubPred]
\qquad
\inference
   {\isSubType{\Gamma}{\sigma_1}{\sigma_2}}
   {\isSubType{\Gamma}{\ttabs{\alpha}{\sigma_1}}{\ttabs{\alpha}{\sigma_2}}}
   [\tsubPoly]
\end{array}
$$

\medskip \judgementHead{Type Checking}{$\hastype{\Gamma}{e}{\sigma}$}

$$
\begin{array}{cc}
\inference
  {  \hastype{\Gamma}{e}{\sigma_2} && \isSubType{\Gamma}{\sigma_2}{\sigma_1} 
  && \isWellFormed{\Gamma}{\sigma_1}
  }
  {\hastype{\Gamma}{e}{\sigma_1}}
  [\tsub]
& 
\inference
  {\hastype{\Gamma}{e_x}{\rtyp_x} && 
   \hastype{\Gamma, x:\rtyp_x}{e}{\rtyp} && 
   \isWellFormed{\Gamma}{\rtyp}
  }
  {\hastype{\Gamma}{\elet{x}{e_x}{e}{}}{\rtyp}}
  [\tlet]
\end{array}
$$
$$
\begin{array}{ccc}

\inference
  {x: \tref{b}{r} \in \Gamma}
  {\hastype{\Gamma}{x}{\tref{b}{\vref = x}}}
  [\tbase]

&

\inference
  {x:\rtyp \in \Gamma}
  {\hastype{\Gamma}{x}{\rtyp}} 
  [\tvariable]
&
\inference
  {}
  {\hastype{\Gamma}{c}{\tc{c}}}
  [\tconst]

\\[0.2in]

\label{tapp}
\inference
   {\hastype{\Gamma}{e_1}{\tfun{x}{\rtyp_x}{\rtyp}} 
   &&  \hastype{\Gamma}{e_2}{\rtyp_x}
   }
   {\hastype{\Gamma}{\eapp{e_1}{e_2}}{\SUBST{\rtyp}{x}{e_2}}}
   [\tapp]

&

\inference
   {\hastype{\Gamma, x:\rtyp_x}{e}{\rtyp} 
    && \isWellFormed{\Gamma}{\rtyp_x}
   }
   {\hastype{\Gamma}{\efunt{x}{\rtyp_x}{e}}{\tfun{x}{\rtyp_x}{\rtyp}}}
   [\tfunction]

& 

\inference
  {\hastype{\Gamma, \alpha}{e}{\sigma}}
  {\hastype{\Gamma}{\etabs{\alpha}{e}}{\ttabs{\alpha}{\sigma}}}
  [\tgen]


\\[0.2in]


\inference
    {\hastype{\Gamma}{e}{\tpabs{\rvar}{\rtyp}{\sigma}} && 
     \hastype{\Gamma}{\efunbar{x:\rtyp_x}{\reft'}}{\rtyp}
    }
    {\hastype{\Gamma}
             {\epapp{e}{\efunbar{x:\rtyp_x}{\reft'}}}
             {\rpinst{\sigma}{\rvar}{\efunbar{x:\rtyp_x}{\reft'}}}
    }
    [\tpinst]
&

\inference
    {\hastype{\Gamma, \rvar:\rtyp}{e}{\sigma} &&
     \isWellFormed{\Gamma}{\rtyp} 
    }
    {\hastype{\Gamma}{\epabs{\rvar}{\rtyp}{e}}{\tpabs{\rvar}{\rtyp}{\sigma}}}
    [\tpgen]

&
\inference
  {\hastype{\Gamma}{e}{\ttabs{\alpha}{\sigma}} && 
   \isWellFormed{\Gamma}{\rtyp}
  }
  {\hastype{\Gamma}{\etapp{e}{\tau}}{\SUBST{\sigma}{\alpha}{\rtyp}}}
  [\tinst]

\end{array}$$
\caption{\textbf{Static Semantics: Well-formedness, Subtyping and Type Checking}}
\label{fig:rules}
\end{figure*}



Figure \ref{fig:rules} summarizes the static semantics of \corelan
as described in ~\citep{vazou13}.
Unlike~\citep{vazou13} that syntactically separates concrete ($p$)
from abstract ($\rvapp{\rvar}{x}$) refinements,
here, for simplicity, we merge both concrete and abstract refinements to
atomic refinements $\areft$.

\mypara{A type environment} $\Gamma$ is a sequence of type bindings $x:\sigma$.
We use environments to define three kinds of judgments:

\begin{itemize}
\item{\emphbf{Well-formedness judgments} (\isWellFormed{\Gamma}{\sigma})}
state that a type schema $\sigma$ is well-formed under environment
$\Gamma$. That is, the judgment states that the refinements in $\sigma$
are boolean expressions in the environment $\Gamma$.

\item{\emphbf{Subtyping judgments} (\isSubType{\Gamma}{\sigma_1}{\sigma_2})}
state that the type schema $\sigma_1$ is a subtype of the type schema
$\sigma_2$ under environment $\Gamma$. That is, the judgment states that
when the free variables of $\sigma_1$ and $\sigma_2$ are bound to values
described by $\Gamma$, the values described by $\sigma_1$ are a subset
of those described by $\sigma_2$.

\item{\emphbf{Typing judgments} (\hastype{\Gamma}{e}{\sigma})} state that
the expression $e$ has the type schema $\sigma$ under environment $\Gamma$.
That is, the judgment states that when the free variables in $e$ are bound
to values described by $\Gamma$, the expression $e$ will evaluate to a value
described by $\sigma$.
\end{itemize}

\mypara{The Well-formedness rules}
check that the concrete and abstract refinements are indeed $\tbbool$-valued
expressions in the appropriate environment. The key rule is \wtBase, which
checks that the refinement $\reft$ is boolean.

\mypara{The Subtyping rules}
stipulate when the set of values described by schema $\sigma_1$ is subsumed
by (\ie contained within) the values described by $\sigma_2$.
The rules are standard except for \tsubBase, which reduces subtyping of
basic types to validity of logical implications, by translating the
refinements $r$ and the environment $\Gamma$ into logical formulas:
%
\begin{align*}
\inter{r}      & \defeq r &
\inter{\Gamma} & \defeq \bigwedge \{ \SUBST{r}{\vref}{x}\ |\ (x, \tref{b}{r}) \in \Gamma \}
\end{align*}
%
Recall that we ensure that the refinements $r$
belong to a decidable logic so that validity
checking can be performed by an off-the-self
SMT solver.

\mypara{Type Checking Rules}
are standard except for \tpgen and \tpinst, which
pertain to abstraction and instantiation of abstract refinements.
%
The rule \tpgen is the same as \tfunction: we simply check the body
$e$ in the environment extended with a binding for the refinement
variable $\rvar$.
%
The rule \tpinst checks that the concrete refinement is of the appropriate
(unrefined) type $\tau$, and then replaces all (abstract) applications of
$\rvar$ inside $\sigma$ with the appropriate (concrete) refinement $\reft'$
with the parameters $\overline{x}$ replaced with arguments at that application.
%
In~\cite{vazou13} we prove the following soundness result for \corelan
which states that well-typed programs cannot crash:

\begin{lemma*}[Soundness of \corelan~\cite{vazou13}]
\label{theorem:core}
If   $\hastype{\emptyset}{e}{\sigma}$
then $e \not \goestostar{\ecrash}$.
\end{lemma*}

\end{comment}
