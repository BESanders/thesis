\section{Syntax and Semantics}\label{sec:check}

Next, we present a core calculus \corelan that formalizes the notion
of abstract refinements. We start with the syntax (\S~\ref{sec:syntax}),
present the typing rules (\S~\ref{sec:typing}), show soundness 
via a reduction to contract calculi \cite{Knowles10,Greenberg11}
(\S~\ref{sec:soundness}), and inference via Liquid types (\S~\ref{sec:infer}).

\subsection{Syntax}\label{sec:syntax}

Figure~\ref{fig:syntax} summarizes the syntax of our core 
calculus \corelan which is a polymorphic $\lambda$-calculus 
extended with abstract refinements. 
%
We write 
$b$, 
$\tref{b}{\reft}$ and 
$\tpp{b}{\areft}$ 
to abbreviate 
$\tpref{b}{\true}{\true}$, 
$\tpref{b}{\true}{\reft}$, and
$\tpref{b}{\areft}{\true}$ respectively. 
We say a type or schema is \emph{non-refined} if all the 
refinements in it are $\true$. We write $\overline{z}$ 
to abbreviate a sequence $z_1 \ldots z_n$.


\mypara{Expressions}
\corelan\ expressions include the standard variables $x$, 
primitive constants $c$, $\lambda$-abstraction $\efunt{x}{\tau}{e}$,
application $\eapp{e}{e}$, type abstraction $\etabs{\alpha}{e}$,
and type application $\etapp{e}{\tau}$. The parameter $\tau$ in 
the type application is a \emph{refinement type}, as described shortly.
The two new additions to \corelan are the refinement abstraction
$\epabs{\rvar}{\tau}{e}$, which introduces a refinement variable 
$\rvar$ (together with its type $\tau$), which can appear in refinements
inside $e$, and the corresponding refinement application $\epapp{e}{e}$.
%
%where the argument, is of the form $\efun{\bar{x}{e}}$ which is
%an abbreviation for $\efun{x_1 \ldots x_n}{e}$.
%which is of the form $\ptype{\bar{\tau}}$ which is an 
%abbreviation for $\ptype{tau_1 \rightarrow \ldots \tau_n}$,
%where each $\tau_i$ is a simple (non-function) type.

\mypara{Refinements}
A \emph{concrete refinement} \reft is a boolean valued expression \reft 
drawn from a strict subset of the language of expressions which
includes only terms that 
(a)~neither diverge nor crash, and 
(b)~can be embedded into an SMT decidable refinement logic including 
the theory of linear arithmetic and uninterpreted functions.
%
An \emph{abstract refinement} \areft is a conjunction of refinement
variable applications of the form $\rvapp{\pi}{e}$.

\mypara{Types and Schemas}
The basic types of \corelan include the base types $\tbint$ and $\tbbool$
and type variables $\alpha$. An \emph{abstract refinement type} $\tau$ is 
either a basic type refined with an abstract and concrete refinements,
$\tpref{b}{\areft}{\reft}$, or 
a dependent function type where the parameter $x$ can appear in the 
refinements of the output type. 
We include refinements for functions, as refined type variables can be 
replaced by function types. However, typechecking ensures these refinements
are trivially true.
%
%type application
%Type application consists of a type constructor, its type arguments
%and its refinement arguments 
%that are used to describe properties between its elements.
%
Finally, types can be quantified over refinement variables and type 
variables to yield abstract refinement schemas.


\begin{figure}[t!]
\centering
$$
\begin{array}{rrcl}
\emphbf{Expressions} \quad 
  & e 
  & ::= 
  &      x 
  \spmid c 
  \spmid \efunt{x}{\tau}{e} 
  \spmid \eapp{e}{e} 
  \spmid \etabs{\alpha}{e} 
  \spmid \etapp{e}{\tau} 
  \spmid \epabs{\rvar}{\tau}{e}
  \spmid \epapp{e}{e} 
  \\[0.05in] 

\emphbf{Abstract Refinements} \quad 
  & \areft 
  & ::= 
  &      \true 
  \spmid \areft \land \rvapp{\rvar}{e}
  \\[0.05in] 

\emphbf{Basic Types} \quad 
  & b 
  & ::= 
  &      \tbint
  \spmid \tbbool
  \spmid \alpha
  \\[0.05in]

\emphbf{Abstract Refinement Types} \quad 
  & \tau 
  & ::= 
  &      \tpref{b}{\areft}{\reft} 
  \spmid \trfun{x}{\tau}{\tau}{\reft}
  \\[0.05in]

\emphbf{Abstract Refinement Schemas} \quad 
  & \sigma
  & ::= 
  &      \tau 
  \spmid \ttabs{\alpha}{\sigma}
  \spmid \tpabs{\rvar}{\tau}{\sigma}
  \\[0.05in]
\end{array}
$$
\caption{\textbf{Syntax of Expressions, Refinements, Types and Schemas}}
\label{fig:syntax}
\end{figure}

\subsection{Static Semantics}\label{sec:typing}

\begin{figure}[p]
\centering
\captionsetup{justification=centering}
\judgementHead{Well-Formedness}{\isWellFormed{\Gamma}{\sigma}}

$$\begin{array}{ccc}
\inference
  {}
  {\isWellFormed{\Gamma}{\true(\vref)}}
  [\wtTrue]
&
\quad
&
\inference
    {\isWellFormed{\Gamma}{\areft(\vref)} && 
     \hastype{\Gamma}{\rvapp{\rvar}{e} \ \vref}{\tbbool}
    }
    {\isWellFormed{\Gamma}{(\areft \wedge \rvapp{\rvar}{e})(\vref)}}
    [\wtRVApp]
\end{array}$$
%
$$\inference
    {\hastype{\Gamma, \vref:b}{\reft}{\tbbool} \quad 
%     \isWellFormed{\Gamma, \vref:b}{\areft(\vref)}
     \hastype{\Gamma, \vref:b}{\areft(\vref)}{\tbbool}
    }
    {\isWellFormed{\Gamma}{\tpref{b}{\areft}{\reft}}}
    [\wtBase]
$$
%
$$
\inference
    {
	%\hastype{\Gamma, v:\tfun{x}{\tau_x}{\tau}}{e}{\tbbool} &&
	\hastype{\Gamma}{\reft}{\tbbool} &&
    \isWellFormed{\Gamma}{\tau_x} &&
	\isWellFormed{\Gamma, x:\tau_x}{\tau}
    }
    {\isWellFormed{\Gamma}{\trfun{x}{\tau_x}{\tau}{\reft}}}
    [\wtFun]
$$
%
$$\begin{array}{ccc}
\inference
  {\isWellFormed{\Gamma, \rvar:\tau}{\sigma}}
  {\isWellFormed{\Gamma}{\tpabs{\rvar}{\tau}{\sigma}}}
  [\wtPred]
&
\quad
&
\inference
    {\isWellFormed{\Gamma, \alpha}{\sigma}}
    {\isWellFormed{\Gamma}{\ttabs{\alpha}{\sigma}}}
    [\wtPoly]
\end{array}$$

\medskip \judgementHead{Subtyping}{\isSubType{\Gamma}{\sigma_1}{\sigma_2}}

$$
\inference
   {\text{SMT-Valid}(\inter{\Gamma} \land \inter{\areft_1\ \vref} \land \inter{\reft_1} 
                 \Rightarrow \inter{\areft_2\ \vref} \land \inter{\reft_2})}
   {\isSubType{\Gamma}{\tpref{b}{\areft_1}{\reft_1}}{\tpref{b}{\areft_2}{\reft_2}}}
   [\tsubBase]
$$
%
$$
\inference
   {%\text{Valid}(\inter{\Gamma}\land \inter{e_1} \Rightarrow \inter{e_2}) \\
	\isSubType{\Gamma}{\tau_2}{\tau_1} &
	\isSubType{\Gamma, x_2:{\tau_2}}{\SUBST{\tau_1'}{x_1}{x_2}}{\tau_2'}	
   }
   {\isSubType{\Gamma}
	  {\trfun{x_1}{\tau_1}{\tau_1'}{\reft_1}}
	  {\trfun{x_2}{\tau_2}{\tau_2'}{\true}}
}[\tsubFun]
$$
%
$$
\begin{array}{ccc}
\inference
   {\isSubType{\Gamma, \rvar:\tau}{\sigma_1}{\sigma_2}}
   {\isSubType{\Gamma}{\tpabs{\rvar}{\tau}{\sigma_1}}{\tpabs{\rvar}{\tau}{\sigma_2}}}
   [\tsubPred]
&
\quad
&
\inference
   {\isSubType{\Gamma}{\sigma_1}{\sigma_2}}
   {\isSubType{\Gamma}{\ttabs{\alpha}{\sigma_1}}{\ttabs{\alpha}{\sigma_2}}}
   [\tsubPoly]
\end{array}
$$

\medskip \judgementHead{Type Checking}{$\hastype{\Gamma}{e}{\sigma}$}

$$\inference
  {  \hastype{\Gamma}{e}{\sigma_2} && \isSubType{\Gamma}{\sigma_2}{\sigma_1} 
  && \isWellFormed{\Gamma}{\sigma_1}
  }
  {\hastype{\Gamma}{e}{\sigma_1}}
  [\tsub]
\quad
\inference
  {}
  {\hastype{\Gamma}{c}{\tc{c}}}
  [\tconst]
$$
$$
\inference
  {x: \tpref{b}{\areft}{\reft} \in \Gamma}
  {\hastype{\Gamma}{x}{\tpref{b}{\areft}{e \land \vref = x}}}
  [\tbase]
\quad
\inference
  {x:\tau \in \Gamma}
  {\hastype{\Gamma}{x}{\tau}} 
  [\tvariable]
$$
$$
\inference
   {\hastype{\Gamma, x:\tau_x}{e}{\tau} 
    && \isWellFormed{\Gamma}{\tau_x}
   }
   {\hastype{\Gamma}{\efunt{x}{\tau_x}{e}}{\tfun{x}{\tau_x}{\tau}}}
   [\tfunction]
\quad
\inference
   {\hastype{\Gamma}{e_1}{\tfun{x}{\tau_x}{\tau}} 
   &&  \hastype{\Gamma}{e_2}{\tau_x}
   }
   {\hastype{\Gamma}{\eapp{e_1}{e_2}}{\SUBST{\tau}{x}{e_2}}}
   [\tapp]
$$
$$
\inference
  {\hastype{\Gamma, \alpha}{e}{\sigma}}
  {\hastype{\Gamma}{\etabs{\alpha}{e}}{\ttabs{\alpha}{\sigma}}}
  [\tgen]
\quad
\inference
  {\hastype{\Gamma}{e}{\ttabs{\alpha}{\sigma}} && 
   \isWellFormed{\Gamma}{\tau}
  }
  {\hastype{\Gamma}{\etapp{e}{\tau}}{\SUBST{\sigma}{\alpha}{\tau}}}
  [\tinst]
$$
$$
\inference
    {\hastype{\Gamma, \rvar:\tau}{e}{\sigma} &&
     \isWellFormed{\Gamma}{\tau} 
     % \tau \mbox{ is non-refined } 
     %\isWellFormed{\Gamma}{\tpabs{p}{\tau}{\pi}} && 
     %p \notin \fv{e}
    }
    {\hastype{\Gamma}{\epabs{\rvar}{\tau}{e}}{\tpabs{\rvar}{\tau}{\sigma}}}
    [\tpgen]
\ \
\inference
    {\hastype{\Gamma}{e}{\tpabs{\rvar}{\tau}{\sigma}} && 
     \hastype{\Gamma}{\efunbar{x:\tau_x}{\reft'}}{\tau}
    }
    {\hastype{\Gamma}
             {\epapp{e}{\efunbar{x:\tau_x}{\reft'}}}
             {\rpinst{\sigma}{\rvar}{\efunbar{x:\tau_x}{\reft'}}}
     %        {\sigma\sub{\eapp{p}{\overline{e_p}}}{\eapp{\reft'}{\overline{e_p}}}}
    }
    [\tpinst]
$$
\caption[Type checking of \corelan.]{Well-formedness, Subtyping and Type Checking of \corelan.}
\label{fig:rules}
\end{figure}



Next, we describe the static semantics of \corelan by describing the typing
judgments and derivation rules. Most of the rules are 
standard~\cite{Ou2004,LiquidPLDI08,Knowles10,GordonTOPLAS2011}; we 
discuss only those pertaining to abstract refinements.

\mypara{Judgments}
A type environment $\Gamma$ is a sequence of type bindings $x:\sigma$.
We use environments to define three kinds of typing judgments:

\begin{itemize}
\item{\emphbf{Wellformedness judgments} (\isWellFormed{\Gamma}{\sigma})} 
state that a type schema $\sigma$ is well-formed under environment
$\Gamma$, that is, the refinements in $\sigma$ are boolean 
expressions in the environment $\Gamma$.

\item{\emphbf{Subtyping judgments} (\isSubType{\Gamma}{\sigma_1}{\sigma_2})} 
state that the type schema $\sigma_1$ is a subtype of the type schema
$\sigma_2$ under environment $\Gamma$, that is, when the free variables
of $\sigma_1$ and $\sigma_2$
are bound to values described by $\Gamma$, the set of values described
by $\sigma_1$ is contained in the set of values described by $\sigma_2$. 

\item{\emphbf{Typing judgments} (\hastype{\Gamma}{e}{\sigma})} state that
the expression $e$ has the type schema $\sigma$ under environment $\Gamma$,
that is, when the free variables in $e$ are bound to values described by 
$\Gamma$, the expression $e$ will evaluate to a value described by $\sigma$.
\end{itemize}

\mypara{Wellformedness Rules}
The wellformedness rules check that the concrete and abstract
refinements are indeed $\tbbool$-valued expressions in the 
appropriate environment.
The key rule is \wtBase, which checks, as usual, that the (concrete) 
refinement $\reft$ is boolean, and additionally, that the abstract
refinement $\areft$ applied to the value $\vref$ is also boolean.
This latter fact is established by \wtRVApp which checks that 
each refinement variable application $\rvapp{\rvar}{e}\ \vref$ 
is also of type \tbbool in the given environment.

\mypara{Subtyping Rules}
The subtyping rules stipulate when the set of values described 
by schema $\sigma_1$ is subsumed by the values described by $\sigma_2$.
The rules are standard except for \tsubVar, which encodes the base types' 
abstract refinements $\areft_1$ and $\areft_2$ with conjunctions of 
\emph{uninterpreted predicates} 
$\inter{\areft_1\ \vref}$ and $\inter{\areft_2\ \vref}$ in the 
refinement logic as follows:
\begin{align*}
\inter{\true\ \vref} & \defeq \true\\
\inter{(\areft \land \rvapp{\rvar}{e})\ \vref} & \defeq \inter{\areft\
\vref} \land \rvar(\inter{e_1},\ldots,\inter{e_n},\vref)
\end{align*}
where $\rvar(\overline{e})$ is a term in the refinement logic corresponding
to the application of the uninterpreted predicate symbol $\rvar$ to the 
arguments $\overline{e}$.
% $\text{Valid}(p)$ (\tsubBase) holds if an SMT determines the formula $p$ 
% is \emph{valid}~\cite{Nelson81}.



\mypara{Type Checking Rules}
The type checking rules are standard except for \tpgen and \tpinst, which
pertain to abstraction and instantiation of abstract refinements.
%
The rule \tpgen is the same as \tfunction: we simply check the body
$e$ in the environment extended with a binding for the refinement 
variable $\rvar$.
%
The rule \tpinst checks that the concrete refinement is of the appropriate
(unrefined) type $\tau$, and then replaces all (abstract) applications of
$\rvar$ inside $\sigma$ with the appropriate (concrete) refinement $\reft'$ 
with the parameters $\overline{x}$ replaced with arguments at that application.
Formally, this is represented as $\rpinst{\sigma}{\rvar}{\efunbar{x:\tau}{\reft'}}$
which is $\sigma$ with each base type transformed as
%%$$\rpinst{\tpref{b}{\areft}{\reft}}{\rvar}{z}
%%  \defeq \tpref{b}{\areft''}{\reft \land \reft''} 
%%  \quad \mbox{where} (\areft'', \reft'') =
%%  \rpapply{\areft}{\rvar}{z}{\true}{\true}$$
\begin{align*}
\rpinst{\tpref{b}{\areft}{\reft}}{\rvar}{z}
  & \defeq \tpref{b}{\areft''}{\reft \land \reft''} \\
\mbox{where} \quad (\areft'', \reft'') 
  & \defeq \rpapply{\areft}{\rvar}{z}{\true}{\true} 
\intertext{$\mathsf{Apply}$ replaces each application of $\rvar$ in 
$\areft$ with the corresponding conjunct in $\reft''$, as}
\rpapply{\true}{\cdot}{\cdot}{\areft'}{\reft'} 
  & \defeq (\areft', \reft') \\
\rpapply{\areft \wedge \rvapp{\rvar'}{e}}{\rvar}{z}{\areft'}{\reft'} 
  & \defeq \rpapply{\areft}{\rvar}{z}{\areft' \land \rvapp{\rvar'}{e}}{\reft'} \\
\rpapply{\areft \wedge \rvapp{\rvar}{e}}{\rvar}{\efunbar{x:\tau}{\reft''}}{\areft'}{\reft'} 
  & \defeq
  \rpapply{\areft}{\rvar}{\efunbar{x:\tau}{\reft''}}{\areft'}{\reft' \wedge \SUBST{\reft''}{\overline{x}}{\overline{e},\vref}}
\end{align*}
In other words, the instantiation can be viewed as two symbolic 
reduction steps: first replacing the refinement variable with the
concrete refinement, and then ``beta-reducing" concrete refinement 
with the refinement variable's arguments. For example, 
$$\rpinst{\tpref{\tbint}{\rvar\ y}{\vref > 10}}
       {\rvar}
       {\efunt{x_1}{\tau_1}{\efunt{x_2}{\tau_2}{x_1 < x_2}}}
\defeq \tref{\tbint}{\vref > 10 \land y < \vref}$$
%%rp(x:tx->t , \rvar, z) = x:tx' -> t'
%%  where tx'      = rp(tx, \rvar, z)
%%        t'       = rp(t , \rvar, z)
%%
%%rp(\a.t, \rvar, z) = \a.t'
%%  where t'       = rp(t, \rvar, z)
%%
%%rp(\p.t, \rvar, z) = \p.t'
%%  where t'       = rp(t, \rvar, z)

%%The other rule that handles abstract refinements is \tcase.
%%This rule initially checks that the expression to be analyzed 
%%has a type application type $T = \tcon{\chi}{e_\chi}{\listOf{T}}{\listOf{e}}$.
%%Then for all cases, the case expression is typechecked in the initial environment, 
%%extended with case binders \listOf{x_i} and the initial expression binder $x$.
%%The types of these binders are gained after unfolding data constructor's type \tc{K_i}. 
%%The unfolding is done by replacing its type variables with actual type arguments
%%of $T$, ie. \listOf{T} 
%%its abstract refinements with actual inferred refinements \listOf{e},
%%and its binders with actual binders \listOf{x_i}.

\subsection{Soundness}\label{sec:soundness}

As hinted by the discussion about refinement variable instantiation,
we can intuitively think of abstract refinement variables as 
\emph{ghost} program variables whose values are boolean-valued 
functions. Hence, abstract refinements are a special case of 
higher-order contracts, that can be statically verified using 
uninterpreted functions. (Since we focus on static checking, 
we don't care about the issue of blame.)
We formalize this notion by translating \corelan programs into
the contract calculus \conlan of \cite{Greenberg11} and use this 
translation to define the dynamic semantics and establish soundness.

\mypara{Translation} 
We translate \corelan schemes $\sigma$ to \conlan schemes $\tx{\sigma}$
as by translating abstract refinements into contracts,
and refinement abstraction into function types:
$$\begin{array}{rclcrcl}
\tx{\true\ \vref} & \defeq 
  & \true  
  & \quad \quad &

\tx{\tpabs{\rvar}{\tau}{\sigma}} & \defeq 
  & \tfun{\rvar}{\tx{\tau}}{\tx{\sigma}} \\

\tx{(\areft \land \rvapp{\rvar}{e})\ \vref} & \defeq 
  & \tx{\areft\ \vref} \land \eapp{\eapp{\rvar}{\overline{e}}}{\vref} 
  & \quad \quad &

\tx{\ttabs{\alpha}{\sigma}} & \defeq 
  & \ttabs{\alpha}{\tx{\sigma}} \\

\tx{\tpref{b}{\areft}{\reft}} & \defeq 
  & \tref{b}{\reft \land \tx{\areft\ \vref}} 
  & \quad \quad &

\tx{\tfun{x}{\tau_1}{\tau_2}} & \defeq 
  & \tfun{x}{\tx{\tau_1}}{\tx{\tau_2}} 
%\tx{\trfun{x}{\tau_1}{\tau_2}{\reft}} \defeq 
%  & \trfun{x}{\tx{\tau_1}}{\tx{\tau_2}}{\tx{\reft}} \\
\end{array}$$
Similarly, we translate \corelan terms $e$ to \conlan 
terms $\tx{e}$ by converting refinement abstraction and application 
to $\lambda$-abstraction and application
$$\begin{array}{rclcrcl}
\tx{x} & \defeq & x & \quad \quad \quad & \tx{c} & \defeq & c \\
\tx{\efunt{x}{\tau}{e}} & \defeq & \efunt{x}{\tx{\tau}}{\tx{e}} & \quad & \tx{\eapp{e_1}{e_2}} & \defeq & \eapp{\tx{e_1}}{\tx{e_2}} \\
\tx{\etabs{\alpha}{e}} & \defeq & \etabs{\alpha}{\tx{e}} & \quad & \tx{\etapp{e}{\tau}} & \defeq & \eapp{\tx{e}}{\tx{\tau}} \\
\tx{\epabs{\rvar}{\tau}{e}} &\defeq & \efunt{\rvar}{\tx{\tau}}{\tx{e}} & \quad & \tx{\epapp{e_1}{e_2}} &\defeq & \eapp{\tx{e_1}}{\tx{e_2}}
\end{array}$$

%%\begin{align*}
%%\tx{\true\ \vref} \defeq 
%%  & \true\\
%%\tx{(\areft \land \rvapp{\rvar}{e})\ \vref} \defeq 
%%  & \tx{\areft\ \vref} \land \eapp{\eapp{\rvar}{\overline{e}}}{\vref}\\
%%\tx{\tpref{b}{\areft}{\reft}} \defeq 
%%  & \tref{b}{\reft \land \tx{\areft\ \vref}} \\
%%\tx{\tfun{x}{\tau_1}{\tau_2}} \defeq 
%%  & \tfun{x}{\tx{\tau_1}}{\tx{\tau_2}} \\
%%%\tx{\trfun{x}{\tau_1}{\tau_2}{\reft}} \defeq 
%%%  & \trfun{x}{\tx{\tau_1}}{\tx{\tau_2}}{\tx{\reft}} \\
%%\tx{\ttabs{\alpha}{\sigma}} \defeq 
%%  & \ttabs{\alpha}{\tx{\sigma}} \\
%%\tx{\tpabs{\rvar}{\tau}{\sigma}} \defeq 
%%  & \tfun{\rvar}{\tx{\tau}}{\tx{\sigma}}
%%\end{align*}
%%\tx{x} \defeq & x \\
%%\tx{c} \defeq & c \\
%%\tx{\efunt{x}{\tau}{e}} \defeq & \efunt{x}{\tx{\tau}}{\tx{e}} \\
%%\tx{\eapp{e_1}{e_2}} \defeq & \eapp{\tx{e_1}}{\tx{e_2}} \\
%%\tx{\etabs{\alpha}{e}} \defeq & \etabs{\alpha}{\tx{e}} \\
%%\tx{\etapp{e}{\tau}} \defeq & \eapp{\tx{e}}{\tx{\tau}} \\
%%\tx{\epabs{\rvar}{\tau}{e}} \defeq & \efunt{\rvar}{\tx{\tau}}{\tx{e}} \\
%%\tx{\epapp{e_1}{e_2}} \defeq & \eapp{\tx{e_1}}{\tx{e_2}}





\mypara{Translation Properties}
We can show by induction on the derivations that the 
type derivation rules of \corelan \emph{conservatively approximate}
those of \conlan. Formally, 

\begin{itemize}
\item If $\isWellFormed{\Gamma}{\tau}$ then $\isWellFormedH{\Gamma}{\tau}$,
\item If $\isSubType{\Gamma}{\tau_1}{\tau_2}$ then $\isSubTypeH{\Gamma}{\tau_1}{\tau_2}$,
\item If $\hastype{\Gamma}{e}{\tau}$ then
$\hastypeH{\tx{\Gamma}}{\tx{e}}{\tx{\tau}}$.
\end{itemize}

\mypara{Soundness} Thus rather than re-prove preservation and progress
for \corelan, we simply use the fact that the type derivations are
conservative to derive the following preservation and progress 
corollaries from \cite{Greenberg11}:
%
\begin{itemize}
\item{\textbf{Preservation: }} 
  If $\hastype{\emptyset}{e}{\tau}$ 
  and $\tx{e} \longrightarrow e'$ 
  then $\hastypeH{\emptyset}{e'}{\tx{\tau}}$

\item{\textbf{Progress: }}
  If $\hastype{\emptyset}{e}{\tau}$, then either
  $\tx{e} \longrightarrow e'$ or 
  $\tx{e}$ is a value.
\end{itemize}
%
Note that, in a contract calculus like \conlan, subsumption is encoded
as a \emph{upcast}. However, if subtyping relation can be statically 
guaranteed (as is done by our conservative SMT based subtyping) 
then the upcast is equivalent to the identity function and can 
be eliminated. Hence, \conlan terms $\tx{e}$ translated from well-typed 
\corelan terms $e$ have no casts.

\subsection{Refinement Inference}\label{sec:infer}

Our design of abstract refinements makes it particularly easy to 
perform type inference via Liquid typing, which is crucial for
making the system usable by eliminating the tedium of instantiating 
refinement parameters all over the code. (With value-dependent 
refinements, one cannot simply use, say, unification to determine
the appropriate instantations, as is done for classical type systems.)
We briefly recall how Liquid types work, and sketch how they are 
extended to infer refinement instantiations.

\mypara{Liquid Types} 
The Liquid Types method infers refinements in three steps. 
%
First, we create refinement \emph{templates} for the unknown, 
to-be-inferred refinement types. The \emph{shape} of the template 
is determined by the underlying (non-refined) type it corresponds to, 
which can be determined from the language's underlying (non-refined) 
type system. 
The template is just the shape refined with fresh refinement variables
$\kappa$ denoting the unknown refinements at each type position. 
For example, from a type ${\tfun{x}{\tbint}{\tbint}}$ we create 
the template ${\tfun{x}{\tref{\tbint}{\kappa_x}}{\tref{\tbint}{\kappa}}}$.
%
Second, we perform type checking using the templates (in place of the
unknown types.) Each wellformedness check becomes a wellformedness
constraint over the templates, and hence over the individual $\kvar$,
constraining which variables can appear in $\kvar$.
Each subsumption check becomes a subtyping constraint
between the templates, which can be further simplified, via syntactic
subtyping rules, to a logical implication query between the variables
$\kappa$.
%
Third, we solve the resulting system of logical implication constraints
(which can be cyclic) via abstract interpretation --- in particular,
monomial predicate abstraction over a set of logical qualifiers
\cite{Houdini,LiquidPLDI08}. The solution is a map from $\kvar$ to
conjunctions of qualifiers, which, when plugged back into the templates,
yields the inferred refinement types.

\mypara{Inferring Refinement Instantiations}
The key to making abstract refinements practical is a means of 
synthesizing the appropriate arguments $\reft'$ for each refinement 
application $\epapp{e}{\reft'}$. 
Note that for such applications, we can, from $e$, determine the 
non-refined type of $\reft'$, which is of the form 
${\tau_1 \rightarrow \ldots \rightarrow \tau_n \rightarrow \tbbool}$.
Thus, $\reft'$ has the template 
${\efunt{x_1}{\tau_1}{\ldots \efunt{x_n}{\tau_n}{\kvar}}}$
where $\kvar$ is a fresh, unknown refinement variable that 
must be solved to a boolean valued expression over $x_1,\ldots,x_n$.
Thus, we generate a \emph{wellformedness} constraint 
${\isWellFormed{x_1:\tau_1, \ldots, x_n:\tau_n}{\kvar}}$
and carry out typechecking with template, which, as before, yields
implication constraints over $\kvar$, which can, as before, be 
solved via predicate abstraction.
Finally, in each refinement template, we replace each $\kvar$ with its
solution $e_\kvar$ to get the inferred refinement instantiations.

%%To infer appropabstract refinements we used the liquid type variables
%%$\kappa$ with explicit arguments to avoid inferring function 
%%expressions:
%%To infer an expression to replace a predicate of type 
%%$\listOf{x_i:\tau_i; \tau}$ in an environment $\Gamma$, 
%%we create a fresh liquid variable $\kappa$ on $\tau$ which is
%%wellformed in the environment $\Gamma$ extended with the bindings
%%\listOf{x_i:\tau_i}.
%%When $\kvar$ is solved via predicate abstraction to an expression 
%%$e_\kvar$, we simply replace $\kvar$ with 
%%the set as the inferred expression 
%%$e =\efun{x_1}{\dots\efun{x_n}{\efun{v}{e_\kappa}}}$.
%add a refinement variable in place of expressions. 
%\mypara{Constants}\jhala{constant-refinements and soundness guarantees?}
%%TODO: 
%%[SKIP] add paragraph on constants to opsem
%%[SKIP] redefine \reft to r? (to emphasize not arbitrary expression?)

