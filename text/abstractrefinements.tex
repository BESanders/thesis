\chapter{Abstract Refinement Types}\label{chapter:abstractrefinements}
\makequote
{The purpose of abstraction is not to be vague,\\
but to create a new semantic level in which one can be absolutely precise.}
{Edsger W. Dijkstra}


\renewcommand{\reft}{\ensuremath{e}\xspace}
\renewcommand\tref[2]{\ensuremath{\left\lbrace \vref : #1\mid #2\right\rbrace}}
\renewcommand\tref[2]{\ensuremath{\left\lbrace \vref : #1\mid #2\right\rbrace}}

\renewcommand\subt{\preceq}
\renewcommand\corelan{$\lambda_\downarrow$\xspace}
\renewcommand\sub[2]{\ensuremath{ \left[ #1 \mapsto #2 \right] }}
\renewcommand\shape{\ensuremath{\text{shape}}\xspace}
\renewcommand\tfun[3]{\ensuremath{{(#1:#2)} \rightarrow #3}}


\renewcommand\ecase[5]{\ensuremath{
	\mathtt{case}\ #5 = #1\ \mathtt{of}\ \{ #2\ #3 \rightarrow #4\}
}}

\renewcommand\corelan{$\lambda_{P}$\xspace}


%\lstinline[language=HaskellUlisses]|showsPrec Vec a|
%@Vec (List a) showsPrec Vec a@

\section{Introduction}\label{sec:intro}

Refinement types offer an automatic means 
of verifying semantic properties of programs, 
by decorating types with predicates from logics 
efficiently decidable by modern SMT solvers.
For example, the refinement type 
@{v: Int | v > 0}@
denotes the basic type \ttcode{Int} refined with a logical
predicate over the ``value variable" \ttcode{v}.
This type corresponds to the set of \ttcode{Int} values 
\ttcode{v} which additionally satisfy the logical predicate, 
\ie the set of positive integers.  
The (dependent) function type 
@x:{v:Int| v > 0} -> {v:Int| v < x}@
describes functions that take a positive argument 
\ttcode{x} and return an integer less than \ttcode{x}.
%
Refinement type checking reduces to \emph{subtyping} queries 
of the form ${\cstr{\Gamma}{\valu}{\tau}{p}{\tau}{q}}$,
where $p$ and $q$ are refinement predicates. 
These subtyping queries reduce to logical \emph{validity} 
queries of the form
${\dbrkts{\Gamma} \wedge p \Rightarrow q}$, which can be 
automatically discharged using SMT solvers~\cite{z3}. 
 
Several groups have shown how refinement types 
can be used to verify properties ranging from partial 
correctness concerns like array bounds 
checking~\cite{pfenningxi98,LiquidPLDI08} and data structure
invariants~\cite{LiquidPLDI09} to the correctness of security
protocols~\cite{GordonTOPLAS2011}, web applications 
\cite{SwamyOAKLAND11} and implementations of cryptographic 
protocols~\cite{FournetCCS11}. 

Unfortunately, the automatic verification 
%% (\ie automatic type \emph{checking} not inference)
offered by refinements has come at a price. 
To ensure decidable checking with SMT solvers, the 
refinements are quantifier-free predicates drawn from a
decidable logic.  This significantly limits expressiveness by 
precluding specifications that enable abstraction over the 
refinements (\ie invariants). For example, consider the 
following higher-order for-loop where @set i x v@ returns 
the vector @v@ updated at index @i@ with the value @x@. 

\begin{code}
     for :: Int -> Int -> a -> (Int -> a -> a) -> a
     for lo hi x body      = loop lo x 
       where loop i x 
               | i < hi    = loop (i+1) (body i x)
               | otherwise = x
     
     initUpto :: Vec a -> a -> Int -> Vec a 
     initUpto a x n = for 0 n a (\i -> set i x) 
\end{code}

We would like to 
verify that @initUpto@ returns a vector whose \emph{first}
@n@ elements are equal to @x@. 
In a first-order setting, we could achieve the above with 
a loop invariant that asserted that at the $\ttcode{i}^{th}$
iteration, the first @i@ elements of the vector were 
already initalized to @x@. 
%
However, in our higher-order setting we require a means 
of \emph{abstracting} over possible invariants, each of which can
\emph{depend on} the iteration index @i@. 
%
Higher-order logics like Coq and Agda permit such quantification 
over invariants. Alas, validity in such logics is well outside the 
realm of decidability, and hence their use precludes automatic 
verification. %type checking.

In this paper, we present \emph{abstract refinement types} 
which enable abstraction (quantification) over the refinements 
of data- and function-types. Our key insight is that we can 
preserve SMT-based decidable type checking by encoding 
abstract refinements as \emph{uninterpreted} propositions 
in the refinement logic. 
%This insight yields the following contributions.
This yields several contributions:
\begin{itemize}
\item First, we illustrate how abstract refinements yield a variety 
of sophisticated means for reasoning about high-level program 
constructs (\S \ref{sec:overview}), including:
\emph{parametric} refinements for type classes,
\emph{index-dependent} refinements for key-value maps,
\emph{recursive} refinements for data structures, and
\emph{inductive} refinements for higher-order traversal routines.

\item Second, we demonstrate that type checking remains 
decidable (\S \ref{sec:check}) by showing a fully automatic
procedure that uses SMT solvers, or to be precise, 
decision procedures based on congruence closure~\cite{Nelson81}
to discharge logical subsumption queries over abstract refinements.

\item Third, we show that the crucial problem of \emph{inferring}
appropriate instantiations for the (abstract) refinement 
parameters boils down to inferring (non-abstract) refinement
types (\S \ref{sec:check}), which we have previously automated 
via the abstract interpretation framework of Liquid Types~\cite{LiquidPLDI08}. 

\item Finally, we have implemented abstract refinements in \toolname, 
a new Liquid Type-based verifier for Haskell. We present 
experiments using \toolname to concisely specify and 
verify a variety of correctness properties of several 
programs ranging from microbenchmarks to some widely 
used libraries (\S \ref{sec:experiments}).
\end{itemize}

\section{Overview}\label{sec:overview}

We start with a high level overview of abstract refinements, 
by illustrating how they can be used to uniformly specify and 
automatically verify various kinds of invariants.

\subsection{Parametric Invariants}\label{sec:overview:parametric}

\mypara{Parametric Invariants via Type Polymorphism}
Suppose we had a generic comparison @(<=) :: a -> a -> Bool@ as in
\ocaml.
We could use it to write: 
$$\centering\begin{tabular}{c}\begin{code}
max     :: a -> a -> a
max x y = if x <= y then y else x 

maximum :: [a] -> a
maximum (x:xs) = foldr max x xs
\end{code}\end{tabular}$$
In essence, the type given for @maximum@ states that
\emph{for any} @a@, if a list of @a@ values is passed
into @maximum@, then the returned result is also an @a@
value.
%
Hence, for example, if a list of \emph{prime} numbers 
is passed in, the result is prime, and if a list of 
\emph{even} numbers is passed in, the result is even. 
Thus, we can use refinement types \cite{LiquidPLDI08} 
to verify
%
$$\centering
\begin{tabular}{c}
\begin{code}
type Even = {v:Int | v % 2 = 0 }

maxEvens :: [Int] -> Even
maxEvens xs = maximum (0 : xs') 
  where xs' = [ x | x <- xs, x `mod` 2 = 0]
\end{code}
\end{tabular}$$
Here the @%@ represents the modulus operator in the refinement logic
\cite{z3} and we type the primitive {@mod :: x:Int -> y:Int -> {v: Int | v = x % y}@}.
Verification proceeds as follows.
Given that {@xs :: [Int]@}, the system has to verify that
{@maximum (0 : xs') :: Even@}.
%
To this end, the type parameter of @maximum@ is instantiated with the 
\emph{refined} type @Even@, yielding the instance:
%
$$\centering
\begin{tabular}{c}
\begin{code}
maximum :: [Even] -> Even
\end{code}
\end{tabular}$$
%
Then, @maximum@'s argument should be proved to have type
{@[Even]@}.
So, the type parameter of @(:)@ 
is instantiated with {@Even@}, yielding the instance: 
%
$$\centering
\begin{tabular}{c}
\begin{code}
(:) :: Even -> [Even] -> [Even]
\end{code}
\end{tabular}$$
%
Finally, the system infers that {@0 :: Even@} and {@xs' :: [Even]@}, 
\ie the arguments of {@(:)@} have the expected types, thereby verifying
the program.
%
The refinement type instantiations can be inferred,
from an appropriate set of logical qualifiers, 
using the abstract interpretation framework of Liquid
Types~\cite{LiquidPLDI08}.
Here, once 
@ v%2 = 0 @ 
is added to the set of qualifiers, either manually or (as done by 
our implementation) by automatically scraping predicates from 
refinements appearing in specification signatures, the 
refinement type instantiations, and hence verification, proceed 
automatically.
%
Thus, parametric polymorphism offers an easy means of encoding 
second-order invariants, \ie of quantifying over or parametrizing 
the invariants of inputs and outputs of functions. 

\mypara{Parametric Invariants via Abstract Refinements}
Instead, suppose that the comparison operator was monomorphic, and only
worked for @Int@ values. The resulting (monomorphic) signatures
$$\centering\begin{tabular}{c}
\begin{code}
max     :: Int -> Int -> Int
maximum :: [Int] -> Int
\end{code}
\end{tabular}$$
preclude the verification of @maxEvens@ (\ie typechecking against the 
signature shown earlier). This is because the new type of @maximum@ 
merely states that \emph{some} @Int@  is returned as output, and not 
necessarily one that enjoys the properties of the values in the input
list. This is a shame, since the property clearly still holds.
We could type
%@max :: forall t <: Int. t -> t -> t@
$$\centering
\begin{tabular}{c}
\begin{code}
max :: forall t <: Int. t -> t -> t
\end{code}
\end{tabular}$$
but this route would introduce the complications
that surround bounded quantification which could render checking 
undecidable~\cite{piercebook}.

To solve this problem, we introduce \emph{abstract refinements} 
which let us 
quantify or parameterize a type over its constituent refinements.
For example, we can type @max@ as
$$\centering
\begin{tabular}{c}
\begin{code}
max :: forall <p::Int->Bool>. Int<p> -> Int<p> -> Int<p>
\end{code}
\end{tabular}$$
where @Int<p>@ is an abbreviation for the refinement type {@{v:Int | p(v)}@}.
Intuitively, an abstract refinement @p@ is encoded in the refinement logic 
as an \emph{uninterpreted function symbol}, which satisfies the
\emph{congruence} axiom~\cite{Nelson81}
%
$$\forall \overline{X}, \overline{Y}: (\overline{X} = \overline{Y})
\Rightarrow P(\overline{X}) = P(\overline{Y})$$
%
Thus, it is trivial to verify, with an SMT solver, that @max@ 
enjoys the above type: the input types ensure that both @p(x)@ and @p(y)@ 
hold and hence the returned value in either branch satisfies 
the refinement  @{v:Int | p(v)}@, thereby ensuring the output 
type. By the same reasoning, we can generalize the type of @maximum@ 
to
$$\centering\begin{tabular}{c}
\begin{code}
maximum :: forall <p :: Int -> Bool>. [Int<p>] -> Int<p>
\end{code}\end{tabular}$$
Consequently, we can recover the verification of @maxEvens@.
Now, instead of instantiating a \emph{type} parameter, we simply instantiate
the \emph{refinement} parameter of @maximum@ with the concrete 
refinement 
@{\v -> v % 2 = 0}@,
after which type checking proceeds as usual \cite{LiquidPLDI08}. 
%
Later, we show how to retain automatic verification by inferring
refinement parameter instantiations via liquid typing
(\S~\ref{sec:infer}).

\mypara{Parametric Invariants and Type Classes}
The example above regularly arises in practice, due to type classes. 
In Haskell, the functions above are typed
%
$$\centering\begin{tabular}{c}\begin{code}
(<=)    :: (Ord a) => a -> a -> Bool
max     :: (Ord a) => a -> a -> a
maximum :: (Ord a) => [a] -> a
\end{code}\end{tabular}$$
%
We might be tempted to ignore the typeclass constraint, 
and treat \ttcode{maximum} as @[a] -> a@. 
This would be quite unsound, as typeclass predicates preclude
universal quantification over refinement types. 
Consider the function @sum :: (Num a) => [a] -> a@ which adds the elements 
of a list.
%Clearly,
The @Num@ class constraint implies that numeric operations occur 
in the function, so
if we pass @sum@ a list of odd numbers, 
we are \emph{not} guaranteed to get back an odd number. 

%Thus, given that we cannot instantiate class-predicated type 
%parameters with arbitrary refinement types, 

Thus, how do we soundly verify the desired type of @maxEvens@ 
without instantiating class predicated type parameters with 
arbitrary refinement types? First, via the same analysis as 
the monomorphic @Int@ case, we establish that
%
$$\centering\begin{tabular}{c}\begin{code}
max:: forall <p::a->Bool>. (Ord a)=> a<p> -> a<p> -> a<p>
maximum:: forall <p::a ->Bool>. (Ord a) => [a<p>] -> a<p>
\end{code}\end{tabular}$$
%
Next, at the call-site for @maximum@ in @maxEvens@ we
instantiate the type variable @a@ with @Int@, and 
the abstract refinement @p@ with @{\v -> v % 2 = 0}@
after which, the verification proceeds as described
earlier (for the @Int@ case).
Thus, abstract refinements allow us to quantify over 
invariants without relying on parametric polymorphism, 
even in the presence of type classes.

\subsection{Index-Dependent Invariants}\label{sec:overview:index}

Next, we illustrate how abstract invariants allow us to 
specify and verify index-dependent invariants of key-value maps. 
To this end, we develop a small library of \emph{extensible vectors} 
encoded, for purposes of illustration, as functions from @Int@ to 
some generic range @a@. Formally, we specify vectors as 
%
$$\centering\begin{tabular}{c}\begin{code}
data Vec a <dom :: Int -> Bool, rng :: Int -> a -> Bool> 
  = V (i:Int<dom> -> a <rng i>)
\end{code}\end{tabular}$$
%
Here, we are parameterizing the definition of the type @Vec@ 
with \emph{two} abstract refinements, @dom@ and @rng@, which
respectively describe the \emph{domain} and \emph{range} of the vector.
That is, @dom@ describes the set of \emph{valid} indices, 
and @r@ specifies an invariant relating each @Int@ index
with the value stored at that index.

\mypara{Creating Vectors}
We can use the following basic functions to create vectors:
%
$$\centering\begin{tabular}{c}\begin{code}
empty :: forall <p::Int->a->Bool>.Vec<{\_ -> False}, p> a
empty = V (\_ -> error "Empty Vec")

create :: x:a -> Vec <{\_ -> True}, {\_ v -> v = x}> a
create x = V (\_ -> x)
\end{code}\end{tabular}$$
%
The signature for @empty@ states that its domain is empty (\ie is
the set of indices satisfying the predicate @False@), and that the
range satisfies @any@ invariant. The signature for @create@,
instead, defines a \emph{constant} vector that maps every index to the
constant @x@.

\mypara{Accessing Vectors}
We can write the following @get@ function for reading the contents
of a vector at a given index:
%
$$\centering\begin{tabular}{c}\begin{code}
get :: forall <d :: Int -> Bool, r :: Int -> a -> Bool>
          i:Int<d> -> Vec<d, r> a -> a<r i>
get i (V f) = f i
\end{code}\end{tabular}$$
%
The signature states that for any domain @d@ and range @r@,
if the index @i@ is a valid index, \ie is of type, \verb+Int<d>+ 
then the returned value is an \verb+a+ that additionally satisfies the
range refinement at the index @i@.
%
The type for @set@, which \emph{updates} the vector at
a given index, is even more interesting, as it allows us to 
\emph{extend} the domain of the vector:
%
$$\centering\begin{tabular}{c}\begin{code}
set :: forall <d :: Int -> Bool, r :: Int -> a -> Bool>
          i:Int<d>
       -> a<r i>
       -> Vec<d && {\k -> k != i}, r> a
       -> Vec<d, r> a
set i v (V f) = V (\k -> if k == i then v else f k)
\end{code}\end{tabular}$$
%
The signature for @set@ requires that 
(a)~the input vector is defined everywhere at @d@ \emph{except}
the index @i@, and 
(b)~the value supplied must be of type @a<r i>@, \ie satisfy the range 
relation at the index @i@ at which the vector is being updated.
The signature ensures that the output vector is defined at
@d@ and each value satisfies the index-dependent range refinement @r@.
%
Note that it is legal to call @get@ with a vector that is \emph{also} 
defined at the index @i@ since, by contravariance, such a vector is a
subtype of that required by (a).


\mypara{Initializing Vectors} Next, we can write the following function,
@init@, that ``loops" over a vector, to @set@ each index to 
a value given by some function.
%
$$\centering\begin{tabular}{c}\begin{code}
initialize :: forall <r :: Int -> a -> Bool>.
              (z: Int -> a<r z>) 
           -> i: {v: Int | v >= 0} 
           -> n: Int 
           -> Vec <{\v -> 0 <= v && v < i}, r> a 
           -> Vec <{\v -> 0 <= v && v < n}, r> a 

initialize f i n a 
  | i >= n    = a
  | otherwise = initialize f (i+1) n (set i (f i) a)
\end{code}\end{tabular}$$
%
The signature requires that 
(a)~the higher-order function @f@ produces values that satisfy 
the range refinement @r@, and 
(b)~the vector is initialized from @0@ to @i@.
%
The function ensures that the output vector is initialized from @0@
through @n@.
%
We can thus verify that
%
$$\centering\begin{tabular}{c}\begin{code}
idVec  :: Vec <{\v -> 0<=v && v<n}, {\i v -> v=i}> Int
idVec n = initialize (\i -> i) 0 n empty
\end{code}\end{tabular}$$
%
\ie @idVec@ returns a vector of size @n@ where each
key is mapped to itself. Thus, abstract refinement types allow us 
to verify low-level idioms such as the incremental initialization 
of vectors, which have previously required 
special analyses~\cite{Gopan05,JhalaMcMillanCAV07,CousotsPOPL11}.

\mypara{Null-Terminated Strings}
%
We can also use abstract refinements to verify code which 
manipulates C-style null-terminated strings, 
represented as @Char@ vectors for ease of exposition.
Formally, a null-terminated string of size @n@ has the type
%
\begin{code}
type NullTerm n 
  = Vec <{\v -> 0<=v<n}, {\i v -> i=n-1 => v='\0'}> Char
\end{code}
%%Vec <{\v -> 0 <= v && v < n},
%%     {\i v -> i = n - 1 => v = '\0'}>
%%    Char
%
The above type describes a length-@n@ vector of characters whose
last element must be a null character, signalling the end of
the string.
%
We can use this type in the specification of a function,
@upperCase@, which iterates through the characters of a string,
uppercasing each one until it encounters the null terminator:
%
%  Vec <{\v -> 0 <= v && v < n},
%       {\i v -> i = n - 1 => v = '\0'}>
%      Char ->
%  Vec <{\v -> 0 <= v && v < n},
%       {\i v -> i = n - 1 => v = '\0'}>
%      Char
% upperCase n s = ucs 0 s where
%   ucs i s = let c = get i s in
%             if c == '\0' 
%               then s
%               else ucs (i + 1) (set i (toUpper c) s)
\begin{code}
upperCase :: n:{v: Int| v>0} -> NullTerm n -> NullTerm n
upperCase n s = ucs 0 s where
  ucs i s = case get i s of
              '\0' -> s
              c    -> ucs (i + 1) (set i (toUpper c) s)
\end{code}
%
Note that the length parameter @n@ is provided solely as a ``witness''
for the length of the string @s@, which allows us to use the length of
@s@ in the type of @upperCase@; @n@ is not used in the
computation.
%
In order to establish that each call to @get@ accesses string @s@
within its bounds, our type system must establish that, at each call
to the inner function @ucs@, @i@ satisfies the type
@{v: Int | 0 <= v && v < n}@.
%
This invariant is established as follows.
%
First, the invariant trivially holds on the first call to @ucs@, as
@n@ is positive and @i@ is $0$.
%
Second, we assume that @i@ satisfies the type
%
@{v: Int | 0 <= v && v < n}@,
%
and, further, we know from the types of @s@ and @get@ that @c@ has the
type @{v: Char | i = n - 1 => v = '\0'}@.
%
Thus, if @c@ is non-null, then @i@ cannot be equal to @n - 1@.
%
This allows us to strengthen our type for @i@ in the else branch to
@{v: Int | 0 <= v && v < n - 1}@ and thus to conclude that the value
@i + 1@ recursively passed as the @i@ parameter to @ucs@ satisfies the
type @{v: Int | 0 <= v && v < n}@, establishing the inductive
invariant and thus the safety of the @upperCase@ function.

\mypara{Memoization} 
Next, let us illustrate how the same expressive signatures allow us to
verify memoizing functions. We can specify to the SMT solver the 
definition of the Fibonacci function via an uninterpreted function 
@fib@ and an axiom:
%
$$\centering\begin{tabular}{c}\begin{code}
measure fib :: Int -> Int
axiom: forall i. fib(i) = i<=1 ? 1 : fib(i-1) + fib(i-2)
\end{code}\end{tabular}$$
%axiom_fib :: i:Int -> {v: Bool | (? v) <=> (fib(i) = i <= 1 ? 1 : fib(i-1) + fib(i-2))}
%
Next, we define a type alias @FibV@ for the vector 
whose values are either @0@ (\ie undefined), or 
equal to the Fibonacci number of the corresponding index. 
%(We use @Int@ instead of @Maybe Int@ as the domain for brevity.)
%
$$\centering\begin{tabular}{c}\begin{code}
type FibV = Vec<{\_->True},{\i v-> v!=0 => v=fib(i)}> Int 
\end{code}\end{tabular}$$
%
Finally, we can use the above alias to verify @fastFib@, 
an implementation of the Fibonacci function, which uses 
a vector memoize intermediate results 
%
$$\centering\begin{tabular}{c}\begin{code}
fastFib :: n:Int -> {v:Int | v = fib(n)}
fastFib n = snd $ fibMemo (create 0) n

fibMemo :: FibV -> i:Int -> (FibV, {v: Int | v = fib(i)})   
fibMemo t i 
  | i <= 1    = (t, 1)
  | otherwise = case get i t of   
                  0 -> let (t1, n1) = fibMemo t  (i-1)
                           (t2, n2) = fibMemo t1 (i-2)
                           n        = n1 + n2 
                       in  (set i n t2,  n)
                  n -> (t, n)
\end{code}\end{tabular}$$
%
%% axiom_fib :: i:Int -> {v: Bool | (? v) <=> (fib(i) = i <= 1 ? 1 : fib(i-1) + fib(i-2))}
%%
%% fibMemo t i 
%%   | i <= 1    
%%   = (t, assume (axiom_fib i) $ 1)
%%   
%%   | otherwise 
%%   = case get i t of   
%%       0 -> let (t1, n1) = fibMemo t  (i-1)
%%                (t2, n2) = fibMemo t1 (i-2)
%%                n        = assume (axiom_fib i) $ n1 + n2
%%            in  (set i n t2,  n)
%%       n -> (t, n)
%
Thus, abstract refinements allow us to define key-value maps with
index-dependent refinements for the domain and range. 
Quantification over the domain and range refinements allows us
to define generic access operations (\eg @get@, @set@,
@create@, @empty@) whose types enable us establish
a variety of precise invariants.

\subsection{Recursive Invariants}\label{sec:overview:rec}

Next, we turn our attention to recursively defined datatypes, and show 
how abstract refinements allow us to specify and verify high-level
invariants that relate the elements of a recursive structure.
Consider the following refined definition for lists:
%
$$\centering\begin{tabular}{c}\begin{code}
data [a] <p :: a -> a -> Bool> where
  []  :: [a]<p>
  (:) :: h:a -> [a<p h>]<p> -> [a]<p>
\end{code}\end{tabular}$$
%data List <p :: a -> a -> Bool> a where
%  Nil  :: List<p> a 
%  Cons :: h:a -> List<p> (a<p h>) -> List<p> a
%
%% data List a <p :: a -> a -> Bool>  
%%   = Cons { head :: a
%%          , tail :: List <p> (a <p head>) }
%%   | Nil
%% \end{code}
%
The definition states that a value of type @[a]<p>@ 
is either empty (@[]@) or constructed from a pair of  
a \emph{head} @h::a@ and a \emph{tail} of a list of 
@a@ values \emph{each} of which satisfies the refinement @(p h)@. 
Furthermore, the abstract refinement @p@ holds recursively
within the tail, ensuring that the relationship @p@ 
holds between \emph{all} pairs of list elements.

Thus, by plugging in appropriate concrete refinements, 
we can define the following aliases, which correspond 
to the informal notions implied by their names:
$$\centering\begin{tabular}{c}\begin{code}
type IncrList a = [a]<{\h v -> h <= v}>
type DecrList a = [a]<{\h v -> h >= v}>
type UniqList a = [a]<{\h v -> h != v}>
\end{code}\end{tabular}$$
%%
%\begin{code}
%type IncList a    = List <{\h v -> h <= v}> a 
%type DecList a    = List <{\h v -> h >= v}> a 
%type UniqueList a = List <{\h v -> h != v}> a 
%\end{code}
%
That is, @IncrList a@ (resp. @DecrList a@) describes a list sorted
in increasing (resp. decreasing) order, and @UniqList a@ describes
a list of \emph{distinct} elements, \ie not containing any duplicates.
We can use the above definitions to verify
%
$$\centering\begin{tabular}{c}\begin{code}
[1, 2, 3, 4] :: IncrList Int
[4, 3, 2, 1] :: DecrList Int
[4, 1, 3, 2] :: UniqList Int
\end{code}\end{tabular}$$
%
%
%\begin{code}
%xs :: IncList Int
%xs = 1 `Cons` 2 `Cons` 3 `Cons` 4 `Cons` Nil
%
%ys :: IncList Int
%ys = 4 `Cons` 3 `Cons` 2 `Cons` 1 `Cons` Nil
%
%zs :: UniqueList Int
%zs = 4 `Cons` 1 `Cons` 3 `Cons` 2 `Cons` Nil
%\end{code}
%
More interestingly, we can verify that the usual algorithms 
produce sorted lists:
%
$$\centering\begin{tabular}{c}\begin{code}
insertSort :: (Ord a) => [a] -> IncrList a 
insertSort []     = []
insertSort (x:xs) = insert x (insertSort xs) 

insert :: (Ord a) => a -> IncrList a -> IncrList a 
insert y []       = [y]
insert y (x:xs) 
  | y <= x        = y : x : xs
  | otherwise     = x : insert y xs
\end{code}\end{tabular}$$
%
%insertSort        :: (Ord a) => List a -> IncList a 
%insertSort Nil           = Nil 
%insertSort (x `Cons` xs) = insert x (insertSort xs) 
%
%insert y Nil                  
%  = y `Cons` Nil 
%insert y (Cons x xs) 
%  | y <= x    = y `Cons` (x `Cons` xs) 
%  | otherwise = x `Cons` (insert y xs)
%
%
Thus, abstract refinements allow us to \emph{decouple} the definition 
of the list from the actual invariants that hold.
This, in turn, allows us to conveniently reuse the same 
underlying (non-refined) type to implement various algorithms 
unlike, say, singleton-type based implementations which require 
up to three different types of lists (with three different ``nil" and ``cons" 
constructors~\cite{Sheard06}). This, makes abstract refinements 
convenient for verifying complex sorting implementations like that of 
@Data.List.sort@ which, for efficiency, use lists with different 
properties (\eg increasing and decreasing).

\mypara{Multiple Recursive Refinements} 
We can define recursive types with multiple parameters. 
For example, consider the following refined version of a type used 
to encode functional maps (\ttcode{Data.Map}):
%
$$\centering\begin{tabular}{c}\begin{code}
data Tree k v <l :: k->k->Bool, r :: k->k->Bool>
  = Bin { key   :: k
        , value :: v 
        , left  :: Tree <l, r> (k <l key>) v 
        , right :: Tree <l, r> (k <r key>) v }
  | Tip
\end{code}\end{tabular}$$
%
The abstract refinements \ttcode{l} and \ttcode{r} relate each \ttcode{key}
of the tree with \emph{all} the keys in the \emph{left} and \emph{right}
subtrees of \ttcode{key}, as those keys are respectively of type 
\ttcode{k <l key>} and \ttcode{k <r key>}.
%
Thus, if we instantiate the refinements with the following predicates
$$\centering\begin{tabular}{c}\begin{code}
type BST k v     = Tree<{\x y -> x> y},{\x y-> x< y}> k v
type MinHeap k v = Tree<{\x y -> x<=y},{\x y-> x<=y}> k v
type MaxHeap k v = Tree<{\x y -> x>=y},{\x y-> x>=y}> k v
\end{code}\end{tabular}$$
then @BST k v@, @MinHeap k v@ and @MaxHeap k v@ 
denote exactly binary-search-ordered, min-heap-ordered, and
max-heap-ordered trees (with keys and values of types @k@ and
@v@).  
%
We demonstrate in (\S~\ref{sec:experiments}) how we use the above types to 
automatically verify ordering properties of complex, full-fledged libraries.

\subsection{Inductive Invariants}\label{sec:overview:induction}

Finally, we explain how abstract refinements allow us to formalize 
some kinds of structural induction within the type system. 

\mypara{Measures} First, let us formalize a notion of \emph{length} for
lists within the refinement logic. To do so, we define a special 
\ttcode{len} measure by structural induction
%
$$\centering\begin{tabular}{c}\begin{code}
measure len :: [a] -> Int 
len []      = 0 
len (x:xs)  = 1 + len(xs)
\end{code}\end{tabular}$$
%
We use the measures to automatically strengthen the 
types of the data constructors\cite{LiquidPLDI09}:
%
$$\centering\begin{tabular}{c}\begin{code}
data [a] where 
  []  :: forall a.{v:[a] | len(v) = 0}
  (:) :: forall a.a -> xs:[a] -> {v:[a]|len(v)=1+len(xs)}
\end{code}\end{tabular}$$
%
Note that the symbol \ttcode{len} is encoded as an \emph{uninterpreted}
function in the refinement logic, and is, except for the congruence axiom,
opaque to the SMT solver. The measures are guaranteed, by construction, 
to terminate, and so we can soundly use them as uninterpreted 
functions in the refinement logic. Notice also, that we can define 
\emph{multiple} measures for a type; in this case we simply conjoin 
the refinements from each measure when refining each data constructor.

With these strengthened constructor types, we can verify, for example,
that @append@ produces a list whose length is the sum of the input lists'
lengths:
%
$$\centering\begin{tabular}{c}\begin{code}
append :: l:[a] -> m:[a] -> {v:[a]|len(v)=len(l)+len(m)}
append []     zs = zs
append (y:ys) zs = y : append ys zs
\end{code}\end{tabular}$$
%
However, consider an alternate definition of @append@ that uses @foldr@
%
$$\centering\begin{tabular}{c}\begin{code}
append ys zs = foldr (:) zs ys 
\end{code}\end{tabular}$$
%
where @foldr :: (a -> b -> b) -> b -> [a] -> b@.
It is unclear how to give @foldr@ a (first-order) refinement type
that captures the rather complex fact that the fold-function 
is ``applied" all over the list argument, or, that it is a catamorphism.
Hence, hitherto, it has not been possible to verify the second definition 
of @append@.


\mypara{Typing Folds} Abstract refinements allow us to 
solve this problem with a very expressive type for \ttcode{foldr} 
whilst remaining firmly within the boundaries of SMT-based 
decidability. We write a slightly modified fold:
%
$$\centering\begin{tabular}{c}\begin{code}
foldr :: forall <p :: [a] -> b -> Bool>. 
             (xs:[a] -> x:a -> b <p xs> -> <p (x:xs)>) 
          -> b<p []> 
          -> ys:[a]
          -> b<p ys>
foldr op b []     = b
foldr op b (x:xs) = op xs x (foldr op b xs) 
\end{code}\end{tabular}$$
%
The trick is simply to quantify over the relationship @p@
that @foldr@ establishes between the input list @xs@ and
the output @b@ value. This is formalized by the type signature,
which encodes an induction principle for lists: 
the base value @b@ must 
(1)~satisfy the relation with the empty list,
and the function @op@ must take 
(2)~a value that satisfies the relationship with the tail 
    @xs@ (we have added the @xs@ as an extra ``ghost"
    parameter to @op@), 
(3)~a head value @x@, and return
(4)~a new folded value that satisfies the relationship with \ttcode{x:xs}.
If all the above are met, then the value returned by @foldr@
satisfies the relation with the input list @ys@.
%
This scheme is not novel in itself~\cite{coq-book}
--- what is new is the encoding, via uninterpreted predicate symbols, 
in an SMT-decidable refinement type system.

\mypara{Using Folds} Finally, we can use the expressive type
for the above @foldr@ to verify various inductive properties 
of client functions:
%
$$\centering\begin{tabular}{c}\begin{code}
length :: zs:[a] -> {v: Int | v = len(zs)}
length = foldr (\_ _ n -> n + 1) 0

append :: l:[a] -> m:[a] -> {v:[a]| len(v)=len(l)+len(m)}
append ys zs = foldr (\_ -> (:)) zs ys 
\end{code}\end{tabular}$$
%
The verification proceeds by just (automatically) instantiating the 
refinement parameter \ttcode{p} of \ttcode{foldr} with the concrete
refinements, via Liquid typing:
%
$$\centering\begin{tabular}{c}\begin{code}
{\xs v -> v = len(xs)}                 -- for length
{\xs v -> len(v) = len(xs) + len(zs)}  -- for append
\end{code}\end{tabular}$$

%%This concludes a tour of the many kinds of expressive specifications
%%that abstract refinements enable, while preserving fully automatic
%%verification. 

\section{Syntax and Semantics}\label{sec:check}


Next, we present a core calculus \corelan that formalizes the notion
of abstract refinements. We start with the syntax (\S~\ref{sec:syntax}),
present the typing rules (\S~\ref{sec:typing}), show soundness 
via a reduction to contract calculi \cite{Knowles10,Greenberg11}
(\S~\ref{sec:soundness}), and inference via Liquid types (\S~\ref{sec:infer}).

\subsection{Syntax}\label{sec:syntax}

Figure~\ref{fig:syntax} summarizes the syntax of our core 
calculus \corelan which is a polymorphic $\lambda$-calculus 
extended with abstract refinements. 
%
We write 
$b$, 
$\tref{b}{\reft}$ and 
$\tpp{b}{\areft}$ 
to abbreviate 
$\tpref{b}{\true}{\true}$, 
$\tpref{b}{\true}{\reft}$, and
$\tpref{b}{\areft}{\true}$ respectively. 
We say a type or schema is \emph{non-refined} if all the 
refinements in it are $\true$. We write $\overline{z}$ 
to abbreviate a sequence $z_1 \ldots z_n$.


\mypara{Expressions}
\corelan\ expressions include the standard variables $x$, 
primitive constants $c$, $\lambda$-abstraction $\efunt{x}{\tau}{e}$,
application $\eapp{e}{e}$, type abstraction $\etabs{\alpha}{e}$,
and type application $\etapp{e}{\tau}$. The parameter $\tau$ in 
the type application is a \emph{refinement type}, as described shortly.
The two new additions to \corelan are the refinement abstraction
$\epabs{\rvar}{\tau}{e}$, which introduces a refinement variable 
$\rvar$ (together with its type $\tau$), which can appear in refinements
inside $e$, and the corresponding refinement application $\epapp{e}{e}$.
%
%where the argument, is of the form $\efun{\bar{x}{e}}$ which is
%an abbreviation for $\efun{x_1 \ldots x_n}{e}$.
%which is of the form $\ptype{\bar{\tau}}$ which is an 
%abbreviation for $\ptype{tau_1 \rightarrow \ldots \tau_n}$,
%where each $\tau_i$ is a simple (non-function) type.

\mypara{Refinements}
A \emph{concrete refinement} \reft is a boolean valued expression \reft 
drawn from a strict subset of the language of expressions which
includes only terms that 
(a)~neither diverge nor crash, and 
(b)~can be embedded into an SMT decidable refinement logic including 
the theory of linear arithmetic and uninterpreted functions.
%
An \emph{abstract refinement} \areft is a conjunction of refinement
variable applications of the form $\rvapp{\pi}{e}$.

\mypara{Types and Schemas}
The basic types of \corelan include the base types $\tbint$ and $\tbbool$
and type variables $\alpha$. An \emph{abstract refinement type} $\tau$ is 
either a basic type refined with an abstract and concrete refinements,
$\tpref{b}{\areft}{\reft}$, or 
a dependent function type where the parameter $x$ can appear in the 
refinements of the output type. 
We include refinements for functions, as refined type variables can be 
replaced by function types. However, typechecking ensures these refinements
are trivially true.
%
%type application
%Type application consists of a type constructor, its type arguments
%and its refinement arguments 
%that are used to describe properties between its elements.
%
Finally, types can be quantified over refinement variables and type 
variables to yield abstract refinement schemas.


\begin{figure}[t!]
\centering
$$
\begin{array}{rrcl}
\emphbf{Expressions} \quad 
  & e 
  & ::= 
  &      x 
  \spmid c 
  \spmid \efunt{x}{\tau}{e} 
  \spmid \eapp{e}{e} 
  \spmid \etabs{\alpha}{e} 
  \spmid \etapp{e}{\tau} 
  \spmid \epabs{\rvar}{\tau}{e}
  \spmid \epapp{e}{e} 
  \\[0.05in] 

\emphbf{Abstract Refinements} \quad 
  & \areft 
  & ::= 
  &      \true 
  \spmid \areft \land \rvapp{\rvar}{e}
  \\[0.05in] 

\emphbf{Basic Types} \quad 
  & b 
  & ::= 
  &      \tbint
  \spmid \tbbool
  \spmid \alpha
  \\[0.05in]

\emphbf{Abstract Refinement Types} \quad 
  & \tau 
  & ::= 
  &      \tpref{b}{\areft}{\reft} 
  \spmid \trfun{x}{\tau}{\tau}{\reft}
  \\[0.05in]

\emphbf{Abstract Refinement Schemas} \quad 
  & \sigma
  & ::= 
  &      \tau 
  \spmid \ttabs{\alpha}{\sigma}
  \spmid \tpabs{\rvar}{\tau}{\sigma}
  \\[0.05in]
\end{array}
$$
\caption{\textbf{Syntax of Expressions, Refinements, Types and Schemas}}
\label{fig:syntax}
\end{figure}

\subsection{Static Semantics}\label{sec:typing}

\begin{figure}[p]
\centering
\captionsetup{justification=centering}
\judgementHead{Well-Formedness}{\isWellFormed{\Gamma}{\sigma}}

$$\begin{array}{ccc}
\inference
  {}
  {\isWellFormed{\Gamma}{\true(\vref)}}
  [\wtTrue]
&
\quad
&
\inference
    {\isWellFormed{\Gamma}{\areft(\vref)} && 
     \hastype{\Gamma}{\rvapp{\rvar}{e} \ \vref}{\tbbool}
    }
    {\isWellFormed{\Gamma}{(\areft \wedge \rvapp{\rvar}{e})(\vref)}}
    [\wtRVApp]
\end{array}$$
%
$$\inference
    {\hastype{\Gamma, \vref:b}{\reft}{\tbbool} \quad 
%     \isWellFormed{\Gamma, \vref:b}{\areft(\vref)}
     \hastype{\Gamma, \vref:b}{\areft(\vref)}{\tbbool}
    }
    {\isWellFormed{\Gamma}{\tpref{b}{\areft}{\reft}}}
    [\wtBase]
$$
%
$$
\inference
    {
	%\hastype{\Gamma, v:\tfun{x}{\tau_x}{\tau}}{e}{\tbbool} &&
	\hastype{\Gamma}{\reft}{\tbbool} &&
    \isWellFormed{\Gamma}{\tau_x} &&
	\isWellFormed{\Gamma, x:\tau_x}{\tau}
    }
    {\isWellFormed{\Gamma}{\trfun{x}{\tau_x}{\tau}{\reft}}}
    [\wtFun]
$$
%
$$\begin{array}{ccc}
\inference
  {\isWellFormed{\Gamma, \rvar:\tau}{\sigma}}
  {\isWellFormed{\Gamma}{\tpabs{\rvar}{\tau}{\sigma}}}
  [\wtPred]
&
\quad
&
\inference
    {\isWellFormed{\Gamma, \alpha}{\sigma}}
    {\isWellFormed{\Gamma}{\ttabs{\alpha}{\sigma}}}
    [\wtPoly]
\end{array}$$

\medskip \judgementHead{Subtyping}{\isSubType{\Gamma}{\sigma_1}{\sigma_2}}

$$
\inference
   {\text{SMT-Valid}(\inter{\Gamma} \land \inter{\areft_1\ \vref} \land \inter{\reft_1} 
                 \Rightarrow \inter{\areft_2\ \vref} \land \inter{\reft_2})}
   {\isSubType{\Gamma}{\tpref{b}{\areft_1}{\reft_1}}{\tpref{b}{\areft_2}{\reft_2}}}
   [\tsubBase]
$$
%
$$
\inference
   {%\text{Valid}(\inter{\Gamma}\land \inter{e_1} \Rightarrow \inter{e_2}) \\
	\isSubType{\Gamma}{\tau_2}{\tau_1} &
	\isSubType{\Gamma, x_2:{\tau_2}}{\SUBST{\tau_1'}{x_1}{x_2}}{\tau_2'}	
   }
   {\isSubType{\Gamma}
	  {\trfun{x_1}{\tau_1}{\tau_1'}{\reft_1}}
	  {\trfun{x_2}{\tau_2}{\tau_2'}{\true}}
}[\tsubFun]
$$
%
$$
\begin{array}{ccc}
\inference
   {\isSubType{\Gamma, \rvar:\tau}{\sigma_1}{\sigma_2}}
   {\isSubType{\Gamma}{\tpabs{\rvar}{\tau}{\sigma_1}}{\tpabs{\rvar}{\tau}{\sigma_2}}}
   [\tsubPred]
&
\quad
&
\inference
   {\isSubType{\Gamma}{\sigma_1}{\sigma_2}}
   {\isSubType{\Gamma}{\ttabs{\alpha}{\sigma_1}}{\ttabs{\alpha}{\sigma_2}}}
   [\tsubPoly]
\end{array}
$$

\medskip \judgementHead{Type Checking}{$\hastype{\Gamma}{e}{\sigma}$}

$$\inference
  {  \hastype{\Gamma}{e}{\sigma_2} && \isSubType{\Gamma}{\sigma_2}{\sigma_1} 
  && \isWellFormed{\Gamma}{\sigma_1}
  }
  {\hastype{\Gamma}{e}{\sigma_1}}
  [\tsub]
\quad
\inference
  {}
  {\hastype{\Gamma}{c}{\tc{c}}}
  [\tconst]
$$
$$
\inference
  {x: \tpref{b}{\areft}{\reft} \in \Gamma}
  {\hastype{\Gamma}{x}{\tpref{b}{\areft}{e \land \vref = x}}}
  [\tbase]
\quad
\inference
  {x:\tau \in \Gamma}
  {\hastype{\Gamma}{x}{\tau}} 
  [\tvariable]
$$
$$
\inference
   {\hastype{\Gamma, x:\tau_x}{e}{\tau} 
    && \isWellFormed{\Gamma}{\tau_x}
   }
   {\hastype{\Gamma}{\efunt{x}{\tau_x}{e}}{\tfun{x}{\tau_x}{\tau}}}
   [\tfunction]
\quad
\inference
   {\hastype{\Gamma}{e_1}{\tfun{x}{\tau_x}{\tau}} 
   &&  \hastype{\Gamma}{e_2}{\tau_x}
   }
   {\hastype{\Gamma}{\eapp{e_1}{e_2}}{\SUBST{\tau}{x}{e_2}}}
   [\tapp]
$$
$$
\inference
  {\hastype{\Gamma, \alpha}{e}{\sigma}}
  {\hastype{\Gamma}{\etabs{\alpha}{e}}{\ttabs{\alpha}{\sigma}}}
  [\tgen]
\quad
\inference
  {\hastype{\Gamma}{e}{\ttabs{\alpha}{\sigma}} && 
   \isWellFormed{\Gamma}{\tau}
  }
  {\hastype{\Gamma}{\etapp{e}{\tau}}{\SUBST{\sigma}{\alpha}{\tau}}}
  [\tinst]
$$
$$
\inference
    {\hastype{\Gamma, \rvar:\tau}{e}{\sigma} &&
     \isWellFormed{\Gamma}{\tau} 
     % \tau \mbox{ is non-refined } 
     %\isWellFormed{\Gamma}{\tpabs{p}{\tau}{\pi}} && 
     %p \notin \fv{e}
    }
    {\hastype{\Gamma}{\epabs{\rvar}{\tau}{e}}{\tpabs{\rvar}{\tau}{\sigma}}}
    [\tpgen]
\ \
\inference
    {\hastype{\Gamma}{e}{\tpabs{\rvar}{\tau}{\sigma}} && 
     \hastype{\Gamma}{\efunbar{x:\tau_x}{\reft'}}{\tau}
    }
    {\hastype{\Gamma}
             {\epapp{e}{\efunbar{x:\tau_x}{\reft'}}}
             {\rpinst{\sigma}{\rvar}{\efunbar{x:\tau_x}{\reft'}}}
     %        {\sigma\sub{\eapp{p}{\overline{e_p}}}{\eapp{\reft'}{\overline{e_p}}}}
    }
    [\tpinst]
$$
\caption[Type checking of \corelan.]{Well-formedness, Subtyping and Type Checking of \corelan.}
\label{fig:rules}
\end{figure}



Next, we describe the static semantics of \corelan by describing the typing
judgments and derivation rules. Most of the rules are 
standard~\cite{Ou2004,LiquidPLDI08,Knowles10,GordonTOPLAS2011}; we 
discuss only those pertaining to abstract refinements.

\mypara{Judgments}
A type environment $\Gamma$ is a sequence of type bindings $x:\sigma$.
We use environments to define three kinds of typing judgments:

\begin{itemize}
\item{\emphbf{Wellformedness judgments} (\isWellFormed{\Gamma}{\sigma})} 
state that a type schema $\sigma$ is well-formed under environment
$\Gamma$, that is, the refinements in $\sigma$ are boolean 
expressions in the environment $\Gamma$.

\item{\emphbf{Subtyping judgments} (\isSubType{\Gamma}{\sigma_1}{\sigma_2})} 
state that the type schema $\sigma_1$ is a subtype of the type schema
$\sigma_2$ under environment $\Gamma$, that is, when the free variables
of $\sigma_1$ and $\sigma_2$
are bound to values described by $\Gamma$, the set of values described
by $\sigma_1$ is contained in the set of values described by $\sigma_2$. 

\item{\emphbf{Typing judgments} (\hastype{\Gamma}{e}{\sigma})} state that
the expression $e$ has the type schema $\sigma$ under environment $\Gamma$,
that is, when the free variables in $e$ are bound to values described by 
$\Gamma$, the expression $e$ will evaluate to a value described by $\sigma$.
\end{itemize}

\mypara{Wellformedness Rules}
The wellformedness rules check that the concrete and abstract
refinements are indeed $\tbbool$-valued expressions in the 
appropriate environment.
The key rule is \wtBase, which checks, as usual, that the (concrete) 
refinement $\reft$ is boolean, and additionally, that the abstract
refinement $\areft$ applied to the value $\vref$ is also boolean.
This latter fact is established by \wtRVApp which checks that 
each refinement variable application $\rvapp{\rvar}{e}\ \vref$ 
is also of type \tbbool in the given environment.

\mypara{Subtyping Rules}
The subtyping rules stipulate when the set of values described 
by schema $\sigma_1$ is subsumed by the values described by $\sigma_2$.
The rules are standard except for \tsubVar, which encodes the base types' 
abstract refinements $\areft_1$ and $\areft_2$ with conjunctions of 
\emph{uninterpreted predicates} 
$\inter{\areft_1\ \vref}$ and $\inter{\areft_2\ \vref}$ in the 
refinement logic as follows:
\begin{align*}
\inter{\true\ \vref} & \defeq \true\\
\inter{(\areft \land \rvapp{\rvar}{e})\ \vref} & \defeq \inter{\areft\
\vref} \land \rvar(\inter{e_1},\ldots,\inter{e_n},\vref)
\end{align*}
where $\rvar(\overline{e})$ is a term in the refinement logic corresponding
to the application of the uninterpreted predicate symbol $\rvar$ to the 
arguments $\overline{e}$.
% $\text{Valid}(p)$ (\tsubBase) holds if an SMT determines the formula $p$ 
% is \emph{valid}~\cite{Nelson81}.



\mypara{Type Checking Rules}
The type checking rules are standard except for \tpgen and \tpinst, which
pertain to abstraction and instantiation of abstract refinements.
%
The rule \tpgen is the same as \tfunction: we simply check the body
$e$ in the environment extended with a binding for the refinement 
variable $\rvar$.
%
The rule \tpinst checks that the concrete refinement is of the appropriate
(unrefined) type $\tau$, and then replaces all (abstract) applications of
$\rvar$ inside $\sigma$ with the appropriate (concrete) refinement $\reft'$ 
with the parameters $\overline{x}$ replaced with arguments at that application.
Formally, this is represented as $\rpinst{\sigma}{\rvar}{\efunbar{x:\tau}{\reft'}}$
which is $\sigma$ with each base type transformed as
%%$$\rpinst{\tpref{b}{\areft}{\reft}}{\rvar}{z}
%%  \defeq \tpref{b}{\areft''}{\reft \land \reft''} 
%%  \quad \mbox{where} (\areft'', \reft'') =
%%  \rpapply{\areft}{\rvar}{z}{\true}{\true}$$
\begin{align*}
\rpinst{\tpref{b}{\areft}{\reft}}{\rvar}{z}
  & \defeq \tpref{b}{\areft''}{\reft \land \reft''} \\
\mbox{where} \quad (\areft'', \reft'') 
  & \defeq \rpapply{\areft}{\rvar}{z}{\true}{\true} 
\intertext{$\mathsf{Apply}$ replaces each application of $\rvar$ in 
$\areft$ with the corresponding conjunct in $\reft''$, as}
\rpapply{\true}{\cdot}{\cdot}{\areft'}{\reft'} 
  & \defeq (\areft', \reft') \\
\rpapply{\areft \wedge \rvapp{\rvar'}{e}}{\rvar}{z}{\areft'}{\reft'} 
  & \defeq \rpapply{\areft}{\rvar}{z}{\areft' \land \rvapp{\rvar'}{e}}{\reft'} \\
\rpapply{\areft \wedge \rvapp{\rvar}{e}}{\rvar}{\efunbar{x:\tau}{\reft''}}{\areft'}{\reft'} 
  & \defeq
  \rpapply{\areft}{\rvar}{\efunbar{x:\tau}{\reft''}}{\areft'}{\reft' \wedge \SUBST{\reft''}{\overline{x}}{\overline{e},\vref}}
\end{align*}
In other words, the instantiation can be viewed as two symbolic 
reduction steps: first replacing the refinement variable with the
concrete refinement, and then ``beta-reducing" concrete refinement 
with the refinement variable's arguments. For example, 
$$\rpinst{\tpref{\tbint}{\rvar\ y}{\vref > 10}}
       {\rvar}
       {\efunt{x_1}{\tau_1}{\efunt{x_2}{\tau_2}{x_1 < x_2}}}
\defeq \tref{\tbint}{\vref > 10 \land y < \vref}$$
%%rp(x:tx->t , \rvar, z) = x:tx' -> t'
%%  where tx'      = rp(tx, \rvar, z)
%%        t'       = rp(t , \rvar, z)
%%
%%rp(\a.t, \rvar, z) = \a.t'
%%  where t'       = rp(t, \rvar, z)
%%
%%rp(\p.t, \rvar, z) = \p.t'
%%  where t'       = rp(t, \rvar, z)

%%The other rule that handles abstract refinements is \tcase.
%%This rule initially checks that the expression to be analyzed 
%%has a type application type $T = \tcon{\chi}{e_\chi}{\listOf{T}}{\listOf{e}}$.
%%Then for all cases, the case expression is typechecked in the initial environment, 
%%extended with case binders \listOf{x_i} and the initial expression binder $x$.
%%The types of these binders are gained after unfolding data constructor's type \tc{K_i}. 
%%The unfolding is done by replacing its type variables with actual type arguments
%%of $T$, ie. \listOf{T} 
%%its abstract refinements with actual inferred refinements \listOf{e},
%%and its binders with actual binders \listOf{x_i}.

\subsection{Soundness}\label{sec:soundness}

As hinted by the discussion about refinement variable instantiation,
we can intuitively think of abstract refinement variables as 
\emph{ghost} program variables whose values are boolean-valued 
functions. Hence, abstract refinements are a special case of 
higher-order contracts, that can be statically verified using 
uninterpreted functions. (Since we focus on static checking, 
we don't care about the issue of blame.)
We formalize this notion by translating \corelan programs into
the contract calculus \conlan of \cite{Greenberg11} and use this 
translation to define the dynamic semantics and establish soundness.

\mypara{Translation} 
We translate \corelan schemes $\sigma$ to \conlan schemes $\tx{\sigma}$
as by translating abstract refinements into contracts,
and refinement abstraction into function types:
$$\begin{array}{rclcrcl}
\tx{\true\ \vref} & \defeq 
  & \true  
  & \quad \quad &

\tx{\tpabs{\rvar}{\tau}{\sigma}} & \defeq 
  & \tfun{\rvar}{\tx{\tau}}{\tx{\sigma}} \\

\tx{(\areft \land \rvapp{\rvar}{e})\ \vref} & \defeq 
  & \tx{\areft\ \vref} \land \eapp{\eapp{\rvar}{\overline{e}}}{\vref} 
  & \quad \quad &

\tx{\ttabs{\alpha}{\sigma}} & \defeq 
  & \ttabs{\alpha}{\tx{\sigma}} \\

\tx{\tpref{b}{\areft}{\reft}} & \defeq 
  & \tref{b}{\reft \land \tx{\areft\ \vref}} 
  & \quad \quad &

\tx{\tfun{x}{\tau_1}{\tau_2}} & \defeq 
  & \tfun{x}{\tx{\tau_1}}{\tx{\tau_2}} 
%\tx{\trfun{x}{\tau_1}{\tau_2}{\reft}} \defeq 
%  & \trfun{x}{\tx{\tau_1}}{\tx{\tau_2}}{\tx{\reft}} \\
\end{array}$$
Similarly, we translate \corelan terms $e$ to \conlan 
terms $\tx{e}$ by converting refinement abstraction and application 
to $\lambda$-abstraction and application
$$\begin{array}{rclcrcl}
\tx{x} & \defeq & x & \quad \quad \quad & \tx{c} & \defeq & c \\
\tx{\efunt{x}{\tau}{e}} & \defeq & \efunt{x}{\tx{\tau}}{\tx{e}} & \quad & \tx{\eapp{e_1}{e_2}} & \defeq & \eapp{\tx{e_1}}{\tx{e_2}} \\
\tx{\etabs{\alpha}{e}} & \defeq & \etabs{\alpha}{\tx{e}} & \quad & \tx{\etapp{e}{\tau}} & \defeq & \eapp{\tx{e}}{\tx{\tau}} \\
\tx{\epabs{\rvar}{\tau}{e}} &\defeq & \efunt{\rvar}{\tx{\tau}}{\tx{e}} & \quad & \tx{\epapp{e_1}{e_2}} &\defeq & \eapp{\tx{e_1}}{\tx{e_2}}
\end{array}$$

%%\begin{align*}
%%\tx{\true\ \vref} \defeq 
%%  & \true\\
%%\tx{(\areft \land \rvapp{\rvar}{e})\ \vref} \defeq 
%%  & \tx{\areft\ \vref} \land \eapp{\eapp{\rvar}{\overline{e}}}{\vref}\\
%%\tx{\tpref{b}{\areft}{\reft}} \defeq 
%%  & \tref{b}{\reft \land \tx{\areft\ \vref}} \\
%%\tx{\tfun{x}{\tau_1}{\tau_2}} \defeq 
%%  & \tfun{x}{\tx{\tau_1}}{\tx{\tau_2}} \\
%%%\tx{\trfun{x}{\tau_1}{\tau_2}{\reft}} \defeq 
%%%  & \trfun{x}{\tx{\tau_1}}{\tx{\tau_2}}{\tx{\reft}} \\
%%\tx{\ttabs{\alpha}{\sigma}} \defeq 
%%  & \ttabs{\alpha}{\tx{\sigma}} \\
%%\tx{\tpabs{\rvar}{\tau}{\sigma}} \defeq 
%%  & \tfun{\rvar}{\tx{\tau}}{\tx{\sigma}}
%%\end{align*}
%%\tx{x} \defeq & x \\
%%\tx{c} \defeq & c \\
%%\tx{\efunt{x}{\tau}{e}} \defeq & \efunt{x}{\tx{\tau}}{\tx{e}} \\
%%\tx{\eapp{e_1}{e_2}} \defeq & \eapp{\tx{e_1}}{\tx{e_2}} \\
%%\tx{\etabs{\alpha}{e}} \defeq & \etabs{\alpha}{\tx{e}} \\
%%\tx{\etapp{e}{\tau}} \defeq & \eapp{\tx{e}}{\tx{\tau}} \\
%%\tx{\epabs{\rvar}{\tau}{e}} \defeq & \efunt{\rvar}{\tx{\tau}}{\tx{e}} \\
%%\tx{\epapp{e_1}{e_2}} \defeq & \eapp{\tx{e_1}}{\tx{e_2}}





\mypara{Translation Properties}
We can show by induction on the derivations that the 
type derivation rules of \corelan \emph{conservatively approximate}
those of \conlan. Formally, 

\begin{itemize}
\item If $\isWellFormed{\Gamma}{\tau}$ then $\isWellFormedH{\Gamma}{\tau}$,
\item If $\isSubType{\Gamma}{\tau_1}{\tau_2}$ then $\isSubTypeH{\Gamma}{\tau_1}{\tau_2}$,
\item If $\hastype{\Gamma}{e}{\tau}$ then
$\hastypeH{\tx{\Gamma}}{\tx{e}}{\tx{\tau}}$.
\end{itemize}

\mypara{Soundness} Thus rather than re-prove preservation and progress
for \corelan, we simply use the fact that the type derivations are
conservative to derive the following preservation and progress 
corollaries from \cite{Greenberg11}:
%
\begin{itemize}
\item{\textbf{Preservation: }} 
  If $\hastype{\emptyset}{e}{\tau}$ 
  and $\tx{e} \longrightarrow e'$ 
  then $\hastypeH{\emptyset}{e'}{\tx{\tau}}$

\item{\textbf{Progress: }}
  If $\hastype{\emptyset}{e}{\tau}$, then either
  $\tx{e} \longrightarrow e'$ or 
  $\tx{e}$ is a value.
\end{itemize}
%
Note that, in a contract calculus like \conlan, subsumption is encoded
as a \emph{upcast}. However, if subtyping relation can be statically 
guaranteed (as is done by our conservative SMT based subtyping) 
then the upcast is equivalent to the identity function and can 
be eliminated. Hence, \conlan terms $\tx{e}$ translated from well-typed 
\corelan terms $e$ have no casts.

\subsection{Refinement Inference}\label{sec:infer}

Our design of abstract refinements makes it particularly easy to 
perform type inference via Liquid typing, which is crucial for
making the system usable by eliminating the tedium of instantiating 
refinement parameters all over the code. (With value-dependent 
refinements, one cannot simply use, say, unification to determine
the appropriate instantations, as is done for classical type systems.)
We briefly recall how Liquid types work, and sketch how they are 
extended to infer refinement instantiations.

\mypara{Liquid Types} 
The Liquid Types method infers refinements in three steps. 
%
First, we create refinement \emph{templates} for the unknown, 
to-be-inferred refinement types. The \emph{shape} of the template 
is determined by the underlying (non-refined) type it corresponds to, 
which can be determined from the language's underlying (non-refined) 
type system. 
The template is just the shape refined with fresh refinement variables
$\kappa$ denoting the unknown refinements at each type position. 
For example, from a type ${\tfun{x}{\tbint}{\tbint}}$ we create 
the template ${\tfun{x}{\tref{\tbint}{\kappa_x}}{\tref{\tbint}{\kappa}}}$.
%
Second, we perform type checking using the templates (in place of the
unknown types.) Each wellformedness check becomes a wellformedness
constraint over the templates, and hence over the individual $\kvar$,
constraining which variables can appear in $\kvar$.
Each subsumption check becomes a subtyping constraint
between the templates, which can be further simplified, via syntactic
subtyping rules, to a logical implication query between the variables
$\kappa$.
%
Third, we solve the resulting system of logical implication constraints
(which can be cyclic) via abstract interpretation --- in particular,
monomial predicate abstraction over a set of logical qualifiers
\cite{Houdini,LiquidPLDI08}. The solution is a map from $\kvar$ to
conjunctions of qualifiers, which, when plugged back into the templates,
yields the inferred refinement types.

\mypara{Inferring Refinement Instantiations}
The key to making abstract refinements practical is a means of 
synthesizing the appropriate arguments $\reft'$ for each refinement 
application $\epapp{e}{\reft'}$. 
Note that for such applications, we can, from $e$, determine the 
non-refined type of $\reft'$, which is of the form 
${\tau_1 \rightarrow \ldots \rightarrow \tau_n \rightarrow \tbbool}$.
Thus, $\reft'$ has the template 
${\efunt{x_1}{\tau_1}{\ldots \efunt{x_n}{\tau_n}{\kvar}}}$
where $\kvar$ is a fresh, unknown refinement variable that 
must be solved to a boolean valued expression over $x_1,\ldots,x_n$.
Thus, we generate a \emph{wellformedness} constraint 
${\isWellFormed{x_1:\tau_1, \ldots, x_n:\tau_n}{\kvar}}$
and carry out typechecking with template, which, as before, yields
implication constraints over $\kvar$, which can, as before, be 
solved via predicate abstraction.
Finally, in each refinement template, we replace each $\kvar$ with its
solution $e_\kvar$ to get the inferred refinement instantiations.

%%To infer appropabstract refinements we used the liquid type variables
%%$\kappa$ with explicit arguments to avoid inferring function 
%%expressions:
%%To infer an expression to replace a predicate of type 
%%$\listOf{x_i:\tau_i; \tau}$ in an environment $\Gamma$, 
%%we create a fresh liquid variable $\kappa$ on $\tau$ which is
%%wellformed in the environment $\Gamma$ extended with the bindings
%%\listOf{x_i:\tau_i}.
%%When $\kvar$ is solved via predicate abstraction to an expression 
%%$e_\kvar$, we simply replace $\kvar$ with 
%%the set as the inferred expression 
%%$e =\efun{x_1}{\dots\efun{x_n}{\efun{v}{e_\kappa}}}$.
%add a refinement variable in place of expressions. 
%\mypara{Constants}\jhala{constant-refinements and soundness guarantees?}
%%TODO: 
%%[SKIP] add paragraph on constants to opsem
%%[SKIP] redefine \reft to r? (to emphasize not arbitrary expression?)


\section{Evaluation}\label{sec:experiments}


\def\benchToy{\texttt{Micro}}            
\def\benchFold{\texttt{Folds}}          
\def\benchVec{\texttt{Vector}}         
\def\benchSort{\texttt{ListSort}}       
\def\benchGsort{\texttt{Data.List.sort}} 
\def\benchSplay{\texttt{Data.Set.Splay}} 
\def\benchMap{\texttt{Data.Map.Base}}  


\def\locToy{32}  
\def\specToy{19}    
\def\cspecToy{948}    
\def\nspecToy{9}    
\def\annToy{4}    
\def\cannToy{115}    
\def\nannToy{1}    
\def\modToy{TBD}    
\def\timeToy{2}    

\def\locFold{0}  
\def\specFold{0}  
\def\annFold{0}  
\def\modFold{TBD}  
\def\timeFold{0}  

\def\locVec{56}   
\def\specVec{56}   
\def\nspecVec{14}   
\def\cspecVec{2203}   
\def\annVec{2}  
\def\cannVec{0}  
\def\nannVec{0}  
\def\modVec{TBD}  
\def\timeVec{14}

%ListSort.hs
\def\locSort{29}  
\def\specSort{4}  
\def\cspecSort{193}  
\def\nspecSort{4}  
\def\annSort{1}  
\def\cannSort{47}  
\def\nannSort{1}  
\def\modSort{TBD}  
\def\timeSort{3}  

%GhcListSort.hs
\def\locGsort{71} 
\def\specGsort{3}%lines
\def\cspecGsort{142}%characters
\def\nspecGsort{3}%number
\def\annGsort{1}
\def\cannGsort{47}
\def\nannGsort{1}
\def\modGsort{TBD}
\def\timeGsort{8} 

%Splay.hs
\def\locSplay{136} 
\def\specSplay{15} 
\def\nspecSplay{15} 
\def\cspecSplay{617} 
\def\annSplay{11} 
\def\nannSplay{6} 
\def\cannSplay{590} 
\def\modSplay{TBD} 
\def\timeSplay{15} 

\def\locMap{1399}   
\def\specMap{119}   
\def\cspecMap{10830}   
\def\nspecMap{86}   
\def\annMap{31}   
\def\cannMap{761}   
\def\nannMap{7}   
\def\modMap{TBD}   
\def\timeMap{235}

%total
\edef\locTot{\number\numexpr
  \locToy + \locFold + \locVec + 
  \locGsort + \locSort + \locSplay + \locMap 
  \relax}   

\edef\specTot{\number\numexpr
  \specToy + \specFold + \specVec + 
  \specGsort + \specSort + \specSplay + \specMap 
  \relax}   

\edef\annTot{\number\numexpr
  \annToy + \annFold + \annVec + 
  \annGsort + \annSort + \annSplay + \annMap 
  \relax}   

\def\modTot{0}   

\edef\timeTot{\number\numexpr
  \timeToy + \timeFold + \timeVec + 
  \timeGsort + \timeSort + \timeSplay + \timeMap 
  \relax}   


\begin{table}[t!]
\centering
% \captionsetup{justification=centering}
\begin{tabular}{lrrrrr}
\hline
\textbf{Program}& \textbf{LOC} & \textbf{Specs} & \textbf{Annot} &  \textbf{Time (s)} \\ 
\hline \hline
\benchToy       & \locToy   & \specToy   & \annToy   &  \timeToy     \\
\benchVec       & \locVec   & \specVec   & \annVec   &  \timeVec     \\
\benchSort      & \locSort  & \specSort  & \annSort  &  \timeSort    \\
\benchGsort     & \locGsort & \specGsort & \annGsort &  \timeGsort   \\
\benchSplay     & \locSplay & \specSplay & \annSplay &  \timeSplay   \\
\benchMap       & \locMap   & \specMap   & \annMap   &  \timeMap     \\
\hline
\textbf{Total}  & \locTot   & \specTot   & \annTot   &  \timeTot     \\
\end{tabular}
\caption[A quantitative evaluation of Abstract Refinements.]{ 
\textbf{(LOC)} is the number of non-comment Haskell source code lines as reported by \textit{sloccount},
%
\textbf{(Specs)} is the number of lines of type specifications,
%
\textbf{(Annot)} is the number of lines of other annotations, including refined
datatype definitions, type aliases and measures, required for verification,
%
\textbf{(Time)} is the time in seconds taken for verification.
}
\label{tab:eval}
\end{table}

In this section, we empirically evaluate the expressiveness and
usability of abstract refinement types by 
implementing abstract refinement in \toolname 
as uninterpreted functions.
%
We use \toolname to 
typecheck a set of challenging benchmark programs.
%
(We defer the task of extending the metatheory to a call-by-name calculus to future work).

\begin{comment}
\mypara{\toolname} We have implemented abstract refinement 
in \toolname, a refinement type checker for Haskell.
\toolname verifies Haskell source one module (.hs file) at a time. 
It takes as input:
\begin{itemize}
\item A \emph{target} Haskell source file, with the desired refinement
types specified as a special form of comment annotation, 
\item An (optional) set of type specifications for imported definitions; these 
can either be put directly in the source for the corresponding modules,
if available, or in special \verb+.spec+ files otherwise. For imported 
functions for which no signature is given, \toolname conservatively uses 
the non-refined  Haskell type. %with the trivial refinement @True@ everywhere.
\item An (optional) set of logical qualifiers, which are predicate 
templates from which refinements are automatically synthesized
\cite{LiquidPLDI08}.
%
Formally, a logical qualifier is a predicate whose variables range
over the program variables, the special value variable $\valu$, and
\emph{wildcards} $\placev$, which \toolname instantiates with the
names of program variables.
%
Aside from the qualifiers given by the user, \toolname also uses
qualifiers mined from the refinement type annotations present in the
program.
\end{itemize}
After analyzing the program, \toolname returns as output:
\begin{itemize}
\item Either \textsc{Safe}, indicating that all the specifications indeed
verify, or \textsc{Unsafe}, indicating there are refinement type errors,
together with the positions in the source code where type checking fails
(\eg functions that do not satisfy their signatures, or callsites where 
the inputs don't conform to the specifications).
%
\item
%
  An HTML file containing the program source code annotated with
  inferred refinement types for all sub-expressions in the program.
%
  The inferred refinement type for each program expression is the
  strongest possible type over the given set of logical qualifiers.
%
  When a type error is reported, the programmer can use the inferred
  types to determine why their program does not typecheck:
  they can examine what properties \toolname \emph{can} deduce about 
  various program expressions and add more qualifiers or alter 
  the program as necessary so that it typechecks.
%
\end{itemize}

\mypara{Implementation}
%
\toolname verifies the contents of a single file (module) at a time as follows. 
%
First, the Haskell source is fed into GHC, which desugars the program
to GHC's ``core" intermediate representation~\cite{VytiniotisJM12}.
%
Second, the desugared program,
the type signatures for the module functions (which are to be verified) and 
the type signatures for externally imported functions (which are assumed to hold)
are sent to the constraint generator, which traverses the core bindings in a
syntax-directed manner to generate subtyping constraints.
%
The resulting constraints are simplified via our subtyping rules
(\S~\ref{sec:check}) into simple logical implication
constraints.
%
Finally, the implication constraints, together with the logical
qualifiers provided by the user and harvested from the type
signatures, are sent into an SMT- and abstract interpretation-based
fixpoint computation procedure that determines if the constraints are
satisfiable \cite{GrafSaidi97,Houdini}.
%
If so, the program is reported to be \emph{safe}.
%
Otherwise, each unsatisfiable constraint is mapped back to the
corresponding program source location that generated it and a
potential error is reported at that line in the program.
\end{comment}

\mypara{Benchmarks}
We have evaluated \toolname over the following list of benchmarks
which, in total, represent the different kinds of reasoning described in
\S~\ref{sec:overview}.
%
While we can prove, and previously have proved~\cite{LiquidPLDI09},
many so-called ``functional correctness" properties of these data
structures using refinement types, in this work we focus on the key
invariants which are captured by abstract refinements.

\begin{itemize}
\item \benchToy, which includes several functions demonstrating 
parametric reasoning with base values, type classes, and higher-order 
loop invariants for traversals and folds, as described in
\S~\ref{sec:overview:parametric} and \S~\ref{sec:overview:induction};

\item \benchVec, which includes the domain- and range-generic @Vec@
  functions and several ``clients"
  that use the generic @Vec@ to implement incremental initialization,
  null-terminated strings, and memoization, as described in
  \S~\ref{sec:overview:index};

\item \benchSort, which includes various textbook sorting algorithms
including insert-, merge- and quick-sort. We verify that the functions
actually produce sorted lists, \ie are of type @IncrList a@, as described in
\S~\ref{sec:overview:rec};

\item \benchGsort, which includes three non-standard, optimized list 
sorting algorithms, as found in the \verb+base+ package. 
These employ lists that are increasing and decreasing, as well as 
lists of (sorted) lists, but we can verify that they also finally 
produce values of type @IncrList a@;

\item \benchSplay, which is a purely functional, top-down splay set 
library from the \verb+llrbtree+ package. We verify that all the 
interface functions take and return binary search trees;

\item \benchMap, which is the widely-used implementation of functional
maps from the \verb+containers+ package. We verify that all the interface
functions preserve the crucial binary search ordering property and various
related invariants.
\end{itemize}
%
Table~\ref{tab:eval} quantitatively summarizes the results of our
evaluation.
%
We now give a qualitative account of our experience using \toolname
by discussing what the specifications and other annotations look like.

\mypara{Specifications are usually simple}
In our experience, abstract refinements greatly simplify writing
specifications for the \emph{majority} of interface or public functions.
For example, for \benchMap, we defined the refined version of the
@Tree@  ADT (actually called @Map@ in the source, we reuse the type from
\S~\ref{sec:overview:rec} for brevity), and then instantiated
it with the concrete refinements for binary-search ordering with the alias
@BST k v@  as described in \S~\ref{sec:overview:rec}.
%%$${\centering\begin{tabular}{c}\begin{code}
%%type BST k v = Tree <{\x y -> x > y}, {\x y-> x < y}> k v
%%\end{code}\end{tabular}}$$
Most refined specifications were just the Haskell types
with the @Tree@ type constructor replaced with the 
alias @BST@. For example, the type of 
@fromList@ is refined from @(Ord k) => [(k, a)] -> Tree k a@ to 
@(Ord k) => [(k, a)] -> BST k a@.
%%$${\centering\begin{tabular}{c}\begin{code}
%%insert   :: (Ord k) => k -> v   -> Tree k v -> Tree k v
%%union    :: (Ord k) => k -> v   -> Tree k v -> Tree k v
%%fromList :: (Ord k) => [(k, a)] -> Tree k a 
%%\end{code}\end{tabular}}$$
%%simply get transformed to
%%$${\centering\begin{tabular}{c}\begin{code}
%%insert   :: (Ord k) => k -> v   -> BST k v -> BST k v  
%%union    :: (Ord k) => k -> v   -> BST k v -> BST k v  
%%fromList :: (Ord k) => [(k, a)] -> BST k a              
%%\end{code}\end{tabular}}$$
Furthermore, intra-module Liquid type inference permits 
the automatic synthesis of necessary stronger types for
private functions.
%%$$
%%{\centering
%%\begin{tabular}{c}\begin{code}
%%balanceL :: kcut:k -> a                -- root key, value
%%            -> BST {v:k | v < kcut} a  -- left  tree
%%            -> BST {v:k | v > kcut} a  -- right tree
%%            -> BST k a                 -- output tree
%%\end{code}
%%\end{tabular}}
%%$$
%but of course, one may explicitly specify these types too.

\mypara{Auxiliary Invariants are sometimes Difficult}
However, there are often rather thorny \emph{internal} functions with tricky
invariants, whose specification can take a bit of work. For example, the
function @trim@ in {\benchMap} has the following behavior (copied verbatim
from the documentation):
``\verb-trim blo bhi t- trims away all subtrees that surely
  contain no values between the range \verb-blo- to \verb-bhi-. 
   The returned tree is either empty or the key of the 
   root is between \verb-blo- and \verb-bhi-."
Furthermore @blo@ (resp. @bhi@) are specified as option 
(\ie @Maybe@) values with @Nothing@ denoting $-\infty$ (resp. $+\infty$). 
%
Fortunately, refinements suffice to encode such properties. 
First, we define measures
%
\begin{code}
  measure isJust     :: Maybe a -> Bool 
  isJust (Just x)    = true
  isJust (Nothing)   = false

  measure fromJust   :: Maybe a -> a 
  fromJustS (Just x) = x 

  measure isBin       :: Tree k v -> Bool
  isBin (Bin _ _ _ _) = true
  isBin (Tip)         = false

  measure key :: Tree k v -> k 
  key (Bin k _ _ _)   = k 
\end{code}
%
which respectively embed the @Maybe@ and @Tree@ root value into the
refinement logic, after which we can type the @trim@ function as
\begin{code}
  trim :: (Ord k) => blo:Maybe k 
                  -> bhi:Maybe k 
                  -> BST k a 
                  -> {v:BST k a | bound(blo, v, bhi)}
\end{code}
where @bound@ is simply a refinement alias
\begin{code}
  refinement bound(lo, v, hi) 
    =  isBin(v) => isJust(lo) => fromJust(lo) < key(v) 
    &&  isBin(v) => isJust(hi) => fromJust(hi) > key(v)
\end{code}
That is, the output refinement states that the root is appropriately 
lower- and upper- bounded if the relevant terms are defined. 
Thus, refinement types allow one to formalize the crucial behavior as
machine-checkable documentation.

\mypara{Code Modifications} On a few occasions we also have to change the
code slightly, typically to make explicit values on which various
invariants depend. Often, this is for a trivial reason; a simple
re-ordering of binders so that refinements for \emph{later} binders 
can depend on earlier ones. Sometimes we need to introduce
``ghost" values so we can write the specifications (\eg the @foldr@ in
\S~\ref{sec:overview:induction}). Another example is illustrated by the use
of list @append@ in @quickSort@. Here, the @append@ only produces a sorted
list if the two input lists are sorted and such that each element in
the first is less than each element in the second. 
We address this with a special @append@ parameterized on @pivot@ 
%append :: k:a -> IncrList {v:a|v<k} -> IncrList {v:a|v>k} -> IncrList a
%append :: k:a -> IncList {v:a|v < k} -> IncList {v:a|v > k} 
%       -> IncList a
\begin{code}
  append :: pivot:a                    
         -> IncrList {v:a | v < pivot}  
         -> IncrList {v:a | v > pivot}  
         -> IncrList a              
  append pivot [] ys     = pivot : ys
  append pivot (x:xs) ys = x : append pivot xs ys
\end{code}
%%after which \toolname infers that
%%$$\centering\begin{tabular}{c}\begin{code}
%%\end{code}\end{tabular}$$
%thereby proving @quickSort@ returns a sorted list.


%\mypara{Challenges for Future Work}
%\subsection{Challenges}
%\begin{verbatim}
%- Error messages
%- Incremental checking
%\end{verbatim}

\section{Conclusion}\label{sec:abstractrefinements:conclusion}
We presented \emph{abstract refinement types} which enable 
quantification over the refinements of data- and 
function-types. Our key insight is that we 
can avail of quantification while preserving SMT-based 
decidability, simply by encoding refinement parameters
as \emph{uninterpreted} propositions within the 
refinement logic.
%
We showed how this mechanism yields a variety 
of sophisticated means for reasoning about programs, including:
\emph{parametric} refinements for reasoning with 
type classes,
\emph{index-dependent} refinements for reasoning about 
key-value maps,
\emph{recursive} refinements for reasoning about 
recursive data types, and
\emph{inductive} refinements for reasoning about 
higher-order traversal routines.
%
We implemented our approach in \toolname and present 
experiments using our tool to verify correctness invariants 
of various programs.

As discussed in~\ref{sec:experiments},
verification many times required code modifications
and definition of ``ghost'' variables (\eg to verify @append@), 
that is, extra arguments not used at run time
but required for specifications.
%
In next chapter we raise this limitation by 
introducing Bounded Refinement Types, 
that impose constrains in the abstract refinements
to further increase the expressiveness of decidable specifications.
%
Abstract and Bounded Refinement Types lead to a relatively complete~\cite{TerauchiPOPL13}
specification system, 
that is the system can express any specification (relatively to the underlying logic)
without the requirement of code modifications and ``ghost'' variables. 


\mypara{Acknowledgments}
The material of this chapter are adapted from the following publication:
\noindent N. Vazou, P. Rondon, and R. Jhala,
``Abstract Refinement Types'', 
ESOP, 2013.
