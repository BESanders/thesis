\section{Overview}
\label{sec:refinementreflection:overview}
\label{sec:examples}

We begin with an overview of refinement reflection and
how it allows us to write proofs \emph{of} and \emph{by}
Haskell functions.

\subsection{Refinement Types}

First, we recall some preliminaries about refinement types
and how they enable shallow specification and verification.

\mypara{Refinement types} are the source program's (here
Haskell's) types decorated with logical predicates drawn
from a(n SMT decidable) logic~\citep{ConstableS87,Rushby98}.
%
For example, we can define the @Nat@ type by refining
Haskell's @Int@ type with a predicate @0 <= v@:
%
\begin{code}
  type Nat = { v:Int | 0 <= v }
\end{code}
%
Here, @v@ names the value described by the type:
the above can be read as the
``set of @Int@ values @v@ that are not less than 0".
The refinement is drawn from the logic of quantifier
free linear arithmetic and uninterpreted functions
(QF-UFLIA~\cite{SMTLIB2}).

\mypara{Specification \& Verification}
%
We can use refinements to define and type the
textbook Fibonacci function as:
%
\begin{code}
  fib :: Nat -> Nat
  fib 0 = 0
  fib 1 = 1
  fib n = fib (n-1) + fib (n-2)
\end{code}
%
Here, the input type's refinement specifies a
\emph{pre-condition} that the parameters must
be @Nat@, which is needed to ensure termination,
and the output types's refinement specifies a
\emph{post-condition} that the result is also a @Nat@.
%
Refinement type checking lets us specify
and (automatically) verify the shallow property
that if @fib@ is invoked with a non-negative
@Int@, then it terminates and yields
a non-negative @Int@.

\mypara{Propositions}
%
We can use refinements to define a data type
representing propositions simply as an alias
for unit, a data type that carries no useful
runtime information:
%
\begin{mcode}
  type $\typp$ = ()
\end{mcode}
%
which can be \emph{refined} with
propositions about the code.
%
For example, the following states the proposition
$2 + 2$ equals $4$.
%
\begin{mcode}
  type Plus_2_2_eq_4 = { v: $\typp$ | 2 + 2 = 4 }
\end{mcode}
%
For clarity, we abbreviate the above type by omitting
the irrelevant basic type $\typp$ and variable @v@:
%
\begin{mcode}
  type Plus_2_2_eq_4 = { 2 + 2 = 4 }
\end{mcode}
%
Function types encode universally quantified propositions:
%
\begin{mcode}
  type Plus_com = x:Int -> y:Int -> { x + y = y + x }
\end{mcode}
%
The parameters @x@ and @y@ refer to input
values. Any inhabitant of the above type is a
proof that @Int@ addition is commutative.

\mypara{Proofs}
%
We \emph{prove} the above theorems by providing inhabitants to type specifications
in forms of Haskell programs. To ease this task \toolname
provides primitives to construct proof terms by
``casting'' expressions to \typp.
%
\begin{mcode}
  data QED = QED

  (**) :: a -> QED -> $\typp$
  _ ** _  = ()
\end{mcode}
%
To resemble mathematical proofs, we make this casting post-fix.
Thus, we write @e ** QED@ to cast @e@ to a value of \typp.
%
For example, we can prove the above propositions by writing
%
\begin{code}
  pf_plus_2_2 :: Plus_2_2_eq_4
  pf_plus_2_2 = trivial ** QED

  pf_plus_comm :: Plus_comm
  pf_plus_comm = \x y -> trivial ** QED

  trivial = ()
\end{code}
%
Via standard refinement type checking, the above code yields
the respective verification conditions (VCs),
%
\begin{align*}
                      2 + 2 & = 4 \\
  \forall \ x,\ y\ .\ x + y & = y + x
\end{align*}
%
which are easily proved valid by the SMT solver, allowing us
to prove the respective propositions.

\mypara{A Note on Bottom:} Readers familiar with Haskell's
semantics may be feeling anxious about whether the
dreaded ``bottom", which inhabits all types, makes our
proofs suspect.
%
Fortunately, as described in \cite{Vazou14}, \toolname
ensures that all terms with non-trivial refinements
provably evaluate to (non-bottom) values, thereby making
our proofs sound.

\subsection{Refinement Reflection}

Suppose we wish to prove properties about the @fib@
function, \eg @fib 2@ equals @1@.
%
\begin{code}
  type fib2_eq_1 = { fib 2 = 1 }
\end{code}
%
%% \NV{By Standard refinement type checking, you mean liquid types, not FStar}
Standard refinement type checking runs into two problems.
%
First, for decidability and soundness, arbitrary user-defined
functions do not belong the refinement logic, \ie we cannot
\emph{refer} to @fib@ in a refinement.
%
Second, the only information that a refinement type checker
has about the behavior of @fib@ is its shallow type
specification @Nat -> Nat@ which is far too weak to verify
@fib2_eq_1@.
%
To address both problems, we use the following annotation,
which sets in motion the three steps of refinement reflection:
%
\begin{code}
  reflect fib
\end{code}

\mypara{Step 1: Definition}
%
The annotation tells \toolname to declare an
\emph{uninterpreted function} @fib :: Int -> Int@
in the refinement logic.
%
By uninterpreted, we mean that the logical @fib@
is \emph{not} connected to the program function
@fib@; in the logic, @fib@
only satisfies the \emph{congruence axiom}
%
$$\forall n, m.\ n = m\ \Rightarrow\ \fib{n} = \fib{m}$$
%
On its own, the uninterpreted function is not
terribly useful, as it does not let us prove
% It lets us prove theorems like
% $$\forall m,\ n.\ m = n \Rightarrow \fib{m} = \fib{n}$$
%
%% \begin{code}
  %% fib_cong :: n:Nat -> m:Nat -> {m=n => fib m = fib n}
  %% fib_cong = trivial ** QED
%% \end{code}
%% %
%but not
@fib2_eq_1@ which requires reasoning about the
\emph{definition} of @fib@.

\mypara{Step 2: Reflection}
%
In the next key step, \toolname reflects the
definition into the refinement type of @fib@
by automatically strengthening the user defined
type for @fib@ to:
%
\begin{code}
  fib :: n:Nat -> { v:Nat | fibP v n }
\end{code}
%
where @fibP@ is an alias for a refinement
\emph{automatically derived} from the
function's definition:
%
\begin{mcode}
  fibP v n = v = if n = 0 then 0 else
                 if n = 1 then 1 else
                 fib(n-1) + fib(n-2)
\end{mcode}

\mypara{Step 3: Application}
%
With the reflected refinement type,
each application of @fib@ in the code
automatically unfolds the @fib@ definition
\textit{once} in the logic.
%
We prove @fib2_eq_1@ by:
%
\begin{code}
  pf_fib2 :: { fib 2 = 1 }
  pf_fib2 = let t0 = #fib# 0 
                t1 = #fib# 1
                t2 = #fib# 2 
            in  ()
\end{code}
%
We write @#f#@ to denote places where the
unfolding of @f@'s definition is important.
%
Via refinement typing, the above proof yields the
following verification condition that is
discharged by the SMT solver, even though @fib@
is uninterpreted:
%
\begin{align*}
   (\fibdef\ (\fib\ 0)\ 0) \ \wedge\ (\fibdef\ (\fib\ 1)\ 1) \ \wedge\ 
   (\fibdef\ (\fib\ 2)\ 2) \  \Rightarrow\ (\fib{2} = 1)
\end{align*}
%
Note that the verification of @pf_fib2@ relies
merely on the fact that @fib@ was applied
to (\ie unfolded at) @0@, @1@ and @2@.
%
The SMT solver automatically \emph{combines}
the facts, once they are in the antecedent.
The following is also verified:
%
\begin{code}
  pf_fib2' :: { fib 2 = 1 }
  pf_fib2' = [ #fib# 0, #fib# 1, #fib# 2 ] ** QED
\end{code}
%
%
Thus, unlike classical dependent typing, refinement
reflection \emph{does not} perform any type-level
computation.

\mypara{Reflection vs. Axiomatization}
%
An alternative \emph{axiomatic} approach,
used by Dafny~\citep{dafny} and
\fstar~\citep{fstar},
is to encode @fib@ using a universally
quantified SMT formula (or axiom):
$$\forall n.\ \fibdef\ (\fib\ n)\ n$$
%
Axiomatization offers greater automation than
reflection. Unlike \toolname, Dafny
%and \fstar
will verify the following by
\emph{automatically instantiating} the above
axiom at @2@, @1@ and @0@:
%
\begin{code}
  axPf_fib2 :: { fib 2 = 1 }
  axPf_fib2 = trivial ** QED
\end{code}

The automation offered by axioms is a bit of a
devil's bargain, as axioms render checking of
the VCs \emph{undecidable}.
%
In practice, automatic axiom instantation can
easily lead to infinite ``matching loops''.
%
For example, the existence of a term \fib{n} in a VC
can trigger the above axiom, which may then produce
the terms \fib{(n-1)} and \fib{(n-2)}, which may then
recursively give rise to further instantiations
\emph{ad infinitum}.
%
To prevent matching loops an expert must carefully
craft ``triggers'' and provide a ``fuel''
parameter~\citep{Amin2014ComputingWA} that can be
used to restrict the numbers of the SMT unfoldings,
which ensure termination, but can cause the axiom
to not be instantiated at the right places.
%
In short, per the authors of Dafny, the
undecidability of the VC checking and its
attendant heuristics makes verification
unpredictable~\citep{Leino16}.

\subsection{Structuring Proofs}

In contrast to the axiomatic approach,
with refinement reflection, the VCs are
deliberately designed to always fall in
an SMT-decidable logic, as function symbols
are uninterpreted.
%
It is up to the programmer to unfold the
definitions at the appropriate places,
which we have found, with careful design
of proof combinators, to be quite
a natural and pleasant experience.
%
To this end, we have developed a library
of proof combinators that permits reasoning
about equalities and linear arithmetic,
inspired by Agda~\citep{agdaequational}.

\mypara{``Equation'' Combinators}
%
We equip \toolname with a family of
equation combinators @op.@ for each
logical operator @op@ in
$\{=, \not =, \leq, <, \geq, > \}$,
the operators in the theory QF-UFLIA.
%
The refinement type of @op.@  \emph{requires}
that $x \odot y$ holds and then \emph{ensures}
that the returned value is equal to @x@.
%
For example, we define @=.@ as:
%
\begin{code}
  (=.) :: x:a -> y:{a| x=y} -> {v:a| v=x}
  x =. _ = x
\end{code}
%
and use it to write the following ``equational" proof:
%
\begin{code}
  eqPf_fib2 :: { fib 2 = 1 }
  eqPf_fib2 =  #fib# 2
            =. #fib# 1 + #fib# 0
            =. 1
            ** QED
\end{code} %$

\mypara{``Because'' Combinators}
%
Often, we need to compose ``lemmata'' into larger
theorems. For example, to prove @fib 3 = 2@ we
may wish to reuse @eqPf_fib2@ as a lemma.
%
To this end, \toolname has a ``because'' combinator:
%
\begin{mcode}
  ($\because$) :: ($\typp$ -> a) -> $\typp$ -> a
  f $\because$ y = f y
\end{mcode}
%
The operator is simply an alias for function
application that lets us write
%
@ x op. y $\because$ p@ (instead of @(op.) x y p@)
where @(op.)@ is extended to accept an \textit{optional} third proof
argument via Haskell's typeclass mechanisms.
%
We use the because combinator to
prove that @fib 3 = 2@ with a Haskell function:
%
\begin{mcode}
  eqPf_fib3 :: { fib 3 = 2 }
  eqPf_fib3 =  #fib# 3
            =. fib 2 + #fib# 1
            =. 2              $\because$ eqPf_fib2
            ** QED
\end{mcode}

\mypara{Arithmetic and Ordering}
%
SMT based refinements let us go well beyond just equational
reasoning. Next, lets see how we can use arithmetic and
ordering to prove that @fib@ is (locally) increasing,
%
\ie for all $n$, $\fib{n} \leq \fib{(n+1)}$
%
\begin{mcode}
  fibUp :: n:Nat -> { fib n <= fib (n+1) }
  fibUp n
    | n == 0
    =  #fib# 0 <. #fib# 1
    ** QED

    | n == 1
    =  fib 1 <=. fib 1 + fib 0 <=. #fib# 2
    ** QED

    | otherwise
    =  #fib# n
    =. fib (n-1) + fib (n-2)
    <=. fib n     + fib (n-2) $\because$ fibUp (n-1)
    <=. fib n     + fib (n-1) $\because$ fibUp (n-2)
    <=. #fib# (n+1)
    ** QED
\end{mcode} %$

\mypara{Case Splitting and Induction}
%
The proof @fibUp@ works by induction on @n@.
%
In the \emph{base} cases @0@ and @1@, we simply assert
the relevant inequalities. These are verified as the
reflected refinement unfolds the definition of
@fib@ at those inputs.
%
The derived VCs are (automatically) proved
as the SMT solver concludes $0 < 1$ and $1 + 0 \leq 1$
respectively.
%
In the \emph{inductive} case, @fib n@ is unfolded
to  @fib (n-1) + fib (n-2)@, which, because of the
induction hypothesis (applied by invoking @fibUp@
at @n-1@ and @n-2@) and the SMT solver's arithmetic
reasoning, completes the proof.

\mypara{Higher Order Theorems}
%
Refinements smoothly accomodate higher-order reasoning.
%
For example, lets prove that every locally increasing
function is monotonic, \ie
if @f z <= f (z+1)@ for all @z@,
then @f x <= f y@ for all @x < y@.
%
\begin{mcode}
  fMono :: f:(Nat -> Int)
        -> fUp:(z:Nat -> {f z <= f (z+1)})
        -> x:Nat
        -> y:{x < y}
        -> {f x <= f y} / [y]
  fMono f inc x y
    | x + 1 == y
    =  f x <=. f (x+1) $\because$ fUp x
           <=. f y
           ** QED

    | x + 1 < y
    =  f x <=. f (y-1) $\because$ fMono f fUp x (y-1)
           <=. f y     $\because$ fUp (y-1)
           ** QED
\end{mcode}
%
We prove the theorem by induction
on @y@, which is specified by the
annotation @/ [y]@ which states
that @y@ is a well-founded
termination metric that decreases
at each recursive call~\citep{Vazou14}.
%
% All reflected functions are proved terminating.
% When the annotation metric is not explicit Liquid Haskell
% successfully uses heuristics to automatically prove termination. 
%
If @x+1 == y@, then we use @fUp x@.
%
Otherwise, @x+1 < y@, and we use
the induction hypothesis \ie apply
@fMono@ at @y-1@, after which
transitivity of the less-than
ordering finishes the proof.
%
We can use the general @fMono@
theorem to prove that @fib@
increases monotonically:
%
\begin{code}
  fibMono :: n:Nat -> m:{n<m} ->
             {fib n <= fib m}
  fibMono = fMono fib fibUp
\end{code}


\subsection{Case Study: Deterministic Parallelism}
\label{sec:detpar}

%% The natural integration of deep verification with a language like Haskell makes
%% it possible to engage in lightweight, incremental verification of program
%% properties.

One benefit of an in-language prover is that it lowers the barrier to {\em
  small} verification efforts that touch only a fraction of the program, and yet
ensure critical invariants that Haskell's type system cannot.  Here we consider
parallel programming, which is commonly considered error prone and entails
proof obligations on the user that typically go unchecked.

The situation is especially precarious with parallel programming frameworks that
claim to be {\em determinstic} and thus usable within purely functional
programs.  These include Deterministic Parallel Java (DPJ \cite{DPJ}), Concurrent
Revisions for .NET~\cite{concurrent-revisions-oopsla}, and Haskell's
LVish~\cite{kuper2014freeze}, Accelerate~\cite{accelerate-icfp13}, and
REPA~\cite{repa-icfp10}.
%
Accelerate's parallel fold function, for instance, claims to be
deterministic---and its purely functional type means the Haskell optimizer will
{\em assume} its referential transparency---but its determinism depends on an
associativity guarantee which must be assured {\em by the programmer} rather than the
type system.
%
Thus simply folding the minus function, @fold (-) 0 arr@, is sufficient to
violate determinism and Haskell's pure semantics.


Likewise, DPJ goes to pains to develop a new type system for parallel
programming, but then provides a ``commutes'' annotation for methods updating
shared state, compromising the {\em guarantee} and going back to trusting the
user. LVish has the same Achilles heel. Consider set insertion:

\begin{code}
  insert :: Ord a => a -> Set s a -> Par s ()
\end{code}

Here @insert@ returns an (effectful) @Par@ computation, which can be run within a
pure function to produce a pure result.  At first glance it would seem that
trusting the implementation of the concurrent set is sufficient to assure a
deterministic outcome.  Yet the interface has an @Ord@ constraint. This
 polymorphic function works with user-defined data types, and thus
user-defined orderings.  What if the user fails to implement a total order?
Then, even a correct implementation of, e.g. a concurrent
skiplist~\cite{concurrent-skiplist}, can reveal
different insertion orders due to concurrency.

%% verifiedInsert :: HasPut e => VerifiedOrd a
%%                => a -> ISet s a -> Par e s ()

% \mypara{LVish}
%% We demonstrate the use of \toolname{} to ensure guarantees of
%% deterministic parallel programming. We choose this case study, because, to the
%% best of our knowledge, there exists no practical deterministic parallel
%% programming system, including user-defined parallel folds, which does not have
%% {\em soundness holes}---due to trusted assumptions of user code.

%% {\em LVish}\cite{kuper2014freeze} is a programming library for Haskell, which
%% exposes effectful parallel programming against lattice-variables (LVars) whose
%% states change monotonically during parallel regions of program execution. LVish
%% programs operate on Haskell data types, and LVish requires the operations on
%% these datatypes to satisfy some first order laws, which cannot be expressed in
%% Haskell. However, we can leverage \toolname to verify these properties for
%% arbitrary user-defined datatypes.

%% LVish provides two implementations of concurrent sets, @PureSet@ and @SLSet@,
%% where the underlying data structure is a size-balanced binary tree and
%% concurrent skiplist respectively. The @insert@ operation on a set requires a
%% total ordering on the elements, we can express that in the type signature by
%% \new{The implementation doesn't change, in fact, the}
%% @VerifiedOrd@ \new{methods do not even need to exist at runtime. A sufficiently
%%   smart compiler could optimize away these proof obligations during code
%%   generation.}
% \RN{Let's save the issue of runtime impact for the eval.}

In summary, parallel programs naturally need to communicate, but the mechanisms
of that communication---such as folds or inserts into a shared
structure---typically carry additional proof obligations.  This in turn makes
parallelism a liability.  But we can remove the risk with verification.

% But what if we could use verification to remove the risk?

% through contributions to shared structures (otherwise they are really separate
% programs)


\mypara{Verified typeclasses}
%
Our solution involves simply changing the @Ord@ constraint above to
@VerifiedOrd@.
\begin{mcode}
  insert :: VerifiedOrd a => a -> Set s a -> Par s ()
\end{mcode}
%
This constraint changes the interface but not the implementation of @insert@.
%
% \NV{Why does insert now requires Verified Ord? Is it using the extra methods
% in the implementation?}
%% \note{VerifiedSemigroup story + lifting + isomorphism ("bootstrapping
%%   instances") + detpar propaganda}
%
%% It is an informal requirement when using
%% typeclasses in GHC that some typeclass laws be satisfied. For example, the @Ord@
%% typeclass in GHC requires that the $\leq$ operation be a total order. Using
%% \toolname, we can extend it to include the required properties of a total order,
%% which we call a @VerifiedOrd@.
The additional methods of the verified type class don't add operational
capabilities, but rather impose additional proof obligations:

\begin{code}
  class Ord a => VerifiedOrd a where
   antisym :: x:a -> y:a -> { x <= y && y <= x => x = y }
   trans   :: x:a -> y:a -> z:a -> { x <= y && y <= z => x <= z }
   total   :: x:a -> y:a -> { x <= y || y <= x }
\end{code}

% ---------------------------------------------------------------
% \mypara{Verified Monoids}

Similarly, we can extend
the @Monoid@ typeclass to a @VerifiedMonoid@, with refinements
expressing @Monoid@ laws.
%
\begin{code}
  class Monoid a => VerifiedMonoid a where
   lident :: x:a -> { mempty <> x = x }
   rident :: x:a -> { x <> mempty = x }
   assoc  :: x:a -> y:a -> z:a -> { x <> (y <> z) = (x <> y) <> z }
\end{code}
The @VerifiedMonoid@ typeclass constraint requires the binary operation
to be associative, thus can be safely used to fold on
an unknown number of processors.
%% A parallel fold requires the underlying binary operation to be associative and
%% have a well-behaved identity element, or a @Monoid@.


%% We can then extend the @ParFoldable@ typeclass to a @VerifiedParFoldable@ which
%% enforces a @VerifiedMonoid@ constraint.

%% \NV{Not sure if the below code adds any information: too difficult to follow,
%%   especially for non Haskell people} \NV{I suggest to say similarly to Verified
%%   Ord and add a link to appendix}
%%\RN{I concur with Niki -- we often whitewash away details of the library for
%%  presentation purposes.  E.g. we are not going to explain effect signatures in
%%  this paper.}

%% \begin{code}
%% class ParFoldable c
%%    => VerifiedParFoldable c where
%%   verifiedPmapFold :: forall m e s a .
%%   ( ParFuture m, HasGet e
%%   , HasPut e, FutContents m a,
%%   , VerifiedMonoid a )
%%   => (ElemOf c -> m e s a) -- compute one
%%                            -- result
%%   -> c                     -- element generator
%%                            -- to consume
%%   -> m e s a
%% \end{code}


%%  -------------------------------------------------------------------------

\mypara{Verified instances for primitive types}
@VerifiedOrd@ instances for primitive types like @Int@, @Double@ are trivial to
write; they just appeal to the SMT solver's built-in theories.
%
For example, the following is a valid totality proof on @Int@.
\begin{code}
  totInt :: x:Int -> y:Int -> {x <= y || y <= x}
  totInt _ _ = trivial ** QED
\end{code}

\mypara{Verified instances for algebraic datatypes}
%
To prove the class laws for user defined algebraic datatypes,
refinement reflection allows for structurally inductive proof terms.
%
For example, we can inductively define Peano numerals
%
\begin{code}
  data Peano = Z | S Peano
\end{code}
%
We can compare two @Peano@ numbers via
\begin{code}
  reflect leq :: Peano -> Peano -> Bool
  leq Z _         = True
  leq (S n) Z     = False
  leq (S n) (S m) = leq n m
\end{code}
%
In \S~\ref{sec:theory} we will describe
exactly how the reflection mechanism (illustrated
via @fibP@) is extended to account for ADTs like @Peano@.
%
\toolname automatically checks
that @leq@ is total~\citep{Vazou14}, which
lets us safely @reflect@ it into the logic.

Next, we prove that @leq@ is total on @Peano@ numbers
%
\begin{mcode}
  totalPeano :: n:Peano -> m:Peano -> {leq n m || leq m n}
             /  [toInt n + toInt m]
  totalPeano Z m = leq Z m ** QED
  totalPeano n Z = leq Z n ** QED
  totalPeano (S n) (S m)
   =  leq (S n) (S m) || leq (S m) (S n)
   =. leq n m || leq m n
   =. True $\because$ totalPeano m n
   ** QED
\end{mcode}
The proof goes by induction, splitting cases on
whether the number is zero or non-zero. Consequently,
we pattern match on the parameters @n@ and @m@, and furnish
separate proofs for each case.
%
In the ``zero" cases, we simply unfold the definition
of @leq@.
%
In the ``successor" case, after unfolding we (literally)
apply the induction hypothesis by using the because operator.
%
The termination hint @[toInt n + toInt m]@,
where @toInt@ maps @Peano@ numbers to integers,
is used to verify well-formedness of the @totalPeano@
proof term.
%
\toolname's termination and totality checker
use the hint to
verify that we are in fact doing induction
properly~(\S~\ref{sec:types-reflection}).

Similarly to @totalPeano@, we can define the rest of the @VerifiedOrd@
proof methods and use them to create the verified instance.
%
\begin{code}
  instance Ord Peano where
    (<=) = leq

  instance VerifiedOrd Peano where
    total = totalPeano
\end{code}
%
Proving all the four @VerifiedOrd@ laws
is a burden on the programmer.
%becomes a burden as the datatype grows more complicated.
%
Since @Peano@ is isomorphic to @Nat@s,
next we present how
to reduce the @Peano@ proofs into the
SMT automated integer proofs.

\mypara{Isomorphisms}
%
In order to reuse proofs for a custom datatype,
we provide a way to translate verified instances between isomorphic types~\cite{barthe2001type}.
%% If our datatype is isomorphic to a nesting of binary sums and products, we
%% should be able to reusing existing proofs.
%% To verify operations on custom data types efficiently, we
%% need to be able lift verified instances on one type to another.
%
We design a typeclass @Iso@ which witnesses the fact that
two types are isomorphic.
%, with respect to the built-in equality in \toolname{}
%which is a congruence.

\begin{mcode}
  class Iso a b where
    to      :: a -> b
    from    :: b -> a
    to$\circ$from :: x:a -> {to (from x) = x}
    from$\circ$to :: x:a -> {from (to x) = x}
\end{mcode}
%
For two isomorphic types @a@ and @b@
we compare instances of @b@ using @a@'s
comparison method.
%
\begin{mcode}
  instance (Ord a, Iso a b) => Ord b where
    x <= y = from x <= from y
\end{mcode}
%
Then, we prove that @VerifiedOrd@ laws are closed under isomorphisms.
%
For example, we prove totality of comparison on @b@s
using the @VerifiedOrd@ totality on @a@s

\begin{mcode}
  isoTotal :: (VerifiedOrd a, Iso a b)
           => x:b -> y:b -> {x <= y || y <= x}
  isoTotal x y
   =  x <= y || y <= x
   =. (from x) <= (from y) || (from y) <= (from x)
      $\because$ total (from x) (from y)
   ** QED
\end{mcode}
%
We use @isoTotal@ to create a verified instance on @b@s.
\begin{mcode}
  instance (VerifiedOrd a, Iso a b) 
         => VerifiedOrd b where
    total   = isoTotal
\end{mcode}
%
With the above technique,
and using Haskell's instances,
getting a @VerifiedOrd@ instance for @Peano@
reduces to definition of an @Iso Nat Peano@.
%\VC{Iso (Either () Peano) Peano}

\mypara{Proof Composition via Products}
Finally, we present a mechanism to automatically
reduce proofs on product types to proofs of the product components.
%
For example, lexicographic ordering preserves the ordering laws.
%
First, we use class instances to define lexicographic ordering.
%
\begin{mcode}
  instance (VerifiedOrd a, VerifiedOrd b) => Ord (a, b) where
    (x1, y1) <= (x2, y2) = if x1 == x2 then y1 <= y2 else x1 <= x2
\end{mcode}
%
Then, we prove that lexicographic ordering
preserves the ordering laws.
%
For example, it preserves totality.
%
\begin{mcode}
  prodTotal :: (VerifiedOrd a, VerifiedOrd b)
            => p:(a, b) -> q:(a, b) -> {p <= q || q <= p}
  prodTotal p@(x1, y1) q@(x2, y2)
   =  p <= q || q <= p
   =. if x1 == x2 then (y1 <= y2 || y2 <= y1) else True 
      $\because$ total x1 x2
   =. if x1 == x2 then True                   else True 
      $\because$ total y1 y2
   ** QED
\end{mcode}
%
Finally, using the @prodTotal@ proof method,
we conclude that each instance defined via the lexicographic
ordering is indeed a verified instance.
%
\begin{mcode}
  instance (VerifiedOrd a, VerifiedOrd b) 
         => VerifiedOrd (a, b) where
    total   = prodTotal
\end{mcode}
%
For example the type @(Peano, Peano)@ is derived to be a @VerifiedOrd@ instance.

In short, we can decompose an algebraic datatype into an isomorphic type using sums and
products to generate verified instances for arbitrary Haskell
datatypes. This could be combined with the Glasgow Haskell Compiler's (GHC) support
for generics~\cite{ghc-generics} to automate the derivation of verified instances
for user datatypes.
In \S\ref{sec:eval-parallelism}, we use these ideas to develop fully safe
interfaces to LVish modules, as well as verifying programming patterns from DPJ.
