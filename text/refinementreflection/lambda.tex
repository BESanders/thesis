
\section{Reasoning About Lambdas}\label{sec:lambdas}


Though \smtlan, as presented so far, is sound and decidable,
it is \emph{imprecise}: our encoding of $\lambda$-abstractions
and applications via uninterpreted functions makes it impossible
to prove theorems that require $\alpha$- and $\beta$-equivalence,
or extensional equality. Next, we show how to address the former
by strengthening the VCs with equalities~\S~\ref{subsec:equivalences},
and the latter by introducing a combinator for safely asserting
extensional equality~\S~\ref{subsec:extensionality}.
%
In the rest of this section, 
for clarity we omit $\smtappname{}{}$ when it is clear from the context.

\subsection{Equivalence}\label{subsec:equivalences}

As soundness relies on satisfiability under a
\bmodel  (see Definition~\ref{def:beta-model}),
we can safely \emph{instantiate} the axioms of
$\alpha$- and $\beta$-equivalence on any set of
terms of our choosing and still preserve soundness
(Theorem~\ref{thm:soundness-smt}).
%
That is, instead of checking the validity
of a VC
${p \Rightarrow q}$,
we check the validity of a \emph{strengthened VC},
${a \Rightarrow p \Rightarrow q}$,
where $a$ is a (finite) conjunction
of \emph{equivalence instances}
derived from $p$ and $q$,
as discussed below.

% Since it is unclear how to reify this axiomatization
% while preserving decidability, we choose (once again)
% to syntactically instantiate the axioms of $\alpha$-,
% $\beta$- and normal form equivalence on the relevant
% candidates.

\mypara{Representation Invariant}
%
The lambda binders,
for each SMT sort, are drawn from a
pool of names $x_i$ where the index
$i=1,2,\ldots$.
%
When representing
$\lambda$ terms we enforce
a \emph{normalization invariant}
that for each lambda term
$\slam{x_i}{e}$, the index $i$
% $x_i$
is greater than any lambda
argument appearing in $e$.

\mypara{$\alpha$-instances}
For each syntactic term
${\slam{x_i}{e}}$, and $\lambda$-binder
$x_j$ such that $i < j$ appearing in the VC,
we generate an \emph{$\alpha$-equivalence instance predicate}
(or \emph{$\alpha$-instance}):
$$\slam{x_i}{e} = \slam{x_j}{e \subst{x_i}{x_j}}$$

% , x_i < x_j \leq \maxlamarg$$
%\leq \maxlamarg$
% that follow the ordering
% $i < j \Leftrightarrow x_i < x_j$.
%
% The number of distinct valid lambda
% arguments is determined by the
% maximum number of $\lambda$s
% syntactically in program's refinements.
% In the implementation, we bound
% this number by $\maxlamarg = 7$.
% %
% In practice we only encountered
% 4 nested lambdas when verifying
% monadic left identity of the
% reader monad.

The conjunction of $\alpha$-instances
can be more precise than De Bruijn
representation, as they let the SMT
solver deduce more equalities via
congruence.
%
For example, consider the VC needed
to prove the applicative laws for @Reader@:
%
\begin{align*}
d & = \slam{x_1}{(\sapp{x}{x_1})} \\
  & \Rightarrow \slam{x_2}{(\sapp{(\slam{x_1}{(\sapp{x}{x_1})})}{x_2})}
              = \slam{x_1}{(\sapp{d}{x_1})}
\end{align*}
%
The $\alpha$ instance
%
${\slam{x_1}{(\sapp{d}{x_1})} = \slam{x_2}{(\sapp{d}{x_2})}}$
%
derived from the VC's hypothesis,
combined with congruence immediately
yields the VC's consequence.

\mypara{$\beta$-instances}
%
For each syntactic term $\smapp{(\slam{x}{e})}{e_x}$,
with $e_x$ not containing any $\lambda$-abstractions,
appearing in the VC,
we generate an \emph{$\beta$-equivalence instance predicate}
(or \emph{$\beta$-instance}):
$$
\smapp{(\slam{x_i}{e})}{e_x} = \SUBST{e}{x_i}{e_x},
  \mbox{ s.t. } e_x \mbox{ is $\lambda$-free}
$$
%
We require the $\lambda$-free restriction as
a simple way to enforce that the reduced
term ${\SUBST{e}{x_i}{e'}}$ enjoys the
representation invariant.
%
%%\NV{This restriction is not implemented, but neither has a normalization function}
%%\RJ{Then why are we discussing it?}

For example, consider the following VC
needed to prove that the bind operator for
lists satisfies the monadic associativity law.
%
$$(\sapp{f}{x} \ebind g) = \smapp{(\slam{y}{(\sapp{f}{y} \ebind g)})}{x}$$
%
The right-hand side of the above VC generates
a $\beta$-instance that corresponds directly
to the equality, allowing the SMT solver to
prove the (strengthened) VC.

%% immediately yields
%% $\beta$-equivalence is required in our benchmarks to for example prove
%% monadic associativity for lists.
%% %
%% An intermediate step in the proof is to verify that
%%
%% Assuming that @h x = f x >>= g@ the above
%% translates to the logical query
%% $$
%% \sapp{h}{x} = \sapp{(\slam{x_1}{(\sapp{h}{x_1})})}{x}
%% $$
%%
%% The right hand side of the query will fire a
%% $\beta$-equivalence instantiation
%% and the assumption
%% $
%% \sapp{(\slam{x_1}{(\sapp{h}{x_1})})}{x} = \sapp{h}{x}
%% $
%% will be added in the environment, allowing SMT to prove the equivalence.

\mypara{Normalization}
%
The combination of $\alpha$- and $\beta$-instances
is often required to discharge proof obligations.
%
For example, when proving that the bind operator
for the @Reader@ monad is associative, we need
to prove the VC:
%
$$\slam{x_2}{(\slam{x_1}{w})} =
  \slam{x_3}{(\smapp{(\slam{x_2}{(\slam{x_1}{w})})}{w})}$$
%
The SMT solver proves the VC via the equalities
corresponding to an $\alpha$ and then $\beta$-instance:
%
\begin{align*}
\slam{x_2}{(\slam{x_1}{w})}
  \ =_{\alpha}\ & \slam{x_3}{(\slam{x_1}{w})} \\
  \ =_{\beta}\ & \slam{x_3}{(\smapp{(\slam{x_2}{(\slam{x_1}{w})})}{w})}
\end{align*}

%%% of monadic
%%% associativity for the reader monad requires to prove a $\lambda$ equality
%%% simplified to
%%% %
%%% \begin{code}
  %%% w:a -> {(\x y -> w) = (\x -> (\z y -> w) w)}
%%% \end{code}
%%% %
%%% The proof for the above code is @trivial@ as
%%% in the logic, the $\lambda$ arguments are renamed to
%%% @(\x2 x1 -> w) = (\x3 -> (\x2 x1 -> w) w)@.
%%% Due to $\alpha$- equivalence the
%%% right hand side is equal to
%%% @\x3 x1 -> w@.
%%% Due to $\beta$-equivalence we get
%%% @((\x2 x1 -> w) w) = \x1 -> w@,
%%% by which and due to congruence axiom
%%% we get the desired equality,
%%% \begin{code}
  %%% \x y -> w               -- lhs
  %%% \x2 x1 -> w             -- representation
  %%% \x3 x1 -> w             -- alpha
  %%% \x3 ->((\x2 x1 -> w) w) -- beta
  %%% \x -> ((\z y -> w) w)   -- rhs
%%% \end{code}

\subsection{Extensionality} \label{subsec:extensionality}

Often, we need to prove that two
functions are equal, given the
definitions of reflected binders.
%
For example, consider
%
\begin{code}
 reflect id
 id x = x
\end{code}
%
\toolname accepts the proof that
@id x = x@ for all @x@:
%
\begin{code}
 id_x_eq_x :: x:a -> {id x = x}
 id_x_eq_x = \x -> #id# x =. x ** QED
\end{code}
%
as ``calling'' @id@ unfolds its definition,
completing the proof.
%
However, consider this $\eta$-expanded variant of
the above proposition:
%
\begin{code}
 type Id_eq_id = {(\x -> id x) = (\y -> y)}
\end{code}
%
\toolname \emph{rejects} the proof:
%
\begin{code}
 fails :: Id_eq_id
 fails =  (\x -> #id# x) =. (\y -> y) ** QED
\end{code}
%
The invocation of @id@ unfolds the
definition, but the resulting equality
refinement @{id x = x}@ is \emph{trapped}
under the $\lambda$-abstraction.
%
That is, the equality is absent from the
typing environment at the \emph{top} level,
where the left-hand side term is compared to @\y -> y@.
%
% NV LHS reads like LiquidHaskell
Note that the above equality requires
the definition of @id@ and hence is
outside the scope of purely the
$\alpha$- and $\beta$-instances.

\newcommand\eqfun{\ensuremath{\texttt{=}\forall}}

\mypara{An Exensionality Operator}
%
To allow function equality via
extensionality, we provide the
user with a (family of)
%
\emph{function comparison operator(s)}
that transform an \emph{explanation} @p@
which is a proof that @f x = g x@ for every
argument @x@, into a proof that @f = g@.
%
\begin{mcode}
  =* :: f:(a -> b) -> g:(a -> b)
     -> exp:(x:a -> {f x = g x})
     -> {f = g}
\end{mcode}
%
Of course, @=*@ cannot be implemented;
its type is \emph{assumed}. We can use
@=*@ to prove @Id_eq_id@ by providing
a suitable explanation:
%
\begin{mcode}
pf_id_id :: Id_eq_id
pf_id_id = (\y -> y) =* (\x -> id x) $\because$ expl ** QED
  where
    expl = (\x -> #id# x =. x ** QED)
\end{mcode}
%
%   =* (\x -> id x) $\because$ exp
%   where
%    exp   :: x:a -> {(\x -> id x) x = (\x -> x) x}
%    exp x = #id# x =. x ** QED
%
The explanation is
% the proof passed as
the second argument to $\because$ which has
the following type that syntactically fires $\beta$-instances:
\begin{code}
  x:a -> {(\x -> id x) x = ((\x -> x) x}
\end{code}

\begin{comment}
While the above operator makes proving function
equality via extensionality \emph{possible}
it can be somewhat \emph{cumbersome}.
%
For example, in the proof of associativity of
the monadic bind operator for the @Reader@
monad three of eight equalities required
such explanations, some of which were under
nested $\lambda$-abstractions.

In future work, it would be interesting to
explore wheth

This proof style made verification of monadic associativity
in reader monad too big as out of the 8 equalities used in
the proof three required extentional equality
one of which was double wrapped in lambdas.

is requires a cumbersome indirect proof.
%
%% Extentional function equality requires
%% as argument an equality on such redexes.
%% %
%% Via $\beta$ equality instantiations,
%% both such redexes will automatically reduce,
%% requiring @exp@ to prove @id x = x@,
%% with is direct.


For each function equality in the main proof
one needs to define an explanation function
that proves the equality for every argument.
%
This proof style made verification of monadic associativity
in reader monad too big as out of the 8 equalities used in
the proof three required extentional equality
one of which was double wrapped in lambdas.

\mypara{Soundly Enforcing Function Equality}
%
The class constraint @Arg a@ is needed
to address the following subtle issue
that arises due to the interaction of
parametric polymorphism and subtyping.
%
Notice that type parameter @a@
appears only in \emph{negative}
positions.
%
At any \emph{use} site, we instantiate
the signature with \emph{fresh} ref
%
Thus \emph{without} the class constraint,
the signature holds \emph{for all} @a@.
At a usage site, as the @a@ appears only
negatively it has no ``lower bound'' (\ie
is not the \emph{supertype} of any type)
and can be instantiated to @{v:a | false}@.
%
Consequently, via function subtyping,
the crucial fact that @f x = g x@ is
\emph{unsoundly} checked (to trivially hold)
under the assumption that @x:{a | False}@
which would let us unsoundly prove that
\emph{any} two functions are equal!

We solve the above problem via the class constraint
@(Arg a)@ which.
%
The constraint forces @a@ to appear \emph{positively}
in the signature, by stipulating that methods of the
type class @Arg a@ can \emph{create}
new values of type @a@.
%
This forces constrains @a@ to be instantiated as
@{a | True}@~\citep{Vazou13}.
\end{comment}

%%% Liquid type inference is smart enough to infer that
%%% since @a@ appears only negative @(=..)@ cannot use any @a@
%%% and thus will not call any of its argument arguments @f@, @g@, nor the @p@.
%%% %
%%% Thus, at each call site of @(=..)@ the type variable `a`
%%%
%%% indicating dead-code (since `a`s will not be used by the callee.)
%%% %
%%% Refining the argument @x:a@ with false at each call-site though
%%% leads to unsoundness, as each proof argument @p@ is a valid proof under
%%% the false assumption.
%%% %
%%% What Liquid inference cannot predict is our intention to call
%%% @f@, @g@ and @p@ at \textit{every possible argument}.

%% But soundness of its usage requires
%% the argument type variable @a@ to be
%% constrained by a type class constraint
%% @Arg a@. for both operational and type theoretic reasons.
%%
%% From \textit{operational} point of view,
%% an implementation of @(=..)@ would require checking
%% equality of @f x = g x@ \textit{forall} arguments @x@ of type @a@.
%% %
%% This equality would hold due to the proof argument @p@.
%% %
%% The only missing point is a way to enumerate all the argument @a@,
%% but this could be provided by a method of the type clas @Arg a@.
%% %
%% Yet, we have not implement @(==.)@ because we do not know how to
%% provide such an implementation that can provably satisfy @(=..)@'s type.
