\section{Refinement Reflection}
\label{sec:formalism}
\label{sec:types-reflection}

Our first step towards formalizing refinement
reflection is a core calculus \corelan with an
\emph{undecidable} type system based on
denotational semantics.
We show how the soundness of the type system
allows us to \emph{prove theorems} using \corelan.

%
%% Note that \smtlan programs are a subset of
%% \corelan programs derivations
%%
%% a subset of \corelan where the that forms a
%% decidable logic of
%% refinements, and use it to obtain \corelan with
%% decidable SMT-based algorithmic typing.

\subsection{Syntax}
\begin{figure}[t!]
\centering
\captionsetup{justification=centering}
\vspace{-5mm}
\centering
$$
\begin{array}{rrcl}
\emphbf{Operators}
  & \odot
  & ::= & = \spmid  <
\\[0.03in]

\emphbf{Constants}
  & c
  & ::=
  & \land \spmid \lnot \spmid \odot \spmid +,-,\dots  \spmid
      \etrue\spmid \efalse \spmid 0, -1, 1, \dots
\\[0.03in]

\emphbf{Values} 
  & w & ::=&  c
             \spmid \efun{x}{\typ}{e} \spmid D\ \overline{w}
\\[0.03in]

\emphbf{Expressions} 
  & e & ::=    & w \spmid x \spmid \eapp{e}{e}  
  \spmid \ecase{x}{e}{\dc}{\overline{x}}{e}
  %% &   & \spmid & \eletb{x}{\gtyp}{e}{e}
\\[0.03in]

\emphbf{Binders} 
  & \bd & ::= & e \spmid \eletb{x}{\gtyp}{\bd}{\bd}
\\[0.03in]

\emphbf{Program} 
  & \prog & ::= & \bd \spmid \erefb{x}{\gtyp}{e}{\prog}
\\[0.03in]

% \emphbf{Logical Labels} 
%  & L & ::= & \la \spmid \lm
%\\[0.03in]

\emphbf{Basic Types} 
  & \btyp
  & ::=
  & \tint \spmid \tbool \spmid T
\\[0.03in]

\emphbf{Refined Types} 
  & \typ
  & ::=   & \tref{v}{\btyp^{[\tlabel]}}{\reft} \spmid \tfun{x}{\typ}{\typ}
\\[0.05in]
\end{array}
$$
\caption{{Syntax of \corelan.}}
\label{fig:syntax}
\vspace{-2mm}
\end{figure}

%%\begin{figure}[t!]
%%\centering
%%
%%\emphbf{Contexts}\hfill{{$\fbox{\textit{C}}$}}
%%$$
%%\begin{array}{rcl}
%%  C & ::=    & \bullet \spmid C\ e \spmid c\ C \spmid D\ \overline{e}\ C\ \overline{e}\\
%%    & \spmid & \ecase{y}{C}{\dc_i}{\overline{x}}{e_i}
%%  \\[0.03in]
%%\end{array}
%%$$
%%
%%\emphbf{Reductions}\hfill{{$\fbox{\goesto{\prog}{\prog'}}$}}
%%\begin{align*}
%%C[\prog]
%%  & \hookrightarrow
%%   C[\prog'],\quad \text{if}\ \goesto{\prog}{\prog'}
%%  \\
%%{c\ v}
%%  & \hookrightarrow
%%   {\ceval{c}{v}}
%%  \\
%%{({\efun{x}{\typ}{e})}\ {e'}}
%%  & \hookrightarrow
%%   {\SUBST{e}{x}{e'}}
%%  \\
%%{\ecase{y}{\dc_j\ \overline{e}}{\dc_i}{\overline{x_i}}{e_i}}
%%  &\hookrightarrow
%%  {\SUBST{\SUBST{e_j}{y}{\dc_j\ \overline{e}}}{\overline{x_i}}{\overline{e}}}
%%\\
%%{\erefb{x}{\gtyp}{e}{\prog}}
%%  & \hookrightarrow
%%  {\SUBST{\prog}{x}{\efix{(\efun{x}{\gtyp}{e})}}}
%%\\
%%{\eletb{x}{\gtyp}{\bd_x}{\bd}}
%%  & \hookrightarrow
%%{\SUBST{\bd}{x}{\efix{(\efun{x}{\gtyp}{\bd_x})}}}
%%\\
%%{\efix{\prog}}
%%  & \hookrightarrow
%%  {\prog\ (\efix{\prog})}
%%\end{align*}
%%\caption{\textbf{Operational Semantics of \corelan}}
%%\label{fig:semantics}
%%\end{figure}

%
Figure~\ref{fig:syntax} summarizes the syntax of \corelan,
which is essentially the calculus \undeclang~\cite{Vazou14}
with explicit recursion and a special $\erefname$ binding form
to denote terms that are reflected into the refinement logic.
%
In \corelan refinements $r$ are arbitrary expressions $e$
(hence $r ::= e$ in Figure~\ref{fig:syntax}).
%
This choice allows us to prove preservation and progress,
but renders typechecking undecidable.
%
In \S~\ref{sec:algorithmic} we will see how to recover
decidability by soundly approximating refinements.

The syntactic elements of \corelan are layered into
primitive constants, values, expressions, binders
and programs.

\mypara{Constants}
The primitive constants of \corelan
include all the primitive logical
operators $\op$, here, the set $\{ =, <\}$.
%
Moreover, they include the
primitive booleans $\etrue$, $\efalse$,
integers $\mathtt{-1}, \mathtt{0}$, $\mathtt{1}$, \etc,
and logical operators $\mathtt{\land}$, $\mathtt{\lor}$, $\mathtt{\lnot}$, \etc.

\mypara{Data Constructors}
%
We encode data constructors as special constants.
% Each data type has an equality predicate $\haseq{T}$
% that is true only if values of type $T$ can be finitely compared.
For example the data type \tintlist, which represents
finite lists of integers, has two data constructors: $\dnull$ (``nil'')
and $\dcons$ (``cons'').
% and satisfies $\haseq{\tintlist}$.

%% NV Arity is not used anywhere
%%Each data type has an arity $\arity{T}$ that represents
%%the exact number of data constructors that return
%%a value of type $T$.
%
%%For example the data type \tintlist, which represents
%%lists of integers, has two data constructors: $\dnull$ (``nil'')
%%and $\dcons$ (``cons'') and so has arity $2$.


\mypara{Values \& Expressions}
%
The values of \corelan include
constants, $\lambda$-abstractions
$\efun{x}{\typ}{e}$, and fully
applied data constructors $D$
that wrap values.
%
The expressions of \corelan
include values and variables $x$,
applications $\eapp{e}{e}$, and
$\mathtt{case}$ expressions.

\mypara{Binders \& Programs}
%
A \emph{binder} $\bd$ is a series of possibly recursive
let definitions, followed by an expression.
%
A \emph{program} \prog is a series of $\erefname$
definitions, each of which names a function
that can be reflected into the refinement
logic, followed by a binder.
%
The stratification of programs via binders
is required so that arbitrary recursive definitions
are allowed but cannot be inserted into the logic
via refinements or reflection.
%
(We \emph{can} allow non-recursive $\mathtt{let}$
binders in $e$, but omit them for simplicity.)

\subsection{Operational Semantics}

Figure~\ref{fig:syntax} summarizes the small
step contextual $\beta$-reduction semantics for
\corelan.
%
%%There are two points to note.
%%%
%%First, we allow for reductions under
%%data constructors, and thus, values may
%%be further reduced.
%%%
%%Second, for simplicity, we treat both
%%$\eletname$ and $\erefname$ as possibly
%%recursive (\ie $\mathtt{let\ rec}$) binders.
%% Note that, for simplicity, we treat each
%% $\eletname$ as a possibly
%% recursive (\ie $\mathtt{let\ rec}$) binder.
%
We write \evalj{e}{e'}{j} if there exist
$e_1,\ldots,e_j$ such that $e$ is $e_1$,
$e'$ is $e_j$ and $\forall i,j, 1 \leq i < j$,
we have $\evals{e_i}{e_{i+1}}$.
%
We write $\evalsto{e}{e'}$ if there exists
some finite $j$ such that $\evalj{e}{e'}{j}$.
%
We define $\betaeq{}{}$ to be the reflexive,
symmetric, transitive closure of $\evals{}{}$.

\mypara{Constants} Application of a constant requires the
argument be reduced to a value; in a single step the
expression is reduced to the output of the primitive
constant operation.
%
For example, consider $=$, the primitive equality
operator on integers.
%
We have $\ceval{=}{n} \defeq =_n$
where $\ceval{=_n}{m}$ equals \etrue
iff $m$ is the same as $n$.
%
%\mypara{Equality}
%
We assume that the equality operator
is defined \emph{for all} values,
and, for functions, is defined as
extensional equality.
%
That is, for all
$f$ and
$f'$
we have
$\evals{(f = f')}{\etrue}
  \quad \mbox{iff} \quad
  \forall v.\ \betaeq{f\ v}{f'\ v}$.
%
We assume source \emph{terms} only contain implementable equalities
over non-function types; the above only appears in \emph{refinements}
and allows us to state and prove facts about extensional
equality~\S~\ref{subsec:extensionality}.
%% % \RJ{CHECK}

%%That is, for all
%%$f \defeq \efun{x}{\typ}{e}$ and
%%$f' \defeq \efun{x}{\typ}{e'}$
%%we have
%%$$\evals{(f = f')}{\etrue}
%%  \quad \mbox{iff} \quad
%%  \forall v.\ \evalsto{(\SUBST{e}{x}{v} = \SUBST{e'}{x}{v})}{\etrue}
%%$$


\subsection{Types}

\corelan types include basic types, which are \emph{refined} with predicates,
and dependent function types.
%
\emph{Basic types} \btyp comprise integers, booleans, and a family of data-types
$T$ (representing lists, trees \etc.)
%
For example the data type \tintlist represents lists of integers.
%
We refine basic types with predicates (boolean valued expressions \refa) to obtain
\emph{basic refinement types} $\tref{v}{\btyp}{\refa}$.
%
Finally, we have dependent \emph{function types} $\tfun{x}{\typ_x}{\typ}$
where the input $x$ has the type $\typ_x$ and the output $\typ$ may
refer to the input binder $x$.
%
We write $\btyp$ to abbreviate $\tref{v}{\btyp}{\etrue}$,
and \tfunbasic{\typ_x}{\typ} to abbreviate \tfun{x}{\typ_x}{\typ} if
$x$ does not appear in $\typ$.
%
We use $r$ to refer to refinements.


\mypara{Denotations}
%
Each type $\typ$ \emph{denotes} a set of expressions $\interp{\typ}$,
that are defined via the dynamic semantics~\cite{Knowles10}.
%
Let $\shape{\typ}$ be the type we get if we erase all refinements
from $\typ$ and $\bhastype{}{e}{\shape{\typ}}$ be the
standard typing relation for the typed lambda calculus.
%
Then, we define the denotation of types as:
\begin{align*}
\interp{\tref{x}{\btyp}{r}} \defeq &
    \{e \mid  \bhastype{}{e}{\btyp},
              \mbox{ if } \evalsto{e}{w}
              \mbox{ then } \evalsto{r\subst{x}{w}}{\etrue} \}\\
\interp{\tfun{x}{\typ_x}{\typ}} \defeq &
    \{e \mid  \bhastype{}{e}{\shape{\tfunbasic{\typ_x}{\typ}}},
              \forall e_x \in \interp{\typ_x}.\ \eapp{e}{e_x} \in \interp{\typ\subst{x}{e_x}}
    \}
\end{align*}


\mypara{Constants}
For each constant $c$ we define its type \constty{c}
such that $c \in \interp{\constty{c}}$.
%
For example,
%
$$
\begin{array}{lcl}
\constty{3} &\doteq& \tref{v}{\tint}{v = 3}\\
\constty{+} &\doteq& \tfun{\ttx}{\tint}{\tfun{\tty}{\tint}{\tref{v}{\tint}{v = x + y}}}\\
\constty{\leq} &\doteq& \tfun{\ttx}{\tint}{\tfun{\tty}{\tint}{\tref{v}{\tbool}{v \Leftrightarrow x \leq y}}}\\
\end{array}
$$
%
So, by definition we get the constant typing lemma
%
\begin{lemma}{[Constant Typing]}\label{lemma:constants}
Every constant $c \in \interp{\constty{c}}$.
\end{lemma}
%
Thus, if $\constty{c} \defeq \tfun{x}{\typ_x}{\typ}$,
then for every value $w \in \interp{\typ_x}$, we require
$\ceval{c}{w} \in \interp{\typ\subst{x}{w}}$.

%% \mypara{Equality}
%% The equality predicate
%% $\haseq{B}$, is defined to be true for \tint and \tbool,
%% and for each type constructor $T$
%% whose values can be finitely compared.
%% %
%% So, by definition we get the equality lemma
%% %
%% \begin{lemma}{[Equality]}\label{lemma:equality}
%% If $\haseq{B}$ then for each value $\bhastype{\emptyset}{w}{B}$
%% \evalsto{w = w}{\etrue}
%% \end{lemma}

\subsection{Refinement Reflection}
\label{subsec:logicalannotations}
%
The simple, but key idea in our work is to
\emph{strengthen} the output type of functions
with a refinement that \emph{reflects} the
definition of the function in the logic.
%
We do this by treating each
%
$\erefname$-binder:
%
${\erefb{f}{\gtyp}{e}{\prog}}$
%
as a $\eletname$-binder:
%
${\eletb{f}{\exacttype{\gtyp}{e}}{e}{\prog}}$
%
during type checking (rule $\rtreflect$ in Figure~\ref{fig:typing}).

\mypara{Reflection}
%
We write \exacttype{\typ}{e} for the \emph{reflection}
of term $e$ into the type $\typ$,  defined by strengthening
\typ as:
%
$$
\begin{array}{lcl}
\exacttype{\tref{v}{\btyp}{r}}{e}
  & \defeq
  & \tref{v}{\btyp}{r \land v = e}\\
\exacttype{\tfun{x}{\typ_x}{\typ}}{\efun{y}{}{e}}
  & \defeq
  & \tfun{x}{\typ_x}{\exacttype{\typ}{e\subst{y}{x}}}
\end{array}
$$
%
As an example, recall from \S~\ref{sec:overview}
that the @reflect fib @ strengthens the type of
@fib@ with the reflected refinement @fibP@.
%% NV In Overview, we have fibP v n = v = fib n && fibR v n
%% NV Here we get the reflection part (fibR v n)
%% NV which we can verify
%% NV at each fix invocation we also get the v = fib n portion
%% NV via the exact rule
%% NV We can not add the v = fib n as a port condition, because
%% NV we cannot prove it.


\mypara{Consequences for Verification}
%
Reflection has two consequences for verification.
%
First, the reflected refinement is \emph{not trusted};
it is itself verified (as a valid output type)
during type checking.
%
Second, instead of being tethered to quantifier
instantiation heuristics or having to program
``triggers'' as in Dafny~\citep{dafny} or
\fstar~\citep{fstar}
%
the programmer can predictably ``unfold'' the
definition of the function during a proof simply
by ``calling'' the function, which we have found
to be a very natural way of structuring
proofs~\S~\ref{sec:evaluation}.


\subsection{Refining \& Reflecting Data Constructors with Measures}
\label{subsec:measures}
\label{subsec:list}

% We reuse the notion of \emph{measures}~\cite{Vazou14}
% to reflect functions over datatypes into the refinement
% logic.

We assume that each data type is equipped with
a set of \emph{measures} which are \emph{unary}
functions whose (1)~domain is the data type, and
(2)~body is a single case-expression over the
datatype~\cite{Vazou14}:
%
$$\emeasb{f}
         {\gtyp}
         {\efun{x}{\typ}{\ecase{y}{x}{\dc_i}{\overline{z}}{e_i}}}$$
%
For example, @len@ measures the size of an $\tintlist$:
%
\begin{code}
  measure len :: [Int] -> Nat
  len = \x -> case x of
                []     -> 0
                (x:xs) -> 1 + len xs
\end{code}

\mypara{Checking and Projection}
%
We assume the existence of measures that
\emph{check} the top-level constructor,
and \emph{project} their individual fields.
%
\NV{Remove this pointer since we removed the text}
In \S~\ref{subsec:embedding} we show how to
use these measures to reflect functions over
datatypes.
%
% Such measures are straightforward to generate
% automatically from the data-type definition.)
%
For example, for lists, we assume the existence of measures:
%
\begin{code}
  isNil []      = True
  isNil (x:xs)  = False

  isCons (x:xs) = True
  isCons []     = False

  sel1 (x:xs)   = x
  sel2 (x:xs)   = xs
\end{code}

\mypara{Refining Data Constructors with Measures}
%
We use measures to strengthen the types
of data constructors, and we use these
strengthened types during construction
and destruction (pattern-matching).
%
Let:
%
(1)~$\dc$ be a data constructor,
   with \emph{unrefined} type
   $\tfun{\overline{x}}{\overline{\gtyp}}{T}$
%
(2)~the $i$-th measure definition with
   domain $T$ is:
%
$$\emeasb{f_i}
         {\gtyp}
         {\efun{x}{\typ}{\ecase{y}{x}{\dc}{\overline{z}}{e_{i}}}}
$$
%
Then, the refined type of $\dc$ is defined:
$$
\constty{\dc} \defeq
   \tfun{\overline{x}}
        {\overline{\typ}}
        {\tref{v}{T}{ \wedge_i f_i\ v = \SUBST{e_{i}}{\overline{z}}{\overline{x}}}}
$$

Thus, each data constructor's output type is refined to reflect
the definition of each of its measures.
%
For example, we use the measures @len@, @isNil@, @isCons@, @sel1@,
and @sel2@ to strengthen the types of $\dnull$ and $\dcons$ to:
%
\begin{align*}
\constty{\dnull}  \defeq & \tref{v}{\tintlist}{r_{\dnull}} \\
\constty{\dcons}  \defeq & \tfun{x}{\tint}
                                   {\tfun{\mathit{xs}}
                                         {\tintlist}
                                         {\tref{v}{\tintlist}{r_\dcons}}}
\intertext{where the output refinements are}
r_{\dnull} \defeq &\ \mathtt{len}\ v = 0
             \land  \mathtt{isNil}\ v
             \land  \lnot \mathtt{isCons}\ v \\
r_{\dcons} \defeq &\ \mathtt{len}\ v = 1 + \mathtt{len}\ \mathit{xs}
             \land  \lnot \mathtt{isNil}\ v
             \land  \mathtt{isCons}\ v \\
             \land & \  \mathtt{sel1}\ v = x
             \land  \mathtt{sel2}\ v = \mathit{xs}
\end{align*}
%
It is easy to prove that Lemma~\ref{lemma:constants}
holds for data constructors, by construction.
%
For example, $\mathtt{len}\ \dnull = 0$ evaluates to $\tttrue$.


%%
%% The above annotation \emph{strengthens} the types of data constructors
%% $\dc_i$ to reflect the
%% behavior of $f$:
%%
%%
%% \preproc{\eletrecoptsmall{f}{\efun{x}{\typ}{\ecase{y}{x}{\dc_i}{\overline{z}}{e_i}}}{\gtyp}{M}{\prog}}
%%
%% %
%% Where \exacttypefun{f}{\gtyp}{e} strengthens the result $v$ of the type \typ to exactly
%% describe that $f\ v = e$:
%% %
%% \begin{align*}
%% \exacttypefun{f}{\tref{v}{\btyp}{r}}{e} &= \tref{v}{\btyp}{r \land f\ v = e}\\
%% \exacttypefun{f}{\tfun{x}{\typ_x}{\typ}}{\efun{y}{}{e}} &=\tfun{x}{\typ_x}{\exacttype{\typ}{e\subst{y}{x}}}
%% \end{align*}

\subsection{Typing Rules}
\begin{figure}[!t]
\centering
\captionsetup{justification=centering}
\emphbf{Typing}\hfill{\fbox{\hastype{\env}{\prog}{\typ}}}\\
$$
\inference{
	\tbind{x}{\gtyp}\in\env
}{
	\hastype{\env}{x}{\gtyp}
}[\rtvar]
\qquad
\inference{
}{
	\hastype{\env}{c}{\constty{c}}
}[\rtconst]
$$
$$
\inference{
	\hastype{\env}{\prog}{\typ'} &
	\issubtype{\env}{\typ'}{\typ}
}{
	\hastype{\env}{\prog}{\typ}
}[\rtsub]
\qquad
\inference{
    % \haseq{\btyp} &
	\hastype{\env}{e}{\tref{v}{\btyp}{\reft_r}}
}{
	\hastype{\env}{e}{\tref{v}{\btyp}{\reft_r\land v = e}}
}[\rtexact]
$$
$$
\inference{
	\hastype{\env, \tbind{x}{\typ_x}}{e}{\typ}
}{
	\hastype{\env}{\efun{x}{\typ}{e}}{\tfun{x}{\typ_x}{\typ}}
}[\rtfun]
\qquad
\inference{
	\hastype{\env}{e_1}{(\tfun{x}{\typ_x}{\typ})} &&
	\hastype{\env}{e_2}{\typ_x}
}{
	\hastype{\env}{e_1\ e_2}{\typ}
}[\rtapp]
$$
$$
\inference
	{\hastype{\env, \tbind{x}{\gtyp_x}}{\bd_x}{\gtyp_x} &
	 \iswellformed{\env, \tbind{x}{\gtyp_x}}{\typ_x} &
	 \hastype{\env, \tbind{x}{\gtyp_x}}{\bd}{\gtyp} &
	 \iswellformed{\env}{\typ}}
	{\hastype{\env}{\eletb{x}{\gtyp_x}{\bd_x}{\bd}}{\typ}}
	[\rtlet]
$$
$$
\inference
	{\hastype{\env}
	 				 {\eletb{f}{\exacttype{\gtyp_f}{e}}{e}{\prog}}
					 {\typ}
	}
	{\hastype{\env}
					 {\erefb{f}{\gtyp_f}{e}{\prog}}
					 {\typ}
	}[\rtreflect]
$$
$$
\inference{
	\hastype{\env}{e}{\tref{v}{T}{e_r}} & \iswellformed{\env}{\typ} \\
	& \forall i. \constty{\dc_i} = \tfunbasic{\overline{\tbind{y_j}{\typ_j}}}{\tref{v}{T}{e_{r_i}}} &
	 \hastype{\env, \overline{\tbind{y_j}{\typ_j}}, \tbind{x}{\tref{v}{T}{e_r \land e_{r_i}}} }{e_i}{\typ}
}{
	\hastype{\env}{\ecase{x}{e}{\dc_i}{\overline{y_i}}{e_i}}{\typ}
}[\rtcase]
$$
\emphbf{Well Formedness}\hfill{\fbox{\iswellformed{\env}{\typ}}}\\

$$
\inference{
  \hastype{\env,\tbind{v}{\btyp}}{\refa}{\tbool^{\tlabel}}
}{
	\iswellformed{\env}{\tref{v}{\btyp}{\refa}}
}[\rwbase]
\qquad
\inference{
	\iswellformed{\env}{\typ_x} &
	\iswellformed{\env,\tbind{x}{\typ_x}}{\typ}
}{
	\iswellformed{\env}{\tfun{x}{\typ_x}{\typ}}
}[\rwfun]
$$

\emphbf{Subtyping}\hfill{\fbox{\issubtype{\env}{\typ_1}{\typ_2}}}\\

$$
\inference{
%\env' \defeq \env,\tbind{v}{\{\btyp^\tlabel | \refa\}} \\
\env' \defeq \env,\tbind{v}{\tref{v}{\btyp^\tlabel}{\refa}} &
\tologicshort{\env'}{\refa'}{\tbool}{\pred'}{}{}{} &
\smtvalid{\vcond{\env'}{\pred'}}
%%	\forall \sub\in\interp{\env}.
%%	\interp{\applysub{\sub}{\tref{v}{\btyp}{\refa_1}}}
%%	\subseteq
%%	\interp{\applysub{\sub}{\tref{v}{\btyp}{\refa_2}}}
}{
	\issubtype{\env}{\tref{v}{\btyp}{\refa}}{\tref{v}{\btyp}{\refa'}}
}[\rsubbase]
$$
$$
\inference{
	\issubtype{\env}{\typ_x'}{\typ_x} &
	\issubtype{\env,\tbind{x}{\typ_x'}}{\typ}{\typ'}
}{
	\issubtype{\env}{\tfun{x}{\typ_x}{\typ}}{\tfun{x}{\typ_x'}{\typ'}}
}[\rsubfun]
$$
\caption{Type checking of \corelan.}
\label{fig:typing}
\end{figure}
%
Next, we present the type-checking
judgments and rules of \corelan.

\mypara{Environments and Closing Substitutions}
A \emph{type environment} $\env$ is a sequence of type bindings
$\tbind{x_1}{\typ_1},\ldots,\tbind{x_n}{\typ_n}$. An environment
denotes a set of \emph{closing substitutions} $\sto$ which are
sequences of expression bindings:
$\gbind{x_1}{e_1}, \ldots, \gbind{x_n}{e_n}$ such that:
$$
\interp{\env} \defeq  \{\sto \mid \forall \tbind{x}{\typ} \in \Env.
                                    \sto(x) \in \interp{\applysub{\sto}{\typ}} \}
$$

\mypara{Judgments}
We use environments to define three kinds of
rules: Well-formedness, Subtyping,
and Typing~\citep{Knowles10,Vazou14}.
%
%\mypara{Well-formedness}
A judgment \iswellformed{\env}{\typ} states that
the refinement type $\typ$ is well-formed in
the environment $\env$.
%
Intuitively, the type $\typ$ is well-formed if all
the refinements in $\typ$ are $\tbool$-typed in $\env$.
%
%\mypara{Subtyping}
A judgment \issubtype{\env}{\typ_1}{\typ_2} states
that the type $\typ_1$ is a subtype of %the type
$\typ_2$ in the environment $\env$.
%
Informally, $\typ_1$ is a subtype of $\typ_2$ if, when
the free variables of $\typ_1$ and $\typ_2$
are bound to expressions described by $\env$,
the denotation of $\typ_1$
is \emph{contained in} the denotation of $\typ_2$.
%
Subtyping of basic types reduces to denotational containment checking.
%
%\mypara{Implication}
%%A judgment \issubref{\Env}{p_1}{p_2} states
%%that the predicate $p_1$ \emph{implies}
%%the predicate $p_2$ in the environment $\Env$.
%
That is, for any closing substitution $\sto$
in the denotation of $\env$, for every expression $e$,
if $e \in \interp{\applysub{\sto}{\typ_1}}$ then
$ e \in \interp{\applysub{\sto}{\typ_2}}$.
%
%\mypara{Typing}
A judgment \hastype{\env}{\prog}{\typ} states that
the program $\prog$ has the type $\typ$ in
the environment $\env$.
That is, when the free variables in $\prog$ are
bound to expressions described by $\env$, the
program $\prog$ will evaluate to a value
described by $\typ$.

\mypara{Rules}
%
All but three of the rules are standard~\cite{Knowles10,Vazou14}.
%
First, rule \rtreflect is used to strengthen the type of each
reflected binder with its definition, as described previously
in \S~\ref{subsec:logicalannotations}.
%
% \NV{FIX:Eq}
% applies only to expressions that can be finitely compared
% (\ie whose type satisfies the \haseq{B} predicate) and
Second, rule \rtexact strengthens the expression with
a singleton type equating the value and the expression
(\ie reflecting the expression in the type).
%
This is a generalization of the ``selfification'' rules
from \cite{Ou2004,Knowles10}, and is required to
equate the reflected functions with their definitions.
%
For example, the application $(\fib\ 1)$ is typed as
${\tref{v}{\tint}{\fibdef\ v\ 1 \wedge v = \fib\ 1}}$ where
the first conjunct comes from the (reflection-strengthened)
output refinement of \fib~\S~\ref{sec:overview}, and
the second conjunct comes from rule~\rtexact.
%
Finally, rule \rtfix is used to type the intermediate
$\texttt{fix}$ expressions that appear, not in the
surface language but as intermediate terms in the
operational semantics.

\mypara{Soundness}
Following \undeclang~\citep{Vazou14}, we can show that
evaluation preserves typing and that typing implies
denotational inclusion.
%
\begin{theorem}{[Soundness of \corelan]}\label{thm:safety}
\begin{itemize}
\item\textbf{Denotations}
If $\hastype{\env}{\prog}{\typ}$ then
$\forall \sto\in \interp{\env}. \applysub{\sto}{\prog} \in \interp{\applysub{\sto}{\typ}}$.
\item\textbf{Preservation}
If \hastype{\emptyset}{\prog}{\typ}
       and $\evalsto{\prog}{w}$ then $\hastype{\emptyset}{w}{\typ}$.
\end{itemize}
\end{theorem}

\subsection{From Programs \& Types to Propositions \& Proofs}

The denotational soundness Theorem~\ref{thm:safety}
lets us interpret well typed programs as proofs of
propositions.

\NV{say that definition is a context that will be used later}
\mypara{``Definitions''}
A \emph{definition} $\defn$ is a sequence of reflected binders:
%
$$\defn \ ::= \ \bullet \spmid \erefb{x}{\gtyp}{e}{\defn}$$
%
A \emph{definition's environment} $\env(\defn)$ comprises
its binders and their reflected types:
%
\begin{align*}
\aenv(\bullet)                    \defeq & \emptyset \\
\aenv(\erefb{f}{\gtyp}{e}{\defn}) \defeq & (f, \exacttype{\gtyp}{e}),\ \env(\defn) \\
\end{align*}
%
A \emph{definition's substitution} $\sto(\defn)$ maps each binder
to its definition:
%
\begin{align*}
\sto(\bullet)                     \defeq & \emptysto \\
\sto(\erefb{f}{\gtyp}{e}{\defn})  \defeq & \extendsto{f}{\efix{f}\ e}{\sto(\defn)}
\end{align*}

\mypara{``Propositions''}
%
A \emph{proposition} is a type
%
$$\tbind{x_1}{\typ_1} \rightarrow \ldots
  \rightarrow \tbind{x_n}{\typ_n}
  \rightarrow \tref{v}{\tunit}{\ppn}$$
%
For brevity, we abbreviate propositions like the above to
%
$\tfun{\overline{x}}{\overline{\typ}}{\ttref{\ppn}}$
%
and we call $\ppn$ the \emph{proposition's refinement}.
%
For simplicity we assume that $\freevars{\typ_i} = \emptyset$.

\mypara{``Validity''}
%
\NV{add termination: proofs should provably terminates}

A proposition $\tfun{\overline{x}}{\overline{\typ}}{\ttref{\ppn}}$
is \emph{valid under} $\defn$ if
%
$$\forall \overline{w} \in \interp{\overline{\typ}}.\
  \evalsto{\applysub{\sto(\defn)}{\SUBST{\ppn}{\overline{x}}{\overline{w}}}}{\etrue}$$
%
That is, the proposition is valid if its refinement
evaluates to $\etrue$ for every (well typed)
interpretation for its parameters $\overline{x}$
under $\defn$.

\mypara{``Proofs''}
%
A binder $\bd$ \emph{proves} a proposition $\gtyp$ under $\defn$ if
$$\hastype{\emptyset}{\defn[\eletb{x}{\typ}{\bd}{\eunit}]}{\tunit}$$
%
That is, if the binder $\bd$ has the proposition's type $\gtyp$
under the definition $\defn$'s environment.

\begin{theorem}{[Proofs]} \label{thm:validity}
If $\bd$ proves $\typ$ under $\defn$ then $\typ$ is valid under $d$.
\end{theorem}

\begin{proof}
As $\bd$ proves $\typ$ under $\defn$, we have
%
\begin{align}
\hastype{\emptyset}{\defn[\eletb{x}{\typ}{\bd}{\eunit}]}{\tunit}
\label{pf:1} \\
%
\intertext{By Theorem~\ref{thm:safety} on \ref{pf:1} we get}
%
\sto(\defn) \in \interp{\env(\defn)} \label{pf:2}\\
%
\intertext{Furthermore,  by the typing rules \ref{pf:1}
implies $\hastype{\env(\defn)}{\bd}{\typ}$ and hence, via Theorem~\ref{thm:safety}}
%
\forall \sub \in \interp{\env(\defn)}.\ \applysub{\sub}{\bd} \in \interp{\applysub{\sub}{\typ}}
\label{pf:3} \\
\intertext{Together, \ref{pf:2} and \ref{pf:3} imply}
\applysub{\sto(\defn)}{\bd} \in \interp{\applysub{\sto(\defn)}{\typ}}
\label{pf:4}
\intertext{By the definition of type denotations, we have}
%
\interp{\applysub{\sto(\defn)}{\typ}}
  \defeq \{ f\ |\ \typ \mbox{ is valid under}\ \defn\}
  \label{pf:5}
\end{align}
By \ref{pf:4}, the above set is not empty, and hence $\typ$ is valid under $\defn$.
\end{proof}


%%%%%  \begin{definition}[Theorem]
%%%%%  Let $\aenv=\aenv(\prog)$ be the set of axioms for some program \prog.
%%%%%  Then
%%%%%  $\typ\defeq\tbind{x_1}{\typ_1} \rightarrow \dots \rightarrow \tbind{x_n}{\typ_n} \rightarrow \ttreft{v}{\btyp}{\refa}$
%%%%%  is a theorem if
%%%%%  $\freevars{\refa} \subseteq \{x_1, \dots, x_n\} \cup \domain{\aenv}$,
%%%%%  and for simplicity $\freevars{\typ_i} = \emptyset$.
%%%%%
%%%%%  Moreover, every expression $e'$ such that  \hastype{\aenv}{e'}{\typ}
%%%%%  provides a proof of the theorem $\typ$ with respect to the definitions in \prog.
%%%%%  \end{definition}
%%%%%  %
%%%%%  In other words, any expression $e'$ provides an evidence that the theorem expressed
%%%%%  by \typ holds.
%%%%%  The above definition is an instance of the Curry-Howard correspondence,
%%%%%  where programs as interpreted as proofs or the theorems expressed by their types.
%%%%%  %
%%%%%  %% Moreover, since the proof of the theorem is any expression $e'$ with type $\typ$
%%%%%  %% the definition states that our proof system is proof irrelevant.
%%%%%
%%%%%  \begin{theorem}
%%%%%  Let $\aenv=\aenv(\prog)$.
%%%%%  For every theorem
%%%%%  $\typ \equiv \tbind{x_1}{\typ_1} \rightarrow \dots \rightarrow \tbind{x_n}{\typ_n} \rightarrow \ttreft{v}{\btyp}{\refa}$
%%%%%  with a proof $e_\typ$ with respect to $p$,
%%%%%  then $\forall x_1\in\interp{\typ_1},\dots, x_n\in\interp{\typ_n}. \evalsto{\applysub{\Theta_\aenv(\prog)}{\refa}}{\etrue}$.
%%%%%  \end{theorem}
%%%%%  \begin{myproof}
%%%%%  Since \hastype{\aenv}{e_\typ}{\typ}
%%%%%  by Theorem~\ref{thm:safety} we have
%%%%%  $\forall \sub \in \interp{\aenv}.
%%%%%  \applysub{\sub}{e_\typ} \in \interp{\applysub{\sub}{\typ}}
%%%%%  $.
%%%%%  %
%%%%%  Since \prog type checks
%%%%%  we have $\Theta_\aenv(\prog)\in\interp{\aenv}$, thus
%%%%%  $\applysub{\Theta_\aenv(\prog)}{\refa} \in \interp{\applysub{\Theta_\aenv(\prog)}{\typ}}$.
%%%%%  %
%%%%%  By the Definition of Type Denotations we have
%%%%%  $\interp{\typ} =  \{f |\forall x_1\in\interp{\typ_1},\dots, x_n\in\interp{\typ_n}. \evalsto{\applysub{\Theta_\aenv(\prog)}{\refa}}{\etrue} \}$.
%%%%%  %
%%%%%  But $e_\typ$ is a witness that the set \interp{\typ} is not empty, thus
%%%%%  $\forall x_1\in\interp{\typ_1},\dots, x_n\in\interp{\typ_n}. \evalsto{\applysub{\Theta_\aenv(\prog)}{\refa}}{\etrue}$.
%%%%%  \end{myproof}

%% Theta_\aenv(\eletrec{f}{e}{\gtyp}{L}{\prog})
 %% & = (f, \efix{f}\ e),\aenv(p) \\
%% \Theta_\aenv(\elet{f}{e}{\gtyp}{L}{\prog})
 %% & = (f, e),\aenv(p) \\
%% \Theta_\aenv(\eletrecopt{f}{e}{\gtyp}{}{\prog})
 %% & = \Theta_\aenv(p) \\
%% \Theta_\aenv(e) &= \emptyset
%% \end{align*}

%% A proposition is \emph{well-formed} under $\defn$ if
%% $\freevars{\refa} \subseteq \{x_1, \dots, x_n\} \cup \domain{\aenv}$,


\mypara{Example: Fibonacci is increasing}
%
In \S~\ref{sec:overview} we verified that
under a definition $\defn$ that includes \fibname,
the term \fibincrname proves
$${\tfun{n}{\tnat}{\ttref{\fib{n} \leq \fib{(n+1)}}}}$$
%
Thus, by Theorem~\ref{thm:validity} we get
%that the \fib{}, as defined in $\defn$ is increasing
%
%$$
%\forall n\in\interp{\tnat}. \evalsto{\fib{n} \leq \fib{(n+1)}}{\etrue}
%$$
%Equivalently
$$
\forall n. \evalsto{0 \leq n}{\etrue} \Rightarrow \evalsto{\fib{n} \leq \fib{(n+1)}}{\etrue}
$$
