

\subsection{Map Fusion}\label{subsec:map_fusion}

Next, we prove the higher order property of map fusion.
First we define and axiomatize function composition and
list map:

\begin{code}
  axiomatize (.)
  (.) :: (b -> c) -> (a -> b) -> a -> c
  (.) f g x = f (g x)

  axiomatize map
  map :: (a -> b) -> L a -> L b
  map f N        = N
  map f (C x xs) = C (f x) (map f xs)
\end{code}


\NV{Say why we need defunctionalization}
\NV{We use app function like HALO (link to the theory)}
\NV{Zombie with rewritting does not allow HIGHER ORDER reasoning}

Then, we specify the map fusion property as a type
specification for the function @map_fusion@
and prove the property by induction on the list argument.
\begin{code}
  type MapFusion F G X
    = {map (F . G) X == (map F . map G) X}

  map_fusion :: f:(a -> a)
             -> g:(a -> a)
             -> xs:L a
             -> MapFusion f g xs
\end{code}

\begin{code}
  map_fusion f g N
    =   ((map f) r. (map g)) N
    ==! (map f) (rmap g N)
    ==! rmap f N
    ==! N
    ==! gmap (f . g) N
    *** QED

  map_fusion f g (C x xs)
    =   rmap (f . g) (C x xs)
    ==! (f . g) x `C` map (f . g) xs
    ==! (f . g) x `C` (map f r. map g) xs
         ? map_fusion f g xs
    ==! (f r. g) x `C` map f (map g xs)
    ==! f (g x)   `C` map f (map g xs)
    ==! gmap f (C (g x) (map g xs))
    ==! (map f) (gmap g (C x xs))
    ==! (map f g. map g) (C x xs)
    *** QED
\end{code}

\subsection{Monadic Laws: Associativity}
As a last example,
we axiomatize the monadic list bind operator
\begin{code}
  axiomatize >>=
  (>>=) :: L a -> (a -> L b) -> L b
  (C x xs) >>= f = f x ++ (xs >>= f)
  Emp      >>= f = N
\end{code}

We use the above definition to inductively prove
associativity of the bind operator

\begin{code}
  type Associative M F G
   {  M >>= F >>= G
   == M >>= (\x -> F x >>= G) }

  associativity
    :: m:L a
    -> f: (a -> L b)
    -> g:(b -> L c)
    -> Associative m f g

  associativity N f g
    =   N r>>= f >>= g
    ==! N r>>= g
    ==! N
    ==! N g>>= (\x -> f x >>= g)
    *** QED

  associativity (C x xs) f g
    =   (C x xs) r>>= f    >>= g
    ==! (f x) ++ (xs >>= f) >>= g
        ? bind_append (f x) (xs >>= f) g
    ==! (f x >>= g) ++ ((xs >>= f) >>= g)
    ==!    (f x >>= g)
        ++ (xs >>= (\y -> f y >>= g))
        ? associativity xs f g
    ==!    (\y -> f y >>= g) x
        ++ (xs >>= (\y -> f y >>= g))
        -- eta-equivalence
    ==! (C x xs) g>>= (\y -> f y >>= g)
    *** QED
\end{code}

In the proof we used the bind-append fusion lemma
\begin{code}
  bind_append
    :: xs:L a
    -> ys:L a
    -> f:(a -> L b)
    -> { (xs ++ ys) >>= f == (xs >>= f) ++ (ys >>= f) }
\end{code}

Moreover, we required $\beta$- and $\eta$-equilvalence on
anonymous functions.
For example, during the proof, we need the equality
@f x >>= g ==! (\x -> f x >>= g) y@.
%
To prove this equality, in the logic,
the anonymous functions are represented as functional variables
axiomatized with extensionality axioms.
%
Thus, in the logic, we define @f'@ and the axioms
@forall x. f' x = f x >>= g@ and
@forall g x. (f' x = g x) => f' = g@.
%
These two axioms are sufficient to prove
1. $\eta$-equivalence that is required in the last step
of the inductive case; and
2. $\beta$-equivalence that is required to prove that our proof
@xs >>= f >>= g ==! xs >>= (\y -> f y >>= g)@
implies the specification.
