\section{Evaluation}\label{sec:evaluation}


\begin{figure}
\begin{itemize}
\item \S~\ref{subsec:ackermann}: Ackermann Properties (Figure~\ref{fig:ackermann})
\item \S~\ref{subsec:fold}: Folding Properties (@universal@ \& @fusion@)
\item \S~\ref{subsec:class}: Class Laws (Figure~\ref{fig:laws}) for @List@ and @Maybe@
\item \S~\ref{subsec:programs}: Proofs and Programs: (@SAT-solve@ \& @unify@)
\end{itemize}
\caption{Summary of Benchmarks}
\label{fig:summary}
\end{figure}


\subsection{Arithmetic Properties}\label{subsec:ackermann}
\begin{figure}
\textbf{Definition}
\[
\begin{array}{lr}
\ack{n}{x}
 \left\{
\begin{array}{l}
\setlength\arraycolsep{0pt}
\begin{array}{ll}
      x+2 &\quad \text{, if}\ n=0 \\
      2   &\quad \text{, if}\ x=0 \\
\end{array} \\
\ack{n-1}{\ack{n}{x-1}}
\end{array}
\right.
&
\iack{h}{n}{x}
 \left\{
\begin{array}{l}
      x \quad \text{, if}\ h=0 \\
      \ack{n}{\iack{h-1}{n}{x}}
\end{array}
\right.
\end{array}
 \]

\textbf{Properties}
$$
\begin{array}{lrcrcl}
1.&                &&            \ack{n+1}{x}   &=& \iack{x}{n}{2}\\
2.&                &&            x + 1          &<& \ack{n}{x}\\
3.&                &&            \ack{n}{x}     &<& \ack{n}{x+1}\\
4.& x < y          &\Rightarrow& \ack{n}{x}     &<& \ack{n}{y}\\
5.& 0 < x          &\Rightarrow& \ack{n}{x}     &<& \ack{n+1}{x}\\
6.& 0 < x, n < m   &\Rightarrow& \ack{n}{x}     &<& \ack{m}{x}\\
7.&                &&            \iack{h}{n}{x} &<& \iack{h+1}{n}{x}\\
8.&                &&            \iack{h}{n}{x} &<& \iack{h}{n}{x+1}\\
9.& x<y            &\Rightarrow& \iack{h}{n}{x} &<& \iack{h}{n}{y}\\
10.&               &\Rightarrow& \iack{h}{n}{x} &<& \iack{h}{n+1}{x}\\
11.& 0<n, l-2 < x  &\Rightarrow& x + l          &<& \ack{n}{x}\\
12.& 0<n, l-2 < x  &\Rightarrow& \iack{l}{n}{x} &<& \ack{n+1}{x}\\
13.&               &&            \iack{x}{n}{y} &<& \ack{n+1}{x+y}\\
\end{array}
$$
\caption{Ackermann Properties~\citep{ackermann}, $\forall n, m, x, y, h, l \geq 0$}
\label{fig:ackermann}
\end{figure}

We used \libname to prove arithmetic properties.
We proved that the fibonacci function is increasing,
as described in \S~\ref{sec:examples}
and the properties of Ackermann function
from~\citep{ackermann} summarized in
Figure~\ref{fig:ackermann}.
%
This figure provides two alternative definitions
for the Ackermann Function. We proved equivalence
of the definition (Prop 1.) and various arithmetic
relations among them.


\spara{Generalizing Increasing Properties}
Property 3. shows that \ack{n}{x} is increasing on $x$.
We derived property 4. by applying the\\
\hbox{@gen_increasing@} theorem
from \S~\ref{sec:examples} with input function
the partially applied Ackermann Function \ack{n}{\star}.
%
We used the same technique to generalize increasing properties
from 8. to 9. on the partially applied funciton \iack{h}{n}{\star}.
%
Property 5. proves that \ack{n}{x} is increasing on the first argument $n$.
To generalize it to 6. we cannot reuse theorem @gen_increasin@,
since it only applies to the first argument,
and we need to define a variant @gen_increasing2@ that
generalizes increasing property on the first argument.

\spara{Existentials}
Properties 11. and 12. are described in~\citep{ackermann}
to hold for almost every $x$.
%
That is, Property 11. is described as
$\exists x_0. x_0 < x \Rightarrow x + l < \ack{n}{x}$.
%
Refinement types cannot express existentials,
thus expressing the above property is (currently)
not feasible, but the minimum $x$ was easy to retrieve.
Thus, we expressed the above statement by specifying $x_0 = l-2$.

\spara{Constructive Proofs}
Property 12. was proved in~\citep{ackermann} using a constructive proofs.
Specifically, it constructed a ``ladder'' if Ackermann invocations
and bounded \iack{h}{n}{x} and \ack{n}{x} with this ``ladder''.
To follow the steps of the proof, we defined a function @ladder@ that
simulated the proofs construction.

\subsection{Folding Properties}\label{subsec:fold}

Next, we proved properties of list folding,
as encoded in Adga~\citep{agdaequational}.
%
The following type encodes the universal property of foldr
\begin{code}
  foldr_univ
    :: f:(a -> b -> b)
    -> h:(L a -> b)
    -> e:b
    -> ys:L a
    -> base:{h [] == e }
    -> step:(x:a -> xs:[a] ->
            {h (x:xs) == f x (h xs)})
    -> {h ys == foldr f e ys }
\end{code}
The \libname proof term of @foldr_univ@
is similar to the one in Agda.
But, there are two major differences.
First, specifications do not support quantifiers:
universal quantification of @x@ and @xs@ in
the step assumption is encoded as functional arguments.
Second, unlike Agda,
\libname does not support optional arguments,
thus to use @foldr_univ@ the proof arguments
@base@ and @step@ that were implicit arguments in Agda
need to be explicitely provided.
%
For example, the following code calls @foldr_univarsal@ to prove
@foldr_fusion@ by explicitely applying proofs for the base and the step arguments
\begin{code}
  foldr_fusion
   :: h:(b -> c)
   -> f:(a -> b -> b)
   -> g:(a -> c -> c)
   -> e:b
   -> z:[a]
   -> x:a -> y:b
   -> fuse: {h (f x y) == g x (h y)})
   -> {(h.foldr f e) z == foldr g (h e) z}

  foldr_fusion h f g e ys fuse
   = foldr_univ g (h.foldr f e) (h e) ys
       (fuse_base h f e)
       (fuse_step h f e g fuse)
\end{code}

As an example, @fuse_base@ is a function with type
\begin{code}
  fuse_base
   :: h:(b -> c)
   -> f:(a -> b -> b)
   -> e:b
   -> { (h.foldr f e) [] == h e }
\end{code}

\subsection{Class Laws}\label{subsec:class}
\input{classlaws}
We used \libname to prove Monoid, Functor, Applicative and Monad Laws,
as in HERMIT~\citep{Farmer15}, as summarized
in Figure~\ref{fig:laws} for user-defined
List and Maybe data types.
%

The purpose of these proofs is to investigate the proving abilities of \libname.
For this purpose, we defined the appropriate class operators on user defined lists,
instead of using Haskell's predefined class instances.
%
In the near future, we plan to embed these proofs
to check the laws on real Haskell instances,
but this requires some engineering from the \liquidHaskell team.


\spara{Maybe is a Monoid}
The library definition of Monoid instance on @Maybe a@
defines @mappend@ as
\begin{code}
  instance (Monoid a)
           => Monoid (Maybe a) where
    mappend (Just x) (Just y)
      = Just (mappend x y)
    ...
\end{code}
%
Since we define @mappend@ as a Haskell function,
rather than an instance,
we have no way to express the above definition.
Instead we define @mappend@ to return the first non @Nothing@
argument, if any:
\begin{code}
  mappend Nothing x  = x
  mappend (Just x) _ = Just x
\end{code}
and prove that satisfies the associativity property.

\spara{Functor}
%
We proved both functor laws of Figure~\ref{fig:laws}
by induction on the argument $xs$.
%
One limitation of \libname name, is that
we know of no way to derive the partial
applied version of these laws.
%
For instance, even though we can prove
$\efmap\ \eid\ xs \equiv \eid\ xs$
we know of no way to derive
$\efmap\ \eid \equiv \eid$.

\spara{Monad}
Monad Laws refer to $\lambda$-functions,
thus their proof require defunctionalization and
the extensionality axioms to prove.

In all,
most of the proofs are straightforward,
using inductive reasoning in the structure of the data constructors
and rewriting the definitions of axiomatized functions.
%
The proofs are Haskell programs, and before checked by \liquidHaskell
they are checked by GHC for simple type errors.
%
Our experience,
during the proof process was that most of the times,
proofs that went thought vanilla GHC type checking,
were actually correct, \ie they were accepted by \liquidHaskell.

\subsection{Combining Proofs and Programs} \label{subsec:programs}
%
We used \libname to prove correctness of Haskell functions
that we could not previously prove,
namely, a SAT solver, and type unification.

\spara{SAT solver}.
We encoded the SAT solver example of~\citep{Zombie}
A solver is a Haskell function that takes as input a logical
@f:Formula@ and returns an assignment that satisfies @f@,
if one exists.

\begin{code}
  solve :: f:Formula
        -> Maybe {a:Assignment | sat a f }
  solve f = find (`sat` f) (assignments f)

  assignments :: Formula -> [Assignment]

  find :: (a -> Bool) -> [a] -> Maybe a

  axiomatize sat
  sat :: Assignment -> Formula -> Bool
  \end{code}

Function @assignments f@ returns all possible assignments of the formula @f@,
@sat a f@ returns True $\iff$ the assignment @a@ satisfies the formula @f@.
%
Verification of @solve@ follows directly from axiomatization of the Haskell function @sat@.

\spara{Term Unification}
As another example, we verified unification of first order terms,
as presented in~\citep{Sjoberg2015}.

We define the Haskell function @unify t1 t2@ that can
1. diverge, 2. return @Nothing@, or 3. return a substitution @su@ such that
@apply su t1 == apply su t2@:

\begin{code}
  inline eq_subst
  eq_subst su t1 t2
    = apply su t1 == apply su t2

  type EQSubst T1 T2 =
    {su:Subst | eq_subst su T1 T2 }

  unify :: t1:Term -> t2:Term
        -> Maybe (EQSubst t1 t2)
\end{code}
%
Note for the specification we axiomatize the Haskell function @apply@,
but we do not axiomatize @unify@.
Thus the user \textit{needs not prove} that @unify@ terminates, which is a complicate proof.
%
As before, we prove correctness by invoking helper lemmas that
are separately proved.
%
For example to prove the post-condition when unifying a variable
@TVar i@ with any type @t@ that does not contain the variable @i@,
we invoke the theorem @theoremVar t2 i@ that inductively proves the requirement

\begin{code}
  unify (TVar i) t2
    | not (i Set_mem freeVars t2)
    = Just $ [(i, t2)]
           `byTheorem` theoremVar t2 i

  theoremVar
    :: t:Term
    -> i:{Int | not (i Set_mem freeVars t) }
    -> { eq_subst [(i, t)] (TVar i) t }
\end{code}	%% $
