\section{Tracking Substitutions}

Then we define the notion of tracking substitutions.
In Figure~\ref{fig:proofs:tracking} we extend the operational
semantics with a state $\sigma$, \ie a mapping from variables to 
expressions that tracks evaluation of its expressions

%\newcommand\teval[4]{\evalt{#1}{#2}{#3}{#4}}
\renewcommand\evalt[4]{{\teval{#1}{#3}{#2}{#4}}}

\begin{figure*}
\hfill\mbox{\evalt{\sigma}{\sigma}{e}{e}}
$$
\begin{array}{ll}
	%% CONTEXT
	\evalt{\sigma}{\sigma'}{\eapp{e_1}{e_2}}{\eapp{e_1'}{e_2}} 
	& \text{if}\ 
	\evalt{\sigma}{\sigma'}{e_1}{e_1'}\\
	%
	\evalt{\sigma}{\sigma'}{\eapp{c}{e}}{\eapp{c}{e'}} 
	& \text{if}\ 
	\evalt{\sigma}{\sigma'}{e}{e'}\\
	%
	\evalt{\sigma}{\sigma'}{\ecase{e}{D_i}{\overline{y_i}}{e_i}{x}}{\ecase{e'}{D_i}{\overline{y_i}}{e_i}{x}}
	&\text{if}\ \evalt{\sigma}{\sigma'}{e}{e'} \\
	%
	\evalt{\sigma}{\sigma'}
		{D\ \overline{e_i}\ e\ \overline{e_j}}
		{D\ \overline{e_i}\ e'\ \overline{e_j}}
	&\text{if}\ \evalt{\sigma}{\sigma'}{e}{e'} \\
	%
	%% EVALUATION
	\evalt{\sigma}{\sigma}{\eapp{c}{v}}{\ceval{c}{v}} &\\
	\evalt{\sigma}{\sigma}{\eapp{\efun{x}{}{e}}{e_x}}{e\sub{x}{e_x}} &\\
	\evalt{\sigma}{\sigma}{\elet{x}{e_x}{e}}{e\sub{x}{e_x}}& \\
	\evalt{\sigma}{\sigma}{\ecase{D_j\ \overline{e}}{D_i}{\overline{y_i}}{e_i}{x}}{e_j\sub{x}{D_j\ 			\overline{e}}\sub{\overline{y_j}}{\overline{e}}} \\
	%% TRACKING
\teval{(x,e_x)\sigma}{x}{(x,e'_x)\sigma'}{x} &\text{if}\ \teval{\sigma}{e_x}{\sigma'}{e'_x}\\
	\teval{(y,e_y)\sigma}{x}{(y,e_y)\sigma'}{e_x}
  &\text{if}\
	\teval{\sigma}{x}{\sigma'}{e_x}\\
	\teval{(x,v)\sigma}{x}{(x,v)\sigma}{v}
  &\text{if}\
  v \not = D\ \overline{e}\\
	\teval{(x,D\ \overline{y})\sigma}{x}{(x,D\ \overline{y})\sigma}{D\ \overline{y}}
  &\\
	\teval{(x,D\ \overline{e})\sigma}{x}{(x,D\ \overline{y})\overline{(y_i, e_i)}\sigma}{D\ \overline{y}}
  &\text{if}\
	\text{fresh}\ \overline{y_i}
  \\
\end{array}
$$
\caption{Tracking Substitutions}
\label{fig:proofs:tracking}
\end{figure*}

First we prove that evaluation to a constant exists \textit{iff} 
tracking evaluation to the same constant exists.
\begin{lemma}\label{lemma:teval}
$\forall\theta, e, c, \exists\theta'. \evals{\thetasub{\theta}{e}}{c} \Leftrightarrow 
	\tevals{\theta}{e}{\theta'}{c}$.
\end{lemma}
\showproofsketch{
\begin{proofsketch}
\begin{itemize} We prove each direction:
\item $\Rightarrow$.
Given the derivation $\evals{\thetasub{\theta}{e}}{v}$, we can track the appearances 
of each expressions $\theta(x_i)$ and its derivatives and replace them with $x_i$.
Thus, given the initial derivation we can transverse it
(left-to-right and post-order); 
for every tracked appearance we use the appropriate rules 
that update the stack every time a tracked expressions evaluates, ie., 
appears in the left hand side of a rule; 
and remove the multiple evaluations of expressions in the stack
and construct the evaluation 
\tevals{\theta}{e}{\theta'}{c}.

Note that if $\theta(x_i)$ goes to a value, then 
$\theta(x_i) \equiv e_0 \hookrightarrow \dots e_i \dots e_n \equiv v$.
By the way we transverse the tree, 
after the stack is updated to $e_k$ and before it is updated to $e_{k+1}$
all tracked computations for $x_i$ are $e_j, j \leq k$.

If $\theta(x_i)$ does not go to a value, it cannot appear in the left hand side of 
a rule, because evaluation would diverge, thus the stack is not updated for $x_i$.

When a tracked expression reaches a value, we use the appropriate value to 
substitute (and untrack) the value.
Since the result of the initial evaluation is a constant, 
then the result of the tracked computation is the same constant. 

\item $\Leftarrow$.
Given $\tevals{\theta}{e}{\theta'}{c}$
we can construct the derivation \evals{\thetasub{\theta}{e}}{c} replacing each query to the 
stack with the initial computation of the expression.
\end{itemize}
\end{proofsketch}
}

Then we define a \textit{bottomize} function \mkbot{\bullet}
that replaces non-evaluated expressions with \ebot:
\begin{definition}{[Bottomize]}
$$
\mkbot{\theta}(x) = 
\left\{
	\begin{array}{ll}
		D\ \overline{\mkbot{\theta}(y)}  & \mbox{if } \theta(x) = D\ \overline{y}\\
		v  & \mbox{if } \theta(x) = v \not = D\ \overline{y}\\
		\ebot & \mbox{otherwise}
	\end{array}
\right.
$$
\end{definition}

Using the bottomize function we show that evaluation does not depend 
on non-evaluated expressions:
\begin{lemma}\label{lemma:mkbot}
If \tevals{\theta}{e}{\theta'}{c}, 
then \evals{\mkbot{\theta'}\ e}{c}.
\end{lemma}
\showproofsketch{
\begin{proofsketch}
Since \tevals{\theta}{e}{\theta'}{c}$(1)$, 
then \tevals{\theta'}{e}{\theta'}{c}$(2)$:
From the evaluation tree $(1)$ we can construct the evaluation tree $(2)$.
The trees differ on store related rules.

Say that in $(1)$ the store in $x$ is updated, for an arbitrary $x$: 
\teval{(x,e_x)\theta_x}{x}{(x,e'_x)\theta_x}{x}
Since $(1)$ is finite, it should be that
$\tevals{\theta_x}{e_x}{\theta_x}{v} (3)$.
Call $v_x = D\ \overline{y}$ if $v = D\ \overline{e}$, $v$ otherwise.
Then in $(1)$ there should be a ``subtree'' with $(3)$
after which the value of $x$ cannot change in the store.
Or $\theta'(x) = v_x$.
We construct $(2)$ by removing the ``subtree'' with $(3)$.
After that all rules that relate store with $x$ will be the same
on $(1)$ and $(2)$.

If $x$ is not updated in $(1)$ then 
$x$ does not appear in the left hand side of a rule; 
thus $\theta'(x) = \theta(x)$.

We construct $\theta''(x)= \left\{
	\begin{array}{ll}
		v  & \mbox{if}\ \theta'(x)= v\\
		\ebot & \mbox{otherwise}
	\end{array}
\right.$

Then \tevals{\theta''}{e}{\theta''}{c}.
If $\theta'(x)$ is not a value, then it does not appear in the left hand side 
of any rule in $(2)$, thus evaluation of $e$ cannot depend on $x$.

Then by Lemma \ref{lemma:teval}, \evals{\thetasub{\theta''}{e}}{c}.
But $\mkbot{\theta'} e = \thetasub{\theta''}{e}$, so \evals{\thetasub{\mkbot{\theta'}}{e}}{c}.
\end{proofsketch}
}

Also, replacing \ebot with any expression yields the same evaluation:
\begin{lemma}\label{lemma:rmbot}
If \evals{\thetasub{\mkbot{\theta}}{e}}{c}, 
then \evals{\thetasub{\theta}{e}}{c}.
\end{lemma}
\showproofsketch{
\begin{proof}
Since \evals{\thetasub{\mkbot{\theta}}{e}}{c}$(1)$, then
\tevals{\mkbot{\theta}}{e}{\theta'}{c}$(2)$.
\ebot expressions in \mkbot{\theta} are not evaluated, 
otherwise $(2)$ would get stuck.
Thus they can be instantiated with any expression.
$\theta$ provides such an instantiation, thus
\tevals{\theta}{e}{\theta''}{c}$(3)$.
By Lemma \ref{lemma:teval}, \evals{\thetasub{\theta}{e}}{c}.
\end{proof}
}

Finally, we define lifting substitutions
%
\begin{definition}{[Lifting Substitutions]}
$
\trackevals{\sto}{\botsto} \doteq 
\exists e, e', \theta' \tevals{\theta}{e}{\theta'}{e'} \land \botsto = \mkbot{\theta'}
$
\end{definition}

and prove the Lifting Lemma
\begin{lemma}{[Lifting]}\label{lemma:proofs:lifting}
$\evals{\thetasub{\sto}{e}}{c}$ iff $\exists \trackevals{\sto}{\botsto}$ s.t. 
$\evals{\thetasub{\botsto}{e}}{c}$.
\end{lemma}
\showproofsketch{
\begin{proofsketch}
The $\Rightarrow$ direction follows immediately from Lemmata~\ref{lemma:teval} and~\ref{lemma:mkbot}.
The $\Leftarrow$ direction follows immediately from Lemmata~\ref{lemma:teval} and~\ref{lemma:rmbot}.
\end{proofsketch}
}


