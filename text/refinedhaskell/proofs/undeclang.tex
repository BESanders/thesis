\section{Declarative Typing: \undeclang}

\subsection{Definitions}
To simplify the metatheory we extend \undeclang so that
\begin{itemize}
\item Supports stratified types, and
\item explicitly contains \ebot, a primitive that has any type, but does not evaluate. 
\end{itemize}


\begin{figure}
$$
\begin{array}{rrcl}

\emphbf{Constants} \quad 
  & c & ::=    & 0,1,-1,\ldots \spmid \etrue, \efalse \\
  &   & \spmid & +,-,\ldots \spmid =, <, \ldots \spmid \ecrash 
  \\[0.05in]

\emphbf{Values} \quad 
  & v & ::= &  c \spmid \efun{x}{\typ}{e} \spmid \edapp{D}{e}
  \\[0.05in] 

\emphbf{Expressions} \quad 
  & e & ::=    & \ebot \spmid v \spmid x \spmid \eapp{e}{e} \spmid \elet{x}{e}{e} \\ 
  &   & \spmid & \ecase{e}{D}{\overline{x}}{e}{x} \\[0.05in] 

\emphbf{Basic Types} \quad 
  & \tbase & ::= & \tint \spmid \tbool \spmid T \\[0.05in] 

\emphbf{Label} \quad 
  & l
  & ::= 
  & \trivial \spmid \finite 
  \\[0.05in] 
  
\emphbf{Types} \quad 
  & \typ & ::= & \tlref{v}{\tbase}{}{e} \spmid \tlref{v}{\tbase}{l}{e} \spmid
  				 \tfunref{x}{\typ}{\typ}{v}{e} \\ 
\end{array}
$$

\hrule width 0.48\textwidth

$$
\begin{array}{rrcl}
\emphbf{Contexts} \quad 
  & C
  & ::= 
  &   	 \bullet 
  \spmid \eapp{C}{e} 
  \spmid \eapp{c}{C} 
  \spmid D\ \overline{e}\ C\ \overline{e}\\
  &&\spmid &
  \ecase{C}{D}{\overline{y}}{e}{x}
  \\[0.05in] 
\end{array}
$$

\caption{\undeclang: Syntax}
\label{fig:undeclang}
\label{fig:operational}
\end{figure}

Then, we define the function \erase{\bullet} that erases the refinements in types and environments:
\begin{align*}
\erase{\tlref{v}{B}{l}{e}}&=B^{l} &
\erase{\emptyset}&=\emptyset\\
\erase{\tfunref{x}{\tau_x}{\tau}{v}{e}}&= \erase{\tau_x} \rightarrow \erase{\tau} &
\erase{x\colon\tau, \Gamma}&= x\colon\erase{\tau},\erase{\Gamma}
\end{align*}

and variable substitution on types:
\begin{align*}
(\tref{v}{B}{l}{e})\sub{y}{e_y} &=\tref{v}{B}{l}{e\sub{y}{e_y}}\\
(\tfunref{x}{\tau_x}{\tau}{v}{e})\sub{y}{e_y} &=
	\tfunref{x}{(\tau_x\sub{y}{e_y})}{(\tau\sub{y}{e_y})}{v}{e\sub{y}{e_y}}\\
\end{align*}


We extend the typing rules with another rule that types \ebot with
\textbf{any} type getting the rules as defined in Figure~\ref{fig:proofs:typing}.
%
\begin{figure}[p]
\centering
\captionsetup{justification=centering}
\judgementHead{Well-Formedness}{\isWellFormed{\Gamma}{\sigma}}

$$\begin{array}{ccc}
\inference
  {}
  {\isWellFormed{\Gamma}{\true(\vref)}}
  [\wtTrue]
&
\quad
&
\inference
    {\isWellFormed{\Gamma}{\areft(\vref)} && 
     \hastype{\Gamma}{\rvapp{\rvar}{e} \ \vref}{\tbbool}
    }
    {\isWellFormed{\Gamma}{(\areft \wedge \rvapp{\rvar}{e})(\vref)}}
    [\wtRVApp]
\end{array}$$
%
$$\inference
    {\hastype{\Gamma, \vref:b}{\reft}{\tbbool} \quad 
%     \isWellFormed{\Gamma, \vref:b}{\areft(\vref)}
     \hastype{\Gamma, \vref:b}{\areft(\vref)}{\tbbool}
    }
    {\isWellFormed{\Gamma}{\tpref{b}{\areft}{\reft}}}
    [\wtBase]
$$
%
$$
\inference
    {
	%\hastype{\Gamma, v:\tfun{x}{\tau_x}{\tau}}{e}{\tbbool} &&
	\hastype{\Gamma}{\reft}{\tbbool} &&
    \isWellFormed{\Gamma}{\tau_x} &&
	\isWellFormed{\Gamma, x:\tau_x}{\tau}
    }
    {\isWellFormed{\Gamma}{\trfun{x}{\tau_x}{\tau}{\reft}}}
    [\wtFun]
$$
%
$$\begin{array}{ccc}
\inference
  {\isWellFormed{\Gamma, \rvar:\tau}{\sigma}}
  {\isWellFormed{\Gamma}{\tpabs{\rvar}{\tau}{\sigma}}}
  [\wtPred]
&
\quad
&
\inference
    {\isWellFormed{\Gamma, \alpha}{\sigma}}
    {\isWellFormed{\Gamma}{\ttabs{\alpha}{\sigma}}}
    [\wtPoly]
\end{array}$$

\medskip \judgementHead{Subtyping}{\isSubType{\Gamma}{\sigma_1}{\sigma_2}}

$$
\inference
   {\text{SMT-Valid}(\inter{\Gamma} \land \inter{\areft_1\ \vref} \land \inter{\reft_1} 
                 \Rightarrow \inter{\areft_2\ \vref} \land \inter{\reft_2})}
   {\isSubType{\Gamma}{\tpref{b}{\areft_1}{\reft_1}}{\tpref{b}{\areft_2}{\reft_2}}}
   [\tsubBase]
$$
%
$$
\inference
   {%\text{Valid}(\inter{\Gamma}\land \inter{e_1} \Rightarrow \inter{e_2}) \\
	\isSubType{\Gamma}{\tau_2}{\tau_1} &
	\isSubType{\Gamma, x_2:{\tau_2}}{\SUBST{\tau_1'}{x_1}{x_2}}{\tau_2'}	
   }
   {\isSubType{\Gamma}
	  {\trfun{x_1}{\tau_1}{\tau_1'}{\reft_1}}
	  {\trfun{x_2}{\tau_2}{\tau_2'}{\true}}
}[\tsubFun]
$$
%
$$
\begin{array}{ccc}
\inference
   {\isSubType{\Gamma, \rvar:\tau}{\sigma_1}{\sigma_2}}
   {\isSubType{\Gamma}{\tpabs{\rvar}{\tau}{\sigma_1}}{\tpabs{\rvar}{\tau}{\sigma_2}}}
   [\tsubPred]
&
\quad
&
\inference
   {\isSubType{\Gamma}{\sigma_1}{\sigma_2}}
   {\isSubType{\Gamma}{\ttabs{\alpha}{\sigma_1}}{\ttabs{\alpha}{\sigma_2}}}
   [\tsubPoly]
\end{array}
$$

\medskip \judgementHead{Type Checking}{$\hastype{\Gamma}{e}{\sigma}$}

$$\inference
  {  \hastype{\Gamma}{e}{\sigma_2} && \isSubType{\Gamma}{\sigma_2}{\sigma_1} 
  && \isWellFormed{\Gamma}{\sigma_1}
  }
  {\hastype{\Gamma}{e}{\sigma_1}}
  [\tsub]
\quad
\inference
  {}
  {\hastype{\Gamma}{c}{\tc{c}}}
  [\tconst]
$$
$$
\inference
  {x: \tpref{b}{\areft}{\reft} \in \Gamma}
  {\hastype{\Gamma}{x}{\tpref{b}{\areft}{e \land \vref = x}}}
  [\tbase]
\quad
\inference
  {x:\tau \in \Gamma}
  {\hastype{\Gamma}{x}{\tau}} 
  [\tvariable]
$$
$$
\inference
   {\hastype{\Gamma, x:\tau_x}{e}{\tau} 
    && \isWellFormed{\Gamma}{\tau_x}
   }
   {\hastype{\Gamma}{\efunt{x}{\tau_x}{e}}{\tfun{x}{\tau_x}{\tau}}}
   [\tfunction]
\quad
\inference
   {\hastype{\Gamma}{e_1}{\tfun{x}{\tau_x}{\tau}} 
   &&  \hastype{\Gamma}{e_2}{\tau_x}
   }
   {\hastype{\Gamma}{\eapp{e_1}{e_2}}{\SUBST{\tau}{x}{e_2}}}
   [\tapp]
$$
$$
\inference
  {\hastype{\Gamma, \alpha}{e}{\sigma}}
  {\hastype{\Gamma}{\etabs{\alpha}{e}}{\ttabs{\alpha}{\sigma}}}
  [\tgen]
\quad
\inference
  {\hastype{\Gamma}{e}{\ttabs{\alpha}{\sigma}} && 
   \isWellFormed{\Gamma}{\tau}
  }
  {\hastype{\Gamma}{\etapp{e}{\tau}}{\SUBST{\sigma}{\alpha}{\tau}}}
  [\tinst]
$$
$$
\inference
    {\hastype{\Gamma, \rvar:\tau}{e}{\sigma} &&
     \isWellFormed{\Gamma}{\tau} 
     % \tau \mbox{ is non-refined } 
     %\isWellFormed{\Gamma}{\tpabs{p}{\tau}{\pi}} && 
     %p \notin \fv{e}
    }
    {\hastype{\Gamma}{\epabs{\rvar}{\tau}{e}}{\tpabs{\rvar}{\tau}{\sigma}}}
    [\tpgen]
\ \
\inference
    {\hastype{\Gamma}{e}{\tpabs{\rvar}{\tau}{\sigma}} && 
     \hastype{\Gamma}{\efunbar{x:\tau_x}{\reft'}}{\tau}
    }
    {\hastype{\Gamma}
             {\epapp{e}{\efunbar{x:\tau_x}{\reft'}}}
             {\rpinst{\sigma}{\rvar}{\efunbar{x:\tau_x}{\reft'}}}
     %        {\sigma\sub{\eapp{p}{\overline{e_p}}}{\eapp{\reft'}{\overline{e_p}}}}
    }
    [\tpinst]
$$
\caption[Type checking of \corelan.]{Well-formedness, Subtyping and Type Checking of \corelan.}
\label{fig:rules}
\end{figure}


%

We define the denotations of types by combining the denotations 
of stratified types:
\begin{definition}{[Type Denotations]}
\begin{align*}
\interp{\tref{x}{\tbase}{}{p}} \defeq & 
    \{e \mid  \hastypebase{\emptyset}{e}{\tbase},
              \mbox{ if } \evals{e}{v} 
              \mbox{ then } \evals{\SUBST{p}{x}{v}}{\etrue} \}\\
\interp{\tlref{v}{\tbase}{\trivial}{p}} \defeq & 
    \interp{\tlref{v}{\tbase}{}{p}} \cap \{ e \mid \evals{e}{v} \}\\
\interp{\tlref{v}{\tbase}{\finite}{p}} \defeq & 
    \interp{\tlref{v}{\tbase}{\trivial}{p}} \cap \{ e \mid \evals{e}{d} \} \\
\interp{\tfun{x}{\typ_x}{\typ}} \defeq & 
    \{e \mid  \hastypebase{\emptyset}{e}{\erase{\tfunbasic{\typ_x}{\typ}}}, 
              \forall e_x \in \interp{\typ_x}.\ \eapp{e}{e_x} \in \interp{\typ\sub{x}{e_x}}
    \}
\end{align*}
\end{definition}

Finally, we define the constraints that should be satisfied by constants:
%
\begin{definition}{[Constants]}\label{def:constants}
For every basic type $T$ there are exactly  $n = \arity{T}$ 
data contractors with result type $T$, namely 
$\{D_T^i | 0 < i \leq n \}$.

\CRASH is an untyped constant.
%
For each constant $c \neq \CRASH$
\newcommand\pcond[1]{\ensuremath{}}
\newcommand\const{\ensuremath{c}}
\begin{enumerate}
\item \hastype{\emptyset}{c}{\constty{\const}} and \iswellformed{}{\constty{c}}
%
\item If $\constty{c} = \tfun{x}{\tau_x}{\tau}$, then for each $v$, 
	$\ceval{\const}{v}$ is defined and 
	if \hastype{\emptyset}{v}{\tau_x} then
	\shastype{}{\interp{c}(v)}{\tau\sub{x}{v}},
	otherwise  $\interp{c}(v) = \CRASH$.
%	
\item If $\constty{c} = \tref{v}{B}{l}{e}$, 
	then 
	$c \in \interp{\constty{c}}$ and 
	$\forall c', c' \neq c. c' \not \in \interp{\constty{c}}$ 
%
\item If $\constty{D_T^i} = \tfun{x_1}{\tau_1}{\dots\tfun{x_n}{\tau_n}{\tau}}$, 
then $\tau_i$ are unrefined and for every $e_i$ with $0 < i \leq n$,
such that \hastype{\emptyset}{e_i}{\tau_i}, 
$D_T^i\ \overline{e_i}\in \interp{\tau\sub{x_i}{e_i}}$.
\end{enumerate}
\end{definition}


\subsection{Denotational Typing}
We define denotational typing as follows:
\begin{align*}
\shastype{\Gamma}{e}{\tau} & \doteq
	\forall \theta . \theta\in\interp{\Gamma}\Rightarrow \theta\ e \in \interp{\theta \ \tau}\\
\sissubtype{\Gamma}{\tau_1}{\tau_2} & \doteq 
	\forall \theta . \theta\in\interp{\Gamma}\Rightarrow \interp{\theta\ \tau_1} \subseteq \interp{\theta\ \tau_2}
\end{align*}

And prove that syntactic typing implies denotational typing, 
\ie a general version of Lemma~\ref{lem:denotation} of the paper.



\begin{lemma}{[Denotation Typing]}\label{lem:proofs:denotation}
\begin{enumerate}
\item If \issubtype{\Gamma}{\tau_1}{\tau_2} then \sissubtype{\Gamma}{\tau_1}{\tau_2}. 
\item If \hastype{\Gamma}{e}{\tau} then \shastype{\Gamma}{e}{\tau}.
\end{enumerate}
\end{lemma} 
\begin{proof}
Helping Lemma:
\begin{lemma}\label{lemma:closesem}
If \evals{e}{e'} then $e' \in \interp{\tau}$ \textit{iff} $e \in \interp{\tau}$.
\end{lemma}
\begin{proofsketch}
Since the validity of $e \in \interp{\tau}$ depends on the evaluated $e$, 
the if direction is evident.
The only if direction follows from the deterministic operational semantics.
\end{proofsketch}

%
\begin{enumerate}
\item \label{proof:ssub} Assume \issubtype{\Gamma}{\tau_1}{\tau_2}. 
We will prove it by induction on the derivation tree:

\begin{itemize}
\item\rsubbase. We have
$$\issubtype{\Gamma}{\tref{v}{B}{l}{e_1}}{\tref{v}{B}{l}{e_2}}$$
By inversion we get 
$$\issubref{\Gamma, v\colon B}{e_1}{e_2}$$
By inversion of \rimpl we have
$$	\forall \theta. \theta\in \interp{\Gamma}\Rightarrow
	\generalconditionImpl{\thetasub{\theta}{e_1}}
						{\thetasub{\theta}{e_2}}
\ (1)$$

We want to prove 
$$\sissubtype{\Gamma}{\tref{v}{B}{l}{e_1}}{\tref{v}{B}{l}{e_2}}$$
Equivalently
$$	
	\forall \theta . \iswellformedtheta{\Gamma}{\theta} \Rightarrow 
	\interp{\theta\ \tref{v}{B}{l}{e_1}} \subseteq \interp{\theta\ \tref{v}{B}{l}{e_2}}
$$

Since the labels are the same it suffices to prove that
\begin{align*}
	\forall \theta . \iswellformedtheta{\Gamma}{\theta} & \Rightarrow 
		\{e \mid \hastype{}{e}{B} 
 			\land 
			\generalconditionInterp{e}{\thetasub{\theta}{e_1\sub{v}{e}}} 
		\}	
	\\& \subseteq 
		\{e \mid \hastype{}{e}{B} 
			\land 
			\generalconditionInterp{e}{\thetasub{\theta}{e_2\sub{v}{e}}}
		 \}	
\end{align*}
Since $e \in \interp{B}$, we have \iswellformed{\Gamma,v\colon B}{\theta,\sub{v}{e}}.
So, from $(1)$ for $\theta := \theta,\sub{v}{e}$
we have 
$$	
	\generalconditionImpl
		{\thetasub{\theta}{e_1\sub{v}{e}}}
		{\thetasub{\theta}{e_2\sub{v}{e}}}
$$
\item\rsubfun Assume
$$
	\issubtype{\Gamma}{\tfunref{x}{\tau_x}{\tau}{v}{e_1}}{\tfunref{x}{\tau'_x}{\tau'}{v}{e_2}}
$$
By inversion we have
$$	
	\issubtype{\Gamma}{\tau'_x}{\tau_x} \qquad
	\issubtype{\Gamma, x \colon \tau'_x}{\tau}{\tau'} 
$$
By IH
$$	
	\sissubtype{\Gamma}{\tau'_x}{\tau_x} \ (1) \qquad
	\sissubtype{\Gamma, x \colon \tau'_x}{\tau}{\tau'} \ (2)
$$
We want to show that 
$$
	\sissubtype{\Gamma}
		{\tfunref{x}{\tau_x}{\tau}{v}{e_1}}
		{\tfunref{x}{\tau'_x}{\tau'}{v}{e_2}}
$$
Equivalently
$$	
	\forall \theta . \iswellformedtheta{\Gamma}{\theta} \Rightarrow 
	\interp{\thetasub{\theta}{\tfunref{x}{\tau_x}{\tau}{v}{e_1}}} 
	\subseteq 
	\interp{\thetasub{\theta}{\tfunref{x}{\tau'_x}{\tau'}{v}{e_2}}}
$$
Equivalently
\begin{align*}
	&\forall \theta. \iswellformedtheta{\Gamma}{\theta} \\&\Rightarrow 
	\{e \mid \hastype{}{e}{\erase{\tau_x} \rightarrow \erase{\tau}} 
	\land 
	\forall e_x \in \interp{\thetasub{\theta}{\tau_x}}. \
	 \eapp{e}{e_x} \in \interp{\thetasub{\theta}{\tau\sub{x}{e_x}}} 
	 \}\\ &
	\subseteq 
	\{e \mid \hastype{}{e}{\erase{\tau'_x} \rightarrow \erase{\tau'}} 
	\land 
	\forall e_x \in \interp{\thetasub{\theta}{\tau'_x}}. \
	 \eapp{e}{e_x} \in \interp{\thetasub{\theta}{\tau'\sub{x}{e_x}}} 
	 \}
\end{align*}
The above holds, as for any $e, e_x$
if $e_x \in \interp{\thetasub{\theta}{\tau_x'}}$
then by $(1)$
$e_x \in \interp{\thetasub{\theta}{\tau_x}}$.
Also, by $(2)$
if $\eapp{e}{e_x} \in \interp{\thetasub{\theta}{\tau\sub{x}{e_x}}}$
then
$\eapp{e}{e_x} \in \interp{\thetasub{\theta}{\tau'\sub{x}{e_x}}}$.
\end{itemize}


\item Assume \hastype{\Gamma}{e}{\tau}. 
We will prove it by induction on the derivation tree.
\begin{itemize}
\item\rtvar Assume \hastype{\Gamma}{e}{\tau}
	where $e \equiv x$.
	By inversion we have
	$$(x,\tau) \in \Gamma$$
	We need to show that 
	$$	\forall \theta . \iswellformedtheta{\Gamma}{\theta} 
		\Rightarrow \thetasub{\theta}{x} \in \interp{\thetasub{\theta}{\tau}}$$
	Which holds by the definition of well-formed substitutions.

\item\rtconst. Assume \hastype{\Gamma}{e}{\tau}
	where $e \equiv c$  and $\tau\equiv\constty{c}$.
	Then \shastype{\Gamma}{e}{\tau} holds by Definition \ref{def:constants}.

\item\rtsub Assume \hastype{\Gamma}{e}{\tau}.
	By inversion
	$$
	\hastype{\Gamma}{e}{\tau'}\ (1) \qquad
	\issubtype{\Gamma}{\tau'}{\tau}\ (2) \qquad
	\iswellformed{\Gamma}{\tau}\ (3)
	$$
%
	By IH on $(1)$ we have
	$$\shastype{\Gamma}{e}{\tau'}\ (4)$$
%
	By \ref{proof:ssub} on $(2)$
	$$\sissubtype{\Gamma}{\tau'}{\tau}\ (5)$$
%
	By $(4)$ and $(5)$ we get
	$$\shastype{\Gamma}{e}{\tau}$$

\item\rtfun Assume \hastype{\Gamma}{e}{\tau},
	where $e \equiv \efun{x}{}{e'}$ and 
	$\tau \equiv\tfun{x}{\tau'_x}{\tau'}$.
	By inversion we get
	$$
	\hastype{\Gamma, x\colon\tau'_x}{e'}{\tau'}\ (1) \qquad
	\iswellformed{\Gamma}{\tau'_x}\ (2)
	$$
	By IH on $(1)$ we have
	$$
	\shastype{\Gamma, x\colon\tau'_x}{e'}{\tau'}\ (3)
	$$
	Equivalently
	$$	
	\forall \theta . \iswellformedtheta{(\Gamma,x\colon\tau'_x)}{(\theta\sub{x}{e_x})} 
		\Rightarrow \thetasub{(\theta\sub{x}{e_x})}{e'} \in 
		\interp{\thetasub{(\theta\sub{x}{e_x})}{\tau'}}\\
	$$
	Or
	$$	
	\forall \theta . \iswellformedtheta{\Gamma}{\theta} \Rightarrow
	\forall e_x . e_x \in \interp{\tau'_x} \Rightarrow
		\thetasub{\theta}{\eapp{e}{e_x}} \in \interp{\thetasub{\theta}{\tau'\sub{x}{e_x}}}\\
	$$
%
	Moreover, $\hastypebase{}{e}{\erase{\tau'_x}\rightarrow{\erase{\tau}}}$.
%
	So,
	$$	
	\forall \theta . \iswellformedtheta{\Gamma}{\theta}. \thetasub{\theta}{e}\in \interp{\thetasub{\theta}{\tau}}
	$$
	Or, $$\shastype{\Gamma}{e}{\tau}$$

\item\rtapp. Assume \hastype{\Gamma}{e}{\tau},
	where $e\equiv\eapp{e_1}{e_2}$ and $\tau\equiv\tau'\sub{x}{e_2}$.
	By inversion:
	$$
	\hastype{\Gamma}{e_1}{(\tfunref{x}{\tau'_{x}}{\tau'}{v}{e_r})}\ (1)\qquad
	\hastype{\Gamma}{e_2}{\tau'_{x}}\ (2)
	$$
	By IH we get
	$$
	\shastype{\Gamma}{e_1}{(\tfunref{x}{\tau'_{x}}{\tau'}{v}{e_r})}\ (3)\qquad
	\shastype{\Gamma}{e_2}{\tau'_{x}}\ (4)
	$$
	So 
	$$\forall \theta. \iswellformedtheta{\Gamma}{\theta}\Rightarrow
	\forall e_x \in \interp{\thetasub{\theta}{\tau'_x}} \Rightarrow
		\eapp{(\thetasub{\theta}{e_1})}{e_x} \in 
		\interp{\thetasub{\theta}{\tau'\sub{x}{e_x}}}
	\ (5)$$
	and
	$$\forall \theta. \iswellformedtheta{\Gamma}{\theta}\Rightarrow
		\thetasub{\theta}{e_2} \in 
		\interp{\thetasub{\theta}{\tau'_x}}
	\ (6)$$
%
	From $(5)$ and $(6)$, we get
	$$\forall \theta. \iswellformedtheta{\Gamma}{\theta}\Rightarrow
		\theta\ e \in \interp{\thetasub{\theta}{\tau}}
	$$
	Or $$\shastype{\Gamma}{e}{\tau}$$

\item\rtlet. Assume \hastype{\Gamma}{e}{\tau}, 
	where $e \equiv\elet{x}{e_x}{e'}$.
	By inversion:
	$$
	\hastype{\Gamma}{e_x}{\tau_{x}}\ (1) \qquad
	\hastype{\Gamma,x\colon\tau_x}{e'}{\tau}\ (2)\qquad
	\iswellformed{\Gamma}{\tau}\ (3)
	$$
	By IH we get
	$$
	\shastype{\Gamma}{e_x}{\tau_{x}}\ (4) \qquad
	\shastype{\Gamma,x\colon\tau_x}{e'}{\tau}\ (5)
	$$
	By $(5)$
	$$\forall \theta'. \iswellformedtheta{\Gamma, x:\tau_x}{\theta'}\Rightarrow
		\thetasub{\theta'}{e'} \in \interp{\thetasub{\theta'}{\tau}}
		\ (6)
	$$
	By $(4)$, 
	$$
	 	\iswellformedtheta{\Gamma}{\theta}
		\Rightarrow 
		\iswellformedtheta{\Gamma, x:\tau_x}{\theta\sub{x}{e_x}}
	\ (7)$$
	From $(6)$, $(7)$ and $(3)$, we get
	$$\forall \theta. \iswellformedtheta{\Gamma}{\theta}\Rightarrow
		\thetasub{\theta}{e'\sub{x}{e_x}} \in \interp{\thetasub{\theta}{\tau}}
	$$
	By Lemma \ref{lemma:closesem}, we get
	$$\forall \theta. \iswellformedtheta{\Gamma}{\theta}\Rightarrow
		\thetasub{\theta}{e} \in \interp{\thetasub{\theta}{\tau}}
	$$
	So, $$\shastype{\Gamma}{e}{\tau}$$
\item\rtbot Assume \hastype{\Gamma}{e}{\tau}, 
	where $e \equiv\ebot$ and $\tau \equiv \tref{v}{B}{}{p}$.
	Since \ebot does not evaluate, 
	$$\forall \theta. \iswellformedtheta{\Gamma}{\theta}\Rightarrow
		\thetasub{\theta}{e} \in \interp{\thetasub{\theta}{\tau}}
	$$
	So, $$\shastype{\Gamma}{e}{\tau}$$

\item\rtcase Assume \hastype{\Gamma}{e}{\tau}, 
	where $e' \equiv \ecase{e}{D^i_T}{\overline{y}}{e_i}{x}$.
	By inversion
$$
	l \not \in \{\finite, \trivial\} \Rightarrow \tau \ \text{is}\ \Div\ (1)\qquad
	\hastype{\Gamma}{e}{\tref{v}{T}{l}{e_T}} \ (2)\qquad
	 \iswellformed{\Gamma}{\tau}\ (3)
$$
$$
\forall i. 0 < i \leq \arity{T}\{
$$
$$
	\constty{D^i_T} = \tfun{y_1}{\tau_1}{\dots\rightarrow\tfun{y_n}{\tau_n}{\tref{v}{T}{}{e_{T_i}}}}\ (4)
$$
$$
		\hastype{\Gamma,  
				\overline{y_i\colon \tau_i},
				x\colon\tlref{v}{T}{\restrictdecidable{\trivial}
				{\addtechnical{}{\ltrivial}}
				}{e_T \land e_{T_i}}}{e_i}{\tau}\ (5) \}
$$

By IH on $(2)$ we get 
$$
	\shastype{\Gamma}{e'}{\tref{v}{T}{l}{e_T}}\ (6)
$$

We fix a $\theta$ such that $\iswellformedtheta{\Gamma}{\theta}$
We split cases on whether \thetasub{\theta}{e'} evaluates to a WNF or not:
\begin{itemize}
\item If \evals{\thetasub{\theta}{e'}}{v}.
By $(6)$, for some $i$ such that $0 < i \leq \arity{T}$, 
%
$\evals{\thetasub{\theta}{e'}}{D^i_T\ \overline{e_j}}$.

By IH on $(4)$ and the Definition~\ref{def:constants}
$$
		\shastype{\Gamma}{e_i\sub{y_i}{e_j}\sub{x}{e'}}{\tau}
$$
Finally, by Lemma~\ref{lemma:closesem}
$$
		\shastype{\Gamma}{e}{\tau}
$$
\item If $\thetasub{\theta}{e'}$, then by $(6)$
$l \not \in \{\finite, \trivial \}$.
Moreover, $e$ diverges so it trivially belongs to the 
interpretation of any \Div type, or by $(1)$
$$
		\shastype{\Gamma}{e}{\tau}
$$
%%$$
%%\interp{\tref{v}{T}{}{p}} \doteq
%%\{
%%e \mid \hastypebase{}{e}{T}, 
%%\evals{e}{D^i_T\overline{e_j}} \Rightarrow
%%\constty{D^i_T} = \tfun{y_1}{\tau_1}{\dots\rightarrow\tfun{y_n}{\tau_n}{\tref{v}{T}{}{q}}} \Rightarrow
%%e_i \in \interp{\tau_i\sub{y_j}{e_j}}, 
%%\evals{p \land q\sub{y_j}{e_j}}{\etrue}
%%\}
%%$$
\end{itemize}
\end{itemize}
\end{enumerate}

\end{proof}

We define \iswellformed{}{\Gamma}
as \iswellformed{}{\emptyset} and if \iswellformed{\Gamma}{\tau} then \iswellformed{}{\Gamma, x:\tau}.
Now, using Lemma~\ref{lem:proofs:denotation} we prove substitution Lemma:
\begin{lemma}{[Substitution]}\label{lemma:substitution}
If \hastype{\Gamma}{e_x}{\tau_x} and \iswellformed{}{\Gamma, x\colon\tau_x ,\Gamma'}, then 
\begin{enumerate}
\item If 
	\issubtype{\Gamma, x\colon\tau_x, \Gamma'}{\tau_1}{\tau_2}
	then
	\issubtype{\Gamma, \sub{x}{e_x}\Gamma'}{\sub{x}{e_x}\tau_1}{\sub{x}{e_x}\tau_2}.
\item If 
	\hastype{\Gamma, x\colon\tau_x, \Gamma'}{e}{\tau}
	then
	\hastype{\Gamma, \sub{x}{e_x}\Gamma'}{\sub{x}{e_x}e}{\sub{x}{e_x}\tau}.
\item If 
	\iswellformed{\Gamma, x\colon\tau_x, \Gamma'}{\tau}
	then
	\iswellformed{\Gamma, \sub{x}{e_x}\Gamma'}{\sub{x}{e_x}\tau}.
\end{enumerate}
\end{lemma}
\begin{proof}
\newcommand\generalconditionImpol[2]{\ensuremath{\evals{#1}{\etrue}\Rightarrow \evals{#2}{\etrue}}}
If \hastype{\Gamma}{e_x}{\tau_x} and \iswellformed{\Gamma, x\colon\tau_x ,\Gamma'}, then 
\begin{enumerate}
\item\label{proof:sub:sub} Assume
	$$\issubtype{\Gamma, x\colon\tau_x, \Gamma'}{\tau_1}{\tau_2}$$
We will prove the lemma by induction on the derivation tree.
\begin{itemize}
\item \rsubbase
Assume \issubtype{\Gamma, x\colon\tau_x, \Gamma'}{\tau_1}{\tau_2}
where $\tau_1 \equiv \tref{v}{B}{l}{e_1}$
and   $\tau_2 \equiv \tref{v}{B}{l}{e_2}$.
By inversion
	$$
	\issubref{\Gamma, x\colon\tau_x, \Gamma',v:B}{e_1}{e_2}
	$$
By inversion
	\begin{align*}
	\forall &\theta, e'_x, \theta',e .
	\iswellformedtheta{\Gamma, x\colon\tau_x, \Gamma',v:B}
		{\theta\sub{x}{e'_x}\theta'\sub{v}{e}}\\& \Rightarrow
	\generalconditionImpl{\thetasub{\theta\sub{x}{e'_x}\theta'\sub{v}{e}}{e_1}\\&}
						 {\thetasub{\theta\sub{x}{e'_x}\theta'\sub{v}{e}}{e_2}}
	\end{align*}

Since \hastype{\Gamma}{e_x}{\tau_x}, so
	\begin{align*}
	\forall &\theta, \theta',e .
	\iswellformedtheta{\Gamma,x\colon\tau_x, \Gamma',v:B}{\theta \sub{x}{e_x}\theta'\sub{v}{e}}\\& \Rightarrow
	\generalconditionImpl{\thetasub{\theta\sub{x}{e_x}\theta'\sub{v}{e}}{e_1}\\&}
						 {\thetasub{\theta\sub{x}{e_x}\theta'\sub{v}{e}}{e_2}}
	\end{align*}
Since \hastype{\Gamma}{e_x}{\tau_x}, so
	\begin{align*}
	\forall &\theta, \theta',e .
	\iswellformedtheta{\Gamma,\sub{x}{e_x}\Gamma',v:B}{\theta\theta'\sub{v}{e}}\\& \Rightarrow
	\generalconditionImpl{\thetasub{\theta\theta'\sub{v}{e}}{e_1\sub{x}{e_x}}\\&}
						 {\thetasub{\theta\theta'\sub{v}{e}}{e_2\sub{x}{e_x}}}
	\end{align*}
So,
	$$
	\issubref{\Gamma, \sub{x}{e_x}\Gamma',v:B}{e_1\sub{x}{e_x}}{e_2\sub{x}{e_x}}
	$$
Or
	$$
	\issubtype{\Gamma, \sub{x}{e_x}\Gamma',v:B}{t_1\sub{x}{e_x}}{t_2\sub{x}{e_x}}
	$$
\item \rsubfun
Assume \issubtype{\Gamma, x\colon\tau_x, \Gamma'}{\tau_1}{\tau_2},
where $\tau_1 \equiv \tfun{y}{\tau_y}{\tau}$
and   $\tau_2 \equiv \tfun{y}{\tau'_y}{\tau'}$.
By inversion
	$$
	\issubtype{\Gamma, x\colon\tau_x, \Gamma'}{\tau'_y}{\tau_y}\ (1) \qquad
	\issubtype{\Gamma, x\colon\tau_x, \Gamma',y\colon\tau'_y}{\tau}{\tau'}\ (2)
	$$
By IH	
	$$
	\issubtype{\Gamma, \sub{x}{e_x}\Gamma'}{\tau'_y\sub{x}{e_x}}{\tau_y\sub{x}{e_x}} 
	$$
	$$
	\issubtype{\Gamma, \sub{x}{e_x}\Gamma',y\colon\tau'_y\sub{x}{e_x}}{\tau\sub{x}{e_x}}{\tau'\sub{x}{e_x}}
	$$
By rule \rsubfun	
	$$
	\issubtype{\Gamma, \sub{x}{e_x}\Gamma'}{\tau_1\sub{x}{e_x}}{\tau_2\sub{x}{e_x}}
	$$
\end{itemize}


\item \label{proof:sub:type} 
Assume 
	\hastype{\Gamma, x\colon\tau_x, \Gamma'}{e}{\tau}.
We will prove the lemma by induction on the derivation tree.
\begin{itemize}
\item \rtvar Assume \hastype{\Gamma, x\colon\tau'_x, \Gamma'}{e}{\tau},
where $e \equiv y$.
By inversion 
$$(y,\tau )\in \Gamma, x\colon\tau'_x, \Gamma'$$
Assume
$$(y,\tau)\in \Gamma$$
By rule \rtvar
$$\hastype{\Gamma,\sub{x}{e_x}\Gamma'}{e}{\tau}$$
Since \iswellformed{}{\Gamma}, $x$ cannot appear in $\tau$
so $\tau\sub{x}{e_x}\equiv\tau$.
Also, $x\neq y$, so $e\sub{x}{e_x}\equiv e$.
So,
$$\hastype{\Gamma,\sub{x}{e_x}\Gamma'}{e\sub{x}{e_x}}{\tau\sub{x}{e_x}}$$
%
Assume
$$y \equiv x$$
By lemma's assumption 
$$\hastype{\Gamma}{e_x}{\tau_x}$$
so
$$\hastype{\Gamma,\sub{x}{e_x}\Gamma'}{e_x}{\tau'_x}$$
Since $x = y$, $e\sub{x}{e_x} \equiv e_x$.
Also, since $x \notin Dom(\Gamma)$ 
it cannot appear in $\tau'_x$,so
$\tau\sub{x}{e_x} \equiv \tau \equiv \tau'_x$.
So,
$$\hastype{\Gamma,\sub{x}{e_x}\Gamma'}{e\sub{x}{e_x}}{\tau\sub{x}{e_x}}$$
%
Otherwise, assume
$$(y,\tau)\in \Gamma'$$
So,
$$(y,\sub{x}{e_x}\tau)\in \sub{x}{e_x}\Gamma'$$
Also, $e\sub{x}{e_x}\equiv e \equiv y$.
By which and rule \rtvar, we get
$$\hastype{\Gamma,\sub{x}{e_x}\Gamma'}{e\sub{x}{e_x}}{\tau\sub{x}{e_x}}$$

\item Case \rtconst.
Assume \hastype{\Gamma, x\colon\tau_x, \Gamma'}{e}{\tau},
where $e \equiv c$ and $\tau\equiv\constty{c}$.
Since $e\sub{x}{e_x} \equiv e$ and $\tau\sub{x}{e_x}\equiv\tau$
$$\hastype{\Gamma,\sub{x}{e_x}\Gamma'}{e\sub{x}{e_x}}{\tau\sub{x}{e_x}}$$

\item\rtsub
Assume \hastype{\Gamma, x\colon\tau_x, \Gamma'}{e}{\tau}.
By inversion
$$
\hastype{\Gamma, x\colon\tau_x, \Gamma'}{e}{\tau'}\ (1)\qquad
\issubtype{\Gamma, x\colon\tau_x, \Gamma'}{\tau'}{\tau} \ (2)
$$
$$
\iswellformed{\Gamma, x\colon\tau_x, \Gamma'}{\tau} \ (3)
$$
By IH, \ref{proof:sub:sub} and \ref{proof:sub:wf}
$$
\hastype{\Gamma, \sub{x}{e_x}\Gamma'}{\sub{x}{e_x}e}{\sub{x}{e_x}\tau'}
$$
$$
\issubtype{\Gamma, \sub{x}{e_x}\Gamma'}{\sub{x}{e_x}\tau'}{\sub{x}{e_x}\tau}
$$
$$
\iswellformed{\Gamma, \sub{x}{e_x}\Gamma'}{\sub{x}{e_x}\tau}
$$
By rule \rtsub
$$\hastype{\Gamma,\sub{x}{e_x}\Gamma'}{e\sub{x}{e_x}}{\tau\sub{x}{e_x}}$$

\item\rtfun Assume \hastype{\Gamma, x\colon\tau_x, \Gamma'}{e}{\tau},
where $e\equiv\efun{y}{e'}$ and $\tau\equiv\tfun{y}{\tau'_y}{\tau'}$.
By inversion
	$$
	\hastype{\Gamma, x\colon\tau_x, \Gamma', y\colon\tau'_y}{e'}{\tau'}\ (1)\qquad
	\iswellformed{\Gamma, x\colon\tau_x, \Gamma'}{\tau'_y}\ (2)
	$$
By IH and \ref{proof:sub:wf}
	$$
	\hastype{\Gamma,\sub{x}{e_x} \Gamma', y\colon\sub{x}{e_x}\tau'_y}{\sub{x}{e_x}e'}{\sub{x}{e_x}\tau'} 	$$
	$$
	\iswellformed{\Gamma, \sub{x}{e_x}\Gamma'}{\sub{x}{e_x}\tau'_y}
	$$
	By rule \rtfun
	$$
	\hastype{\Gamma,\sub{x}{e_x} \Gamma'}{\sub{x}{e_x}e}{\sub{x}{e_x}\tau}
	$$
	
\item\rtapp Assume \hastype{\Gamma, x\colon\tau_x, \Gamma'}{e}{\tau},
where $e\equiv\eapp{e_1}{e_2}$ and $\tau\equiv\tau'\sub{y}{e_2}$.
By inversion
	$$
	\hastype{\Gamma, x\colon\tau_x, \Gamma'}{e_1}{\tfun{y}{\tau'_y}{\tau'}}\ (1)\qquad
	\hastype{\Gamma, x\colon\tau_x, \Gamma'}{e_2}{{\tau'_y}}\ (2)
	$$
By IH 
	$$
	\hastype{\Gamma,\sub{x}{e_x} \Gamma'}{\sub{x}{e_x}e_1}{\sub{x}{e_x}\tfun{y}{\tau'_y}{\tau'}} \qquad
	$$
	$$
	\hastype{\Gamma,\sub{x}{e_x} \Gamma'}{\sub{x}{e_x}e_2}{\sub{x}{e_x}{\tau'_y}}
	$$
	By rule \rtapp
	$$
	\hastype{\Gamma,\sub{x}{e_x} \Gamma'}{\sub{x}{e_x}e}{\sub{x}{e_x}\tau}
	$$

\item\rtlet Assume \hastype{\Gamma, x\colon\tau_x, \Gamma'}{e}{\tau},
where $e\equiv\elet{y}{e_y}{e'}$.
By inversion
	$$
	\hastype{\Gamma, x\colon\tau_x, \Gamma'}{e_y}{\tau_y}\ (1) \qquad
	\hastype{\Gamma, x\colon\tau_x, \Gamma', y\colon\tau_y}{e}'{\tau}\ (2)
	$$
	$$
	\iswellformed{\Gamma, x\colon\tau_x, \Gamma'}{\tau}\ (3)
	$$
By IH and \ref{proof:sub:wf}	
	$$
	\hastype{\Gamma, \sub{x}{e_x}\Gamma'}{e_y}{\tau_y}\ (4) \qquad
	\hastype{\Gamma, \sub{x}{e_x}\Gamma', y\colon\tau_y}{e}'{\tau}\ (5)
	$$
	$$
	\iswellformed{\Gamma, \sub{x}{e_x}\Gamma'}{\tau}\ (6)
	$$
So, 	$$
	\hastype{\Gamma,\sub{x}{e_x} \Gamma'}{\sub{x}{e_x}e}{\sub{x}{e_x}\tau}
	$$

\item\rtcase This case is similar to \rtlet.
\item\rtbot Assume \hastype{\Gamma, x\colon\tau_x, \Gamma'}{e}{\tau},
where $e\equiv\ebot$.
By inversion
$$	\iswellformed{\Gamma, x\colon\tau_x, \Gamma'}{\tau}$$
By \ref{proof:sub:wf}
$$	\iswellformed{\sub{x}{e_x}\Gamma'}{\sub{x}{e_x}\tau}$$
By rule \rtbot
	$$
	\hastype{\Gamma,\sub{x}{e_x} \Gamma'}{\sub{x}{e_x}e}{\sub{x}{e_x}\tau}
	$$
\end{itemize}




\item \label{proof:sub:wf}
Assume \iswellformed{\Gamma, x\colon\tau_x, \Gamma'}{\tau}.
We will prove it by induction on the derivation.
\begin{itemize}
\item \rwbase
Assume \iswellformed{\Gamma, x\colon\tau_x, \Gamma'}{\tau},
where $\tau\equiv\tref{v}{B}{l}{e}$.
By inversion
$$\hastypebase{\erase{\Gamma, x\colon\tau_x, \Gamma'},v\colon B}{e}{\tbool}$$
So,
$$\hastypebase{\erase{\Gamma, \sub{x}{e_x}\Gamma'},v\colon B}{e\sub{x}{e_x}}{\tbool}$$
By rule \rwbase
$$\iswellformed{\Gamma, \sub{x}{e_x}\Gamma'}{\tref{v}{B}{l}{e\sub{x}{e_x}}}$$
Or 
$$\iswellformed{\Gamma, \sub{x}{e_x}\Gamma'}{\tau\sub{x}{e_x}}$$
\item \rwfun
Assume \iswellformed{\Gamma, x\colon\tau_x, \Gamma'}{\tau},
where $\tau\equiv \tfun{y}{\tau'_y}{\tau'}$.
By inversion, we get
$$
	\iswellformed{\Gamma, x\colon\tau_x, \Gamma'}{\tau_x} \qquad
	\iswellformed{\Gamma, x\colon\tau_x, \Gamma', y \colon \tau'_y}{\tau'}
$$
By IH
$$
	\iswellformed{\Gamma, \sub{x}{e_x} \Gamma'}{\tau_x\sub{x}{e_x}}\qquad
	\iswellformed{\Gamma, \sub{x}{e_x}(\Gamma', y \colon \tau'_y)}{\tau'\sub{x}{e_x}}
$$
Due to $\alpha$-renaming, $x \neq y$, so
$$
	\iswellformed{\Gamma, \sub{x}{e_x} \Gamma'}{\tau'_y\sub{x}{e_x}}\qquad
	\iswellformed{\Gamma, \sub{x}{e_x}\Gamma', y \colon \sub{x}{e_x}\tau'_y}{\tau'\sub{x}{e_x}}
$$
By \rwfun
$$
	\iswellformed{\Gamma, \sub{x}{e_x} \Gamma'}{\tfun{y}{\tau'_y\sub{x}{e_x}}{\tau'\sub{x}{e_x}}}
$$
Or
$$
	\iswellformed{\Gamma, \sub{x}{e_x} \Gamma'}{\tau\sub{x}{e_x}}
$$
\end{itemize}
\end{enumerate}
\end{proof}


\subsection{Soundness}
Figure~\ref{fig:proofs:botomless} defines a \botomless{\bullet} predicate on expressions:

%\def\figone{%
\begin{figure*}[t!]
$$
\botomless{c} \qquad\botomless{x} \qquad \lnot \botomless{\ebot}
$$
$$
\botomless{D\ \overline{e_i}} \Leftrightarrow \bigwedge\botomless{e_i} \qquad
\botomless {\efun{x}{}{e}} \Leftrightarrow \botomless{e}
$$
$$
\botomless {e_1 \ e_2} \Leftrightarrow \botomless{e_1} \land \botomless{e_2} 
$$
$$
\botomless {\elet{x}{e_1}{e_2}} \Leftrightarrow \botomless{e_1} \land \botomless{e_2}
$$
$$
\botomless {\ecase{e}{D_i}{\overline{x}}{e_i}{x}} \Leftrightarrow \botomless{e} \land \bigwedge\botomless{e_i}
$$
\caption{\botomless{e}}
\label{fig:proofs:botomless}
\end{figure*}
%\global\let\figone\relax}
We prove Preservation and Progress only on expressions that do not contain \ebot:
%
\begin{lemma}[Preservation]\label{lemma:preservation}
If \hastype{\emptyset}{e}{\tau}, \botomless{e} and \eval{e}{e'} then \hastype{\emptyset}{e'}{\tau}.
\end{lemma}
\begin{proof}
Helping Lemmata:
\begin{lemma}\label{lemma:wftypes}
If \hastype{\Gamma}{e}{\tau} and \iswellformed{}{\Gamma} then \iswellformed{\Gamma}{\tau}.
\end{lemma}
\begin{proofsketch}
By case split on the derivation \iswellformed{\Gamma}{\tau}
\end{proofsketch}
\begin{lemma}\label{lemma:eval}
If \eval{e}{e'} then
	\issubtype{\emptyset}{\tau\sub{x}{e'}}{\tau\sub{x}{e}}
\end{lemma}
\begin{proofsketch}
By case split on the derivation \issubtype{\Gamma}{\tau\sub{x}{e'}}{\tau\sub{x}{e}}
\end{proofsketch}

Assume \botomless{e} and \hastype{\emptyset}{e}{\tau} and \eval{e}{e'}. 
We will prove the lemma by induction on the derivation tree. 
\begin{itemize}
\item Cases \rtvar, \rtconst, \rtfun and \rtbot trivially hold
       as there is no $e'$ for which \eval{e}{e'}. 

\item Case \rtsub. Assume \hastype{\emptyset}{e}{\tau}.
By inversion
$$	\hastype{\emptyset}{e}{\tau'} \ (1) \qquad
	\issubtype{\emptyset}{\tau'}{\tau}\ (2) \qquad
	\iswellformed{\emptyset}{\tau}\ (3)
$$

By IH on $(1)$
$$	\hastype{\emptyset}{e'}{\tau'} $$
By which, $(2), (3)$ and \rtsub
$$	\hastype{\emptyset}{e'}{\tau}$$

\item Case \rtlet. Assume \hastype{\emptyset}{e}{\tau}, 
where $e \equiv \elet{x}{e_x}{e_0}$. Then 
 $e' \equiv e_0\sub{x}{e_x}$.
By inversion
$$
	\hastype{\emptyset}{e_x}{\tau_{x}} \ (1) \qquad
	\hastype{x\colon\tau_x}{e_0}{\tau} \ (2) \qquad
	\iswellformed{\emptyset}{\tau} \ (3)
$$

By $(1)$, $(2)$ and Lemma \ref{lemma:substitution}, 
$$\hastype{\emptyset}{e'}{\tau\sub{x}{e_x}} \ (4)$$
By $(3)$ $x$ does not appear free in $\tau$, so, $\tau\sub{x}{e_x} \equiv \tau$ and
$$\hastype{\emptyset}{e'}{\tau}$$

\item Case \rtapp. Assume
$$	\hastype{\emptyset}{e}{\tau}\ (1)$$
where $e \equiv \eapp{e_1}{e_2}$, and
	  $\tau\equiv\tau'\sub{x}{e_2}$

By inversion
$$	
	\hastype{\emptyset}{e_1}{(\tfun{x}{\tau_{x}}{\tau'})}\ (2) \qquad
	\hastype{\emptyset}{e_2}{\tau_{x}}\ (3)
$$

We split cases on the structure of $e$.
\begin{itemize}
\item $e\equiv \eapp{c}{v_2}$.
Then, $e'\equiv\interp{c}(v_2)$.
By Definition \ref{def:constants},
$$\hastype{\emptyset}{e'}{\tau}$$

\item $e\equiv \eapp{c}{e_2}$ where $e_2$ is botomless and not a value, 
Then, by (3) and Lemma~\ref{lemma:progress},
\eval{e_2}{e_2'}, and $e' \equiv \eapp{e_1}{e_2'}$.
%
By IH on $(3)$
$$	\hastype{\emptyset}{e_2'}{\tau_{x}}$$
By which, $(2)$ and rule \rtapp we get
$$\hastype{\emptyset}{e'}{\tau'\sub{x}{e_2'}}\ (4)$$
By Lemma \ref{lemma:eval}
$$
	\issubtype{\emptyset}{\tau'\sub{x}{e_2'}}{\tau'\sub{x}{e_2}}\ (5)
$$
By $(1)$ and Lemma \ref{lemma:wftypes}, since \iswellformed{}{\emptyset}
$$
	\iswellformed{\emptyset}{\tau'\sub{x}{e_2}}\ (6)
$$
By $(4), (5), (6)$ and rule \rtsub
$$	\hastype{\emptyset}{e'}{\tau}$$

\item $e \equiv \eapp{\efun{x}{e_x}}{e_2}$.
Then, $e' \equiv e_x\sub{x}{e_2}$.

By inversion on $(2)$
$$
	\hastype{x\colon\tau_x}{e_x}{\tau'}
$$
By which, $(3)$ and Lemma \ref{lemma:substitution} (since \iswellformed{}{x\colon\tau_x})
$$\hastype{\emptyset}{e'}{\tau'}$$

\item $e \equiv \eapp{e_1}{e_2}$, where $e_1$ is botomless and not a value.
Then, by $(2)$ and Lemma \ref{lemma:progress}, \eval{e_1}{e_1'} and 
$e'\equiv\eapp{e_1'}{e_2}$
By IH on $(2)$
$$	\hastype{\emptyset}{e_1'}{(\tfun{x}{\tau_{x}}{\tau'})}
$$
By which, $(3)$ and rule \rtapp we get
$$	\hastype{\emptyset}{e'}{\tau}$$
\end{itemize}
\item Case \rtcase, assume $\hastype{\emptyset}{e}{\tau}$, 
where $e \equiv \ecase{e_0}{D_i}{\overline{y}}{e}{x}$.
By inversion
$$	\hastype{\emptyset}{e_T}{\tref{v}{T}{l}{e_T}}\ (1) $$
$$	 \iswellformed{\emptyset}{\tau}\ (2)$$
$$	\forall i, 0 < i \leq \arity{T}. (
		\constty{D^i_T} = \tfun{x_1}{\tau_1}{\dots\tfun{x_n}{\tau_n}{\tref{v}{T}{l}{e_{T_i}}}}\ (3)
$$
$$		\theta = \overline{\sub{x_i}{y_i}}\ (4) \qquad
		\hastype{x\colon\tlref{v}{T}{}{e_t \land e_{T_i}}, 
						\overline{y_i\colon \theta\ \tau_i}}{e_i}{\tau}\ (5)	
	)
$$
We split cases on the structure of $e_T$.
\begin{itemize}
\item Assume that $e_T \equiv D^i_T\ \overline{e}$,
then $e' \equiv e_i \sub{x}{e} \overline{\sub{y_i}{e_i}}$.

By $(5)$
$$
		\hastype{\overline{y_i\colon \theta\ \tau_i}, 
		x\colon\tlref{v}{T}{l}{e_t \land \theta\ e_{T_i}}}{e_i}{\tau}\	
$$

By inversion on $(1)$ 
\hastype{\emptyset}{e_j}{\tau_j\overline{\sub{x}{e}}} 
and
\hastype{\emptyset}{D^i_T\ \overline{e}}{\tref{v}{T}{}{e_{T_i}}\overline{\sub{x}{e}}}. 
So,
\hastype{\emptyset}{e_j}{\tau_j\overline{\sub{x}{y}\sub{y}{e}}} 
and
\hastype{\emptyset}{D^i_T\ \overline{e}}{\tref{v}{T}{}{e_{T_i}}\overline{\sub{x}{y}\sub{y}{e}}}. 
And,
\shastype{\emptyset}{e_j}{\tau_j\overline{\sub{x}{y}\sub{y}{e}}} 
and
\shastype{\emptyset}{D^i_T\ \overline{e}}{\tref{v}{T}{}{e_{T_i}}\overline{\sub{x}{y}\sub{y}{e}}}. 

Finally, by Definition~\ref{def:constants}
\shastype{\emptyset}{D^i_T\ \overline{e}}{\tref{v}{T}{}{e_{T_i} \land e_t}\overline{\sub{x}{y}\sub{y}{e}}}. 

Then, by Lemma \ref{lemma:substitution}

\hastype{\emptyset}{e'}{\tau \overline{\sub{y_i}{e_i}}\sub{x}{e}}.

Finally, by $(2)$, $\tau \overline{\sub{y_i}{e_i}}\sub{x}{e} \equiv \tau$, so
$$
\hastype{\emptyset}{e'}{\tau}.
$$

\item Otherwise, by $(1)$ and Lemma \ref{lemma:progress} \eval{e_0}{e'_0}.
So $e' \equiv \ecase{e'_0}{D_i}{\overline{y}}{e}{x}$.
By IH \hastype{\emptyset}{e'_0}{\tref{v}{T}{}{e_T}}, 
by which and $(1) - (6)$ $$\hastype{\emptyset}{e'}{\tau}$$
\end{itemize}

\end{itemize}
\end{proof}
\begin{lemma}[Progress]\label{lemma:progress}
If \hastype{\emptyset}{e}{\tau}, \botomless{e} and $e \neq v$ 
then there exists an $e'$ such that \botomless{e'} and \eval{e}{e'}.
\end{lemma}
\begin{proof}
Assume \hastype{\emptyset}{e}{\tau}.
We will prove the Lemma by induction on the derivation tree.
\begin{itemize}
\item Case \rtvar cannot occur, as $\Gamma = \emptyset$
\item Case \rtbot is trivial, 
		as $\lnot \botomless{e}$.
\item Cases \rtconst and \rtfun are trivial, 
		as $e = v$.
\item Case \rtsub. Assume \hastype{\emptyset}{e}{\tau}.
By inversion
$$	\hastype{\emptyset}{e}{\tau'}$$
By IH 
either $e \equiv v$ or there exists an botomless $e'$ such that \eval{e}{e'}.
\item Case \rtapp. Assume $$\hastype{\emptyset}{e}{\tau}\ (1)$$
where $e\equiv\eapp{e_1}{e_2}$ and $\tau\equiv\tau'\sub{x}{e_2}$.
By inversion
$$
	\hastype{\emptyset}{e_1}{(\tfun{x}{\tau_{x}}{\tau})}\ (2)\qquad
	\hastype{\emptyset}{e_2}{\tau_{x}}\ (3)
$$

We split cases on the structure of $e$.
\begin{itemize}
\item $e\equiv \eapp{c}{v_2}$.
Then, $e'\equiv\interp{c}(v_2)$ which is botomless by Definition of constants.

\item $e\equiv \eapp{c}{e_2}$ where $e_2$ is not a value, 
By IH on $(3)$ \eval{e_2}{e_2'} and  $e' \equiv \eapp{e_1}{e_2'}$

\item $e \equiv \eapp{\efun{x}{e_x}}{e_2}$.
Then, $e' \equiv e_x\sub{x}{e_2}$, which does not contain bottom.

\item $e \equiv \eapp{e_1}{e_2}$, where $e_1 \neq v$.
Then, by IH on $(2)$ \eval{e_1}{e_1'} and 
$e'\equiv\eapp{e_1'}{e_2}$.
\end{itemize}

\item Case \rtlet. Assume \hastype{\emptyset}{e}{\tau}, where 
$e \equiv \elet{x}{e_x}{e_0}$, then $e'\equiv e_0\sub{x}{e_x}$ which is botomless.

\item Case \rtcase. Assume \hastype{\emptyset}{e}{\tau}, where
$e \equiv \ecase{e_T}{D_{T_i}}{\overline{y}}{e_i}{x}$.
By inversion, 
$$
	\hastype{\Gamma}{e}{\tref{v}{T}{e_T}}\ (1)
$$
We split cases on the structure of $e_T$
\begin{itemize}
\item If $e_T$ is a value, then by $(1)$ it is of the form $e_T \equiv D_{T_i} \overline{e}$,
so $e' \equiv e_i \sub{x}{e_T}\overline{\sub{y}{e}} $
\item Otherwise, by IH there exists $e'_T$ such that \evals{e_T}{e'_T}, 
so $e' \equiv \ecase{e'_T}{D_{T_i}}{\overline{y}}{e_i}{x}$.
\end{itemize}
\end{itemize}
\end{proof}

We combine the above to prove 
\textit{Soundness of \undeclang}, \ie Theorem~\ref{thm:safety} in the paper:
%
\begin{theorem}{[Soundness of \undeclang]}\label{thm:proofs:safety}
If \hastype{\emptyset}{e}{\tau} and \botomless{e}, then 
\begin{itemize}
\item\textbf{Type-Preservation:} If 
       $\evals{e}{v}$ then $\hastypet{\emptyset}{v}{\typ}$.
\item\textbf{Crash-Freedom:} $\evals{e\not}{\ecrash}$.
\end{itemize}
\end{theorem}
\begin{proof}
1. 
Since \botomless{e} there exists by Lemma \ref{lemma:progress} a bottomless evaluation sequence 
$$
e \equiv e_0 \eval{}{} e_1 \eval{}{} \dots \eval{}{} \dots e_n \equiv v
$$
The Theorem is proven by $n$ applications of Preservation Lemma.

2. If $\evals{e}{\CRASH}$, then by Preservation \hastype{\emptyset}{\CRASH}{\tau}
which cannot happen, as \CRASH by definition is an untyped constant.
\end{proof}
