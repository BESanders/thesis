The terms ``Haskell'' and ``pointer arithmetic'' 
rarely occur in the same sentence. 
Thus, from a verification point of view, the 
single most important aspect of the \bytestring 
library~\cite{bytestring}, our first
case study, is its pervasive intermingling of
high level abstractions like higher-order loops,
folds, and fusion, with low-level pointer 
manipulations in order to achieve high-performance. 
%
%% From the package description, \bytestring is, 
%% ``A time and space-efficient implementation of byte vectors using packed
%% Word8 arrays, suitable for high performance use, both in terms of large
%% data quantities, or high speed requirements. Byte vectors are encoded as
%% strict Word8 arrays of bytes, held in a ForeignPtr, and can be passed
%% between C and Haskell with little effort."
%
\bytestring is an appealing target for evaluating
\toolname, as refinement types are an ideal way to 
statically ensure the correctness of the delicate 
pointer manipulations, errors in which lie below 
the scope of dynamic protection.

The library spans $8$ files (modules) totaling about 3600 lines.
We used \toolname to verify the library by giving precise 
types describing the sizes of internal pointers and bytestrings. 
These types are used in a modular fashion to verify the 
implementation of functional correctness properties of 
higher-level API functions which are built using 
lower-level internal operations. 
Next, we show the key invariants and how
\toolname reasons precisely about pointer
arithmetic and higher-order codes.

\spara{Key Invariants}
A (strict) @ByteString@ is a triple of a @pay@load pointer, 
an @off@set into the memory buffer referred to by the pointer 
(at which the string actually ``begins") and a @len@gth 
corresponding to the number of bytes in the string, which is 
the size of the buffer \emph{after} the @off@set, that
corresponds to the string.
%
We define a measure for the \emph{size} of 
a @ForeignPtr@'s buffer, and use it to define 
the key invariants as a refined datatype 
%
\begin{code}
measure fplen  :: ForeignPtr a -> Int
data ByteString = PS
 { pay :: ForeignPtr Word8
 , off :: {v:Nat | v       <= (fplen pay)}
 , len :: {v:Nat | off + v <= (fplen pay)} }
\end{code}
%
The definition states that 
the offset is a @Nat@ no bigger than the size of 
the @payload@'s buffer, and that
the sum of the @off@set and non-negative @len@gth
is no more than the size of the payload buffer.
Finally, we encode a @ByteString@'s size as a measure.
%
\begin{code}
measure bLen   :: ByteString -> Int
bLen (PS p o l) = l
\end{code}

\spara{Specifications}
We define a type alias for a @ByteString@ whose length is the same
as that of another, and use the alias to type the API 
function @copy@, which clones @ByteString@s.

\begin{code}
type ByteStringEq B 
  = {v:ByteString | (bLen v) = (bLen B)}
copy :: b:ByteString -> ByteStringEq b 
copy (PS fp off len) 
  = unsafeCreate len $ \p -> 
      withForeignPtr fp $ \f ->
        memcpy len p (f `plusPtr` off) 
\end{code}

\spara{Pointer Arithmetic}
The simple body of @copy@ abstracts a fair bit of internal work. 
@memcpy sz dst src@, implemented in \C and accessed via the FFI is a potentially
dangerous, low-level operation, that copies @sz@ bytes starting
\emph{from} an address @src@ \emph{into} an address @dst@. 
Crucially, for safety, the regions referred to be @src@ and @dst@ 
must be larger than @sz@. We capture this requirement by defining
a type alias @PtrGE a N@ denoting GHC pointers that refer to a region
bigger than @N@ bytes, and then specifying that the destination
and source buffers for @memcpy@ are large enough.

\begin{code}
type PtrN a N = {v:Ptr a| N <= (plen v)}
memcpy :: sz:CSize -> dst:PtrN a siz 
                   -> src:PtrN a siz 
                   -> IO () 
\end{code}

The actual output for @copy@ is created and filled in using the 
internal function @unsafeCreate@ which is a wrapper around %@create@.

% -- | Create ByteString of size @l@ and use
% --   action @f@ to fill it's contents.
\begin{code}
create :: l:Nat -> f:(PtrN Word8 l -> IO ())
       -> IO (ByteStringN l)
create l f = do
    fp <- mallocByteString l
    withForeignPtr fp $ \p -> f p
    return $! PS fp 0 l
\end{code}

% We include the comment to illustrate how the 
% refinement type captures the natural language 
% requirement in a machine checkable manner.
%
The type of @f@ specifies that the action
will only be invoked on a pointer of length at least 
@l@, which is verified by propagating the types of
@mallocByteString@ and @withForeignPtr@. 
%
The fact that the action is only invoked on such pointers 
is used to ensure that the value @p@ in the body of @copy@ 
is of size @l@. This, and the @ByteString@ 
invariant that the size of the payload @fp@ 
exceeds the sum of @off@ and @len@, ensures 
that the call to @memcpy@ is safe.

\spara{Higher Order Loops} 
@mapAccumR@ combines a @map@ and a @foldr@ over a @ByteString@. 
The function uses non-trivial recursion, and demonstrates 
the utility of abstract-interpretation based inference. 
%
\begin{code}
mapAccumR f z b 
  = unSP $ loopDown (mapAccumEFL f) z b
\end{code}
%
To enable fusion \cite{streamfusion} 
@loopDown@ uses a higher order @loopWrapper@ 
to iterate over the buffer with a @doDownLoop@ action:
%
\begin{code}
doDownLoop f acc0 src dest len 
  = loop len (len-1) (len-1) acc0
  where
   loop (w::Int) s d acc
     | s < 0 
     = return (acc :*: d+1 :*: len - (d+1))
     | otherwise       
     = do x <- peekByteOff src s
          case f acc x of
            (acc' :*: NothingS) -> 
                 loop (w-1) (s-1) d acc'
            (acc' :*: JustS x') -> 
                 pokeByteOff dest d x'
              >> loop (w-1) (s-1) (d-1) acc'
\end{code}

The above function iterates across the @src@ and @dst@ 
pointers from the right (by repeatedly decrementing the 
offsets @s@ and @d@ starting at the high @len@ down to @-1@). 
Low-level reads and writes are carried out using the 
potentially dangerous @peekByteOff@ and @pokeByteOff@ 
respectively. To ensure safety, we type these low level 
operations with refinements stating that they are only 
invoked with valid offsets @VO@ into the input buffer @p@.

\begin{code}
type VO P    = {v:Nat | v < plen P}
peekByteOff :: p:Ptr b -> VO p -> IO a
pokeByteOff :: p:Ptr b -> VO p -> a -> IO ()
\end{code}

The function @doDownLoop@ is an internal function.
Via abstract interpretation~\cite{LiquidPLDI08}, 
\toolname infers that
%
(1)~@len@ is less than the sizes of @src@ and @dest@,
(2)~@f@ (here, @mapAccumEFL@) always returns a @JustS@, so
(3)~source and destination offsets satisfy $\mathtt{0 \leq s, d < {len}}$,
(4)~the generated @IO@ action returns a triple @(acc :*: 0 :*: len)@,
%
thereby proving the safety of the accesses in @loop@ \emph{and}
verifying that @loopDown@ and the API function @mapAccumR@ 
return a \bytestring whose size equals its input's.
 
To prove \emph{termination}, we add a \emph{witness} @w@.
Though @s@ decreases at each call, it is \emph{not} a @Nat@ 
as it reaches @-1@. 
The system infers that @w@ decreases and \emph{is} a @Nat@
as it equals @s+1@, thus proving termination.

\spara{Nested Data}
Finally, consider @group@, which
splits a string like @"aart"@ into the list
@["aa","r","t"]@, \ie a list of
(a)~non-empty @ByteString@s whose 
(b)~total length equals that of the input. 
To specify these requirements, we define a measure for 
the total length of strings in a list and use it to
write an alias for a list of \emph{non-empty} strings
whose total length equals that of another string:

\begin{code}
measure bLens :: [ByteString] -> Int 
bLens ([])     = 0
bLens (x:xs)   = bLen x + bLens xs

type ByteStringNE 
  = {v:ByteString | bLen v > 0}
type ByteStringsEq B
  = {v:[ByteStringNE] | bLens v = bLen b}
\end{code}
%
\toolname uses the above to verify that
%
\begin{code}
group :: b:ByteString -> ByteStringsEq b
group xs
 | null xs   = []
 | otherwise = let x        = unsafeHead xs
                   xs'      = unsafeTail xs
                   (ys, zs) = spanByte x xs' 
               in (y `cons` ys) : group zs
\end{code}
%
The example illustrates why refinements are critical for
proving termination. \toolname determines that @unsafeTail@ 
returns a \emph{smaller} @ByteString@ than its input, and that
each element returned by @spanByte@ is no bigger than the 
input, concluding that @zs@ is smaller than @xs@, and hence
checking the body under the termination-weakened environment.

To see why the output type holds, let's look at @spanByte@,
which splits strings into a pair:
%
\begin{code}
spanByte c ps@(PS x s l) 
  = inlinePerformIO $ withForeignPtr x $
        \p -> go l (p `plusPtr` s) 0
  where
    go (w::Int) p i 
      | i >= l    = return (ps, empty)
      | otherwise = do
          c' <- peekByteOff p i
          if c /= c'
            then let b1 = unsafeTake i ps
                     b2 = unsafeDrop i ps
                 in  return (b1, b2)
            else go (w-1) p (i+1)
\end{code}
%
Via inference, \toolname verifies the safety of 
the pointer accesses, and determines that the 
sum of the lengths of the output pair of 
@ByteString@s equals that of the input @ps@.
Termination follows by inferring that the sum of 
the witness @w@ and @i@ equals @l@.

