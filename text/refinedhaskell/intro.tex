When we started the \toolname project, 
we built on the theory of standard refinement types 
as implemented for example for ML~\cite{pfenningxi98,GordonRefinement09,LiquidPLDI08}.
%
Standard refinement types were developed for the eager, 
\emph{call-by-value} languages, but we presumed that
the order of evaluation would surely prove irrelevant
and that the soundness guarantees would translate 
to Haskell's lazy, \emph{call-by-need} regime.
%
We were wrong.

\begin{itemize}
\item We start this chapter by showing that standard refinement 
systems crucially rely on a property of eager languages:
%
when analyzing any term, one can assume that \emph{all} the
free variables appearing in the term are bound to \emph{values}.
This property lets us check each term in an environment where 
the free variables are logically constrained according to 
their refinements.
%
Unfortunately, this property does not hold for lazy evaluation, 
where free variables can be lazily substituted with arbitrary 
(potentially diverging) expressions, which breaks 
soundness (\Sref{sec:refinedhaskell:overview}).

The two natural paths towards soundness are blocked by challenging problems.
%
The first path is to \emph{conservatively ignore} free variables 
except those that are guaranteed to be values \eg by pattern 
matching, @seq@ or strictness annotations.  
While sound, this leads to a drastic loss of precision. 
%
The second path is to \emph{explicitly} reason about divergence 
within the refinement logic. This would be sound and 
precise -- however it is far from obvious to us how 
to re-use and extend existing SMT machinery for 
this purpose. (\Sref{sec:refinedhaskell:conclusion})

\item Next, we present a novel approach that
enables sound and precise checking with existing 
SMT solvers, using a \emph{stratified} type system that 
labels binders as potentially diverging or not 
(\Sref{sec:typing}).
%
While previous stratified systems~\cite{ConstableS87}
would suffice for soundness, we show how to recover 
precision by using refinement types to develop a 
notion of \emph{terminating fixpoint} combinators 
that allows the type system to automatically 
verify that a wide variety of recursive functions 
actually terminate (\Sref{sec:haskell}).

\item Finally, we provide an extensive empirical
evaluation of our approach on more than $10,000$ 
lines of widely used complex Haskell libraries.
We have implemented our approach in \toolname, 
and use it to prove termination of 
libraries verified in Chapter~\ref{chapter:tool}.
%
\toolname is able to prove 96\% of 
all recursive functions terminating, requiring a modest 
1.7 lines of termination annotations per 100 lines 
of code, thereby enabling the sound, precise, 
and automated verification of functional 
correctness properties of real-world Haskell code 
(\Sref{sec:refinedhaskell:evaluation}).
\end{itemize}
