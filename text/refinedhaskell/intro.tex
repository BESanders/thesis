\section{Introduction}\label{sec:intro}

% Refinement types are unsound under Haskell's lazy semantics.
%
Refinement types encode invariants
by composing types with SMT-decidable refinement 
predicates~\cite{Rushby98,pfenningxi98},
generalizing Floyd-Hoare Logic 
(\eg EscJava~\cite{ESCJava})
for functional languages.
%
For example
%
\begin{code}
   type Pos = {v:Int | v >  0}
   type Nat = {v:Int | v >= 0}
\end{code}
%
are the basic type @Int@ refined with logical predicates 
that state that ``the values'' @v@ described by the type 
are respectively strictly positive and non-negative.
%
We encode \emph{pre-} and \emph{post-}conditions (contracts) using 
refined function types like 
%
\begin{code}
   div :: n:Nat -> d:Pos -> {v:Nat | v <= n}
\end{code}
%
which states that the function @div@ \emph{requires} inputs that are 
respectively non-negative and positive, and \emph{ensures} that the 
output is less than the first input @n@.
If a program containing @div@ statically type-checks, we can rest assured that
executing the program will not lead to any unpleasant divide-by-zero errors.
%
By combining types and SMT based validity checking,
refinement types have automated the verification of 
programs with recursive datatypes, higher-order 
functions, and polymorphism.
%
Several groups have used refinements to statically 
verify properties ranging from simple array safety~\cite{pfenningxi98,LiquidPLDI08}
to functional correctness of data structures~\cite{LiquidPLDI09}, 
security protocols~\cite{GordonTOPLAS2011},  
and compiler correctness~\cite{fstar}.

Given the remarkable effectiveness of the technique, 
we embarked on the project of developing a refinement 
type based verifier for  Haskell.
The previous systems were all developed for eager, 
\emph{call-by-value} languages, but we presumed that
the order of evaluation would surely prove irrelevant, 
and that the soundness guarantees would translate 
to Haskell's lazy, \emph{call-by-need} regime.

We were wrong.
% To our surprise, we were totally wrong. 
%
Our first contribution is to show that standard refinement 
systems crucially rely on a property of eager languages:
%
when analyzing any term, one can assume that \emph{all} the
free variables appearing in the term are bound to \emph{values}.
This property lets us check each term in an environment where 
the free variables are logically constrained according to 
their refinements.
%
Unfortunately, this property does not hold for lazy evaluation, 
where free variables can be lazily substituted with arbitrary 
(potentially diverging) expressions, which breaks 
soundness (\Sref{sec:overview}).

The two natural paths towards soundness are blocked by challenging problems.
%
The first path is to \emph{conservatively ignore} free variables 
except those that are guaranteed to be values \eg by pattern 
matching, @seq@ or strictness annotations.  
While sound, this leads to a drastic loss of precision. 
%
The second path is to \emph{explicitly} reason about divergence 
within the refinement logic. This would be sound and 
precise -- however it is far from obvious to us how 
to re-use and extend existing SMT machinery for 
this purpose. (\Sref{sec:conclusion})

%% DV: I reworded slightly. 
%% While sound and precise, we do not this route 
%% makes the refinement logic three-valued, making it 
%% impossible to use existing SMT machinery (\Sref{sec:conclusion}).

Our second contribution is a novel approach that
enables sound and precise checking with existing 
SMT solvers, using a \emph{stratified} type system that 
labels binders as potentially diverging or not 
(\Sref{sec:typing}).
%
While previous stratified systems~\cite{ConstableS87}
would suffice for soundness, we show how to recover 
precision by using refinement types to develop a 
notion of \emph{terminating fixpoint} combinators 
that allows the type system to automatically 
verify that a wide variety of recursive functions 
actually terminate (\Sref{sec:haskell}).

Our third contribution is an extensive empirical
evaluation of our approach on more than $10,000$ 
lines of widely used complex Haskell libraries.
We have implemented our approach in \toolname, 
an SMT based verifier for Haskell. 
%
\toolname is able to prove 96\% of 
all recursive functions terminating, requiring a modest 
1.7 lines of termination annotations per 100 lines 
of code, thereby enabling the sound, precise, 
and automated verification of functional 
correctness properties of real-world Haskell code 
(\Sref{sec:evaluation}).

%%% MOVE TO RELATED
%%% ES: In other words, eager semantics embed an 
%%% ES: implicit reasoning about divergence.
%%% ES: %
%%% ES: To restore soundness under lazy evaluation
%%% ES: we use refinement types themselves to explicitly reason about divergence
%%% ES: and annotate our types with labels carrying termination proofs.
%%% ES: %
%%% ES: We contrast our approach, which \emph{allows} diverging computations,
%%% ES: but \emph{tracks} them as Haskell tracks side-effects, with
%%% ES: total functional programming, \eg in the dependently-typed setting,
%%% ES: where all computations are \emph{forced} to terminate.

%%% Local Variables: 
%%% mode: latex
%%% TeX-master: "main"
%%% End: 
