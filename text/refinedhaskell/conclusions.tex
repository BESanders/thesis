\section{Conclusions \& Alternative Approaches}\label{sec:refinedhaskell:conclusion}

Our goal is to use the recent advances in SMT solving to 
build automated refinement type-based verifiers for 
Haskell.
%
In this paper, we have made the following advances 
towards the goal. 
%
First, we demonstrated how the classical technique
for generating VCs from refinement subtyping queries
is unsound under lazy evaluation.
%
Second, we have presented a solution that addresses 
the unsoundness by stratifying types into those that 
are inhabited by terms that may diverge, those that must reduce 
to Haskell values, and those that must reduce to finite values, 
and have shown how refinement types may themselves 
be used to soundly verify the stratification. 
%
Third, we have developed an implementation of our 
technique in \toolname and have evaluated the tool 
on a large corpus comprising 10KLOC of widely used 
Haskell libraries. Our experiments empirically 
demonstrate the practical effectiveness of our
approach: using refinement types, we were able 
to prove 96\% of recursive functions as 
terminating, and to crucially use this information 
to prove a variety of functional correctness properties.

\mypara{Limitations}
While our approach is demonstrably effective 
\emph{in practice}, it relies critically on 
proving termination, which, while independently 
useful, is not wholly satisfying 
\emph{in theory}, as adding divergence shouldn't 
\emph{break} a safety proof.
%to quote~\cite{McMillanPersonal}: 
%\emph{``adding divergence shouldn't break your safety proofs.''}
%
%In our approach, we can prove a program safe, 
Our system can prove a program safe, 
but if the program is modified by making 
some functions non-deterministically diverge,
then, since we rely on termination, we
may no longer be able to prove safety.
%
Thus, in future work, it would be valuable to 
explore \emph{other} ways to reconcile laziness 
and refinement typing. We outline some routes 
and the challenging obstacles along them.


%% \mypara{1. Reject Inconsistent Environements}
%% We may be tempted to point the finger of blame at the
%% ``inconsistency'' itself. Unfortunately, this would be 
%% misguided -- inconsistencies are not a bug but a crucial 
%% feature of refinement type systems. 
%% %
%% They enable, among other things, \emph{path sensitivity} by
%% incorporating information from run-time tests (guards) and 
%% hence let us verify that expressions that throw catastrophic
%% exceptions (\eg @error e@) are indeed unreachable dead code 
%% and will not fail at run-time.
%% 
%% \mypara{2. CPS Transformation}
%% We might use a CPS transformation~\cite{PlotkinTCS75,WadlerICFP03} 
%% to convert the program into call-by-value.
%% We confess to be somewhat wary of the prospect of translating 
%% inferred types and errors \emph{back} to the source level after 
%% such a transformation.
%% Previous experience shows that the ability to map types and 
%% errors to source is critical for usability.

%% \mypara{3. Strictness Analysis}
%% We may want some form of \emph{strictness} or 
%% \emph{dependency analysis}~\cite{Mycroft80} to statically 
%% predict which binders must be evaluated, and only use
%% refinements for those binders in the environment. 
%% This route is problematic for many reasons. 
%% %
%% First, we aim to prove fine-grained functional correctness 
%% properties of programs. In addition to the well known limitations
%% of strictness analysis~\cite{HaskellWiki}, it 
%% is unclear how to develop an analysis that is sensitive
%% to the precise semantic (``path'') conditions under which 
%% the evaluation of different binders will be forced.
%% %
%% Second, and more importantly, it is often useful to add
%% \emph{ghost} values into the program for the sole purpose 
%% of making refinement types \emph{complete}~\cite{TerauchiPOPL13}. 
%% By construction, these values are not used by the program, 
%% and would be thrown away by a strictness analysis, thereby
%% precluding verification.

\mypara{A. Convert Lazy To Eager Evaluation}
%
One alternative might be to translate the program from lazy to eager evaluation,
for example, to replace every (thunk) $e$ with an abstraction $\efun{()}{}{e}$,
and every use of a lazy value $x$ with an application $x\ ()$. 
After this, we could simply assume eager evaluation, and so the usual refinement
type systems could be used to verify Haskell. Alas, no. 
While sound, this translation
doesn't solve the problem
of reasoning about divergence. 
%%While this translation
%%does soundly reject the @explode@ example, 
%%it doesn't solve the problem
%%of reasoning about divergence. 
A dependent function type
${\tfun{x}{\tint}{\tlref{\vv}{\tint}{}{\vv>\x}}}$
would be transformed to
${\tfun{x}{(\tfunbasic{()}{\tint})}
          {\tlref{\vv}{\tint}{}{\vv > \x\ ()}}}$.
%
%%%\begin{code}
%%%  f :: x:Int -> {v:Int | v > x}
%%%\end{code}
%%%%
%%%would be transformed to
%%%%
%%%\begin{code}
%%%  f :: x:(() -> Int) -> {v:Int | v > x ()}
%%%\end{code}
%
The transformed type is problematic as it uses 
arbitrary function applications in the refinement logic!
%
The type is only sensible if $x\ ()$ provably reduces to a value, 
bringing us back to square one.

%%% This is highly problematic, because now we have function 
%%% applications in the logic! Now it seems that "x" is a 
%%% pretty harmless function but not really, because we're 
%%% essentially back in the same world where we have to be 
%%% *sure* that "x ()" actually reduces to a value! 
%%% 
%%% That is to say, essentially, this is the same as the 
%%% "direct" approach of, 
%%% 
%%%    x:Int -> {y | (isvalue x) => y > x}
%%% 
%%% 
%%% That is, the simple refinement in the original CBN is converted to 
%%% 
%%% 
%%% %
%%% Unfortunately, this doesn't 
%%% 
%%% that require CBV evaluation
%%% 
%%%   That use call by name and a strict language.  Now all the old techniques should work. 
%%% 
%%% convert the program from lazy to ea
%% We might use a CPS transformation~\cite{PlotkinTCS75,WadlerICFP03} 
%% to convert the program into call-by-value.
%% We confess to be somewhat wary of the prospect of translating 
%% inferred types and errors \emph{back} to the source level after 
%% such a transformation.
%% Previous experience shows that the ability to map types and 
%% errors to source is critical for usability.

\mypara{B. Explicit Reasoning about Divergence}
%
Another alternative is to enrich the refinement logic
% It is not really clear that it is THE only 
% Thus, the only other alternative is to enrich the refinement logic
with a \emph{value predicate} $\isvalue{x}$ that is true when 
``$x$ is a value'' and use the SMT solver to 
\emph{explicitly} reason about divergence.
%
(Note that $\isvalue{x}$ is equivalent to introducing a 
$\ebot$ constant denoting divergence, and 
writing $(x \not =\ \ebot)$.)
%
Unfortunately, this $\isvalue{x}$ predicate takes the VCs 
outside the scope of the standard efficiently decidable logics 
supported by SMT solvers.
%
To see why, recall 
the subtyping query %(\ref{sub:good}) 
from @good@. % in \Sref{sec:refinedhaskell:overview}. 
With explicit value predicates, 
this subtyping reduces to the VC:
%
\begin{align}
{(\isvalue{x} \Rightarrow x \geq 0)}, \
{(\isvalue{y} \Rightarrow y \geq 0)} 
\Rightarrow
{(v = y+1)}   \Rightarrow {(v > 0)}\label{vc:good:explicit}
\end{align}
%
To prove the above valid, we require the knowledge 
that $(v = y+1)$ implies that $y$ is a value, \ie that 
$\isvalue{y}$ holds.
%
This fact, while obvious to a \emph{human} reader, is 
outside the decidable theories of linear arithmetic
of the existing SMT solvers.
%
Thus, existing solvers would be unable to prove (\ref{vc:good:explicit}) 
valid, causing us to reject @good@.

%%%%%%%%%%%%%%%%%%%%%%%%%%%%%%%%%%%%%%%%%%%%%%%%%%%%%%%%%%%%%%%%%%%%%%%%%%%%%%%
%%%%%%%%%%%%%%%%%%%%%%%%%%%%%%%%%%%%%%%%%%%%%%%%%%%%%%%%%%%%%%%%%%%%%%%%%%%%%%%
%%%%%%%%%%%%%%%%%%%%%%%%%%%%%%%%%%%%%%%%%%%%%%%%%%%%%%%%%%%%%%%%%%%%%%%%%%%%%%%

\mypara{Possible Fix: Explicit Reasoning With Axioms?}
%
One possible fix for the above would be to specify a collection of
\emph{axioms} that characterize how the value predicate behaves with 
respect to the other theory operators. 
%
For example, we might specify axioms like: 
%
\begin{align*}
\forall x,y,z. (x = y + z)\ &\Rightarrow\ (\isvalue{x} \wedge \isvalue{y} \wedge \isvalue{z})\\
% \forall x,y,z. (x = y - z)\ &\Rightarrow\ (\isvalue{x} \wedge \isvalue{y} \wedge \isvalue{z})\\\
\forall x,y. (x < y )\ &\Rightarrow\ (\isvalue{x} \wedge \isvalue{y})
% &\mathit{etc.}
\end{align*}
%
\etc. However, this is a non-solution for several reasons. 
First, it is not clear what a complete set of axioms is.
Second, there is the well known loss of predictable checking
that arises when using axioms, as one must rely on various 
brittle, syntactic matching and instantiation heuristics~\cite{simplifyj}. 
%
It is unclear how well these heuristics will work with the
sophisticated linear programming-based algorithms used to 
decide arithmetic theories. 
%
Thus, proper support for value predicates could require 
significant changes to existing decision procedures, 
making it impossible to use existing SMT solvers.


%%%%%%%%%%%%%%%%%%%%%%%%%%%%%%%%%%%%%%%%%%%%%%%%%%%%%%%%%%%%%%%%%%%%%%%%%%%%%%%
%%%%%%%%%%%%%%%%%%%%%%%%%%%%%%%%%%%%%%%%%%%%%%%%%%%%%%%%%%%%%%%%%%%%%%%%%%%%%%%
%%%%%%%%%%%%%%%%%%%%%%%%%%%%%%%%%%%%%%%%%%%%%%%%%%%%%%%%%%%%%%%%%%%%%%%%%%%%%%%

\mypara{Possible Fix: Explicit Reasoning With Types?}
%
Another possible fix would be to encode the behavior of the
value predicates within the refinement types for different 
operators, after which the predicate itself could be treated 
as an \emph{uninterpreted function} in the refinement 
logic~\cite{bradleybook}. For instance, we could type 
the primitives:
%
\begin{code}
 (+) :: x:Int -> y:Int -> {v | v  =  x + y && Val x && Val y}
 (<) :: x:Int -> y:Int -> {v | v <=> x < y && Val x && Val y}
\end{code}
%
While this approach requires \emph{no} changes to the SMT 
machinery, it makes specifications complex and verbose. 
%
%% (and not unlike having to sprinkle explicit ``non-null'' 
%% checks all over pointer manipulating programs!)
%
We cannot just add the value predicates to the primitives' 
specifications. Consider 
%
\begin{code}
 choose b x y = if b then x+1 else y+2
\end{code}
%
To reason about the output of @choose@ we must type it as:
%
\begin{code}
 choose :: Bool -> x:Int -> y:Int 
        -> {v|(v > x && Val x)||(v > y && Val y)}  
\end{code}
%
Thus, the value predicates will pervasively clutter all 
signatures with strictness information, making the system 
unpleasant to use.

%%%%%%%%%%%%%%%%%%%%%%%%%%%%%%%%%%%%%%%%%%%%%%%%%%%%%%%%%%%%%%%%%%%%%
%%%%%%%%%%%%%%%%%%%%%%%%%%%%%%%%%%%%%%%%%%%%%%%%%%%%%%%%%%%%%%%%%%%%%
%%%%%%%%%%%%%%%%%%%%%%%%%%%%%%%%%%%%%%%%%%%%%%%%%%%%%%%%%%%%%%%%%%%%%

\mypara{Divergence Requires 3-Valued Logic}
Finally, for either ``fix'', the value predicate poses a 
model-theoretic problem: 
what is the meaning of $\isvalue{x}$? 
%what meaning do we give $\isvalue{x}$? 
%
One sensible approach is to extend the universe with a family of 
\emph{distinct} $\bot$ constants, such that $\isvalue{\bot}$ is false.
%
These constants lead inevitably into a three-valued logic 
(in order to give meaning to formulas like $\bot = \bot$).
%
Thus, even if we were to find a way to reason with the value 
predicate via axioms or types, we would have to ensure that 
we properly handled the 3-valued logic within 
existing 2-valued SMT solvers.

\mypara{Future Work}
Thus, in future work it would be worthwhile to address the above 
technical and usability problems to enable explicit reasoning with 
the value predicate.
%
This explicit system would be \emph{more expressive} than our 
stratified approach, \eg would let us check 
%
\begin{code}
  let x = collatz 10 in 12 `div` x + 1
\end{code}
%
by encoding strictness inside the logic. Nevertheless, we suspect
such a verifier would use stratification to eliminate the value
predicate in the common case.
%
At any rate, until these hurdles are crossed, we can take comfort in
stratified refinement types and can just \emph{eagerly}
use termination to prove safety for \emph{lazy} languages.

%% If we could address the above problems
%% Thus, at this point, even though the natural route is to reason explicitly
%% with value predicates, as they would address the theoretical
%% robustness-to-divergence limitation described above, 
%% it is unclear how the above problems can be solved, 
%% and we believe they may be promising directions for 
%% future work.
%
%
%
%% Thus, at this point, even though the natural route is to reason explicitly
%% with value predicates, as they would address the theoretical
%% robustness-to-divergence limitation described above, 
%% it is unclear how the above problems can be solved, 
%% and we believe they may be promising directions for 
%% future work.
%% 
%% %
%% Of course, that does not mean they are unsolvable, just 
%% that the presence of value predicates means we cannot 
%% use \emph{existing}, off-the-shelf SMT solvers to 
%% achieve our goal of a sound and predictable 
%% refinement type checker for Haskell.


%%% Local Variables: 
%%% mode: latex
%%% TeX-master: "main"
%%% End: 
