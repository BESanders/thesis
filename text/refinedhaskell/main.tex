%\documentclass{llncs}
\documentclass{sigplanconf}
%%% \usepackage[top=1.5cm,bottom=1.5cm,left=2.3cm,right=2cm]{geometry}

\pagestyle{plain}
\usepackage{times}
%\usepackage[nocompress]{cite} % AR: comment this out if you wish hyperrefs 
\usepackage{hyperref}

\usepackage{amsmath,amssymb, latexsym}

\usepackage{amsmath}
\usepackage{amsthm}

\usepackage{comment}
\usepackage{flushend}

\newtheorem{lemma}{Lemma}
\newtheorem{corollary}{Corollary}
\newtheorem{theorem}{Theorem}
\newtheorem*{theorem*}{Theorem}
\newtheorem{definition}{Definition}
\newtheorem*{lemma*}{Lemma}

\usepackage{amsfonts}
\usepackage{amssymb}
\usepackage{mathtools}
\usepackage{commands}
\usepackage{liquidHaskell}
\usepackage[inference]{semantic}

\usepackage{enumerate}
%\def\url{}
\usepackage{xspace}
\usepackage{epsfig}
\usepackage{booktabs}
\usepackage{listings}
\usepackage{comment}
\usepackage{ifthen}

\newcommand{\isTechReport}{false} % true or false
\newcommand\includeProof[1]{%
  \ifthenelse{\equal{\isTechReport}{true}}
    {{#1}}
    {\ignorespaces}
\xspace}
\newcommand\includeBytestring[1]{}

%%%%%\usepackage{fancyvrb}
%%%%%\DefineVerbatimEnvironment{code}{Verbatim}{fontsize=\small}
%%%%%\DefineVerbatimEnvironment{example}{Verbatim}{fontsize=\small}
%%%%%\newcommand{\ignore}[1]{}

% command to end a proof or definition:
%\def\qed{\rule{0.4em}{1.4ex}}
\def\qed{\hfill$\Box$}

% space at the beginning of an environment:
\def\@envspa{\hspace{0.3em}}
\def\@sa{\hspace{-0.2em}}
\def\@sb{\hspace{0.5em}}
\def\@sc{\hspace{-0.1em}}

\def\sk{\smallskip}		% space before and after theorems

\newtheorem{notation}{Notation}{\itshape}{}
\newtheorem{invariant}{Invariant}
\newtheorem*{hypothesis}{Hypothesis}
\newtheorem{technical}{}
\usepackage{thmtools}

\declaretheoremstyle[%
  spaceabove=-6pt,%
  spacebelow=6pt,%
  headfont=\normalfont\itshape,%
  postheadspace=1em,%
  qed=\qedsymbol,%
  headpunct={}
]{mystyle} 
\declaretheorem[name={Proof Sketch:},style=mystyle,unnumbered,
]{proofsketch}


%\newcommand{\proofsketch}{\noindent {\bf Proof Sketch: }}
%\newcommand{\qed}{\hfill\rule{2mm}{2mm}}
\newcommand\proofsketchend{\hfill\ensuremath{\qedhere}}
\newcommand\liquidtech{
	\href{http://goto.ucsd.edu/~rjhala/liquid/liquid_types_techrep.pdf}
		 {liquid types technical report}
}

\usepackage{listings}

% uncomment next line to restore colors
% \def\withcolor{}

\ifdefined\withcolor
	\definecolor{haskellblue}{rgb}{0.0, 0.0, 1.0}
	\definecolor{haskellblue}{rgb}{1.0, 0.0, 0.0}
	\definecolor{gray_ulisses}{gray}{0.55}
	\definecolor{castanho_ulisses}{rgb}{0.71,0.33,0.14}
	\definecolor{preto_ulisses}{rgb}{0.41,0.20,0.04}
	\definecolor{green_ulisses}{rgb}{0.0,0.4,0.0}
\else
	\definecolor{haskellblue}{gray}{0.1}
	\definecolor{haskellred}{gray}{0.1}
	\definecolor{gray_ulisses}{gray}{0.1}
	\definecolor{castanho_ulisses}{gray}{0.1}
	\definecolor{preto_ulisses}{gray}{0.1}
	\definecolor{green_ulisses}{gray}{0.1}
\fi


\def\codesize{\normalsize}
\newcommand\showfocus[1]{\color{purple}{\textbf{#1}}}

\lstdefinelanguage{HaskellUlisses} {
	basicstyle=\ttfamily\footnotesize,
	sensitive=true,
	morecomment=[l][\color{gray_ulisses}\ttfamily\codesize]{--},
	%% morecomment=[s][\color{gray_ulisses}\ttfamily\codesize]{\{-}{-\}},
	morestring=[b]",
	moredelim=[is][\showfocus]{\#}{\#},
	stringstyle=\color{haskellred},
	showstringspaces=false,
	numberstyle=\codesize,
	numberblanklines=true,
	showspaces=false,
	breaklines=true,
	showtabs=false,
    literate={
           {`}{{{$^{\backprime}{}$}}}1
           {'}{{{$^{\prime}{}$}}}1
           % {QED}{{{\color{lcolor}QED}}}3
           % {***}{{{\color{lcolor}***}}}3
           {?}{{{$\therefore$}}}1
           % {<}{{$<$}}1
           % {>}{{$>$}}1
           {<=}{{$\leq$}}2
           {>>=}{>>=}3
           {>=}{{$\geq$}}2
           {theta}{{$\theta$}}1
           {env}{{$\Gamma$}}1
           {|-}{{$\vdash$}}1
           {<=!}{{{\color{lcolor}<=!}}}3
           {!=}{{$\neq$}}2
           {forall}{{$\forall$}}1
           {->}{{$\rightarrow$}}2
           {=*}{{$\eqfun$}}2
           {<=>}{{$\Leftrightarrow$}}3
           {=>}{{$\Rightarrow$}}2
           {<:}{{$\preceq$}}1
           {mempty}{{$\mempty$}}1
           {mappend}{{$\mappend$}}1
           {<>}{{$\mappend$}}1
           {stringMempty}{{$\stringMempty$}}1
           {<+>}{{$\stringMappend$}}1
           {stringMappend}{{$\stringMappend$}}1
           {listMempty}{{[]}}1
           {listMappend}{{++}}2
           {epsilon}{{$\epsilon$}}1
           {eta}{{$\eta$}}1
           {&&&}{&&&}3
           {&&}{{$\land$}}1
           {_m}{{${}_m$}}1
           {_n}{{${}_n$}}1
           {m^+}{{m${}^{+}$}}2
           {SetMem}{{$\in$}}1
           {Set_cup}{{$\cup$}}1
           {Set_cap}{{$\cap$}}1
           {Set_emp}{{$\emptyset$}}1
           {Set_sub}{{$\subseteq$}}1
           },
	emph=
	{[1]
		FilePath,IOError,abs,acos,acosh,all,and,any,appendFile,approxRational,asTypeOf,asin,
		asinh,atan,atan2,atanh,basicIORun,break,catch,ceiling,chr,compare,concat,concatMap,
		const,cos,cosh,curry,cycle,decodeFloat,denominator,digitToInt,div,divMod,drop,
		dropWhile,either,elem,encodeFloat,enumFrom,enumFromThen,enumFromThenTo,enumFromTo,
		error,even,exp,exponent,fail,filter,flip,floatDigits,floatRadix,floatRange,floor,
		fmap,foldl,foldl1,foldr,foldr1,fromDouble,fromEnum,fromInt,fromInteger,
		fromRational,fst,gcd,getChar,getContents,getLine,head,id,inRange,index,init,intToDigit,
		interact,ioError,isAlpha,isAlphaNum,isAscii,isControl,isDenormalized,isDigit,isHexDigit,
		isIEEE,isInfinite,isLower,isNaN,isNegativeZero,isOctDigit,isPrint,isSpace,isUpper,iterate,
		last,lcm,length,lex,lexDigits,lexLitChar,lines,log,logBase,lookup,map,mapM,mapM_,max,
		maxBound,maximum,maybe,min,minBound,minimum,mod,negate,not,notElem,numerator,odd,
		or,pi,pred,primExitWith,print,product,properFraction,putChar,putStr,putStrLn,quot,
		quotRem,range,rangeSize,read,readDec,readFile,readFloat,readHex,readIO,readInt,readList,readLitChar,
		readLn,readOct,readParen,readSigned,reads,readsPrec,realToFrac,recip,rem,repeat,replicate,
		reverse,round,scaleFloat,scanl,scanl1,scanr,scanr1,seq,sequence,sequence_,show,showChar,showInt,
		showList,showLitChar,showParen,showSigned,showString,shows,showsPrec,significand,signum,sin,
		sinh,snd,span,splitAt,sqrt,subtract,succ,sum,tail,take,takeWhile,tan,tanh,threadToIOResult,toEnum,
		toInt,toInteger,toLower,toRational,toUpper,truncate,uncurry,undefined,unlines,until,unwords,unzip,
		unzip3,userError,words,writeFile,zip,zip3,zipWith,zipWith3,listArray,doParse,for,initTo,
        maxEvens,create,get,set,initialize,idVec,fastFib,fibMemo,
        insert,union,split,size,fromList,initUpto,trim,quickSort,insertSort,append,upperCase,
        copy, group, doDownLoop, mapAccumR, peekByteOff,
        pokeByteOff,spanByte, 
        good, bad, foo, explode, 
        fib, ack, 
        tLen,
        memcpy,writeChar,unsafeWrite,unsafeFreeze,
        singleton
	},
	emphstyle={[1]\color{haskellblue}},
	emph=
	{[2]
		Bool,Char,Double,Either,Float,IO,Integer,Int,Maybe,Ordering,Rational,Ratio,ReadS,ShowS,String,
		Word8,Nat,NonZero,Nat64,Text,ByteString,ByteStringSZ,ByteStringN,
        Ptr,ForeignPtr,CSize
        InPacket,Tree,Prop,TreeEq,TreeLt,Vec,
        NullTerm,IncrList,DecrList,UniqList,BST,MinHeap,MaxHeap,
        PtrN,ByteStringN,ByteStringEq,VO,ByteStringsEq,ByteStringNE
	},
	emphstyle={[2]\color{castanho_ulisses}},
	emph=
	{[3]
		case,class,data,deriving,do,else,if,return,def,import,in,infixl,infixr,instance,let,
		module,measure,predicate,of,primitive,then,refinement,type,where,lazy, type, bound
	},
	emphstyle={[3]\color{preto_ulisses}\textbf},
	emph=
	{[4]
		quot,rem,div,mod,elem,notElem,seq
	},
	emphstyle={[4]\color{castanho_ulisses}\textbf},
	emph=
	{[5]
		PS,Tip,Node,EQ,False,GT,Just,LT,Left,Nothing,Right,True,Show,Eq,Ord,Num
	},
	emphstyle={[5]\color{green_ulisses}}
}

%%%ORIG
%%%\lstnewenvironment{code}
%%%{\textbf{Haskell Code} \hspace{1cm} \hrulefill \lstset{language=HaskellUlisses}}
%%%{\hrule\smallskip}

%V1
%\lstnewenvironment{code}
%{\smallskip \lstset{language=HaskellUlisses}}
%{\smallskip}

\lstnewenvironment{code}
{\lstset{language=HaskellUlisses}}
{}

\lstnewenvironment{mcode}
{\lstset{language=HaskellUlisses,columns=fullflexible,keepspaces,mathescape}}
{}


\lstMakeShortInline[language=HaskellUlisses,mathescape,keepspaces,mathescape,basicstyle=\ttfamily\normalsize,breakatwhitespace]@


%\lstMakeShortInline[language=HaskellUlisses,basicstyle=\ttfamily\normalsize,breakatwhitespace]@

\begin{document}

\title{
	\ifthenelse{\equal{\isTechReport}{true}}
    {{Technical Report:}}
    {}
    Refinement Types For Haskell
    \thanks{This work was supported by NSF grants 
      CNS-0964702, CNS-1223850, CCF-1218344, CCF-1018672,
      and a generous gift from Microsoft Research.
    }
  } 
\newcommand\showproof[1]{\texttt{proved}}
\newcommand\showproofsketch[1]{#1}%{\texttt{proof sketch}}
\newcommand\showprooftodo[1]{\texttt{TODO}}



\authorinfo{Niki Vazou \and Eric L. Seidel \and Ranjit Jhala}{UC San Diego}{}
\authorinfo{Dimitrios Vytiniotis \and Simon Peyton-Jones}{Microsoft Research}{}

\maketitle

%%\NV{Fonts:
%%SPJ:I would prefer to use math-italic for all non-terminals
%%Particular terminal symbols like Int, Bool, and keywords like 'measure' can be typewriter font
%%Types can be \tau instead of t}


\conferenceinfo{ICFP~'14}{September 1--6, 2014, Gothenburg, Sweden} 
\copyrightyear{2014} 
\copyrightdata{978-1-4503-2873-9/14/09} 
%% \doi{nnnnnnn.nnnnnnn} 


We present \emph{abstract refinement types} which enable 
quantification over the refinements of data- and 
function-types. Our key insight is that we 
can avail of quantification while preserving SMT-based 
decidability, simply by encoding refinement parameters
as \emph{uninterpreted} propositions within the 
refinement logic.
%
We illustrate how this mechanism yields a variety 
of sophisticated means for reasoning about programs, including:
\emph{parametric} refinements for reasoning with 
type classes,
\emph{index-dependent} refinements for reasoning about 
key-value maps,
\emph{recursive} refinements for reasoning about 
recursive data types, and
\emph{inductive} refinements for reasoning about 
higher-order traversal routines.
%
We have implemented our approach in a refinement
type checker for Haskell, and present experiments using our tool
to verify correctness invariants of various programs.
%including some \textsc{GHC} libraries.

\section{Introduction}\label{sec:introduction}

Refinement types enable specification of complex invariants 
by extending the base type system with \emph{refinement predicates} 
drawn from decidable logics. For example,
%
\begin{code}
  type Nat = {v:Int | 0 <= v}
  type Pos = {v:Int | 0 <  v}
\end{code}
%
are refinements of the basic type @Int@ with a logical predicate 
that states the \emph{values} @v@ being described must be 
\emph{non-negative} and \emph{postive} respectively. 
%
We can specify \emph{contracts} of functions by refining function types. 
For example, the contract for @div@
%
\begin{code}
  div :: n:Nat -> d:Pos -> {v:Nat | v <= n}
\end{code}
%
states that @div@ \emph{requires} a non-negative dividend @n@ and a positive
divisor @d@, and \emph{ensures} that the result is less than the dividend.
%
If a program (refinement) type checks, we can be sure that @div@ will never 
throw a divide-by-zero exception.

What are refinement types good for?
%
While there are several papers describing the \emph{theory} behind  
refinement types 
~\cite{Zenger97,pfenningxi98,ORS92,flanagan06,GordonTOPLAS2011,fstar,LiquidPLDI08}, 
even for \toolname~\cite{LiquidICFP14}, there is rather less 
literature on how the approach can be \emph{applied} to large, real-world
codes. In particular, we try to answer the following questions:
%
\begin{enumerate}
  \item What properties can be specified with refinement types?
  \item What inputs are provided and what feedback is received?
  \item What is the process for modularly verifying a library?
  \item What are the limitations of refinement types? 
\end{enumerate}

In this paper, we attempt to investigate these questions, by using the
refinement type checker \toolname, to specify and verify a variety of 
properties of over 10,000 lines of Haskell code from various popular 
libraries, including @containers@, \hbox{@hscolor@,} @bytestring@, @text@, 
@vector-algorithms@ and @xmonad@. 
%
First (\S~\ref{sec:liquidhaskell}), 
we present a high-level overview of \toolname, through a tour 
of its features.
%
Second, we present a qualitative discussion of the kinds of properties
that can be checked -- ranging from generic application independent 
criteria like totality (\S~\ref{sec:totality}), 
\ie that a function is defined for all inputs (of a given type),  
and termination, 
(\S~\ref{sec:termination}) 
\ie that a recursive function cannot diverge,
to application specific concerns like memory safety (\S~\ref{sec:memory-safety}) 
and functional correctness properties (\S~\ref{sec:structures}).
%
Finally (\S~\ref{sec:evaluation}), we present a quantitative evaluation of the approach, with a view
towards measuring the efficiency and programmer's effort required for
verification, 
and we discuss various limitations of the approach which could
provide avenues for further work.


%%% Local Variables: 
%%% mode: latex
%%% TeX-master: "main"
%%% End: 

\section{Overview}\label{sec:boundedrefinements:overview}

We start with a high level overview of bounded refinement types.
%
We first present a short introduction to refinement type specifications, 
to make this chapter self contained.
%
Then, we introduce bounded refinements,
and show how they permit \emph{modular} higher-order specifications.
%
Finally, we describe how they are implemented via an elaboration
process that permits \emph{automatic} first-order verification.

\subsection{Preliminaries}

\mypara{Refinement Types} let us precisely specify subsets of values,
by conjoining base types with logical predicates that constrain the values.
We get decidability of type checking, by limiting these predicates to
decidable, quantifier-free, first-order logics, including the theory
of linear arithmetic, uninterpreted functions, arrays, bit-vectors
and so on. 
%
For example, the refinement types
%
\begin{code}
  type Pos     = {v:Int | 0 < v}
  type IntGE x = {v:Int | x <= v}
\end{code}
%
specify subsets of @Int@ corresponding to values
that are positive or larger than some other value @x@
respectively. 
%
We can use refinement types to specify contracts
like pre- and post-conditions by suitably refining the input
and output types of functions.

\mypara{Preconditions} are specified by refining input types.
We specify that the function @assert@ must
\emph{only} be called with @True@,
%
where the type @TRUE@ contains only the singleton @True@:
%
\begin{code}
  type TRUE = {v:Bool | v <=> True}

  assert         :: TRUE -> a -> a
  assert True x  = x
  assert False _ = error "Provably Dead Code"
\end{code}

\mypara{We can specify post-conditions} by refining output types.
For example, a primitive @Int@ comparison operator @leq@ can be
assigned a type that says that the output is @True@ iff the
first input is actually less than or equal to the second:
%
\begin{mcode}
  leq :: x:Int -> y:Int -> {v:Bool | v  <=> x <= y}
\end{mcode}

\mypara{Refinement Type Checking} proceeds by checking that at each
application, the types of the actual arguments are \emph{subtypes}
of those of the function inputs, in the environment (or context) in
which the call occurs.
%
Consider the function:
%
\begin{code}
  checkGE     :: a:Int -> b:IntGE a -> Int
  checkGE a b = assert cmp b
    where cmp = a `leq` b
\end{code}
%
To verify the call to @assert@ we check that
the actual parameter @cmp@ is a subtype of @TRUE@,
under the assumptions given by the input types for
@a@ and @b@.
%
Via subtyping~\cite{LiquidICFP14} the check reduces to establishing
the validity of the \emph{verification condition}~(VC)
%
\begin{code}
  a <= b => (cmp <=> a <= b) => v = cmp => (v<=>true)
\end{code}
%
The first antecedent comes from the input type of @b@, the
second from the type of @cmp@ obtained from the output of @leq@,
the third from the \emph{actual} input passed to @assert@,
and the goal comes from the input type \emph{required} by @assert@.
%
An SMT solver \cite{NelsonOppen} readily establishes the validity
of the above VC, thereby verifying @checkGE@.

\begin{comment}

% \subsection{Abstract Refinements}

\mypara{First order refinements prevent modular specifications.}
Consider the function that returns the largest element of a list:
%
\begin{code}
  maximum         :: List Int -> Int
  maximum [x]     = x
  maximum (x:xs)  = max x (maximum xs)
    where max a b = if a < b then b else a
\end{code}
%
How can one write a first-order refinement type specification for
@maximum@ that will let us verify the below code?
%
\begin{code}
  posMax :: List Pos -> Pos
  posMax = maximum
\end{code}
%
%   posMax :: List Neg -> Neg
%   posMax = maximum
%
Any suitable specification would have to enumerate the
situations under which @maximum@ may be invoked
breaking modularity.

\mypara{Abstract Refinements} overcome the above modularity
problems \cite{vazou13}.
%
The main idea is that we can type @maximum@ by observing
that it returns \emph{one of} the elements in its input list.
Thus, if every element of the list enjoys some refinement @p@
then the output value is also guaranteed to satisfy @p@.
%
Concretely, we can type the function as:
%
\begin{code}
maximum :: forall<p::Int->Bool>. List Int<p> -> Int<p>
\end{code}
%
where informally, @Int<p>@ stands for @{v:Int | p v}@,
and @p@ is an \emph{uninterpreted function} in the refinement
logic~\cite{NelsonOppen}.
%
The signature states that for any refinement @p@ on @Int@,
the input is a list of elements satisfying @p@
and returns as output an integer satisfying @p@.
%
In the sequel, we will drop the explicit quantification
of abstract refinements; all free abstract refinements
will be \emph{implicitly} quantified at the top-level
(as with classical type parameters.)

\paragraph{Abstract Refinements Preserve Decidability.}
Abstract refinements do not require the use of higher-order
logics. Instead, abstractly refined signatures (like @maximum@)
can be verified by viewing the abstract refinements @p@ as
uninterpreted functions that only satisfy the axioms of
congruence, namely:
%
\begin{code}
  forall x y. x = y => p x <=> p y
\end{code}
%
As the quantifier free theory of uninterpreted functions
is decidable \cite{NelsonOppen}, abstract refinement type
checking remains decidable \cite{vazou13}.

\paragraph{Abstract Refinements are Automatically Instantiated} at call-sites,
via the abstract interpretation framework of Liquid Typing~\cite{vazou13}.
Each instantiation yields fresh refinement variables on
which subtyping constraints are generated; these constraints
are solved via abstract interpretation yielding the instantiations.
%
Hence, we verify @posMax@ % and @negMax@
by instantiating:
%
\begin{code}
  p |-> \ v -> 0 < v   -- at posMax
\end{code}
  % p |-> \ v -> v < 0   -- at negMax

\end{comment}

\subsection{Bounded Refinements}

Refinement types hit various expressiveness walls, 
as for decidability, refinements are constraint to 
first order, decidable logics.
%
Consider the following example
from~\cite{TerauchiPOPL13}.
%
@find@ takes as input a predicate @q@, a continuation
@k@ and a starting number @i@; it proceeds to compute
the smallest @Int@ (larger than @i@) that satisfies
@q@, and calls @k@ with that value.
%
@ex1@ passes @find@ a continuation that checks that the
``found'' value equals or exceeds @n@.
%
\begin{code}
  ex1 :: (Int -> Bool) -> Int -> ()
  ex1 q n = find q (checkGE n) n

  find q k i
    | q i       = k i
    | otherwise = find q k (i + 1)
\end{code}

\mypara{Verification fails} as there is no way to specify that
@k@ is only called with arguments greater than @n@.
%
First, the variable @n@ is not in scope at the function
definition so we cannot refer to it.
%
Second, we could try to say that @k@ is invoked with values
greater than or equal to @i@, which gets substituted with @n@
at the call-site. Alas, due to the currying order, @i@ too is
not in scope at the point where @k@'s type is defined so
the type for @k@ cannot depend upon @i@.

\mypara{Can Abstract Refinements Help?} Lets try to
use Abstract Refinements, from chapter~\ref{chapter:abstractrefinements},
to abstract over the refinement that @i@ enjoys, and
assign @find@ the type:
%
\begin{code}
  find :: (Int -> Bool) -> (Int<p> -> a) -> Int<p> -> a
\end{code}
%
which states that for any refinement @p@, the function takes
an input @i@ which satisfies @p@ and hence that the continuation
is also only invoked on a value which trivially enjoys @p@, namely @i@.
%
At the call-site in @ex1@ we can instantiate
\begin{equation}
\cc{p} \mapsto \lambda \cc{v} \rightarrow \cc{n} \leq \cc{v} \label{eq:inst:find}
\end{equation}
%
This instantiated refinement is satisfied by the parameter @n@ and is
sufficient to verify, via function subtyping, that @checkGE n@ will
only be called with values satisfying @p@, and hence larger than @n@.

\mypara{The function find is ill-typed} as the signature requires that
at the recursive call site, the value @i+1@ \emph{also}
satisfies the abstract refinement @p@.
%
While this holds for the example we have in mind~(\ref{eq:inst:find}),
it does not hold \emph{for all} @p@, as required by the type of @find@!
%
Concretely, @{v:Int | v = i + 1}@ is in general \emph{not} a subtype of
@Int<p>@, as the associated VC
% Concretely, the recursive call generates the VC
%
\begin{equation}
    ... \Rightarrow \cc{p i} \Rightarrow \cc{p (i+1)} \label{eq:vc:find}
\end{equation}
%
%\begin{equation}
%p\ i \Rightarrow p\ (i + 1) \label{eq:vc:find}
%\end{equation}
%
is \emph{invalid} -- the type checker thus (soundly!) rejects @find@.

\mypara{We must Bound the Quantification} of @p@ to limit
it to refinements satisfying some constraint, in this case
that @p@ is \emph{upward closed}. In the dependent setting,
where refinements may refer to program values, bounds
are naturally expressed as constraints between refinements.
% Horn clauses over refinements.
%
We define a bound, @UpClosed@
%
which states that @p@ is a refinement that is \emph{upward closed},
\ie satisfies @forall x. p x =>  p (x+1)@,
and use it to type @find@ as:
%
\begin{code}
  bound UpClosed (p :: Int -> Bool)
    = \x -> p x => p (x+1)

  find :: (UpClosed p) => (Int -> Bool)
                       -> (Int<p> -> a)
                       ->  Int<p> -> a
\end{code}
%
This time, the checker is able to use the bound to
verify the VC~(\ref{eq:vc:find}).
%
We do so in a way that refinements (and thus VCs) remain quantifier
free and hence, SMT decidable~(\S~\ref{sec:overview:implementation}).

\mypara{At the call} to @find@ in the body of @ex1@, we perform
the instantiation~(\ref{eq:inst:find}) which generates the
\emph{additional} VC
%
\hbox{@n <=  x => n <=  x+1@}
%
by plugging in the concrete refinements to the bound constraint.
%
The SMT checks the validity of the VC
and hence this instantiation, thereby statically
verifying @ex1@, \ie that the assertion inside
@checkGE@ cannot fail.
%

\subsection{Bounds for Higher-Order Functions}

Next, we show how bounds expand the scope of refinement typing by
letting us write precise modular specifications for various canonical
higher-order functions.

\subsubsection{Function Composition}\label{sec:compose}

First, consider @compose@. What is a modular specification
for @compose@ that would let us verify that @ex2@ enjoys
the given specification?
%
\begin{code}
  compose f g x = f (g x)

  type Plus x y = {v:Int | v = x + y}
  
  ex2    :: n:Int -> Plus n 2
  ex2    = incr `compose` incr

  incr   :: n:Int -> Plus n 1
  incr n = n + 1
\end{code}

\mypara{The challenge is to chain the dependencies} between the
input and output of @g@ and the input and output of @f@ to
obtain a relationship between the input and output of the
composition. We can capture the notion of chaining in a bound:
%
%% f -> p
%% g -> q
\begin{code}
  bound Chain p q r = \x y z ->
        q x y => p y z => r x z
\end{code}
%
which states that for any @x@, @y@ and @z@, if
%
(1) @x@ and @y@ are related by @q@, and
(2) @y@ and @z@ are related by @p@, then
(3) @x@ and @z@ are related by @r@.

We use @Chain@ to type @compose@ using three abstract
refinements @p@, @q@ and @r@, relating the arguments
and return values of @f@ and @g@ to their composed value.
%
(Here, @c<r x>@ abbreviates @{v:c | r x v}@).

\begin{code}
  compose :: (Chain p q r) => (y:b -> c<p y>)
                           -> (x:a -> b<q x>)
                           -> (w:a -> c<r w>)
\end{code}

\mypara{To verify} @ex2@ we instantiate, at the call to @compose@,
%
\begin{code}
  p, q |-> \x v -> v = x + 1
     r |-> \x v -> v = x + 2
\end{code}
%
The above instantiation satisfies the bound, as shown by the validity
of the VC derived from instantiating @p@, @q@, and @r@ in @Chain@:
%
\begin{code}
  y = x + 1 => z = y + 1 => z == x + 2
\end{code}
%
and hence, we can check that @ex2@ implements its specified type.


\subsubsection{List Filtering}

Next, consider the list @filter@ function.
%
What type signature for @filter@ would let us check @positives@?
\begin{code}
  filter q (x:xs)
    | q x         = x : filter q xs
    | otherwise   = filter q xs
  filter _ []     = []

  positives       :: [Int] -> [Pos]
  positives       = filter isPos
    where isPos x = 0 < x
\end{code}
%
Such a signature would have to relate the @Bool@ returned by
@f@ with the property of the @x@ that it checks for.
%
Typed Racket's latent predicates~\cite{typedracket}
account for this idiom, but are a special construct
limited to @Bool@-valued ``type'' tests, and not
arbitrary invariants.
%
Another approach is to avoid the so-called
``Boolean Blindness'' that accompanies
@filter@ by instead using option types
and @mapMaybe@.

\mypara{We overcome blindness using a witness} bound:
%
\begin{code}
  bound Witness p w = \x b -> b => w x b => p x
\end{code}
%
which says that @w@ \emph{witnesses} the
refinement @p@. That is, for any boolean @b@ such
that @w x b@ holds, if @b@ is @True@ then @p x@ also holds.

\mypara{We can give} @filter@ a type that says that the output values
enjoy a refinement @p@ as long as the test predicate @q@ returns
a boolean witnessing @p@:
%
\begin{code}
  filter :: (Witness p w) => (x:a -> Bool<w x>)
                          -> List a
                          -> List a<p>
\end{code}

\mypara{To verify} @positives@ we infer the following type and
instantiations for the abstract refinements @p@ and @w@ at the
call to @filter@:
%
\begin{code}
  isPos :: x:Int -> {v:Bool | v <=> 0 < x}
  p     |-> \v    -> 0 < v
  w     |-> \x b  -> b <=> 0 < x
\end{code}

\subsubsection{List Folding}

Next, consider the list fold-right function. Suppose we
wish to prove the following type for @ex3@:
%
\begin{code}
  foldr :: (a -> b -> b) -> b -> List a -> b
  foldr op b []     = b
  foldr op b (x:xs) = x `op` foldr op b xs

  ex3 :: xs:List a -> {v:Int | v == len xs}
  ex3 = foldr (\_ -> incr) 0
\end{code}
%
where @len@ is a \emph{logical} or \emph{measure}
function used to represent the number of elements of
the list in the refinement logic~\ref{subsec:measures}:
%
\begin{code}
  measure len :: List a -> Nat
  len []      = 0
  len (x:xs)  = 1 + len xs
\end{code}

\mypara{We specify induction as a bound.} Let
(1)~@inv@ be an abstract refinement relating a list @xs@
    and the result @b@ obtained by folding over it and
(2)~@step@ be an abstract refinement relating the
    inputs @x@, @b@ and output @b'@ passed to and
    obtained from the accumulator @op@ respectively.
%
We state that @inv@ is closed under @step@ as:
%
\begin{code}
  bound Inductive inv step = \x xs b b' ->
        inv xs b => step x b b' => inv (x:xs) b'
\end{code}
%

\mypara{We can give} @foldr@ a type that says that the
function \emph{outputs} a value that is built inductively
over the entire \emph{input} list:
%
\begin{code}
  foldr :: (Inductive inv step)
        => (x:a -> acc:b -> b<step x acc>)
        -> b<inv []>
        -> xs:List a
        -> b<inv xs>
\end{code}
%
That is, for any invariant @inv@ that is inductive
under @step@, if the initial value @b@ is @inv@-related
to the empty list, then the folded output is @inv@-related
to the input list @xs@.

\mypara{We verify} @ex3@ by inferring, at the call to @foldr@
%
\begin{code}
  inv  |-> \xs v   -> v  == len xs
  step |-> \x b b' -> b' == b + 1
\end{code}
%
The SMT solver validates the VC obtained by plugging the
above into the bound.
%
Instantiating the signature for @foldr@ yields precisely the
output type desired for @ex3@.

%% \nv{addition:}
Previously,~\ref{chapter:abstractrefinements} describes a way to type @foldr@
using abstract refinements that required the operator @op@
to have one extra ghost argument.
Bounds let us express induction without ghost arguments.

\subsection{Implementation}\label{sec:overview:implementation}

To implement bounded refinement typing, we must solve two
problems. Namely, how do we
%
(1)~\emph{check} and
%
(2)~\emph{use}
%
functions with bounded signatures?
%
We solve both problems via an insight inspired
by the way typeclasses are implemented in Haskell.
%
\begin{enumerate}
%
\item \emphbf{A Bound Specifies} a function type
whose inputs are unconstrained and whose output is
some value that carries the refinement corresponding
to the bound's body.
%
\item \emphbf{A Bound is Implemented} by a ghost
function that returns @True@, but is defined
in a context where the bound's constraint holds when
instantiated to the concrete refinements at the context.
%
\end{enumerate}

\mypara{We elaborate bounds into ghost functions} satisfying
the bound's type.
%
To \emph{check} bounded functions, we need to
\emph{call} the ghost function to materialize the
bound constraint at particular values of interest.
%
Dually, to \emph{use} bounded functions, we need to
\emph{create} ghost functions whose outputs are
guaranteed to satisfy the bound constraint.
%
This elaboration reduces \emph{bounded} refinement
typing to the simpler problem
of \emph{unbounded} abstract refinement typing.
%
The formalization of our elaboration is described in
\S~\ref{sec:check}.
%
Next, we illustrate the elaboration by explaining how
it addresses the problems of checking and using bounded
signatures like @compose@.

\mypara{We Translate Bounds into Function Types} called the
bound-type where the inputs are unconstrained, and the
outputs satisfy the bound's constraint.
%
For example, the bound @Chain@ used to type @compose@ in
\S~\ref{sec:compose}, corresponds to a function type, yielding
the translated type for @compose@:
%
\begin{code}
  type ChainTy p q r
    =  x:a -> y:b -> z:c -> {v:Bool | q x y => p y z => r x z}

  compose :: (ChainTy p q r) 
          -> (y:b -> c<p y>)
          -> (x:a -> b<q x>)
          -> (w:a -> c<r w>)
\end{code}

\mypara{To Check Bounded Functions} we view the bound constraints
as extra (ghost) function parameters (cf. type class dictionaries),
that satisfy the bound-type. Crucially, each expression where a
subtyping constraint would be generated (by plain refinement typing)
is wrapped with a ``call'' to the ghost to materialize the constraint
at values of interest. For example we elaborate @compose@ into:
%
\begin{code}
  compose dollarchain f g x =
    let t1 = g x
        t2 = f t1
        _  = dollarchain x t1 t2   -- materialize
    in  t2
\end{code}
%
In the elaborated version @dollarchain@ is the ghost parameter %
corresponding to the bound. As is standard \cite{LiquidPLDI08},
we perform ANF-conversion to name intermediate values, and then
wrap the function output with a call to the ghost to materialize
the bound's constraint. Consequently, the output of compose, namely
@t2@, is checked to be a subtype of the specified output type,
in an environment \emph{strengthened} with the bound's constraint
instantiated at @x@, @t1@ and @t2@. This subtyping reduces to a
quantifier free VC:
%
\begin{code}
      q x t1
  =>  p t1 t2
  => (q x t1 => p t1 t2 => r x t2)
  =>  v = t2 => r x v
\end{code}
%
whose first two antecedents are due to the types of @t1@ and @t2@
(via the output types of @g@ and @f@ respectively) and the third
comes from the call to @dollarchain@.
%
The output value @v@ has the singleton refinement that
states it equals to @t2@ and finally the VC states that the
output value @v@ must be related to the input @x@ via @r@.
%
An SMT solver validates this decidable VC easily, thereby
verifying @compose@.

Our elaboration inserts materialization calls \emph{for all}
variables (of the appropriate type) that are in scope at the
given point. This could introduce upto $n^k$ calls where $k$
is the number of parameters in the bound and $n$ the number
of variables in scope. In practice (\eg in @compose@) this
number is small (\eg 1) since we limit ourselves to variables
of the appropriate types.

%% Colin's comment: how about strictness and termination?
To preserve semantics we ensure that none of these materialization
calls can diverge, by carefully constraining the structure of
the arguments that instantiate the ghost functional parameters.

\mypara{At Uses of Bounded Functions} our elaboration uses
the bound-type to create lambdas with appropriate parameters
that just return @true@. For example, @ex2@ is elaborated to:
%
\begin{code}
  ex2 = compose (\_ _ _ -> true) incr incr
\end{code}
%
This elaboration seems too na\"ive to be true: how do we
ensure that the function actually satisfies the bound type?

Happily, that is automatically taken care of by function subtyping.
%
Recalling the translated type for @compose@, the elaborated lambda
@(\_ _ _ ->  true)@ is constrained to be a subtype of @ChainTy p q r@.
%
In particular, given the call site instantiation
%
\begin{mcode}
  p $\mapsto$ \ y z -> z == y + 1
  q $\mapsto$ \ x y -> y == x + 1
  r $\mapsto$ \ x z -> z == x + 2
\end{mcode}
%
this subtyping constraint reduces to the quantifier-free VC:
%
\begin{align*}
\inter{\Gamma}
  \Rightarrow \mathtt{true}
  \Rightarrow \cc{(z == y + 1)}
   \Rightarrow \cc{(y == x + 1)}\notag
   \Rightarrow \cc{(z == x + 2)} % \label{vc:bounded:ex2}
\end{align*}
%
%
where $\Gamma$ contains assumptions about the various binders in
scope.
%
The above VC is easily proved valid by an SMT solver, thereby
verifying the subtyping obligation defined by the bound, and hence,
that @ex2@ satisfies the given type.

\newcommand\hnull{\ensuremath{\text{[]}}\xspace}

\subsection{Measures: From Integers to Data Types}\label{sec:measures}

\begin{figure}
\centering
\captionsetup{justification=centering}
$$
\begin{array}{lrcl}
{\emphbf{Definition}} &
  \mathit{def} & ::=  &  \mathtt{measure} \ f :: \tau \\
              & &      &  \quad eq_1 \ldots eq_n       \\[0.05in]

{\emphbf{Equation}}   & 
  \mathit{eq}  & ::=  &   f\ (D\ \overline{x}) = r    \\[0.15in] 

{\emphbf{Equation to Type}} &
\quad \embed{f\ (D\ \overline{x}) = r} & \defeq & D :: \overline{\tbind{x}{\tau}} \rightarrow \tref{\mathtt{v}}{\tau}{}{f\ \mathtt{v} = r}
\end{array}
$$
\caption{Syntax of Measures.}
\label{fig:measures}
\end{figure}



So far, all our examples have used only integer and boolean expressions in refinements.
To describe properties of algebraic data types, we use \emph{measures},
introduced in prior work on Liquid Types~\cite{LiquidPLDI09}.
%
Measures are inductively defined functions that can be used in refinements and
provide an efficient way to axiomatize properties of data types.
%
For example, @emp@ determines whether a list is empty:
%
\begin{code}
  measure emp  :: [Int] -> Bool
    emp []     = true
    emp (x:xs) = false
\end{code}
The syntax for measures deliberately looks like Haskell, but it is \emph{far} more
restricted, and should really be considered as a separate language.
A measure has exactly one argument and is defined by a list of equations,
each of which has a simple pattern on the left hand side (Figure~\ref{fig:measures}).
The right-hand side of the equation is a refinement expression $r$.
Measure definitions are typechecked in the usual way; we omit the typing rules which are standard.
(Our metatheory does not support type polymorphism,
so here we simply reason about lists of integers;
however, our implementation supports polymorphism.)

\paragraph{Denotational semantics}
The denotational semantics of types in \hlang in \Sref{sec:den-sem} is readily extended to
support measures.  In \hlang a refinement $r$ is an arbitrary expression and
calls to a measure are evaluated in the usual way by pattern matching.
For example, with the above definition of @emp@ it is straightforward to show that
\begin{align}
  \mathtt{[1, 2, 3]} \dcolon \tref{\mathtt{v}}{[\tint]}{}{\mathtt{not}\ (\mathtt{emp}\ \mathtt{v})} \label{type:len}
\end{align}
as the refinement @not (emp ([1, 2, 3]))@ evaluates to $\tttrue$.

\mypara{Measures as Axioms}
How can we reason about invocations of measures in the decidable logic of VCs?
A natural approach is to treat a measure like @emp@ as an uninterpreted function
and add logical axioms that capture its behaviour. This looks easy: each equation 
of the measure definition corresponds to an axiom, thus:
%
\begin{align*}
\ttemp\ \hnull &= \tttrue\\
\forall \ttx, \ttxs.\, \ttemp\ (\ttx:\ttxs) &= \ttfalse
\end{align*}
%
Under these axioms the judgement~\ref{type:len} is indeed valid. 
% % Measures as data constructor refinements

\mypara{Measures as Refinements in Types of Data Constructors}
Axiomatizing measures is \emph{precise}; that is, 
the axioms exactly capture the meaning of measures.
Alas, axioms render SMT solvers \emph{inefficient}, and render the VC mechanism \emph{unpredictable}, 
as one must rely on various brittle syntactic matching and instantiation heuristics~\cite{simplifyj}.

Instead, we use a different approach that is \emph{both} precise \emph{and} efficient.
The key idea is this: \emph{instead of translating each measure equation into an axiom, 
we translate each equation into a refined type for the corresponding data constructor}~\citep{LiquidPLDI09}.
This translation is given in Figure~\ref{fig:measures}.
For example, the definition of the measure @emp@ yields the following refined types for the list data constructors:
$$
\begin{array}{lcl}
\hnull  & :: & \ttreft{v}{[\tint]}{emp\ v = true}\\
{:}  & :: & \tfun{\ttx}{\tint}{\tfun{\ttxs}{[\tint]}{\ttreft{v}{[\tint]}{emp\ v = false}}}
\end{array}
$$
These types ensure that:
%
~(1) each time a list value is \emph{constructed}, 
its type carries the appropriate emptiness information. 
Thus our system is able to statically decide that 
(\ref{type:len}) is valid and
~(2) each time a list value is \emph{matched}, 
the appropriate emptiness information is used to 
improve precision of pattern matching, as we see next.

\mypara{Using Measures}
\label{sec:pattern-match}
As an example, we use the measure @emp@ to 
provide an appropriate type for the @head@ function:
%
\begin{code}
  head    :: {v:[Int] | not (emp v)} -> Int 
  head xs = case xs of
              (x:_) -> x
              []    -> error "yikes"  

  error   :: {v:String | false} -> a
  error   = undefined
\end{code}
%
@head@ is safe as its input type stipulates that it will only 
be called with lists that are \emph{not} @[]@, and so
@error "..."@ is dead code.
%
The call to @error@ generates the subtyping query
%
\begin{align*}
   \tbind{\ttxs}{\tref{\ttxs}{[\tint]}{\trivial}{\lnot (\ttemp\ \ttxs)}}, \
   \tbind{\ttb}{\tttref{\ttb}{[\tint]}{\trivial}{(\ttemp\ \ttxs)= true}} 	
	 & \vdash \subtref{\tttrue}{\ttfalse} 
\end{align*}
%
The match-binder $\ttb$ holds the result of the 
match~\cite{SulzmannCJD07}. In the \texttt{[]} case,
we assign it the refinement of the type of \texttt{[]} 
which is $(\ttemp\ \ttxs) = \tttrue$. %~\cite{LiquidPLDI09}.
%
Since the call is done inside a @case-of@ expressions 
both @xs@ and @b@ are in WHNF,
thus they have \Wnf types. 
  
The verifier \emph{accepts} the program as the above subtyping reduces to the valid VC:
\begin{align*}
\lnot (\ttemp\ \ttxs) \wedge ((\ttemp\ \ttxs)= \tttrue) \Rightarrow\ & \tttrue \Rightarrow\ \ttfalse
\end{align*}
%
Thus, our system supports idiomatic 
Haskell, \eg taking the @head@ of an infinite list:
%
\begin{code}
  ex x     = head (repeat x)
  
  repeat   :: Int -> {v:[Int] | not (emp v)}
  repeat y = y : repeat y
\end{code}
%

\mypara{Multiple Measures}
If a type has multiple measures, we simply refine each data constructor's type
with the \emph{conjunction} of the refinements from each measure.
%
For example, consider a measure that computes the length of a list:
\begin{code}
  measure len  :: [Int] -> Int
    len ([])   = 0
    len (x:xs) = 1 + len xs
\end{code}
%
Using the translation of Figure~\ref{fig:measures},
we get the following types for list's data constructors.
%
\begin{align*}
\text{[]}  & ::  \ttreft{v}{[\tint]}{len\ v = 0}\\
{:}  & ::  \tfun{\ttx}{\tint}{\tfun{\ttxs}{[\tint]}{\ttreft{v}{[\tint]}{len\ v = 1 + (len\ xs)}}}\\
\intertext{The final types for list data are the 
conjunction of the refinements from $\mathtt{len}$ and $\mathtt{emp}$:}\\
\text{[]}  & ::  \ttreft{v}{[\tint]}{emp\ v = true \land len\ v = 0}\\
{:}  & ::  \tfun{\ttx}{\tint}{\tfun{\ttxs}{[\tint]}
           {\ttreft{v}{[\tint]}{emp\ v = false \land len\ v = 1 + (len\ xs)}}}
\end{align*}



\section{Declarative Typing: \undeclang}\label{sec:language}\label{sec:undec}

Next, we formalize our stratified refinement type system, in two steps.
%
First, in this section, we present a core calculus \undeclang, 
with a general $\beta$-reduction semantics. We describe the syntax,
operational semantics, and sound but undecidable declarative typing 
rules for \undeclang. 
%
Second, in \Sref{sec:typing}, we describe \logiclang, a subset 
of \undeclang that forms a decidable logic of refinements, and 
use it to obtain \declang with decidable SMT-based algorithmic typing.

\subsection{Syntax}\label{sec:undec:syntax} 

\begin{figure}
\hrule width 0.48\textwidth \vspace{0.05in}
$$
\begin{array}{rrcl}

\emphbf{Constants} \quad 
  & c & ::=    & 0,1,-1,\ldots \spmid \etrue, \efalse \\
  &   & \spmid & +,-,\ldots \spmid =, <, \ldots \spmid \ecrash 
  \\[0.05in]

\emphbf{Values} \quad 
  & \val & ::= &  c \spmid \efun{x}{\typ}{e} \spmid \edapp{D}{e}
  \\[0.05in] 

\emphbf{Expressions} \quad 
  & e & ::=    & \val \spmid x \spmid \eapp{e}{e} \spmid \elet{x}{e}{e} \\ 
  &   & \spmid & \ecase{e}{D}{\overline{x}}{e}{x} \\[0.05in] 

\emphbf{Refinements} \quad 
  & r & ::= &   e \\[0.05in] 

\emphbf{Basic Types} \quad 
  & \tbase & ::= & \tint \spmid \tbool \spmid \ttct \\[0.05in] 

\emphbf{Types} \quad 
  & \typ & ::= & \tref{v}{\Base}{}{r} \spmid \tfunref{x}{\typ}{\typ}{v}{e} \\[0.1in]
\end{array}
$$
% \hrule width 0.48\textwidth

$$
\begin{array}{rrcl}
\emphbf{Contexts} \quad 
  & C
  & ::= 
  &   	 \bullet 
  \spmid \eapp{C}{e} 
  \spmid \eapp{c}{C} 
  \spmid D\ \overline{e}\ C\ \overline{e}\\
  &&\spmid &
  \ecase{C}{D}{\overline{y}}{e}{x}
  \\[0.05in] 
\end{array}
$$

\judgementHead{Reduction}{\eval{e}{e}}

$$
\begin{array}{rcl}
\eval{C[e]&}{&C[e']} \qquad \text{if}\ \eval{e}{e'} \\
	\eval{\eapp{c}{v}&}{& \ceval{c}{v}}\\
\eval{\eapp{(\efun{x}{\tau_x}{e})}{e_x}&}{&e\sub{x}{e_x}}\\
	\eval{\elet{x}{e_x}{e}&}{&e\sub{x}{e_x}} \\
	\eval{\ecase{D_j\ \overline{e}}{D_i}{\overline{y_i}}{e_i}{x}&}
	{&e_j\sub{x}{D_j\ \overline{e}}\sub{\overline{y_j}}{\overline{e}}} \\
\end{array}
$$

\caption{\undeclang: Syntax and Operational Semantics}
\label{fig:undeclang}
\label{fig:operational}
\end{figure}


Figure~\ref{fig:undeclang} summarizes the syntax of \undeclang, 
which is essentially the calculus \hlang~\cite{Knowles10} 
\emph{without} the dynamic checking features (like casts), but 
\emph{with} the addition of data constructors.
%
In \undeclang, as in \hlang, refinement expressions $r$ are not drawn from a decidable 
logical sublanguage, but can be arbitrary expressions $e$
(hence $r ::= e$ in Figure~\ref{fig:undeclang}). 
This choice allows us to prove preservation and progress, 
but renders typechecking undecidable. 
 
%The syntactic elements of \undeclang are layered into 
%primitive constants, values, and expressions.

\spara{Constants}
The primitive constants of \undeclang include  
$\tttrue$, $\ttfalse$, $\mathtt{0}$, $\mathtt{1}$, $\mathtt{-1}$, \etc,
and arithmetic and logical operators like $\mathtt{+}$, $\mathtt{-}$, 
$\mathtt{\leq}$,$\mathtt{/}$, $\mathtt{\land}$, $\mathtt{\lnot}$.
%
In addition, we include a special \emph{untypable} constant $\ecrash$ 
that models ``going wrong''. Primitive operations return a $\ecrash$
when invoked with inputs outside their domain, \eg when $\mathtt{/}$ 
is invoked with $\mathtt{0}$ as the divisor, or when $\mathtt{assert}$ is 
applied to $\mathtt{false}$.

\spara{Data Constructors}
We encode data constructors as special constants. 
Each data type has an arity $\arity{T}$ that represents
the exact number of data constructors that return a value of 
type $T$.
%
For example the data type \tintlist, which represents 
lists of integers, has two data constructors: $\dnull$ and $\dcons$,
\ie has arity $2$.
%%$D^\tintlist_1 \defeq \dnull$ and
%% $D^\tintlist_2 \defeq \dcons$.


\spara{Values \& Expressions}
The values of \undeclang include constants, 
$\lambda$-abstractions $\efun{x}{\typ}{e}$, and 
fully applied data constructors $D$ that wrap expressions.
%
The expressions of \undeclang include values, as well as
variables $x$, 
applications $\eapp{e}{e}$, 
and the $\mathtt{case}$ 
and $\mathtt{let}$ expressions.

\subsection{Operational Semantics}

Figure~\ref{fig:operational} summarizes the small 
step contextual $\beta$-reduction semantics for 
\undeclang.
%
Note that we allow for reductions under data constructors, 
and thus, values may be further reduced.
%
We write \evalj{e}{e'}{j} if there exist $e_1,\ldots,e_j$ such that
$e$ is $e_1$, $e'$ is $e_j$ and $\forall i,j, 1 \leq i < j$, we have
\eval{e_i}{e_{i+1}}.
%
We write \evals{e}{e'} if there exists some (finite) $j$ such that
$\evalj{e}{e'}{j}$.

\spara{Constants} Application of a constant requires the
argument be reduced to a value; in a single step the 
expression is reduced to the output of the primitive 
constant operation. 
%
For example, consider $=$, the primitive equality operator 
on integers. We have $\ceval{=}{n} \defeq =_n$
where $\ceval{=_n}{m}$ equals \etrue iff $m$ is the same as $n$.
%%as follows:
%%%
%%$$
%%\ceval{=}{n} \defeq =_n \qquad \ceval{=_n}{m} \defeq \begin{cases} \etrue, & \mbox{if}\ m = n \\
%%                                                                     \efalse, & \mbox{otherwise}
%%                                                       \end{cases}
%%$$

%% \begin{align*}
%% \interp{=}(n) \defeq & =_n \\
%% \interp{=_n}(m) \defeq & \begin{cases} \etrue, & \mbox{if}\ m = n \\
%%                                        \efalse, & \mbox{otherwise}
%%                          \end{cases}
%% \end{align*}
%%
%%\spara{Data Constructors} \RJ{This para is cancelled with $\beta$-reduction, right?}
%%%
%%There is no rule to evaluate application of data constructors.
%%Like Haskell, our semantics do not force evaluation of expressions 
%%wrapped in data constructors, thus, $D \ e_1\ \dots \ e_n$ is a 
%%value, without requiring $e_i$s to be values.


\subsection{Types}

\undeclang types include basic types, which are \emph{refined} with predicates, 
and dependent function types.
%
\emph{Basic types} $\tbase$ comprise integers, booleans, and a family of data-types 
$T$ (representing lists, trees \etc.)
%
For example the data type \tintlist represents lists of integers.
%
We refine basic types with predicates (boolean valued expressions $e$) to obtain
\emph{basic refinement types} $\tref{v}{\tbase}{}{e}$.
%
Finally, we have dependent \emph{function types} $\tfun{x}{\typ_x}{\typ}$ 
where the input $x$ has the type $\typ_x$ and the output $\typ$ may
refer to the input binder $x$.

\spara{Notation} We write $\tbase$ to abbreviate $\tref{v}{\tbase}{}{\etrue}$, 
and \tfunbasic{\typ_x}{\typ} to abbreviate \tfun{x}{\typ_x}{\typ} if 
$x$ does not appear in $\typ$. 
% We use $p$, $q$, and $r$ for 
% refinements, and 
We use $\_$ for unused binders.
We write $\tref{v}{\tnat}{l}{r}$ to abbreviate $\tref{v}{\tint}{l}{0 \leq v \wedge r}$.


\spara{Denotations}
%
Each type $\typ$ \emph{denotes} a set of expressions $\interp{\typ}$,
that are defined via the dynamic semantics~\cite{Knowles10}.
%
Let \erase{\typ} be the type we get if we erase all refinements 
from $\typ$ and $\hastypebasesmall{\Env}{e}{\erase{\typ}}$ be the 
standard typing relation for the typed lambda calculus.
%
Then, we define the denotation of types as: 
\begin{align*}
\interp{\tref{x}{\tbase}{}{r}} \defeq & 
    \{e \mid  \hastypebasesmall{\emptyset}{e}{\tbase},
              \mbox{ if } \evals{e}{w} 
              \mbox{ then } \evals{\SUBST{r}{x}{w}}{\etrue} \}\\
\interp{\tfun{x}{\typ_x}{\typ}} \defeq & 
    \{e \mid  \hastypebasesmall{\emptyset}{e}{\erase{\tfunbasic{\typ_x}{\typ}}}, 
              \forall e_x \in \interp{\typ_x}.\ \eapp{e}{e_x} \in \interp{\typ\sub{x}{e_x}}
    \}
\end{align*}

%% The meaning of the refined type $\tref{v}{b}{}{e_r}$
%% is all the closed expressions $e$ of basic type $b$ for 
%% which $e_r$ holds, 
%% in the sense that if \evals{e}{v} for some value $v$, 
%% then $e_r\sub{v}{e}$ evaluates to \etrue.
%% %
%% A dependent function type $\tfunref{x}{\typ_x}{\typ}{}{}$ 
%% is interpreted as the set of closed expressions of
%% simple type \tfunbasic{\typ_x}{\typ}. 
%% that give output in $\interp{\typ\sub{x}{e_x}}$ 
%% %whenever their input $e_x$ is in \interp{\typ_x}

\spara{Constants}
For each constant $c$ we define its type \constty{c}
such that $c \in \interp{\constty{c}}$. 
%
For example,
%Each constant $c$ we define is in the denotation of its type \constty{c}:
%
%Each constant $c$ is has type \constty{c} such that
%$c \in \interp{\constty{c}}$. 
%
%% For example,
%
$$
\begin{array}{lcl}
\constty{3} &\doteq& \tttref{v}{\tint}{}{v = 3}\\
\constty{+} &\doteq& \tfun{\ttx}{\tint}{\tfun{\tty}{\tint}{\tttref{v}{\tint}{}{v = x + y}}}\\
\constty{/} &\doteq& \tfunbasic{\tint}{\tfunbasic{\tttref{v}{\tint}{}{v > 0}}{\tint}}\\
\constty{\eerror{\typ}} &\doteq& \tfunbasic{\tttref{v}{\tint}{}{\efalse}}{\typ}
\end{array}
$$
%
So, by definition we get the constant typing lemma
%
\begin{lemma}{[Constant Typing]}\label{lemma:constants}
Every constant $c \in \interp{\constty{c}}$.
\end{lemma}
%
Thus, if $\constty{c} \defeq \tfun{x}{\typ_x}{\typ}$, then for every value 
$w \in \interp{\typ_x}$, we require that $\ceval{c}{w} \in \interp{\typ\sub{x}{w}}$.
%
For every value $w \not \in \interp{\typ_x}$, it suffices to define $\ceval{c}{w}$
as \ecrash, a special untyped value.

\spara{Data Constructors}
%As discussed in ~\Sref{sec:measures}, 
The types of data constructor constants are refined 
with predicates that track the semantics of the 
\emph{measures} associated with the data type.
%
For example, as discussed in \Sref{sec:measures} 
we use @emp@ to refine the list data constructors' types:
$$
\begin{array}{lcl}
\constty{\dnull}  & \defeq & \tttref{v}{\tintlist}{}{\eisNull{v}}\\
\constty{\dcons}  & \defeq & \tfunbasic{\tint}{\tfunbasic{\tintlist}{\tttref{v}{\tintlist}{}{\lnot (\eisNull{v})}}}
\end{array}
$$
%
By construction it is easy to prove that Lemma~\ref{lemma:constants}
holds for data constructors.
%
For example, $\ttemp\ \dnull$ goes to $\tttrue$.
%%We \emph{compose} multiple measures for a type by 
%%refining the constructors with the \emph{conjunction} 
%%of each measure's refinements.
%


\subsection{Type Checking}\label{subsec:typing}

\begin{figure}[p]
\centering
\captionsetup{justification=centering}
\judgementHead{Well-Formedness}{\isWellFormed{\Gamma}{\sigma}}

$$\begin{array}{ccc}
\inference
  {}
  {\isWellFormed{\Gamma}{\true(\vref)}}
  [\wtTrue]
&
\quad
&
\inference
    {\isWellFormed{\Gamma}{\areft(\vref)} && 
     \hastype{\Gamma}{\rvapp{\rvar}{e} \ \vref}{\tbbool}
    }
    {\isWellFormed{\Gamma}{(\areft \wedge \rvapp{\rvar}{e})(\vref)}}
    [\wtRVApp]
\end{array}$$
%
$$\inference
    {\hastype{\Gamma, \vref:b}{\reft}{\tbbool} \quad 
%     \isWellFormed{\Gamma, \vref:b}{\areft(\vref)}
     \hastype{\Gamma, \vref:b}{\areft(\vref)}{\tbbool}
    }
    {\isWellFormed{\Gamma}{\tpref{b}{\areft}{\reft}}}
    [\wtBase]
$$
%
$$
\inference
    {
	%\hastype{\Gamma, v:\tfun{x}{\tau_x}{\tau}}{e}{\tbbool} &&
	\hastype{\Gamma}{\reft}{\tbbool} &&
    \isWellFormed{\Gamma}{\tau_x} &&
	\isWellFormed{\Gamma, x:\tau_x}{\tau}
    }
    {\isWellFormed{\Gamma}{\trfun{x}{\tau_x}{\tau}{\reft}}}
    [\wtFun]
$$
%
$$\begin{array}{ccc}
\inference
  {\isWellFormed{\Gamma, \rvar:\tau}{\sigma}}
  {\isWellFormed{\Gamma}{\tpabs{\rvar}{\tau}{\sigma}}}
  [\wtPred]
&
\quad
&
\inference
    {\isWellFormed{\Gamma, \alpha}{\sigma}}
    {\isWellFormed{\Gamma}{\ttabs{\alpha}{\sigma}}}
    [\wtPoly]
\end{array}$$

\medskip \judgementHead{Subtyping}{\isSubType{\Gamma}{\sigma_1}{\sigma_2}}

$$
\inference
   {\text{SMT-Valid}(\inter{\Gamma} \land \inter{\areft_1\ \vref} \land \inter{\reft_1} 
                 \Rightarrow \inter{\areft_2\ \vref} \land \inter{\reft_2})}
   {\isSubType{\Gamma}{\tpref{b}{\areft_1}{\reft_1}}{\tpref{b}{\areft_2}{\reft_2}}}
   [\tsubBase]
$$
%
$$
\inference
   {%\text{Valid}(\inter{\Gamma}\land \inter{e_1} \Rightarrow \inter{e_2}) \\
	\isSubType{\Gamma}{\tau_2}{\tau_1} &
	\isSubType{\Gamma, x_2:{\tau_2}}{\SUBST{\tau_1'}{x_1}{x_2}}{\tau_2'}	
   }
   {\isSubType{\Gamma}
	  {\trfun{x_1}{\tau_1}{\tau_1'}{\reft_1}}
	  {\trfun{x_2}{\tau_2}{\tau_2'}{\true}}
}[\tsubFun]
$$
%
$$
\begin{array}{ccc}
\inference
   {\isSubType{\Gamma, \rvar:\tau}{\sigma_1}{\sigma_2}}
   {\isSubType{\Gamma}{\tpabs{\rvar}{\tau}{\sigma_1}}{\tpabs{\rvar}{\tau}{\sigma_2}}}
   [\tsubPred]
&
\quad
&
\inference
   {\isSubType{\Gamma}{\sigma_1}{\sigma_2}}
   {\isSubType{\Gamma}{\ttabs{\alpha}{\sigma_1}}{\ttabs{\alpha}{\sigma_2}}}
   [\tsubPoly]
\end{array}
$$

\medskip \judgementHead{Type Checking}{$\hastype{\Gamma}{e}{\sigma}$}

$$\inference
  {  \hastype{\Gamma}{e}{\sigma_2} && \isSubType{\Gamma}{\sigma_2}{\sigma_1} 
  && \isWellFormed{\Gamma}{\sigma_1}
  }
  {\hastype{\Gamma}{e}{\sigma_1}}
  [\tsub]
\quad
\inference
  {}
  {\hastype{\Gamma}{c}{\tc{c}}}
  [\tconst]
$$
$$
\inference
  {x: \tpref{b}{\areft}{\reft} \in \Gamma}
  {\hastype{\Gamma}{x}{\tpref{b}{\areft}{e \land \vref = x}}}
  [\tbase]
\quad
\inference
  {x:\tau \in \Gamma}
  {\hastype{\Gamma}{x}{\tau}} 
  [\tvariable]
$$
$$
\inference
   {\hastype{\Gamma, x:\tau_x}{e}{\tau} 
    && \isWellFormed{\Gamma}{\tau_x}
   }
   {\hastype{\Gamma}{\efunt{x}{\tau_x}{e}}{\tfun{x}{\tau_x}{\tau}}}
   [\tfunction]
\quad
\inference
   {\hastype{\Gamma}{e_1}{\tfun{x}{\tau_x}{\tau}} 
   &&  \hastype{\Gamma}{e_2}{\tau_x}
   }
   {\hastype{\Gamma}{\eapp{e_1}{e_2}}{\SUBST{\tau}{x}{e_2}}}
   [\tapp]
$$
$$
\inference
  {\hastype{\Gamma, \alpha}{e}{\sigma}}
  {\hastype{\Gamma}{\etabs{\alpha}{e}}{\ttabs{\alpha}{\sigma}}}
  [\tgen]
\quad
\inference
  {\hastype{\Gamma}{e}{\ttabs{\alpha}{\sigma}} && 
   \isWellFormed{\Gamma}{\tau}
  }
  {\hastype{\Gamma}{\etapp{e}{\tau}}{\SUBST{\sigma}{\alpha}{\tau}}}
  [\tinst]
$$
$$
\inference
    {\hastype{\Gamma, \rvar:\tau}{e}{\sigma} &&
     \isWellFormed{\Gamma}{\tau} 
     % \tau \mbox{ is non-refined } 
     %\isWellFormed{\Gamma}{\tpabs{p}{\tau}{\pi}} && 
     %p \notin \fv{e}
    }
    {\hastype{\Gamma}{\epabs{\rvar}{\tau}{e}}{\tpabs{\rvar}{\tau}{\sigma}}}
    [\tpgen]
\ \
\inference
    {\hastype{\Gamma}{e}{\tpabs{\rvar}{\tau}{\sigma}} && 
     \hastype{\Gamma}{\efunbar{x:\tau_x}{\reft'}}{\tau}
    }
    {\hastype{\Gamma}
             {\epapp{e}{\efunbar{x:\tau_x}{\reft'}}}
             {\rpinst{\sigma}{\rvar}{\efunbar{x:\tau_x}{\reft'}}}
     %        {\sigma\sub{\eapp{p}{\overline{e_p}}}{\eapp{\reft'}{\overline{e_p}}}}
    }
    [\tpinst]
$$
\caption[Type checking of \corelan.]{Well-formedness, Subtyping and Type Checking of \corelan.}
\label{fig:rules}
\end{figure}



Next, we present the type-checking judgments and rules of \undeclang. 

\spara{Environments and Closing Substitutions}
A \emph{type environment} $\Env$ is a sequence of type bindings 
$\tbind{x_1}{\typ_1},\ldots,\tbind{x_n}{\typ_n}$. An environment
denotes a set of \emph{closing substitutions} $\sto$ which are 
sequences of expression bindings: 
$\gbind{x_1}{e_1}, \ldots, \gbind{x_n}{e_n}$ such that:
$$
\interp{\Env} \defeq  \{\sto \mid \forall \tbind{x}{\typ} \in \Env. 
                                    \sto(x) \in \interp{\thetasub{\sto}{\typ}} \}
$$

\spara{Judgments}
We use environments to define three kinds of
rules: Well-formedness, Subtyping, 
and Typing~\cite{Knowles10,GordonTOPLAS2011}.
%
%\spara{Well-formedness}
A judgment \undeciswellformed{\Env}{\typ} states that 
the refinement type $\typ$ is well-formed in 
the environment $\Env$.
%
Intuitively, the type $\typ$ is well-formed if all
the refinements in $\typ$ are $\tbool$-typed in $\Env$.
%
%\spara{Subtyping} 
A judgment \undecissubtype{\Env}{\typ_1}{\typ_2} states 
that the type $\typ_1$ is a subtype of %the type 
$\typ_2$ in the environment $\Env$.
%
Informally, $\typ_1$ is a subtype of $\typ_2$ if, when 
the free variables of $\typ_1$ and $\typ_2$ 
are bound to expressions described by $\Env$,
the denotation of $\typ_1$ 
is \emph{contained in} the denotation of $\typ_2$. 
%
Subtyping of basic types reduces to denotational containment checking.
%
%\spara{Implication} 
%%A judgment \issubref{\Env}{p_1}{p_2} states 
%%that the predicate $p_1$ \emph{implies} 
%%the predicate $p_2$ in the environment $\Env$.
%
That is, for any closing substitution $\sto$
in the denotation of $\Env$, for every expression $e$, 
if $e \in \interp{\thetasub{\sto}{\typ_1}}$ then 
$ e \in \interp{\thetasub{\sto}{\typ_2}}$.
%
%\spara{Typing}
A judgment \undechastype{\Env}{e}{\typ} states that
the expression $e$ has the type $\typ$ in 
the environment $\Env$.
That is, when the free variables in $e$ are 
bound to expressions described by $\Env$, the 
expression $e$ will evaluate to a value 
described by $\typ$.

\mypara{Soundness}
Following \hlang~\cite{Knowles10}, we use the (undecidable) \rsubbase to show that each step 
of evaluation preserves typing, and that if an expression
is not a value, then it can be further evaluated:
%
\begin{itemize}
\item\textbf{Preservation:} 
	If \undechastype{\emptyset}{e}{\typ} and \eval{e}{e'}, 
	then \undechastype{\emptyset}{e'}{\typ}. 
\item\textbf{Progress:}
	If \undechastype{\emptyset}{e}{\typ} and $e \not = w$,
	then \eval{e}{e'}. 
\end{itemize}
%
We combine the above to prove that evaluation preserves 
typing, and that a well typed term will not \ecrash.
%
\begin{theorem}{[Soundness of \undeclang]}\label{thm:safety}
\begin{itemize}
\item\textbf{Type-Preservation:} If \undechastype{\emptyset}{e}{\typ}, %NV with the v -> w edit this didn't fit in 1 line
       $\evals{e}{w}$ then $\undechastype{\emptyset}{w}{\typ}$.
\item\textbf{Crash-Freedom:} If \undechastype{\emptyset}{e}{\typ} 
        then $\evals{e\not}{\ecrash}$.
\end{itemize}
\end{theorem}

We prove the above following the overall recipe of~\cite{Knowles10}. 
Crash-freedom follows from type-preservation and as \ecrash has no type.
%
The Substitution Lemma, in particular, follows from a connection between
the typing relation and type denotations:

\begin{lemma}{[Denotation Typing]}\label{lem:denotation}
If $\undechastype{\emptyset}{e}{\typ}$ then $e \in \interp{\typ}$.
\end{lemma} 

%%% Local Variables: 
%%% mode: latex
%%% TeX-master: "main"
%%% End: 

\renewcommand\rimpl{\rsubbase}
\renewcommand{\rtdimp}{\rsubbased}
%
\section{Algorithmic Typing: \declang}\label{sec:typing}

% Here is the syntax and typing rules for \lambda-D and QF-EUFLIAD

\renewcommand\restrictdecidable[2]{#1}
\newcommand\samerule[1]{}


\begin{figure}[t!]
\centering
\captionsetup{justification=centering}
$$
\begin{array}{rrcl}
\multicolumn{4}{l}{\text{\emphbf{Expressions}, \emphbf{Values}, \emphbf{Constants}, \emphbf{Basic types}:
see Figure~\ref{fig:undeclang}}} \\[0.05in]

\emphbf{Types} \quad 
  & \tau
  & ::=  
  & 	 \tref{v}{\ttbase}{}{\refi}
  \spmid \tref{v}{\ttbase}{l}{\refi}\\
  &&
  \spmid & \tfunref{x}{\tau}{\tau}{v}{\refi}   \\[0.05in]

\emphbf{Labels} \quad 
  & l
  & ::= 
  & \trivial \spmid \finite 
  \\[0.05in] 

\emphbf{Refinements} \quad 
  & \refi
  & ::=
  & p
  \\[0.05in] 

\emphbf{Predicates} \quad 
  & p
  & ::= 
  &   	 \lterm = \lterm
  \spmid \lterm < \lterm
  \spmid       p \land p
  \spmid \lnot p\\
  &&
  \spmid &   	 n 
  \spmid x 
  \spmid f \ \overline{\lterm}
  % \spmid D \ \overline{\lterm}
  \spmid \lterm \oplus \lterm
  \\ && \spmid &
  \etrue
  \spmid \efalse 
  \\[0.05in] 

\emphbf{Measures} \quad 
  & \multicolumn{3}{l}{f,g,h}
  \\[0.05in] 
  
\emphbf{Operators} \quad 
  & \oplus
  & ::= 
  &   	 + 
  \spmid -  
  \spmid \dots
  \\[0.05in] 
  
\emphbf{Integers} \quad 
  & n
  & ::= 
  &   	 0 
  \spmid 1
  \spmid -1
  \spmid \dots
  \\[0.05in] 

\emphbf{Domain} \quad 
  & d
  & ::= 
  & n 
  \spmid c_w 
  \spmid D\ \overline{d}  
  \spmid \etrue
  \spmid \efalse
  \\[0.05in] 
  
\emphbf{Model} \quad 
  & \sigma & ::= & \gbind{x_1}{d_1},\ldots,\gbind{x_n}{d_n}
  \\[0.05in] 

\emphbf{Lifted Values} \quad 
  & \botv
  & ::= 
  &   	 c 
  \spmid \efun{x}{}{e} 
  \spmid D \ \overline{\botv}
  \spmid \ebot
\end{array}
$$
\caption{Syntax of \declang}
\label{fig:declang:syntax}
\end{figure}


\begin{figure}[t!]
\centering
\captionsetup{justification=centering}

All rules as in Figure~\ref{fig:typing} except as follows: \\

\judgementHead{Well-Formedness}{\deciswellformed{\Gamma}{\tau}}
$$
\inference{
	\dechastype{\Gamma, \tbind{v}{\Base}}{\restrictdecidable{p}{r}}{\FinTy{\tbool}}
}{
	\deciswellformed{\Gamma}{\tref{v}{\Base}{}{\restrictdecidable{p}{r}}}
}[\rwbased]
%%%% \samerule{\qquad
%%%% \inference{
%%%% 	\deciswellformed{\Gamma}{\tau_x} &&
%%%% 	\deciswellformed{\Gamma, x \colon \tau_x}{\tau}
%%%% }{
%%%% 	\deciswellformed{\Gamma}{\tfunref{x}{\tau_x}{\tau}{v}{e}}
%%%% }[\rwfun]
%%%% }
$$

%%\judgementHead{Implication}{\issubref{\Gamma}{\restrictdecidable{p}{e}}{\restrictdecidable{p}{e}}}
%%\restrictdecidable{
%%$$
%%\inference{
%%  \VCOND{\Env}{p_1}{p_2}\ \mbox{is u-valid}
%%}{
%%	\issubref{\Env}{p_1}{p_2}
%%}[\rtdimp]
%%$$}{$$
%%\inference{
%%	\forall \theta. \deciswellformed{\Gamma}{\theta} \land
%%				\evals{\theta\ e_1}{\etrue} 
%%	\Rightarrow \evals{\theta\ e_2}{\etrue}
%%}{
%%	\issubref{\Gamma}{e_1}{e_2}
%%}[\rimpl]
%%$$}
\judgementHead{Subtyping}{\decissubtype{\Gamma}{\tau_1}{\tau_2}}

$$
\inference{
  \VCOND{\Env, v:B}{p_1}{p_2}\ \mbox{is valid}
}{
	\decissubtype{\Gamma}
		{\tref{v}{B}{}{\restrictdecidable{p_1}{e_1}}}
		{\tref{v}{B}{}{\restrictdecidable{p_2}{e_2}}}
}[\rsubbased]
\samerule{\qquad
\inference{
	\decissubtype{\Gamma}{\tau'_x}{\tau_x} &&
	\decissubtype{\Gamma, x \colon \tau'_x}{\tau}{\tau'}
}{
	\decissubtype{\Gamma}{\tfunref{x}{\tau_x}{\tau}{v}{e_1}}{\tfunref{x}{\tau'_x}{\tau'}{v}{e_2}}
}[\rsubfun]
}$$

\judgementHead{Typing}{\dechastype{\Gamma}{e}{\tau}}
\samerule{$$
\inference{
	(x,\tau) \in \Gamma 
}{
	\dechastype{\Gamma}{e}{\tau}
}[\rtvar]
$$

$$
\inference{
}{
	\dechastype{\Gamma}{c}{\constty{c}}
}[\rtconst]
\qquad
\inference{
	\dechastype{\Gamma}{e}{\tau'} &&
	\decissubtype{\Gamma}{\tau'}{\tau} &&
	\deciswellformed{\Gamma}{\tau} &&
}{
	\dechastype{\Gamma}{e}{\tau}
}[\rtsub]
$$}
$$
\samerule{
\inference{
	\dechastype{\Gamma, x\colon\tau_x}{e}{\tau} &&
	\deciswellformed{\Gamma}{\tau_x}
}{
	\dechastype{\Gamma}{\efun{x}{e}}{(\tfun{x}{\tau_x}{\tau})}
}[\rtfun]
\qquad}
\inference{
	\dechastype{\Gamma}{e_1}{(\tfunref{x}{\tau_{x}}{\tau}{v}{e_v})} &&
	\dechastype{\Gamma}{\restrictdecidable{y}{e_2}}{\tau_{x}}
}{
	\dechastype{\Gamma}{\eapp{e_1}{\restrictdecidable{y}{e_2}}}{\tau\sub{x}{\restrictdecidable{y}{e_2}}}
}[\rtapp-D]
$$
\samerule{
$$
\inference{
	\dechastype{\Gamma}{e_x}{\tau_{x}} &&
 in Figure~\ref{fig:}	\dechastype{\Gamma,x\colon\tau_x}{e}{\tau} &&
	\deciswellformed{\Gamma}{\tau}
}{
	\dechastype{\Gamma}{\elet{x}{e_x}{e}}{\tau}
}[\rtlet]
$$
}
$$\inference{
	l \not \in \{\finite, \trivial\} \Rightarrow \tau \ \text{is}\ \Div &&
	\dechastype{\Gamma}{e}{\tref{v}{T}{l}{r}} &&
	 \deciswellformed{\Gamma}{\tau}\\
	\forall i. \constty{D_i} = \overline{y_j}{\tau_j} \rightarrow \tref{v}{T}{}{r_i} &&
	\dechastype{\Gamma,  \overline{\tbind{y_j}{\tau_j}},
				\tbind{x}{\tref{v}{T}{\restrictdecidable{\trivial}
				{\ltrivial}
				}{r \land r_i}}}{e_i}{\tau}	
}{
	\dechastype{\Gamma}{\ecase{e}{D_i}{\overline{y_j}}{e_i}{x}}{\tau}
}[\rtcased]$$
\caption{Typechecking for \declang}
\label{fig:declang:typing}
\end{figure}



While \undeclang is sound, it cannot be \emph{implemented} 
thanks to the undecidable denotational containment rule \rsubbase
(Figure~\ref{fig:refinedhaskell:typing}).
%
Next, we go from \undeclang to \declang, a core calculus 
with sound, SMT-based algorithmic type-checking in four 
steps.
%
First, we show how to restrict the language of 
refinements to an SMT-decidable sub-language 
\logiclang~(\Sref{sec:typing:logic}).
%
Second, we \emph{stratify} the types to specify 
whether their inhabitants may diverge, must reduce
to values, or must reduce to finite values~(\Sref{sec:typing:stratify}).
%
Third, we show how to \emph{enforce} the stratification
by encoding recursion using special fixpoint combinator 
constants (\Sref{sec:typing:termination}).
%
Finally, we show how to use \logiclang and the 
stratification to approximate the undecidable \rsubbase
with a decidable verification condition \rsubbased, thereby 
obtaining the algorithmic system \declang~(\Sref{sec:typing:vc}).


\subsection{Refinement Logic: \logiclang}\label{sec:typing:logic}

Figure~\ref{fig:declang:syntax} summarizes the syntax of \declang.
Refinements $r$ are now predicates $p$, drawn from 
\logiclang, the decidable logic of equality, 
uninterpreted functions and linear arithmetic
~\cite{Nelson81}.
%
Predicates $p$ include linear arithmetic constraints,
function application where function symbols correspond
to measures (as described in \Sref{sec:measures}), and 
boolean combinations of sub-predicates.

\mypara{Well-Formedness} 
For a predicate to be well-formed it should be boolean and
arithmetic operators 
should be applied to integer terms, measures should be applied 
to appropriate arguments 
(\ie \eisNull\ is applied to \tintlist),
and equality or inequality to basic 
(integer or boolean) terms.
%
Furthermore, we require that refinements, and thus measures,
always evaluate to a value.
%
We capture these requirements by assigning appropriate types 
to operators and measure functions, after which we require that
each refinement $r$ has type $\FinTy{\tbool}$ (rule \rwbased in
Figure~\ref{fig:declang:typing}).

\mypara{Assignments}
Figure~\ref{fig:declang:syntax} defines the elements $d$
of the domain \dom 
of integers, booleans, and data constructors that wrap
elements from \dom.
%
The domain \dom also contains a constant $c_w$
for each value $w$ of \undeclang that does 
not otherwise belong in \dom (\eg functions or other primitives).
%
An \emph{assignment} $\sigma$ is a map from variables 
to $\dom$.

%% \NV{DONE:Can we just drop the measure + axioms step?}
\mypara{Satisfiability \& Validity}
We interpret boolean predicates in the logic over the domain 
\dom.
%
We write \lmodels{\sigma}{p} %(resp. \umodels{\sigma}{p}) 
if $\sigma$ is a model of $p$.
%% where the interpretations of 
%%measures respect the measure axioms (resp. measures are 
%%uninterpreted).
%
We omit the formal definition for space.
%
A predicate $p$ is \emph{satisfiable}  %(resp. \emph{u-satisfiable}) 
if there \emph{exists} $\lmodels{\sigma}{p}$. %(resp. $\umodels{\sigma}{p}$).
%
A predicate $p$ is \emph{valid} %(resp. \emph{u-valid}) 
if \emph{for all} assignments $\lmodels{\sigma}{p}$. % (resp. \umodels{\sigma}{p}).
%
%Note that if $p$ is u-valid then $p$ is trivially valid.


%\begin{lemma}\label{ref:valid} If $p$ is u-valid then $p$ is valid.
%\end{lemma}

\mypara{Connecting Evaluation and Logic}
To prove soundness, we need to formally connect the
notion of logical models with the evaluation of a 
refinement to \etrue.
%
We do this in several steps, briefly outlined for brevity
(the detailed proof is in~\cite{vazou14techrep}).
%
First, we introduce a primitive \emph{bottom expression} 
\ebot that can have \emph{any} \Div type, but does not evaluate.
%%(\NV{only values can be applied to primitive constants, 
%%so $c\ \ebot$ will just not evaluate.}
%%\ie yields \ebot when applied to any primitive constant).
%
Second, we define \emph{lifted values} \botv 
(Figure~\ref{fig:declang:syntax}), which are values that
contain \ebot.
%
Third, we define \emph{lifted substitutions} 
$\botsto$, which are mappings from variables to 
lifted values.
%
Finally, we show how to \emph{embed} a lifted substitution 
\botsto into a \emph{set of} assignments \embed{\botsto} 
where, intuitively speaking, each \ebot is replaced by
some arbitrarily chosen element of \dom.
%
%Equipped with the above notions, we prove the following which
%connects evaluation and logical satisfaction.
Now, we can connect evaluation and logical satisfaction:
%
\begin{theorem}\label{thm:equiv}
If \dechastype{\emptyset}{\botsto(p)}{\FinTy{\tbool}}, then
$\evals{\botsto(p)}{\etrue}\ 
\mbox{iff}\ \ 
\forall \sigma \in \embed{\botsto}. \lmodels{\sigma}{p}$.
\end{theorem}

\mypara{Restricting Refinements to Predicates}
Our goal is to restrict \rimpl so that only predicates 
from the decidable logic \logiclang (not arbitrary expressions)
appear in implications \isimplied{\Env}{\p_1}{\p_2}.
%
Towards this goal, as shown in Figures~\ref{fig:declang:syntax}
and~\ref{fig:declang:typing}, 
we restrict the syntax and well-formedness of types to contain
only predicates
%
and we convert the program to ANF after which we can 
restrict the application rule \rtappd to applications 
to variables, which ensures that refinements remain 
within the logic after substitution~\cite{LiquidPLDI08}.
%
Recall, that this is not enough to ensure that refinements do converge, 
as under lazy evaluation,
even binders can refer to potentially divergent values.

\subsection{Stratified Types}\label{sec:typing:stratify}

The typing rules for \declang are given in Figure~\ref{fig:declang:typing}.
Instead of \emph{explicitly} reasoning about divergence or 
strictness in the refinement logic, which leads to significant
theoretical and practical problems, as discussed in \Sref{sec:refinedhaskell:conclusion}, 
we choose to reason \emph{implicitly} about divergence within the type system.
%
Thus, the second critical step in our path to \declang is the 
stratification of types into those inhabited by potentially
diverging terms, terms that only reduce to values, and 
terms which reduce to finite values.
%
Furthermore, the stratification crucially allows us to prove 
Theorem~\ref{thm:equiv}, which requires that refinements do 
not diverge (\eg by computing the length of an infinite list)
by ensuring that inductively defined measures are only applied 
to finite values.
%
Next, we describe how we stratify types with labels and 
then type the various constants, in particular the fixpoint 
combinators, to enforce stratification.

\mypara{Labels}
We specify stratification using two \emph{labels} for types.
%
The label \trivial\ (resp. \finite) is assigned to types given 
to expressions that reduce (using $\beta$-reduction from Figure~\ref{fig:operational})  
to a value $w$ (resp. \emph{finite} value,
\ie an element of the inductively defined \dom).
%
Formally,
%
\begin{align}
  \mbox{\emphbf{\Wnf types}} \quad 
\interp{\tref{\vv}{\Base}{\trivial}{r}} \defeq & 
    \interp{\tref{\vv}{\Base}{}{r}} \cap \{ e \mid \evals{e}{w} \}
    \label{eq:trivial} \\
  \mbox{\emphbf{\Fin types}} \quad 
\interp{\tref{\vv}{\Base}{\finite}{r}} \defeq & 
    \interp{\tref{\vv}{\Base}{}{r}} \cap \{ e \mid \evals{e}{d} \} 
    \label{eq:finite} 
\end{align}
%
Unlabelled types are assigned to expressions that may diverge.
%
Note that for any $\Base$ and refinement $r$ we have
$$
\interp{\tref{\vv}{\Base}{\finite}{r}} \subseteq
\interp{\tref{\vv}{\Base}{\trivial}{r}} \subseteq
\interp{\tref{\vv}{\Base}{}{r}}
$$ 
%
The first two sets are \emph{equal} for $\tint$ and $\tbool$, 
and \emph{unequal} for (lazily) constructed data types $T$. 
%
We need not stratify function types (\ie they are \Div types)
as binders with function types do not appear inside the VC, and 
are not applied to measures.

\mypara{Enforcing Stratification}\label{sec:typing:termination}
%
We enforce stratification in two steps.
%
First, the $\rtcased$ rule uses the operational semantics of case-of to 
type-check each case in an environment where the scrutinee
$x$ is assumed to have a \Wnf type. 
%
All the other rules, not mentioned in Figure~\ref{fig:declang:typing},
remain the same as in Figure~\ref{fig:refinedhaskell:typing}.
%
Second, we create stratified variants for the primitive constants 
and \emph{separate} fixpoint combinator constants for 
(arbitary, potentially non-terminating) recursion (\efix{}) 
and bounded recursion (\etfix{}).

\mypara{Stratified Primitives}
First, we restrict the primitive operators whose output 
types are refined with logical operators, so they are only 
invoked on finite arguments (so that the corresponding 
refinements are guaranteed to not diverge).
%
\begin{align*}
  \constty{\mathtt{\n}} \defeq & \tlref{\vv}{\tint}{\finite}{\vv = \n} \\
  \constty{\mathtt{=}} \defeq & \tfun{x}
                                     {\tbase^\finite}
                                     {\tfun{y}
                                           {\tbase^\finite}
                                           {\tlref{\vv}{\tbool}{\finite}{\vv \Leftrightarrow \x = \y}}} \\
  \constty{\mathtt{+}} \defeq & \tfun{x}
                                     {\tint^\finite}
                                     {\tfun{y}
                                           {\tint^\finite}
                                           {\tlref{\vv}{\tint}{\finite}{\vv = \x + \y}}}\\
  \constty{\mathtt{\land}} \defeq & \tfun{x}
                                         {\tbool^\finite}
                                         {\tfun{y}
                                               {\tbool^\finite}
                                               {\tlref{\vv}{\tbool}{\finite}{\vv \Leftrightarrow \x \land \y}}}
\end{align*}
%
It is easy to prove that the above primitives respect 
their stratification labels, \ie belong in the denotations
of their types. 

Note that the above types are restricted in that they can only be applied to finite arguments.
%
In future work~\ref{chapter:conclusion}, we could address this issue with unrefined versions of primitive types
that soundly allow operation on arbitrary arguments.
%
For example, with the current type for $\mathtt{+}$, addition of potentially 
diverging expressions is rejected.
%
Thus, we could define an unrefined signature 
\begin{align*}
  \constty{\mathtt{+}} \defeq & \tfun{x}
                                     {\tint}
                                     {\tfun{y}
                                           {\tint}
                                           {\tint}}
\end{align*}
and allow the two types of $\mathtt{+}$ to co-exist (as an intersection type),
where the type checker would choose the precise refined type 
if and only if both of $\mathtt{+}$'s arguments are finite.

\mypara{Diverging Fixpoints ($\efix{\typ}$)}
Next, note that the only place where divergence enters the picture is 
through the fixpoint combinators used to encode recursion. 
%
For any function or basic type $\typ \defeq \typ_1 \rightarrow \ldots \rightarrow \typ_n$,
we define the \emph{result} to be the type $\typ_n$.

For each $\typ$ whose result is a \Div type, 
there is a \emph{diverging fixpoint} combinator
$\efix{\typ}$, such that
%
\begin{align*}
  \ceval{\efix{\typ}}{f} \defeq & f\ (\efix{\tau}\ f) \\
  \constty{\efix{\typ}}   \defeq & (\typ \rightarrow \typ) \rightarrow \typ
\end{align*}
%
\ie, $\efix{\typ}$ yields recursive functions of type $\typ$.
Of course, $\efix{\typ}$ belongs in the denotation of
its type~\cite{PLC} \emph{only if} the result type is 
a \Div type (and \emph{not} when the result is a 
\Wnf or \Fin type). 
%
Thus, we restrict diverging fixpoints to functions with \Div result types.

\mypara{Indexed Fixpoints ($\tfixn{\typ}{n}$)}
For each type $\typ$ whose result is a \Fin 
type, we have a family of \emph{indexed} 
fixpoints combinators $\tfixn{\typ}{n}$: 
%
\begin{align*}
  \ceval{\tfixn{\typ}{n}}{f} \defeq & {\efun{m}{}{f\ m\ (\tfixn{\typ}{m}\ f)}}\\
%% (n:nat → τ_n → τ) → τ_n
\constty{\tfixn{\typ}{n}} \defeq & 
  \tfunbasic{(\tfun{n}
                   {\FinTy{\tnat}}
                   {\tfunbasic{\decr{\typ}{n}}{\typ}})}
            {\decr{\typ}{n}} \\
\mbox{where,}\ \decr{\typ}{n} \defeq & \tfunbasic{\tref{v}{\tnat}{\finite}{v < n}}{\typ}
\end{align*}
%
$\decr{\typ}{n}$ is a \emph{weakened} version of $\typ$ 
that can only be invoked on inputs \emph{smaller} than $n$.
%
Thus, we enforce termination by requiring that $\tfixn{\typ}{n}$ is 
\emph{only} called with $m$ that are \emph{strictly smaller than} $n$. 
% 
As the indices are well-founded $\tnat$s, evaluation will terminate. 
  
\mypara{Terminating Fixpoints ($\etfix{\typ}$)}
Finally, we use the indexed combinators to define the 
\emph{terminating} fixpoint combinator $\etfix{\typ}$ as:
\begin{align*}
  \ceval{\etfix{\typ}}{f} \defeq & {\efun{n}{}{f\ n\ (\tfixn{\typ}{n}\ f)}}\\
%% (n:nat → τ_n → τ) → m:nat⇓ → τ
\constty{\etfix{\typ}} \defeq & 
  \tfunbasic{(\tfun{n}
                   {\FinTy{\tnat}}
                   {\tfunbasic{\decr{\typ}{n}}{\typ}})}
            {\tfunbasic{\FinTy{\tnat}}{\typ}}
\end{align*}
%
Thus, the top-level call to the recursive function
requires a $\FinTy{\tnat}$ parameter $n$ that acts
as a \emph{starting} index, after which, all ``recursive''
calls are to combinators with \emph{smaller} indices, 
ensuring termination.

\mypara{Example: Factorial} 
Consider the factorial function:
$$\efac \defeq 
\efun{n}{}{\efun{f}{}{
	\ecaseinstance{\_}{(n = 0)}{
		 \begin{array}{l}
			 \ealt{\etrue}{1}\\ 
			 \ealt{\_}{n \times f (n-1)}	
		\end{array}
	}
}}$$
Let $\typ \defeq {\FinTy{\tnat}}$. 
We prove termination by typing 
\begin{align*}
  \emptyset \vdash_D \etfix{\typ}\ \efac : &\ \tfunbasic{\FinTy{\tnat}}{\typ} \\
\intertext{To understand \emph{why}, note that $\tfixn{\typ}{n}$ is 
only called with arguments strictly smaller than $n$ 
}
% fix fac n 
\etfix{\typ}\ \efac\ n 
%   = fixn fac n
% \stepsym^* & \ \tfixn{\typ}{n}\ \efac\ n\\
%   = fac n (fixn fac)
\stepsym^* & \ \efac\ n\ (\tfixn{\typ}{n}\ \efac)\\
%   = n * (fixn fac (n-1))
\stepsym^* & \ n \times (\tfixn{\typ}{n}\ \efac\ (n-1))\\
%   = n * (fac (n-1) (fix_n-1 fac))
\stepsym^* & \ n \times (\efac\ (n-1)\ (\tfixn{\typ}{n-1}\ \efac))\\
%   = n * (n-1) * (fix_n-1 fac (n-2))
\stepsym^* & \ n \times n-1 \times (\tfixn{\typ}{n-1}\ \efac\ (n-2))\\
%   = n * (n-1) * (fac (n-2) (fix_n-2 fac))
%   = n * (n-1) * (n-2) * (fix_n-2 fac (n-3))
%   = n * (n-1) * ... * (fix_1 fac 0)
\stepsym^* & \ n \times n-1 \times \ldots \times (\tfixn{\typ}{1}\ \efac\ 0)\\
\stepsym^* & \ n \times n-1 \times \ldots \times (\efac\ 0\ (\tfixn{\typ}{0}\ \efac))\\
%   = n * (n-1) * ... * 1
\stepsym^* & \ n \times n-1 \times \ldots \times 1
\end{align*}

\mypara{Soundness of Stratification}
To formally \emph{prove} that stratification is soundly 
enforced, it suffices to prove that the Denotation 
Lemma~\ref{lem:denotation} holds for \declang.
%
This, in turn, boils down to proving that each 
(stratified) constant belongs in its type's denotation, 
\ie each $c \in \interp{\constty{c}}$
or that the Lemma~\ref{lemma:constants} holds for \declang.
%
The crucial part of the above is proving that 
the indexed and terminating fixpoints inhabit 
their types' denotations.
%

\begin{theorem}{[Fixpoint Typing]} 
\begin{itemize}
  \item $\efix{\typ} \in \interp{\constty{\efix{\typ}}}$, 
  \item $\forall n. \etfixn{\typ}{f}{n} \in \interp{\constty{\etfixn{\typ}{f}{n}}}$,
  \item $\etfix{\typ} \in \interp{\constty{\etfix{\typ}}}$.
\end{itemize}
\end{theorem}

With the above we can prove soundness of Stratification as a corollary 
Denotation Lemma~\ref{lem:denotation}, given the interpretations of 
the stratified types. 
%
\begin{corollary}{[Soundness of Stratification]}\label{cor:stratification} 
\begin{enumerate}
  \item If $\dechastype{\emptyset}{e}{\FinTy{\typ}}$, then  evaluation of $e$ is finite.
  \item If $\dechastype{\emptyset}{e}{\WnfTy{\typ}}$, then  $e$ reduces to WHNF.
  \item\label{col:ref} If $\dechastype{\emptyset}{e}{\ttreft{\vv}{\typ}{\p}}$, then \p cannot diverge.
\end{enumerate}
\end{corollary}

Finally, as a direct implication the well-formedness rule \rwbased 
we conclude \ref{col:ref}, \ie that refinements cannot diverge.

\subsection{Verification With Stratified Types}\label{sec:typing:vc}

We can put the pieces together to obtain an algorithmic implication 
rule \rtdimp instead of the undecidable \rimpl (from Figure~\ref{fig:refinedhaskell:typing}).
%
Intuitively, each closing substitution $\sto$ will correspond to 
a set of logical assignments $\embed{\sto}$. 
%
Thus, we will translate $\Env$ into logical
formula $\embed{\Env}$ and 
denotation inclusion into logical implication 
such that:
%
\begin{itemize}
\item $\sto \in \interp{\Env}$ \textit{iff} all $\sigma \in \embed{\sto}$ 
     satisfy $\embed{\Env}$, and 
%%\item $\evals{\sto(p_i)}{\etrue}$ iff each $\sigma \in \embed{\sto}$ 
%%     satisfies $p_i$, and
\item ${\sto}{\ttreft{\vv}{\mathit{B}}{\p_1}}\subseteq {\sto}{\ttreft{\vv}{\mathit{B}}{\p_2}}$ 
		\textit{iff} all $\sigma \in \embed{\sto}$ satisfy $p_1 \Rightarrow p_2$.
\end{itemize} 

%%Thus, we will translate $\Env$, $p_1$, and $p_2$ into logical
%%formulas $\embed{\Env}$, $\embed{p_1}$, and $\embed{p_2}$ such 
%%that:
%%%
%%\begin{itemize}
%%\item $\evals{\sto(p_i)}{\etrue}$ iff each $\sigma \in \embed{\sto}$ 
%%     satisfies $\embed{p_i}$, and
%%\item $\sto \in \interp{\Env}$ iff each $\sigma \in \embed{\sto}$ 
%%     satisfies $\embed{\Env}$.
%%\end{itemize}

\mypara{Translating Refinements \& Environments}
To translate environments into logical formulas, recall that $\sto \in \interp{\Env}$
iff for each $\tbind{x}{\typ} \in \Env$, we have 
$\sto(x) \in \interp{\thetasub{\sto}{\typ}}$. Thus, 
%
\begin{align*}
\embed{\tbind{x_1}{\typ_1},\ldots,\tbind{x_n}{\typ_n}} \defeq &
  \embed{\tbind{x_1}{\typ_1}} \wedge \ldots \wedge \embed{\tbind{x_n}{\typ_n}}\\
\intertext{How should we translate a single binding? Since a binding denotes}
\interp{\tref{x}{\ttbase}{}{p}} \defeq &
  \{ e \mid \mbox{if } \evals{e}{w}\ \mbox{then}\ \evals{\SUBST{p}{x}{w}}{\etrue} \}\\
\intertext{a direct translation would require a logical value 
  predicate $\isvalue{x}$, which we could use to obtain the logical 
  translation}
\embed{\tref{x}{\ttbase}{}{p}} \defeq & \lnot \isvalue{x} \vee p
\end{align*}
%
This translation poses several theoretical and 
practical problems that preclude the use of existing 
SMT solvers (as detailed in \Sref{sec:conclusion}). 
However, our stratification guarantees 
(cf. (\ref{eq:trivial}), (\ref{eq:finite}))
that labeled types reduce to values and 
so we can simply conservatively translate 
the \Div and labeled (\Wnf, \Fin) bindings as:
$$\embed{\tref{x}{\ttbase}{}{p}}\ \defeq\ \etrue \qquad  \embed{\tref{x}{\ttbase}{l}{p}}\ \defeq\ p$$


\mypara{Soundness} We prove soundness by showing that
the decidable implication \rtdimp approximates 
the undecidable \rimpl. 

\begin{theorem}{}\label{thm:approximation} If \decissubref{\Env}{p_1}{p_2} is
  valid then 
  $\undecissubtype{\Env}{\ttreft{\vv}{\Base}{\p_1}}{\ttreft{\vv}{\Base}{\p_2}}$.
\end{theorem}

%% \begin{theorem}{[Approximation]}\label{thm:approximation} Let $\VC \defeq \VCOND{\Env}{p_1}{p_2}$.
%%   \begin{itemize}
%%     \item If \VC is u-valid then \VC is valid, 
%%     %\item If \VC is valid then \isimplied{\Env}{p_1}{p_2}, 
%%     \item If \decissubref{\Env}{p_1}{p_2} then \isimplied{\Env}{p_1}{p_2}.
%%   \end{itemize}
%% \end{theorem}

To prove the above, let ${\VC \defeq \decissubref{\Env}{p_1}{p_2}}$. 
We prove that if the \VC is valid then 
\undecissubtype{\Env}{\ttreft{\vv}{\base}{\p_1}}{\ttreft{\vv}{\base}{\p_2}}.
%%
%%First, note that
%%if $\VC$ is u-valid then it is valid as the addition of axioms preserves
%%validity. Next, we prove that if the \VC is valid then 
%%\undechastype{\Env}{\ttreft{\vv}{\base}{\p_1}}{\ttreft{\vv}{\base}{\p_2}}.
%
This fact relies crucially on a notion of \emph{tracking evaluation} 
which allows us to reduce a closing substitution $\sto$ to a lifted substitution 
$\botsto$, written $\trackevals{\sto}{\botsto}$, after which we prove:

\begin{lemma}{[Lifting]} 
$\evals{\sto(e)}{c}$ iff $\exists \trackevals{\sto}{\botsto}$ s.t. $\evals{\botsto(e)}{c}$.
\end{lemma}

We combine the Lifting Lemma and the equivalence Theorem~\ref{thm:equiv} 
to prove that the validity of the \VC demonstrates 
the denotational containment
$\forall \sto\in \interp{\Env}. 
  		 \interp{\thetasub{\sto}{\ttreft{\vv}{\miBase}{\p_1}}} 
  		$ $\subseteq$ $\interp{\thetasub{\sto}{\ttreft{\vv}{\miBase}{\p_2}}}$.
%
The soundness of algorithmic typing follows from
Theorems~\ref{thm:approximation} and~\ref{thm:refinedhaskell:safety}:

\begin{theorem} {[Soundness of \declang]} 
\begin{itemize}
\item\textbf{Approximation:} If \dechastype{\emptyset}{e}{\typ} then
  \undechastype{\emptyset}{e}{\typ}.
\item\textbf{Crash-Freedom:} If \dechastype{\emptyset}{e}{\typ} 
        then $\evals{e\not}{\ecrash}$.
\end{itemize}
\end{theorem}

To prove approximation we need to prove that Lemma~\ref{lemma:constants} holds for
each constant, and thus it holds for data 
constructors.
In the metatheory we assume a stronger notion 
of validity that respects the measure axioms. 
%
%\RJ{CHECK}
However, since our implementation does not use axioms and instead, 
without loss of precision, treats measures as uninterpreted 
during SMT validity checking, we omit further discussion of axioms 
for clarity.


%% \NV{DONE: remove Uninterpreted Validity Checking}
%%\mypara{Uninterpreted Validity Checking}
%%We reduce the undecidable implication 
%%to checking u-validity of the \VC, \ie, validity 
%%where measures are uninterpreted.
%%%
%%This trivially implies validity of the \VC.
%%%
%%Recall that we use u-validity instead of validity to ensure 
%%predictable and efficient checking, as then the 
%%\VC belongs to \logiclang.
%%%
%%The elimination of the measure axioms is crucial 
%%for efficient and practical verification. 
%%%
%%% As demonstrated by our evaluation (\Sref{sec:evaluation}),
%%The absence of the axioms does not lead to loss 
%%of precision in practice; the semantics of the 
%%axioms are encoded in the refinement types of 
%%the data constructors, and hence already \emph{instantiated} 
%%inside (the environment and) the VC during type checking.
%%

%%% Local Variables: 
%%% mode: latex
%%% TeX-master: "main"
%%% End: 

% \newcommand\rectyp{\FinTy{\tnat}}

\newcommand\factyp{\typ}

\section{Implementation in \toolname}\label{sec:haskell}

We have implemented \declang in \toolname. 
% 
In \S~\ref{sec:termination} we saw real world termination checks.
%
Here claim soundness of \toolname's termination checker, 
as the checker derives as a the transition from \declang to \lhaskell.

\subsection{Termination}

%Let ${\factyp \defeq \tfunbasic{\FinTy{\tnat}}{\FinTy{\tnat}}}$.
Haskell's recursive functions of type ${\tfunbasic{\FinTy{\tnat}}{\typ}}$ 
are represented, in GHC's Core \cite{SulzmannCJD07} as
${\mathtt{let\ rec}\ f = \ \efun{n}{}{e}}$ which is operationally
equivalent to ${\mathtt{let}\ f = \ \etfix{\typ}\ (\efun{n}{}{\efun{f}{}{e}})}$.
Given the type of $\etfix{\typ}$, checking that $f$ has 
type $\tfunbasic{\FinTy{\tnat}}{\typ}$ reduces to checking
$e$ in a \emph{termination-weakened environment} where
$$\tbind{f}{\tfunbasic{\tref{v}{\tnat}{\finite}{v < n}}{\typ}}$$
%
Thus, \toolname proves termination just as \declang 
does: by checking the body in the above environment, 
where the recursive binder is called with $\tnat$ 
inputs that are strictly smaller than $n$.

\mypara{Default Metric}
For example, \toolname proves that 
%
\begin{code}
  fac n = if n == 0 then 1 else n * fac (n-1)
\end{code}
%
has type $\tfunbasic{\FinTy{\tnat}}{\FinTy{\tnat}}$ 
by typechecking the body of @fac@ 
in a termination-weakened environment 
${\mathtt{fac}\ : \tfunbasic{\tref{\ttv}{\tnat}{\finite}{\ttv < \ttn}}{\FinTy{\tnat}}}$.
The recursive call generates the query:
\begin{align*}
\tbind{\ttn}{\{0 \leq \ttn\}}, \lnot (\ttn = 0) \vdash_D &\  \subt{\ttref{v=n-1}}{\ttref{0 \leq v \wedge v < n}}\\ 
%\tlref{n}{\tint}{\finite}{0 \leq n}, \lnot (n = 0) \vdash_D &\  \subt{\ttref{v=n-1}}{\ttref{0 \leq v \wedge v < n}}\\ 
\intertext{Which reduces to the valid VC:}
0 \leq \ttn \wedge \lnot (\ttn = 0) \Rightarrow &\   (\ttv = \ttn-1) \Rightarrow (0 \leq \ttv \wedge \ttv < \ttn)
\end{align*}
%
proving that $\mathtt{fac}$ terminates, in essence because the
\emph{first parameter} forms a \emph{well-founded decreasing metric}.

\mypara{Refinements Enable Termination} 
Consider Euclid's GCD:
%
\begin{code}
  gcd :: a:Nat -> {v:Nat | v < a} -> Nat 
  gcd a 0 = a
  gcd a b = gcd b (a `mod` b)
\end{code}
%
Here, the first parameter is decreasing, but this requires
the fact that the second parameter is smaller than the first 
and that @mod@ returns results smaller than its second 
parameter. Both facts are easily expressed as refinements, 
but elude non-extensible checkers~\cite{Giesl11}.

\mypara{Explicit Termination Metrics}
The indexed-fixpoint combinator technique is easily extended to 
cases where some parameter \emph{other} than the first is the 
well-founded metric. For example, consider: 
% As an example, consider the tail-recursive factorial:
%
\begin{code}
  tfac     :: Nat -> n:Nat -> Nat / [n] 
  tfac x n | n == 0    = x
           | otherwise = tfac (n*x) (n-1)
\end{code}
%
We specify that the \emph{last parameter} is decreasing by 
specifying an explicit termination metric @/ [n]@ in the 
type signature.
%
\toolname \emph{desugars} the 
termination metric into a new $\tnat$-valued \emph{ghost parameter} @d@ 
whose value is always equal to the termination metric @n@:
\begin{code}
  tfac :: d:Nat -> Nat -> {n:Nat | d = n} -> Nat 
  tfac d x n | n == 0    = x
             | otherwise = tfac (n-1) (n*x) (n-1)
\end{code}
%
Type checking, as before, checks the body in an environment where 
the first argument of @tfac@ is weakened, \ie, requires proving @d > n-1@.
%
So, the system needs to know that the ghost argument @d@ 
represents the decreasing metric.
%
We capture this information in the type signature of @tfac@ where the \emph{last} 
argument exactly specifies that @d@ is the termination metric @n@, \ie, @d = n@.
%
Note that since the termination metric can depend on any argument, 
it is important to refine the last argument,
so that all arguments are in scope, with the fact that @d@ is the termination metric.

To generalize, desugaring of termination metrics proceeds as follows.
Let $f$ be a recursive function with parameters $\overline{x}$ and 
termination metric $\mu(\overline{x})$. Then \toolname will
\begin{itemize}
\item add a $\tnat$-valued ghost first parameter $d$ in the definition of $f$, 
\item weaken the last argument of $f$ with the refinement $d = \mu(\overline{x})$, %and
\item at each recursive call of $f\ \overline{e}$, 
apply $\mu(\overline{e})$ as the first argument.
\end{itemize}  
%
%%As will shall see this technique can be used 
%%when the termination metric is any logical expression.

\mypara{Explicit Termination Expressions} 
Let us now apply the previous technique in a function where
none of the parameters themselves decrease across recursive calls,
but there is some \emph{expression} that forms the decreasing metric.
%
%Sometimes, none of the parameters themselves decrease across recursive calls,
%but there is some \emph{expression} that forms the decreasing metric.
%
Consider @range lo hi@ (as in~\S~\ref{sec:termination}), which returns the list of 
@Int@s from @lo@ to @hi@:
%
We generalize the explicit metric specification to 
\emph{expressions} like @hi-lo@. \toolname \emph{desugars} the 
expression into a new $\tnat$-valued \emph{ghost parameter} 
whose value is always equal to @hi-lo@, that is:
\begin{code}
  range :: lo:Nat -> {hi:Nat | hi >= lo} -> [Nat] 
         / [hi-lo]
  range lo hi 
    | lo < hi = lo : range (lo + 1) hi
    | _       = [] 
\end{code}
%
Here, neither parameter is decreasing (indeed, the first one
is \emph{increasing}) but @hi-lo@ decreases across each call. 
%
We generalize the explicit metric specification to 
\emph{expressions} like @hi-lo@. \toolname \emph{desugars} the 
expression into a new $\tnat$-valued \emph{ghost parameter} 
whose value is always equal to @hi-lo@, that is:
%
\begin{code}
  range lo hi = go (hi-lo) lo hi
    where 
      go :: d:Nat -> lo:Nat 
         -> {hi:Nat | d = hi - lo} -> [Nat]
      go d lo hi
       | lo < hi = l : go (hi-(lo+1)) (lo+1) hi 
       | _       = []
\end{code}
%
After which, it proves @go@ terminating, by showing 
that the first argument @d@ is a \tnat that decreases across each 
recursive call (as in @fac@ and @tfac@).

\mypara{Recursion over Data Types}
The above strategy generalizes easily to functions that recurse
over (finite) data structures like arrays, lists, and trees.
In these cases, we simply use \emph{measures} to project the 
structure onto \tnat, thereby reducing the verification to 
the previously seen cases. For each user defined type, \eg
%
\begin{code}
  data L [sz] a = N | C a (L a)
\end{code}
%
we can define a \emph{measure}
%
\begin{code}
  measure sz  :: L a -> Nat
    sz (C x xs) = 1 + (sz xs)
    sz N        = 0
\end{code}
%
We prove that @map@ terminates using the type:
%
\begin{code}
  map :: (a -> b) -> xs:L a -> L b / [sz xs]
  map f (C x xs) = C (f x) (map f xs)
  map f N        = N
\end{code}
%
That is, by simply using @(sz xs)@  as the 
decreasing metric.

\mypara{Generalized Metrics Over Datatypes}
Finally, in many functions there is no single argument 
whose (measure) provably decreases. For example, consider:
%
\begin{code}
  merge :: xs:L a -> ys:L a -> L a / [sz xs + sz ys]
  merge (C x xs) (C y ys)
    | x < y     = x `C` (merge xs  (y `C` ys))
    | otherwise = y `C` (merge (x `C` xs)  ys)
\end{code}
%
from the homonymous sorting routine. Here, neither parameter
decreases, but the \emph{sum} of their sizes does. 
%
As before \toolname desugars the decreasing expression into 
a ghost parameter and thereby proves termination (assuming, 
of course, that the inputs were finite lists, \ie 
$\FinTy{\mathtt{L}}\ a$).

\mypara{Automation: Default Size Measures}
Structural recursion on the first argument is a common pattern 
in \lhaskell code.
%
\toolname automates termination proofs for this common case,
by allowing users to specify a \emph{size measure} 
for each data type, (\eg @sz@ for @L a@).
%
Now, if \emph{no} termination metric is given, by default 
\toolname assumes that the \emph{first} argument whose type
has an associated size measure decreases.
%
Thus, in the above, we need not specify metrics for @fac@ 
or @gcd@ or @map@ as the size measure is automatically 
used to prove termination. 
%
This simple heuristic allows us to {automatically}
prove 67\% of recursive functions terminating.

%%% \mypara{Summary}
%%% To sum up, 
%%% %
%%% \begin{itemize}
%%% \item No termination check for functions marked @lazy@, 
%%% \item If no explicit termination metrice, then first 
%%%       argument with size measure used by default,
%%% \item Otherwise, explicit termination metric desugared 
%%%       into ghost @nat@ parameter that is used to prove 
%%%       termination.
%%% \end{itemize}

\subsection{Non-termination}

By default, \toolname checks that every function is 
terminating. We show in \Sref{sec:refinedhaskell:evaluation} that 
this is in fact the overwhelmingly common case in practice.
%
However, annotating a function as @lazy@ deactivates 
\toolname's termination check (and marks the result as a 
\Div type).
%
This allows us to check functions that are 
non-terminating, and allows \toolname to prove safety 
properties of programs that manipulate \emph{infinite} 
data, such as streams, which arise idiomatically with 
Haskell's lazy semantics.
% 
For example, consider the classic @repeat@ function:
%
\begin{code}
  repeat x = x `C` repeat x
\end{code}
%
We cannot use the $\etfix{}$ combinators to 
represent this kind of recursion, and hence, 
use the non-terminating $\efix{}$ combinator 
instead. 
%
I \toolname, we use the @lazy@ keyword to denote 
potentially diverging definitions 
defined using the non-terminating $\efix{}$ combinator.


\begin{comment}
Abstract Streams

Let us see how we can use refinements to statically 
distinguish between finite and infinite streams. 
The direct, \emph{global} route of using a measure
%
\begin{code}
  measure inf    :: L a -> Prop 
    inf (C x xs) = inf xs
    inf N        = false 
\end{code}
%
to describe infinite lists is unavailable as such 
a measure, and hence, the corresponding refinement
would be non-terminating.
%
Instead, we describe infinite lists in \emph{local} 
fashion, by stating that each \emph{tail} is non-empty.

\mypara{Step 1: Abstract Refinements} 
We can parametrize a datatype with abstract 
refinements that relate sub-parts of the 
structure \cite{vazou13}. 
For example, we parameterize the list type as:
%
\begin{code}
  data L a <p :: L a -> Prop> 
    = N | C a {v: L<p> a | (p v)}
\end{code}
%
which parameterizes the list with a refinement 
@p@ which holds \emph{for each} tail of the list, 
\ie holds for each of the second arguments to 
the @C@ constructor in each sub-list.


\mypara{Step 2: Measuring Emptiness} Now, we can write a measure that 
states when a list is \emph{empty}
%
\begin{code}
  measure emp  :: L a -> Prop 
    emp N        = true
    emp (C x xs) = false
\end{code}
%
As described in \Sref{sec:typing}, \toolname translates the 
abstract refinements and measures into refined types for 
@N@ and @C@.

\mypara{Step 3: Specification \& Verification}
Finally, we can use the abstract refinements and measures to 
write a type alias describing a refined version of @L a@ 
representing infinite streams:
%
\begin{code}
  type Stream a = 
    {xs: L <{\v -> not(emp v)}> a | not(emp xs)}
\end{code}
%
We can now type @repeat@ as:
%
\begin{code}
  lazy repeat :: a -> Stream a
  repeat x    = x `C` repeat x 
\end{code}
%
The @lazy@ keyword \emph{deactivates} termination checking, and 
marks the output as a \Div type.
%
Even more interestingly, we can prove safety properties of 
infinite lists, for example:
%
\begin{code}
  take            :: Nat -> Stream a -> L a
  take 0 _        = N
  take n (C x xs) = x `C` take (n-1) xs
  take _ N        = error "never happens"
\end{code}
%
\toolname proves, similar to the @head@ example from
\Sref{sec:overview}, that we never match a @N@ when 
the input is a @Stream@.

\mypara{Finite vs. Infinite Lists}
%
Thus, the combination of refinements 
and labels allows our stratified type 
system to specify and verify whether 
a list is finite or infinite.
%
Note that:
%
$\FinTy{\mathtt{L}}\ a$ represents
\emph{finite} lists \ie those 
produced using the (inductive) 
terminating fixpoint combinators,
%
$\WnfTy{\mathtt{L}}\ a$ represents 
(potentially) infinite lists which 
are guaranteed to reduce to values, 
\ie non-diverging computations that
yield finite or infinite lists,
and
$\DivTy{\mathtt{L}}\ a$ represents 
computations that may diverge or 
produce a finite or infinite list.

\end{comment}

\subsection{User Specifications and Type Inference}

In program verification it is common that the user provides functional
specification that the code should satisfy.
In \toolname these specifications can be provided as type signatures 
for @let@-bound variables.
%
Consider the typechecking rules of Figure~\ref{fig:typing}
that is used by \declang.
%
$$
\inference{
	\hastype{\Gamma}{e_x}{\tau_{x}} &&
	\hastype{\Gamma,\tbind{x}{\tau_x}}{e}{\tau} &&
	\iswellformed{\Gamma}{\tau}
}{
	\hastype{\Gamma}{\elet{x}{e_x}{e}}{\tau}
}[\rtlet]
$$
%
Note that \rtlet \emph{guesses} an appropriate type $\tau_x$
for $e_x$ and binds it to $x$ to typecheck $e$.

\toolname allows the user to specify the type $\tau_x$ for top level bindings.
%
For every binding \elet{x}{e_x}{\dots}, if the user provides a type specification $\tau_x$,
\toolname checks using the appropriate environment 
(1)~that the specified type is well-formed and 
(2)~that the expression $e_x$ typechecks under the specification $\tau_x$.
%
For the other top level bindings, \ie those without user-provided specifications, 
as well as all local bindings, \toolname uses the Liquid Types~\citep{LiquidPLDI08} 
framework to infer refinement types, thus greatly reducing the number of annotations 
required from the user.



\section{Evaluation}\label{sec:realworld:evaluation}

% We implemented our technique by extending \toolname~\cite{vazou13}. 
% Next, we describe the tool, the benchmarks, 
% and a quantitative summary of our results.
% We then present a qualitative discussion 
% of how \toolname was used to verify safety, 
% termination, and functional correctness 
% properties of a large library, and 
% discuss the strengths and limitations 
% unearthed by the study.

% \subsection{Benchmarks and Results}

% Our goal was to use \toolname to verify a suite of real-world Haskell 
% programs, to evaluate whether our approach is
% %
% %\begin{itemize}
% %  \item 
%     \emph{efficient}  enough for large programs,
% %  \item 
%     \emph{expressive} enough to  specify key correctness properties, and
% %  \item 
%     \emph{precise}    enough to  verify idiomatic Haskell codes.
% %\end{itemize}
% %
% %\mypara{Benchmarks}
% Thus, we used these libraries as benchmarks:
% %
% \begin{itemize}
% \item \texttt{GHC.List} and \texttt{Data.List}, which together implement many standard
%       list operations; we verify various
%       size related properties,
% \item \texttt{Data.Set.Splay}, which implements a splay-tree
%       based functional set data type; we verify that all interface 
%       functions terminate and return well ordered trees,
% \item \texttt{Data.Map.Base}, which implements a functional 
%       map data type; we verify that all interface functions 
%       terminate and return binary-search ordered trees~\cite{vazou13}, 
% \item \libvectoralgos, which includes a suite of 
%       ``imperative'' (\ie monadic) array-based sorting algorithms; 
%       we verify the correctness of vector accessing, indexing, and slicing.
% \item \bytestring, a library for manipulating byte arrays, we
%       verify termination, low-level memory safety, and high-level
%       functional correctness properties\includeProof{(\ref{sec:bytestring})},
% \item \libtext, a library for high-performance unicode text 
%       processing; we verify various pointer safety and 
%       functional correctness properties (\ref{sec:text}),
%       during which we find a subtle bug.
% \end{itemize}
% %
% We chose these benchmarks as they represent a wide spectrum of idiomatic
% Haskell codes: the first three are widely used libraries based on recursive 
% data structures, the fourth and fifth perform subtle, low-level arithmetic 
% manipulation of array indices and pointers, and the last is a rich, high-level
% library with sophisticated application-specific invariants. 
% These last three libraries are especially representative as they pervasively 
% intermingle high level abstractions like higher-order loops, folds, and fusion, 
% with low-level pointer manipulations in order to deliver high-performance. 
% They are an appealing target for \toolname, as refinement types are an ideal 
% way to statically enforce critical invariants that are outside the scope of
% run-time checking as even Haskell's highly expressive type system.

We now present a quantitative evaluation of \toolname.

\begin{table*}[ht!]
\begin{scriptsize}
\caption[A quantitative evaluation of \toolname]{A quantitative evaluation of our experiments. 
  \textbf{Version} is version of the checked library.
  \textbf{LOC} is the number of non-comment lines of source code as reported by \texttt{sloccount}.
  \textbf{Mod}    is the number of modules in the benchmark and \textbf{Fun} is the number
                  of functions.
  \textbf{Specs}  is the number (/ line-count) of type specifications and aliases,
                  data declarations, and measures provided.
  \textbf{Annot}  is the number (/ line-count) of other annotations provided,
                  these include invariants and hints for
                  the termination checker.
  \textbf{Qualif} is the number (/ line-count) of provided qualifiers.
  \textbf{Time (s)}   is the time, in seconds, required to run \toolname. }
\label{table:realworldhaskell:results}
\centering
\begin{tabular}{|l|r|rrr|rrr|r|}
\hline
\textbf{Module} &\textbf{Version}           & \textbf{LOC} & \textbf{Mod} & \textbf{Fun} & \textbf{Specs} & \textbf{Annot} & \textbf{Qualif} & \textbf{Time (s)}\\
\hline\hline
\textsc{GHC.List} &      {7.4.1}    & 309          & 1            & 66           & 29 / 38        & 6 / 6          & 0 / 0           & 15 \\
\textsc{Data.List} &    {4.5.1.0}    & 504          & 1            & 97           & 15 / 26        & 6 / 6          & 3 / 3           & 11 \\
\hline
\textsc{Data.Map.Base} &   {0.5.0.0} & 1396         & 1            & 180          & 125 / 173      & 13 / 13        & 0 / 0           & 174 \\
\hline
\textsc{Data.Set.Splay} &  {0.1.1} & 149          & 1            & 35           & 27 / 37        & 5 / 5          & 0 / 0           & 27 \\
\hline
\textsc{HsColour} &{1.20.0.0}          & 1047         & 16           & 234          & 19 / 40        & 5 / 5          & 1 / 1           & 196 \\
\hline
\textsc{XMonad.StackSet} & {0.11}  & 256          & 1            & 106          & 74 / 213       & 3 / 3          & 4 / 4           & 27 \\
\hline
\textsc{ByteString} &    {0.9.2.1}   & 3505         & 8            & 569          & 307 / 465      & 55 / 55        & 47 / 124        & 294 \\
\hline
\textsc{Text}        &    {0.11.2.3}  & 3128         & 17           & 493          & 305 / 717      & 52 / 54        & 49 / 97         & 499 \\
\hline
\textsc{Vector-Algorithms}& {0.5.4.2} & 1218         & 10           & 99           & 76 / 266       & 9 / 9          & 13 / 13         & 89 \\
\hline
\textbf{Total}            & & 11512        & 56           & 1879         & 977 / 1975     & 154 / 156      & 117 / 242       & 1336 \\
\hline
\end{tabular}
\end{scriptsize}
\end{table*}


\subsection{Results}

We have used the following libraries as benchmarks:
%
\begin{itemize}
\item \texttt{GHC.List} and \texttt{Data.List}, which together implement many standard
      list operations; we verify various
      size related properties,
\item \texttt{Data.Set.Splay}, which implements a splay-tree
      based functional set data type; we verify that all interface 
      functions terminate and return well ordered trees,
\item \texttt{Data.Map.Base}, which implements a functional 
      map data type; we verify that all interface functions 
      terminate and return binary-search ordered trees~\cite{vazou13}, 
\item \texttt{HsColour}, a syntax highlighting program for Haskell code, we
      verify totality of all functions (\S~\ref{sec:totality}),
\item \texttt{XMonad}, a tiling window manager for X11, we verify the uniqueness
      invariant of the core datatype, as well as some of the QuickCheck
      properties (\S~\ref{sec:xmonad}),
\item \bytestring, a library for manipulating byte arrays, we
      verify termination, low-level memory safety, and high-level
      functional correctness properties (\S~\ref{sec:bytestring}),
\item \libtext, a library for high-performance unicode text 
      processing; we verify various pointer safety and 
      functional correctness properties (\S~\ref{sec:text}),
      during which we find a subtle bug,
\item \libvectoralgos, which includes a suite of 
      ``imperative'' (\ie monadic) array-based sorting algorithms; 
      we verify the correctness of vector accessing, indexing, and slicing \etc.
\end{itemize}

Table~\ref{table:realworldhaskell:results} summarizes our experiments, which covered 56 modules
totaling 11,512 non-comment lines of source code and 1,975 lines of specifications.
%
The results are on a machine with an Intel Xeon X5660 and 32GB of RAM~(no benchmark required more than 1GB.)
%
The upshot is that \toolname is very effective on real-world code bases.
%
The total overhead due to hints, \ie the sum of \bfAnnot and \bfQualif, is 3.5\% of \bfLOC.
%
The specifications themselves are machine checkable versions of the comments 
placed around functions describing safe usage and behavior, and required around
two lines to express on average.
%
% Our default metric, namely the first parameter with an associated size measure,
% suffices to prove \todonum\% of (recursive) functions terminating.
%
% \todonum\% require a hint (\ie the position of the decreasing argument) or a
% witness (\todonum\% required both), and the remaining \todonum\% were marked as potentially diverging.
%
% Of the \todonum functions marked as potentially diverging, we suspect \todonum
% actually terminate but were unable to prove so.
%
While there is much room for improving the running times, the tool is fast enough 
to be used interactively, verify a handful of API functions and associated helpers 
in isolation.

\subsection{Limitations}\label{sec:discussion}
%% \NV{
%% * When applying LiquidHaskell to a given library were any improvements to the tool necessary to complete the analysis, or was the tool sufficiently complete such that other users (not the maintainers) could have performed the same analysis?
%% }\NV{
%% * How practical is organizing and keeping source copies of all imported modules?
%% }\NV{
%% * When writing code, how can a developer know what constructs will present verification difficulties of the sort in HsColour?
%% }

Our case studies also highlighted several limitations
of \toolname. 
In most cases, we could alter the code slightly to 
facilitate verification. 

\mypara{Ghost parameters} are sometimes needed in 
order to materialize values that are not needed 
for the computation, but are necessary to prove 
various specifications. For example, the @piv@ 
parameter in the @append@ function for red-black trees
(\S~\ref{sec:redblack}).
%
Bounded Refinement Types (Chapter~\ref{boundedrefinements})
provide a complete, but unfortunately not elegant way to 
eliminated ghost parameters.

\mypara{Fixed-width integer and floating-point numbers}
\toolname uses the theories of linear arithmetic and
real numbers to reason about numeric operations. In some
cases this causes us to lose precision, \eg when we have to
approximate the behavior of bitwise operations. We could
address this shortcoming by using the theory of bit-vectors
to model fixed-width integers, but we are unsure of the
effect this would have on \toolname's performance.

\begin{comment}
\mypara{Higher-order functions} 
must sometimes be \emph{specialized} because the 
original type is not precise enough. 
%
For example, the \texttt{concat} function that
concatenates a list of input @ByteString@s %@xs@,
pre-allocates the output region by computing the 
total size of the input.
%
\begin{code}
  len = sum . map length $ xs
\end{code}
%$
Unfortunately, the type for @map@ is not sufficiently
precise to conclude that the value @len@ equals 
@bLens xs@, se we must manually specialize
the above into a single recursive traversal that 
computes the lengths.
%
Rather than complicating the type system with a very
general higher-order type for @map@ we suspect the 
best way forward will be to allow the user to specify
inlining in a clean fashion.
\end{comment}

\mypara{Functions as Data} Several libraries like \lbtext
encode data structures like (finite) streams using functions,
in order to facilitate fusion. Currently, it is not possible
to describe sizes of these structures using measures, as this
requires describing the sizes of input-output chains starting
at a given seed input for the function. In future work, it will 
be interesting to extend the measure mechanism to support 
multiple parameters (\eg a stream and a seed) in order to reason
about such structures. 

\mypara{Lazy binders} sometimes get in the way of verification. 
A common pattern in Haskell code is to define \emph{all} 
local variables in a single @where@ clause and use them 
only in a subset of all branches. 
%
\toolname flags a few such definitions as \emph{unsafe},
not realizing that the values will only be demanded in a
specific branch. 
%
Currently, we manually transform the code by pushing 
binders inwards to the usage site. % (via @let@).
This transformation could be easily automated.

\mypara{Assumes} 
which can be thought of as ``hybrid" run-time checks,
had to be placed in a couple of cases where the verifier 
loses information. 
%
One source is the introduction of assumptions about
mathematical operators that are currently conservatively 
modeled in the refinement logic (\eg that multiplication 
is commutative and associative). 
These may be removed by using more advanced non-linear 
arithmetic decision procedures.

\mypara{Error messages} are a crucial part of any type-checker.
Currently, we report error locations in the provided source
file and output the failed constraint(s). 
% 
In the future errors should be reported using an
interactive interface with features including 
code and type completion and counter example drive
error explanation. 


%% TECHREP
%% \RJ{CUT: No context} Another is the introduction 
%% of axioms about the semantics of uninterpreted 
%% measures, like @numchars@ used to model the 
%% number of (unicode) characters in a packed 
%% array~\cite{techreport}.
%% %
%% \RJ{CUT: Not enough context} Another source is current 
%% limitations in handling Haskell features like 
%% rank-2 polymorphism \eg like the @liquidAssume@ 
%% in @mapAccumL@ for @Text@.



%%% Local Variables: 
%%% mode: latex
%%% TeX-master: "main"
%%% End: 

\chapter{Related Work}\label{chapter:related}

\section{Refinement Types in Practice}\label{sec:realword:related}

Next, we situate \toolname with 
existing Haskell verifiers.

\spara{Dependent Types} are the basis of many verifiers, 
or more generally, proof assistants.
%
Verification of haskell code is possible with
``full'' dependently typed systems like Coq~\cite{coq-book}, 
Agda~\cite{norell07}, Idris~\cite{Brady13}, Omega~\cite{Sheard06}, and
 {$\lambda_\rightarrow$}~\cite{LohMS10}.
 %
 While these systems are highly expressive,
their expressiveness comes at the cost of making logical validity checking undecidable
thus rendering verification cumbersome.	
 %
 Haskell itself can be considered a dependently-typed language,
 as type level computation is allowed via 
 Type Families~\cite{McBride02},
 Singleton Types\cite{Weirich12}, 
 Generalized Algebraic  Datatypes (GADTs)~\cite{JonesVWW06, SchrijversJSV09}, 
 and type-level functions~\cite{ChakravartyKJ05}.
 %
Again, 
verification in haskell itself turns out to be quite painful~\cite{LindleyM13}.

\spara{Refinement Types} are a form of dependent types where 
invariants are encoded via a combination of types and predicates
from a restricted \emph{SMT-decidable} 
logic~\cite{Rushby98,pfenningxi98,Dunfield07,GordonTOPLAS2011}. 
%
\toolname uses Liquid Types~\cite{LiquidPLDI09} 
that restrict the invariants even more
to allow type inference, a crucial feature of a usable type system.
%
Even though the language of refinements is restricted,
as we presented, the combination of
Abstract Refinements~\cite{vazou13} 
with sophisticated measure definitions 
allows specification and verification of a wide variety 
of program properties.

\spara{Static Contract Checkers} 
like ESCJava~\cite{ESCJava} are a classical way of verifying 
correctness through assertions and pre- and post-conditions. 
%
\cite{XuPOPL09} describes a static contract checker for 
Haskell that uses symbolic execution to unroll procedures
upto some fixed depth, yielding weaker ``bounded'' soundness
guarantees.
% 

Similarly, Zeno~\cite{ZENO} is an automatic Haskell 
prover that combines unrolling with heuristics for rewriting
and proof-search. 
%%Based on rewriting, it is sound but 
%%``Zeno might loop forever'' when faced with 
%%non-termination.
%
Finally, the Halo~\cite{halo} contract checker encodes 
Haskell programs into first-order logic by directly 
modeling the code's denotational semantics,
again, requiring heuristics for instantiating axioms 
describing functions' behavior.
%


\spara{Totality Checking}
is feasible by GHC itself, via an option flag that warns of any incomplete patterns.
%
Regrettably, GHC's warnings are local, \ie
GHC will raise a warning for @head@'s partial definition, 
but not for its caller, as the programmer would desire.
%%(2)~ and preservative:
%%a warning will be raised for any incomplete pattern
%%without an attempt to reason if it is reachable or not.
%
Catch~\cite{catch}, 
a fully automated tool that tracks incomplete patterns,
addresses the above issue
%
by computing functions' pre- and post-conditions.
Moreover, catch statically analyses the code 
to track reachable incomplete patterns.
%
\toolname allows more precise analysis than catch, 
thus, by assigning the appropriate
types to $\star$Error functions (\S~\ref{sec:totality}) 
it tracks reachable incomplete patters 
%we get catch analysis
as a side-effect of verification.

\spara{Termination Analysis}
is crucial for \toolname's soundness 
and is implemented in a technique inspired by~\cite{XiTerminationLICS01}, 
%
Various other authors have proposed techniques to verify termination of
recursive functions, either using the ``size-change
principle''~\cite{JonesB04,Sereni05}, or by annotating types with size indices
and verifying that the arguments of recursive calls have smaller
indices~\cite{HughesParetoSabry96,BartheTermination}.
%
To our knowledge, none of the above analyses have been empirically
evaluated on large and complex real-world libraries.

AProVE~\cite{Giesl11} implements a powerful, fully-automatic
termination analysis for Haskell based on term-rewriting.
%
Compared to AProVE,
encoding the termination proof via 
refinements provides advantages that are crucial in 
large, real-world code bases. 
Specifically, refinements
let us
%
(1) prove termination over a subset 
    (not all) of inputs; many functions (\eg @fac@) 
    terminate only on @Nat@ inputs and not all @Int@s,
%
(2) encode pre-conditions, 
    post-conditions, and auxiliary invariants that 
    are essential for proving termination, (\eg @qsort@),
%
(3) easily specify non-standard 
    decreasing metrics and prove termination, (\eg @range@).
%
In each case, the code could be (significantly) 
\emph{rewritten} to be amenable to AProVE but this defeats
the purpose of an automatic checker.
%


\section{Refinement Types for Haskell}\label{sec:refinedhaskell:related}

Next we situate our work with closely related lines of research.

\spara{Dependent Types} are the basis of many verifiers, 
or more generally, proof assistants.
%
In this setting arbitrary terms may appear inside types,
so to prevent logical inconsistencies, and enable
the checking of type equivalence, all terms must
terminate.
%
``Full'' dependently typed systems like Coq~\cite{coq-book}, 
Agda~\cite{norell07}, and Idris~\cite{Brady13} typically use 
\emph{structural} checks where recursion is allowed on 
sub-terms of ADTs to ensure that \emph{all} terms terminate.
%
We differ in that, since the refinement logic is
restricted, we do not require that all functions terminate,
and hence, we can prove properties of possibly diverging 
functions like @collatz@ as well as lazy functions like @repeat@.
%
Recent languages like Aura~\citep{AURA} and Zombie~\citep{Zombie}
allow general recursion, but constrain the logic to a terminating 
sublanguage, as we do, to avoid reasoning 
about divergence in the logic.
%
In contrast to us, the above systems crucially assume 
\emph{call-by-value} semantics to ensure that binders are bound
to values, \ie cannot diverge.





   Haskell itself can be used to \emph{fake} ``lightweight'' dependent 
   types~\citep{ChakravartyKJ05,JonesVWW06,Weirich12}.
   In this style, the invariants are expressed in 
   a restricted~\citep{Jia10} total 
   index language and relationships (\eg $x<y$ and $y<z$) 
   are combined (\eg $x<z$) by explicitly constructing
   a term denoting the consequent from terms 
   denoting the antecedents.
   %
   On the plus side this ``constructive'' approach
   ensures soundness. 
   It is impossible to witness inconsistencies, 
   as doing so triggers diverging computations.
   %
   However, it is not easy to use restricted indices
   with explicitly constructed relations to verify 
   complex properties~\citep{hasochism}.


\spara{Refinement Types} are a form of dependent types where 
invariants are encoded via a combination of types and predicates
from a restricted \emph{SMT-decidable} 
logic~\cite{Rushby98,pfenningxi98,Dunfield07,GordonTOPLAS2011}. 
%
The restriction makes it safe to support arbitrary recursion, 
which has hitherto never been a problem for refinement types.
%
However, we show that this is because all the above systems 
implicitly assume that all free variables are bound to values, 
which is only guaranteed under CBV and, as we have seen, leads 
to unsoundness under lazy evaluation.



\spara{Tracking Divergent Computations}
The notion of type stratification to track potentially 
diverging computations dates to at least~\citep{ConstableS87} 
which uses $\bar{\typ}$ to encode diverging terms, and types 
$\efix{}$ as $(\bar{\typ}\rightarrow\bar{\typ}) \rightarrow \bar{\typ}$).
%
More recently, \cite{Capretta05} tracks diverging 
computations within a \emph{partiality monad}.
%
Unlike the above, we use refinements to 
obtain terminating fixpoints (\etfix{}), which let us prove 
the vast majority (of sub-expressions) in real world libraries 
as non-diverging, avoiding the restructuring that would
be required by the partiality monad.

\spara{Termination Analyses}
Various authors have proposed techniques to verify termination 
of recursive functions, either using the ``size-change principle'' 
\cite{JonesB04,Sereni05}, or by annotating types with size indices 
and verifying that the arguments of recursive calls have smaller 
indices~\cite{HughesParetoSabry96,BartheTermination}.
%
Our use of refinements to encode terminating fixpoints is most 
closely related to~\cite{XiTerminationLICS01}, but this work 
also crucially assumes CBV semantics for soundness.

AProVE~\cite{Giesl11} implements a powerful, fully-automatic
termination analysis for Haskell based on term-rewriting.
%
While we could use an external analysis like AProVE,
we have found that encoding the termination proof via 
refinements provided advantages that are crucial in 
large, real-world code bases. Specifically, refinements
let us
%
(1) prove termination over a subset 
    (not all) of inputs; many functions (\eg @fac@) 
    terminate only on @Nat@ inputs and not all @Int@s,
%
(2) encode pre-conditions, 
    post-conditions, and auxiliary invariants that 
    are essential for proving termination, (\eg @gcd@),
%
(3) easily specify non-standard 
    decreasing metrics and prove termination, (\eg @range@).
%
In each case, the code could be (significantly) 
\emph{rewritten} to be amenable to AProVE but this defeats
the purpose of an automatic checker.
%
Finally, none of the above analyses have been empirically
evaluated on large and complex real-world libraries.


\spara{Static Contract Checkers} 
like ESCJava~\cite{ESCJava} are a classical way of verifying 
correctness through assertions and pre- and post-conditions. 
%
Side-effects like modifications of global variables are a 
well known issue for static checkers for imperative languages;
the standard approach is to use an effect analysis to determine
the ``modifies clause'' \ie the set of globals modified by a procedure.
%
Similarly, one can view our approach as implicitly computing 
the non-termination effects.
%
%
\cite{XuPOPL09} describes a static contract checker for 
Haskell that uses symbolic execution to unroll procedures
upto some fixed depth, yielding weaker ``bounded'' soundness
guarantees.
% 

%
Similarly, Zeno~\cite{ZENO} is an automatic Haskell 
prover that combines unrolling with heuristics for rewriting
and proof-search. 
Based on rewriting, it is sound but 
``Zeno might loop forever'' when faced with 
non-termination.
%
Finally, the Halo~\cite{halo} contract checker encodes 
Haskell programs into first-order logic by directly 
modeling the code's denotational semantics,
again, requiring heuristics for instantiating axioms 
describing functions' behavior. Halo's translation of Haskell
programs directly encodes constructors as uninterpreted functions,
axiomatized to be injective (as the denotational semantics requires).
This heavyweight encoding is more precise than predicate abstraction 
but leads to model-theoretic problems (outlined in the Halo paper) and 
affects the efficiency of the encoding when scaling to larger programs 
(see also \ref{sec:refinedhaskell:conclusion}, paragraph B) in the lack of specialized 
decisions procedures.
%
Unlike any of the above, our type-based approach does 
not rely on heuristics for unrolling recursive procedures, 
or instantiating axioms. 
%
Instead we are based on decidable SMT validity 
checking and abstract interpretation~\cite{LiquidPLDI08} 
which makes the tool predictable and the overall workflow
scale to the verification of large, real-world
code bases.

\section{Abstract Refinement Types}\label{sec:related}

The notion of type refinements was introduced by Freeman and
Pfenning~\cite{FreemanPfenning91}, with refinements limited to
restrictions on the structure of algebraic datatypes, for which
inference is decidable.
%
Our present notion of refinement types has its roots in the
\emph{indexed types} of Xi and Pfenning~\cite{pfenningxi98}, wherein
data types' ranges are restricted by \emph{indices}, analogous to our
refinement predicates, drawn from a decidable domain; in the example
case explored by Xi and Pfenning, types were indexed by terms from
Presburger arithmetic.
%
Since then, several approaches to developing richer refinement type
systems and accompanying methods for type checking have been
developed.
%
Knowles and Flanagan~\cite{Knowles10} allow refinement predicates to
be arbitrary terms of the language being typechecked and present a
technique for deciding some typing obligations statically and
deferring others to runtime.
%; Gronksi \etal~\cite{Gronski06} present animplementation of such a system.
%
Findler and Felleisen's~\cite{Findler02} higher-order contracts, which
extend Eiffel's~\cite{MeyerBook} first-order contracts --- ordinary
program predicates acting as dynamic pre- and post-conditions --- to
the setting of higher-order programs, eschew any form of static
checking, and can be seen as a dynamically-checked refinement type
system.
%
Bengtson \etal~\cite{GordonTOPLAS2011} present a refinement type
system in which type refinements are drawn from a decidable logic,
making static type checking tractable.
%
Greenberg \etal~\cite{Greenberg11} gives a rigorous treatment of the
metatheoretic properties of such a refinement type system.

Refinement types have been applied to the verification of a variety of
program properties~\cite{pfenningxi98,Dunfield,GordonTOPLAS2011,FournetCCS11}.
%
In the most closely related work to our own, Kawaguchi \etal~\cite{LiquidPLDI09} 
introduce \emph{recursive} and \emph{polymorphic} refinements for data
structure properties.
%
The present work unifies and generalizes these two somewhat ad-hoc notions 
into a single, strictly and significantly more expressive mechanism of
abstract refinements.

%  Higher-order logics: Coq/HTT/F*/Agda which have explicit predicates, quantification 
A number of higher-order logics and corresponding verification tools
have been developed for reasoning about programs.
%
Example of systems of this type include NuPRL \cite{Constable86},
%F$_{<:}$ \cite{Cardelli91},
Coq \cite{coq-book}, F$^\star$ \cite{SwamyCFSBY11} and Agda \cite{norell07}
which support the development and verification of higher-order, 
pure functional programs.
%
While these systems are highly expressive, their expressiveness comes at the
cost of making logical validity checking undecidable.
%
To help automate validity checking, both built-in and user-provided
tactics are used to attempt to discharge proof obligations; however,
the user is ultimately responsible for manually proving any
obligations which the tactics are unable to discharge.

\section{Bounded Refinement Types}\label{sec:abstractrefinements:related}

\paragraph{Higher order Logics and Dependent Type Systems}
%
including
NuPRL~\citep{Constable86},
Coq~\citep{coq-book}, Agda~\citep{norell07},
and even to some extent, \haskell~\citep{JonesVWW06, McBride02},
occupy the maximal extreme of the expressiveness spectrum.
However, in these settings, checking requires explicit
proof terms which can add considerable programmer overhead.
%
Our goal is to eliminate the programmer overhead of
proof construction by restricting specifications to
decidable, first order logics and to see how far
we can go without giving up on expressiveness.
%
The \fstar system enables full dependent typing via
SMT solvers via a higher-order universally quantified
logic that permit specifications similar to ours
(\eg @compose@, @filter@ and @foldr@).
%% https://github.com/FStarLang/FStar/
%
While this approach is at least as expressive
as bounded refinements it has two drawbacks.
%
First, due to the quantifiers, the generated VCs
fall outside the SMT decidable theories.
This renders the type system undecidable (in theory),
forcing a dependency on the solver's unpredictable
quantifier instantiation heuristics (in practice).
%
Second, more importantly, % perhaps more importantly,
the higher order
predicates must be \emph{explicitly} instantiated,
placing a heavy annotation burden on the programmer.
%
In contrast, bounds permit decidable
checking, and are automatically instantiated
via Liquid Types.


\paragraph{Our notion of Refinement Types}
%
has its roots in the predicate subtyping
of PVS~\cite{Rushby98} and \emph{indexed types}
(DML~\cite{pfenningxi98}) where types are constrained
by predicates drawn from a logic.
%
To ensure decidable checking several refinement
type systems including~\citep{pfenningxi98,Dunfield07,LiquidICFP14}
restrict refinements to decidable, quantifier free logics.
%
While this ensures predictable checking and inference~\cite{LiquidPLDI08}
it severely limits the language of specifications, and makes it hard to
fashion simple higher order abstractions like @filter@ (let alone the more
complex ones like relational algebras and state transformers.)

\paragraph{To Reconcile Expressiveness and Decidability}
%
\catalyst~\citep{catalyst} permits a form of
higher order specifications where refinements
are relations which may themselves be parameterized
by other relations, which allows for example, a
way to precisely type @filter@ by suitably
composing relations.
%
However, to ensure decidable checking, \catalyst
is limited to relations that can be specified as
catamorphisms over inductive types, precluding
for example, theories like arithmetic.
More importantly, (like \fstar), \catalyst provides
no inference: higher order relations must be
\emph{explicitly} instantiated.
%
Bounded refinements build directly upon
abstract refinements~\citep{vazou13},
a form of refinement polymorphism
analogous to parametric polymorphism.
%
While \cite{vazou13} adds expressiveness via
abstract refinements, without bounds we cannot
specify any \emph{relationships between} the
abstract refinements. The addition of bounds
makes it possible to specify and verify the examples
shown in this paper,
while preserving decidability and inference.

\paragraph{Our Relational Algebra Library} builds on a long
line of work on type safe database access.
%
The HaskellDB~\citep{haskellDB}
showed how phantom types could be used to eliminate
certain classes of errors.
%
Haskell's HList library~\citep{heterogeneous}
extends this work with type-level computation
features to encode heterogeneous lists, which
can be used to encode database schema, and
(unlike HaskellDB) statically reject accesses
of ``missing'' fields.
%
The HList implementation is non-trivial,
requiring new type-classes for new operations
(\eg @append@ing lists); \citep{thepipower}
shows how a dependently typed language greatly
simplifies the implementation.
%
Much of this simplicity can be recovered in
Haskell using the @singleton@ library~\citep{Weirich12}.
%
Our goal is to show that bounded refinements
are expressive enough to permit the construction
of rich abstractions like a relational algebra
and generic combinators for safe database access
while using SMT solvers to provide decidable
checking and inference. Further, unlike the
HList based approaches, refinements they can
be used to \emph{retroactively} or \emph{gradually}
verify safety; if we erase the types we still
get a valid Haskell program operating over
homogeneous lists.


\paragraph{Our Approach for Verifying Stateful Computations} using monads
indexed by pre- and post-conditions is inspired by the method of
Filli\^atre~\citep{Filliatre98}, which was later enriched with
separation logic in Ynot~\citep{ynot}. In future work it would
be interesting to use separation logic based refinements to specify
and verify the complex sharing and aliasing patterns allowed by Ynot.
%
\fstar encodes stateful computations in a special Dijkstra
Monad~\citep{dijkstramonad} that replaces the two assertions with
a single (weakest-precondition) predicate transformer which
can be composed across sub-computations to yield a transformer
for the entire computation.
%
Our \RIO approach uses the idea of indexed monads but
has two concrete advantages.
%
First, we show how bounded refinements alone suffice to
let us fashion the \RIO abstraction from scratch.
%
Consequently, second, we automate inference of pre- and
post-conditions and loop invariants as refinement instantiation
via Liquid Typing~\citep{LiquidPLDI08}.


\section{Refinement Reflection}\label{sec:refinementreflection:related}

% We compare refinement reflection to the most closely related
% lines of work in the vast literature on program verification.

\mypara{SMT-Based Verification}
%
SMT-solvers have been extensively used to automate
program verification via Floyd-Hoare logics~\cite{Nelson81}.
%
Our work is inspired by Dafny's Verified
Calculations~\citep{LeinoPolikarpova16},
a framework for proving theorems in
Dafny~\citep{dafny}, but differs in
%
(1)~our use of reflection instead of axiomatization and
(2)~our use of refinements to compose proofs.
%
Dafny, and the related \fstar~\citep{fstar}
which like \toolname, uses types to compose
proofs, offer more automation by translating
recursive functions to SMT axioms.
However, unlike reflection, this axiomatic
approach renders typechecking and verification
undecidable (in theory) and leads to
unpredictability and divergence
(in practice)~\citep{Leino16}.
%\NV{CHECL Relational-F*, Barthe et al, from POPL 2014, and EasyCrypt}

%% In a work more closely related to
%% ours, \fstar uses refinement types
%% for program verification supporting
%% expressiveness of fully dependent types.
%% %
%% As in Dafny, \fstar directly translates
%% recursive functions to axioms in the logic
%% thus suffers from the ``butterfly effect''
%% and allows the user to explicitly write SMT tactics to control it.

%% Leino \etal~\citep{Leino16}
%% name this problem as the ``butterfly
%% effect'', in which minor modifications
%% to the program source cause significant
%% instabilities in verification and propose
%% trigger selection strategies to address it.
%% %
%% We avoid the ``butterfly effect'' by not
%% directly axiomatizing functions into logic.
%% Instead the information about the function's
%% body is exactly captured in function's result
%% type and user needs to explicitly invoke the function to push
%% the function's definition information into the logic.


\mypara{Dependent types}
%
Our work is also inspired by dependently typed
systems like Coq~\citep{coq-book} and
Agda~\citep{agda}.
%
Reflection shows how deep specification
and verification in the style of Coq and Agda
can be \emph{retrofitted} into existing languages
via refinement typing.
%
Furthermore, we can use SMT to significantly
automate reasoning over important theories like
arithmetic, equality and functions.
%
It would be interesting to investigate how
the tactics and sophisticated proof search
of Coq \etc can be adapted to the refinement setting.

% which allow for arbitrary expressiveness of the type system
% in the cost of automatic verification.
%
%% The syntax of \libname's operators is inspired by
%% Equational Reasoning in Agda~\citep{agda}.
%% Here we extended these equational operators
%% to support linear arithmetic and, for example, prove
%% properties of Ackermann function.
%% %
%% Unlike Adga, proof term are explicit in \libname,
%% we do not use heuristics to infer proofs.


\mypara{Dependent Types for Non-Terminating Programs}
%
Zombie~\citep{Zombie, Sjoberg2015} integrates
dependent types in non terminating programs
and supports automatic reasoning for equality.
%
Vazou \etal have previously~\citep{Vazou14} shown
how Liquid Types can be used to check
non-terminating programs.
%
Reflection makes \toolname at least as
expressive as Zombie, \emph{without}
having to axiomatize the theory of
equality within the type system.
%
Consequently, in contrast to Zombie,
SMT based reflection lets \toolname
verify higher-order specifications
like @foldr_fusion@.

% which lets us use SMT automation
% to verify deep specifications of
% non-trivial programs.
%
% Our current extension is orthogonal to the
% previous work: our system remains sound as
% long as logical terms provably terminate.
%
% We get automation from SMT solvers for not only
% the theory of equality, but also linear arithmetic.
%
% \NV{Zombie with rewritting does not allow HIGHER ORDER reasoning}

\mypara{Dependent Types in Haskell}
%
Integration of dependent types into Haskell
has been a long standing goal that dates back
to Cayenne~\citep{cayenne}, a Haskell-like,
fully dependent type language with undecidable
type checking.
%
In a recent line of work~\citep{EisenbergS14}
Eisenberg \etal aim to allow fully dependent
programming within Haskell, by making
``type-level programming ... at least as
  expressive as term-level programming''.
%
Our approach differs in two significant ways.
%
First, reflection allows SMT-aided verification,
which drastically simplifies proofs over key theories
like linear arithmetic and equality.
%
Second, refinements are completely erased at run-time.
That is, while both systems automatically lift Haskell
code to either uninterpreted logical functions
or type families, with refinements, the logical
functions are not accessible at run-time and
promotion cannot affect the semantics of
the program.
%
As an advantage (resp. disadvantage), refinements
cannot degrade (resp. optimize)
the performance of programs.

\mypara{Proving Equational Properties}
% of Haskell Programs}
%
Several authors have proposed tools for proving
(equational) properties of (functional) programs.
%
Systems~\citep{sousa16} and \citep{KobayashiRelational15}
extend classical safety verification algorithms,
respectively based on Floyd-Hoare logic and Refinement Types,
to the setting of relational or $k$-safety properties
that are assertions over $k$-traces of a program.
%
Thus, these methods can automatically prove that
certain functions are associative, commutative \etc.
but are restricted to first-order properties and
are not programmer-extensible.
%
Zeno~\citep{ZENO} generates proofs by term
rewriting and Halo~\citep{halo} uses an axiomatic
encoding to verify contracts.
%
Both the above are automatic, but unpredictable and not
programmer-extensible, hence, have been limited to
far simpler properties than the ones checked here.
%
HERMIT~\citep{Farmer15} proves equalities by rewriting
the GHC core language, guided by user specified scripts.
%
In contrast, our proofs are simply Haskell programs,
we can use SMT solvers to automate reasoning, and,
most importantly, we can connect the validity of
proofs with the semantics of the programs.

% \RJ{say: hermit does typeclass laws}
%
% \NV{TO ADD Naoki and class laws for TFP}
%
%% Compared to these systems, our proofs are
%% expressed as Haskell programs and do not
%% require the user to learn a different
%% tactic languages.
%% %
%% Moreover, our system is more general
%% as it allows for both equational
%% and linear arithmetic proofs.
%% %
%% On the other hand, \libname requires
%% explicit proofs and does not currently
%% support any automatic heuristics.

\mypara{Deterministic Parallelism}
%
Deterministic parallelism has plenty of theory but relatively few practical
implementations.  Early discoveries were based on limited producer-consumer
communication, such as single-assignment variables \cite{Tesler-1968,IStructures}, Kahn
process networks~\cite{kahn-1974}, and synchronous dataflow~\cite{lee-sdf}.
Other models use synchronous updates to shared state, as in
Esterel~\cite{synchronous-overview} or PRAM.  Finally, work on type systems for
permissions management \cite{permission-types,habanero-java-permissions},
supports the development of {\em non-interfering} parallel programs that access
disjoint subsets of the heap in parallel.  Parallel functional programming is
also non-interfering~\cite{manticore,multicore-ghc}.
%
Irrespective of which theory is used to support deterministic parallel
programming, practical implementations such as Cilk~\cite{cilk} or Intel
CnC~\cite{cnc} are limited by host languages with type systems insufficient to
limit side effects, much less prove associativity.  Conversely, dependently
typed languages like Agda and Idris do not have parallel programming APIs and
runtime systems.

% synchronous languages such as Esterel


\section{String Matcher}\label{sec:stringmatcher:related}


\paragraph{SMT-Based Verification}
%
SMT solvers have been extensively used to automate
reasoning on verification languages like
Dafny~\cite{dafny}, Fstar~\cite{fstar} and Why3~\cite{why3}.
%
These languages are designed for verification,
thus have limited support for commonly used language
features like parallelism and optimized libraries
that we use in our verified implementation.
%
Refinement Types~\cite{ConstableS87,FreemanPfenningDONTCITE91,Rushby98}
on the other hand, target existing general purpose languages,
such as
ML~\cite{pfenningxi98,GordonRefinement09,LiquidPLDI08},
C~\cite{deputy,LiquidPOPL10},
Haskell~\cite{Vazou14},
Racket~\cite{RefinedRacket}
and Scala~\cite{refinedscala}.
However, before Refinement Reflection~\cite{reflection} was introduced,
they only allowed ``shallow'' program specifications,
that is, properties that only talk about behaviors of program functions
but not functions themselves.
%

\paragraph{Dependent Types}
Unlike Refinement Types, dependent type systems,
like Coq~\cite{coq-book}, Adga~\cite{agda} and Isabelle/HOL~\cite{isabelle} allow for ``deep'' specifications
which talk about program functions,
such as the program equivalence reasoning we presented.
%
Compared to (Liquid) Haskell,
these systems allow for tactics and heuristics
that automate proof term generation
but lack SMT automations and
general-purpose language features,
like non-termination, exceptions and IO.
%
Zombie~\cite{Zombie,Sjoberg2015} and Fstar~\cite{fstar} allow dependent types to
coexist with divergent and effectful programs,
but still lack the optimized libraries,
like @ByteSting@, which come
with a general purpose language
with long history, like Haskell.



\paragraph{Parallel Code Verification}
Dependent type theorem provers have been used before to
verify parallel code.
%
BSP-Why~\cite{bspwhy} is an extension to Why2 that is using both Coq and SMTs
to discharge user specified verification conditions.
%
Daum~\cite{daum07} used Isabelle to formalize the semantics
of a type-safe subset of C, 
by extending Schirmer's~\cite{schirmer06}
formalization of sequential imperative languages.
%
Finally, Swierstra~\cite{wouter10} formalized mutable arrays in Agda
and used them to reason about distributed maps and sums.

One work  closely related to ours is
SyDPaCC~\cite{SyDPaCC}, a Coq library that
automatically parallelizes list homomorphisms
by extracting parallel Ocaml versions of user provided Coq functions.
%
Unlike SyDPaCC, we are not automatically generating the parallel
function version, because of engineering limitations
(\S~\ref{sec:evaluation}).  Once these are addressed, code extraction
can be naturally implemented by turning
Theorem~\ref{theorem:two-level} into a Haskell type class with a
default parallelization method.
%
SyDPaCC used maximum prefix sum as a case study,
whose morphism verification is
much simpler than our string matching case study.
%
Finally, our implementation consists of
2K lines of Liquid Haskell, which we consider verbose and aim to
use tactics to simplify.
On the contrary, the SyDPaCC implementation
requires three different languages:
2K lines of Coq with tactics, 600 lines of Ocaml and 120 lines of C,
and is considered ``very concise''.



\section{Conclusions \& Alternative Approaches}\label{sec:refinedhaskell:conclusion}

Our goal is to use the recent advances in SMT solving to 
build automated refinement type-based verifiers for 
Haskell.
%
In this paper, we have made the following advances 
towards the goal. 
%
First, we demonstrated how the classical technique
for generating VCs from refinement subtyping queries
is unsound under lazy evaluation.
%
Second, we have presented a solution that addresses 
the unsoundness by stratifying types into those that 
are inhabited by terms that may diverge, those that must reduce 
to Haskell values, and those that must reduce to finite values, 
and have shown how refinement types may themselves 
be used to soundly verify the stratification. 
%
Third, we have developed an implementation of our 
technique in \toolname and have evaluated the tool 
on a large corpus comprising 10KLOC of widely used 
Haskell libraries. Our experiments empirically 
demonstrate the practical effectiveness of our
approach: using refinement types, we were able 
to prove 96\% of recursive functions as 
terminating, and to crucially use this information 
to prove a variety of functional correctness properties.

\mypara{Limitations}
While our approach is demonstrably effective 
\emph{in practice}, it relies critically on 
proving termination, which, while independently 
useful, is not wholly satisfying 
\emph{in theory}, as adding divergence shouldn't 
\emph{break} a safety proof.
%to quote~\cite{McMillanPersonal}: 
%\emph{``adding divergence shouldn't break your safety proofs.''}
%
%In our approach, we can prove a program safe, 
Our system can prove a program safe, 
but if the program is modified by making 
some functions non-deterministically diverge,
then, since we rely on termination, we
may no longer be able to prove safety.
%
Thus, in future work, it would be valuable to 
explore \emph{other} ways to reconcile laziness 
and refinement typing. We outline some routes 
and the challenging obstacles along them.


%% \mypara{1. Reject Inconsistent Environements}
%% We may be tempted to point the finger of blame at the
%% ``inconsistency'' itself. Unfortunately, this would be 
%% misguided -- inconsistencies are not a bug but a crucial 
%% feature of refinement type systems. 
%% %
%% They enable, among other things, \emph{path sensitivity} by
%% incorporating information from run-time tests (guards) and 
%% hence let us verify that expressions that throw catastrophic
%% exceptions (\eg @error e@) are indeed unreachable dead code 
%% and will not fail at run-time.
%% 
%% \mypara{2. CPS Transformation}
%% We might use a CPS transformation~\cite{PlotkinTCS75,WadlerICFP03} 
%% to convert the program into call-by-value.
%% We confess to be somewhat wary of the prospect of translating 
%% inferred types and errors \emph{back} to the source level after 
%% such a transformation.
%% Previous experience shows that the ability to map types and 
%% errors to source is critical for usability.

%% \mypara{3. Strictness Analysis}
%% We may want some form of \emph{strictness} or 
%% \emph{dependency analysis}~\cite{Mycroft80} to statically 
%% predict which binders must be evaluated, and only use
%% refinements for those binders in the environment. 
%% This route is problematic for many reasons. 
%% %
%% First, we aim to prove fine-grained functional correctness 
%% properties of programs. In addition to the well known limitations
%% of strictness analysis~\cite{HaskellWiki}, it 
%% is unclear how to develop an analysis that is sensitive
%% to the precise semantic (``path'') conditions under which 
%% the evaluation of different binders will be forced.
%% %
%% Second, and more importantly, it is often useful to add
%% \emph{ghost} values into the program for the sole purpose 
%% of making refinement types \emph{complete}~\cite{TerauchiPOPL13}. 
%% By construction, these values are not used by the program, 
%% and would be thrown away by a strictness analysis, thereby
%% precluding verification.

\mypara{A. Convert Lazy To Eager Evaluation}
%
One alternative might be to translate the program from lazy to eager evaluation,
for example, to replace every (thunk) $e$ with an abstraction $\efun{()}{}{e}$,
and every use of a lazy value $x$ with an application $x\ ()$. 
After this, we could simply assume eager evaluation, and so the usual refinement
type systems could be used to verify Haskell. Alas, no. 
While sound, this translation
doesn't solve the problem
of reasoning about divergence. 
%%While this translation
%%does soundly reject the @explode@ example, 
%%it doesn't solve the problem
%%of reasoning about divergence. 
A dependent function type
${\tfun{x}{\tint}{\tlref{\vv}{\tint}{}{\vv>\x}}}$
would be transformed to
${\tfun{x}{(\tfunbasic{()}{\tint})}
          {\tlref{\vv}{\tint}{}{\vv > \x\ ()}}}$.
%
%%%\begin{code}
%%%  f :: x:Int -> {v:Int | v > x}
%%%\end{code}
%%%%
%%%would be transformed to
%%%%
%%%\begin{code}
%%%  f :: x:(() -> Int) -> {v:Int | v > x ()}
%%%\end{code}
%
The transformed type is problematic as it uses 
arbitrary function applications in the refinement logic!
%
The type is only sensible if $x\ ()$ provably reduces to a value, 
bringing us back to square one.

%%% This is highly problematic, because now we have function 
%%% applications in the logic! Now it seems that "x" is a 
%%% pretty harmless function but not really, because we're 
%%% essentially back in the same world where we have to be 
%%% *sure* that "x ()" actually reduces to a value! 
%%% 
%%% That is to say, essentially, this is the same as the 
%%% "direct" approach of, 
%%% 
%%%    x:Int -> {y | (isvalue x) => y > x}
%%% 
%%% 
%%% That is, the simple refinement in the original CBN is converted to 
%%% 
%%% 
%%% %
%%% Unfortunately, this doesn't 
%%% 
%%% that require CBV evaluation
%%% 
%%%   That use call by name and a strict language.  Now all the old techniques should work. 
%%% 
%%% convert the program from lazy to ea
%% We might use a CPS transformation~\cite{PlotkinTCS75,WadlerICFP03} 
%% to convert the program into call-by-value.
%% We confess to be somewhat wary of the prospect of translating 
%% inferred types and errors \emph{back} to the source level after 
%% such a transformation.
%% Previous experience shows that the ability to map types and 
%% errors to source is critical for usability.

\mypara{B. Explicit Reasoning about Divergence}
%
Another alternative is to enrich the refinement logic
% It is not really clear that it is THE only 
% Thus, the only other alternative is to enrich the refinement logic
with a \emph{value predicate} $\isvalue{x}$ that is true when 
``$x$ is a value'' and use the SMT solver to 
\emph{explicitly} reason about divergence.
%
(Note that $\isvalue{x}$ is equivalent to introducing a 
$\ebot$ constant denoting divergence, and 
writing $(x \not =\ \ebot)$.)
%
Unfortunately, this $\isvalue{x}$ predicate takes the VCs 
outside the scope of the standard efficiently decidable logics 
supported by SMT solvers.
%
To see why, recall 
the subtyping query %(\ref{sub:good}) 
from @good@. % in \Sref{sec:refinedhaskell:overview}. 
With explicit value predicates, 
this subtyping reduces to the VC:
%
\begin{align}
{(\isvalue{x} \Rightarrow x \geq 0)}, \
{(\isvalue{y} \Rightarrow y \geq 0)} 
\Rightarrow
{(v = y+1)}   \Rightarrow {(v > 0)}\label{vc:good:explicit}
\end{align}
%
To prove the above valid, we require the knowledge 
that $(v = y+1)$ implies that $y$ is a value, \ie that 
$\isvalue{y}$ holds.
%
This fact, while obvious to a \emph{human} reader, is 
outside the decidable theories of linear arithmetic
of the existing SMT solvers.
%
Thus, existing solvers would be unable to prove (\ref{vc:good:explicit}) 
valid, causing us to reject @good@.

%%%%%%%%%%%%%%%%%%%%%%%%%%%%%%%%%%%%%%%%%%%%%%%%%%%%%%%%%%%%%%%%%%%%%%%%%%%%%%%
%%%%%%%%%%%%%%%%%%%%%%%%%%%%%%%%%%%%%%%%%%%%%%%%%%%%%%%%%%%%%%%%%%%%%%%%%%%%%%%
%%%%%%%%%%%%%%%%%%%%%%%%%%%%%%%%%%%%%%%%%%%%%%%%%%%%%%%%%%%%%%%%%%%%%%%%%%%%%%%

\mypara{Possible Fix: Explicit Reasoning With Axioms?}
%
One possible fix for the above would be to specify a collection of
\emph{axioms} that characterize how the value predicate behaves with 
respect to the other theory operators. 
%
For example, we might specify axioms like: 
%
\begin{align*}
\forall x,y,z. (x = y + z)\ &\Rightarrow\ (\isvalue{x} \wedge \isvalue{y} \wedge \isvalue{z})\\
% \forall x,y,z. (x = y - z)\ &\Rightarrow\ (\isvalue{x} \wedge \isvalue{y} \wedge \isvalue{z})\\\
\forall x,y. (x < y )\ &\Rightarrow\ (\isvalue{x} \wedge \isvalue{y})
% &\mathit{etc.}
\end{align*}
%
\etc. However, this is a non-solution for several reasons. 
First, it is not clear what a complete set of axioms is.
Second, there is the well known loss of predictable checking
that arises when using axioms, as one must rely on various 
brittle, syntactic matching and instantiation heuristics~\cite{simplifyj}. 
%
It is unclear how well these heuristics will work with the
sophisticated linear programming-based algorithms used to 
decide arithmetic theories. 
%
Thus, proper support for value predicates could require 
significant changes to existing decision procedures, 
making it impossible to use existing SMT solvers.


%%%%%%%%%%%%%%%%%%%%%%%%%%%%%%%%%%%%%%%%%%%%%%%%%%%%%%%%%%%%%%%%%%%%%%%%%%%%%%%
%%%%%%%%%%%%%%%%%%%%%%%%%%%%%%%%%%%%%%%%%%%%%%%%%%%%%%%%%%%%%%%%%%%%%%%%%%%%%%%
%%%%%%%%%%%%%%%%%%%%%%%%%%%%%%%%%%%%%%%%%%%%%%%%%%%%%%%%%%%%%%%%%%%%%%%%%%%%%%%

\mypara{Possible Fix: Explicit Reasoning With Types?}
%
Another possible fix would be to encode the behavior of the
value predicates within the refinement types for different 
operators, after which the predicate itself could be treated 
as an \emph{uninterpreted function} in the refinement 
logic~\cite{bradleybook}. For instance, we could type 
the primitives:
%
\begin{code}
 (+) :: x:Int -> y:Int -> {v | v  =  x + y && Val x && Val y}
 (<) :: x:Int -> y:Int -> {v | v <=> x < y && Val x && Val y}
\end{code}
%
While this approach requires \emph{no} changes to the SMT 
machinery, it makes specifications complex and verbose. 
%
%% (and not unlike having to sprinkle explicit ``non-null'' 
%% checks all over pointer manipulating programs!)
%
We cannot just add the value predicates to the primitives' 
specifications. Consider 
%
\begin{code}
 choose b x y = if b then x+1 else y+2
\end{code}
%
To reason about the output of @choose@ we must type it as:
%
\begin{code}
 choose :: Bool -> x:Int -> y:Int -> {v|(v > x && Val x)||(v > y && Val y)}  
\end{code}
%
Thus, the value predicates will pervasively clutter all 
signatures with strictness information, making the system 
unpleasant to use.

%%%%%%%%%%%%%%%%%%%%%%%%%%%%%%%%%%%%%%%%%%%%%%%%%%%%%%%%%%%%%%%%%%%%%
%%%%%%%%%%%%%%%%%%%%%%%%%%%%%%%%%%%%%%%%%%%%%%%%%%%%%%%%%%%%%%%%%%%%%
%%%%%%%%%%%%%%%%%%%%%%%%%%%%%%%%%%%%%%%%%%%%%%%%%%%%%%%%%%%%%%%%%%%%%

\mypara{Divergence Requires 3-Valued Logic}
Finally, for either ``fix'', the value predicate poses a 
model-theoretic problem: 
what is the meaning of $\isvalue{x}$? 
%what meaning do we give $\isvalue{x}$? 
%
One sensible approach is to extend the universe with a family of 
\emph{distinct} $\bot$ constants, such that $\isvalue{\bot}$ is false.
%
These constants lead inevitably into a three-valued logic 
(in order to give meaning to formulas like $\bot = \bot$).
%
Thus, even if we were to find a way to reason with the value 
predicate via axioms or types, we would have to ensure that 
we properly handled the 3-valued logic within 
existing 2-valued SMT solvers.

\mypara{Future Work}
Thus, in future work it would be worthwhile to address the above 
technical and usability problems to enable explicit reasoning with 
the value predicate.
%
This explicit system would be \emph{more expressive} than our 
stratified approach, \eg would let us check 
%
\begin{code}
  let x = collatz 10 in 12 `div` x + 1
\end{code}
%
by encoding strictness inside the logic. Nevertheless, we suspect
such a verifier would use stratification to eliminate the value
predicate in the common case.
%
At any rate, until these hurdles are crossed, we can take comfort in
stratified refinement types and can just \emph{eagerly}
use termination to prove safety for \emph{lazy} languages.

%% If we could address the above problems
%% Thus, at this point, even though the natural route is to reason explicitly
%% with value predicates, as they would address the theoretical
%% robustness-to-divergence limitation described above, 
%% it is unclear how the above problems can be solved, 
%% and we believe they may be promising directions for 
%% future work.
%
%
%
%% Thus, at this point, even though the natural route is to reason explicitly
%% with value predicates, as they would address the theoretical
%% robustness-to-divergence limitation described above, 
%% it is unclear how the above problems can be solved, 
%% and we believe they may be promising directions for 
%% future work.
%% 
%% %
%% Of course, that does not mean they are unsolvable, just 
%% that the presence of value predicates means we cannot 
%% use \emph{existing}, off-the-shelf SMT solvers to 
%% achieve our goal of a sound and predictable 
%% refinement type checker for Haskell.


%%% Local Variables: 
%%% mode: latex
%%% TeX-master: "main"
%%% End: 


\subsection*{Acknowledgements}
We thank 
Kenneth Knowles, 
Kenneth L. McMillan, 
Andrey Rybalchenko, 
Philip Wadler, 
and the reviewers for their 
excellent suggestions and feedback about 
earlier versions of this paper.

{
\bibliographystyle{plain}
\bibliography{sw}
}

\ifthenelse{\equal{\isTechReport}{true}}
{
\appendix
\section{Declarative Typing: \undeclang}

\subsection{Definitions}
To simplify the metatheory we extend \undeclang so that
\begin{itemize}
\item Supports stratified types, and
\item explicitly contains \ebot, a primitive that has any type, but does not evaluate. 
\end{itemize}


\begin{figure}
$$
\begin{array}{rrcl}

\emphbf{Constants} \quad 
  & c & ::=    & 0,1,-1,\ldots \spmid \etrue, \efalse \\
  &   & \spmid & +,-,\ldots \spmid =, <, \ldots \spmid \ecrash 
  \\[0.05in]

\emphbf{Values} \quad 
  & v & ::= &  c \spmid \efun{x}{\typ}{e} \spmid \edapp{D}{e}
  \\[0.05in] 

\emphbf{Expressions} \quad 
  & e & ::=    & \ebot \spmid v \spmid x \spmid \eapp{e}{e} \spmid \elet{x}{e}{e} \\ 
  &   & \spmid & \ecase{e}{D}{\overline{x}}{e}{x} \\[0.05in] 

\emphbf{Basic Types} \quad 
  & \tbase & ::= & \tint \spmid \tbool \spmid T \\[0.05in] 

\emphbf{Label} \quad 
  & l
  & ::= 
  & \trivial \spmid \finite 
  \\[0.05in] 
  
\emphbf{Types} \quad 
  & \typ & ::= & \tlref{v}{\tbase}{}{e} \spmid \tlref{v}{\tbase}{l}{e} \spmid
  				 \tfunref{x}{\typ}{\typ}{v}{e} \\ 
\end{array}
$$

\hrule width 0.48\textwidth

$$
\begin{array}{rrcl}
\emphbf{Contexts} \quad 
  & C
  & ::= 
  &   	 \bullet 
  \spmid \eapp{C}{e} 
  \spmid \eapp{c}{C} 
  \spmid D\ \overline{e}\ C\ \overline{e}\\
  &&\spmid &
  \ecase{C}{D}{\overline{y}}{e}{x}
  \\[0.05in] 
\end{array}
$$

\caption{\undeclang: Syntax}
\label{fig:undeclang}
\label{fig:operational}
\end{figure}

Then, we define the function \erase{\bullet} that erases the refinements in types and environments:
\begin{align*}
\erase{\tlref{v}{B}{l}{e}}&=B^{l} &
\erase{\emptyset}&=\emptyset\\
\erase{\tfunref{x}{\tau_x}{\tau}{v}{e}}&= \erase{\tau_x} \rightarrow \erase{\tau} &
\erase{x\colon\tau, \Gamma}&= x\colon\erase{\tau},\erase{\Gamma}
\end{align*}

and variable substitution on types:
\begin{align*}
(\tref{v}{B}{l}{e})\sub{y}{e_y} &=\tref{v}{B}{l}{e\sub{y}{e_y}}\\
(\tfunref{x}{\tau_x}{\tau}{v}{e})\sub{y}{e_y} &=
	\tfunref{x}{(\tau_x\sub{y}{e_y})}{(\tau\sub{y}{e_y})}{v}{e\sub{y}{e_y}}\\
\end{align*}


We extend the typing rules with another rule that types \ebot with
\textbf{any} type getting the rules as defined in Figure~\ref{fig:proofs:typing}.
%
\begin{figure}[p]
\centering
\captionsetup{justification=centering}
\judgementHead{Well-Formedness}{\isWellFormed{\Gamma}{\sigma}}

$$\begin{array}{ccc}
\inference
  {}
  {\isWellFormed{\Gamma}{\true(\vref)}}
  [\wtTrue]
&
\quad
&
\inference
    {\isWellFormed{\Gamma}{\areft(\vref)} && 
     \hastype{\Gamma}{\rvapp{\rvar}{e} \ \vref}{\tbbool}
    }
    {\isWellFormed{\Gamma}{(\areft \wedge \rvapp{\rvar}{e})(\vref)}}
    [\wtRVApp]
\end{array}$$
%
$$\inference
    {\hastype{\Gamma, \vref:b}{\reft}{\tbbool} \quad 
%     \isWellFormed{\Gamma, \vref:b}{\areft(\vref)}
     \hastype{\Gamma, \vref:b}{\areft(\vref)}{\tbbool}
    }
    {\isWellFormed{\Gamma}{\tpref{b}{\areft}{\reft}}}
    [\wtBase]
$$
%
$$
\inference
    {
	%\hastype{\Gamma, v:\tfun{x}{\tau_x}{\tau}}{e}{\tbbool} &&
	\hastype{\Gamma}{\reft}{\tbbool} &&
    \isWellFormed{\Gamma}{\tau_x} &&
	\isWellFormed{\Gamma, x:\tau_x}{\tau}
    }
    {\isWellFormed{\Gamma}{\trfun{x}{\tau_x}{\tau}{\reft}}}
    [\wtFun]
$$
%
$$\begin{array}{ccc}
\inference
  {\isWellFormed{\Gamma, \rvar:\tau}{\sigma}}
  {\isWellFormed{\Gamma}{\tpabs{\rvar}{\tau}{\sigma}}}
  [\wtPred]
&
\quad
&
\inference
    {\isWellFormed{\Gamma, \alpha}{\sigma}}
    {\isWellFormed{\Gamma}{\ttabs{\alpha}{\sigma}}}
    [\wtPoly]
\end{array}$$

\medskip \judgementHead{Subtyping}{\isSubType{\Gamma}{\sigma_1}{\sigma_2}}

$$
\inference
   {\text{SMT-Valid}(\inter{\Gamma} \land \inter{\areft_1\ \vref} \land \inter{\reft_1} 
                 \Rightarrow \inter{\areft_2\ \vref} \land \inter{\reft_2})}
   {\isSubType{\Gamma}{\tpref{b}{\areft_1}{\reft_1}}{\tpref{b}{\areft_2}{\reft_2}}}
   [\tsubBase]
$$
%
$$
\inference
   {%\text{Valid}(\inter{\Gamma}\land \inter{e_1} \Rightarrow \inter{e_2}) \\
	\isSubType{\Gamma}{\tau_2}{\tau_1} &
	\isSubType{\Gamma, x_2:{\tau_2}}{\SUBST{\tau_1'}{x_1}{x_2}}{\tau_2'}	
   }
   {\isSubType{\Gamma}
	  {\trfun{x_1}{\tau_1}{\tau_1'}{\reft_1}}
	  {\trfun{x_2}{\tau_2}{\tau_2'}{\true}}
}[\tsubFun]
$$
%
$$
\begin{array}{ccc}
\inference
   {\isSubType{\Gamma, \rvar:\tau}{\sigma_1}{\sigma_2}}
   {\isSubType{\Gamma}{\tpabs{\rvar}{\tau}{\sigma_1}}{\tpabs{\rvar}{\tau}{\sigma_2}}}
   [\tsubPred]
&
\quad
&
\inference
   {\isSubType{\Gamma}{\sigma_1}{\sigma_2}}
   {\isSubType{\Gamma}{\ttabs{\alpha}{\sigma_1}}{\ttabs{\alpha}{\sigma_2}}}
   [\tsubPoly]
\end{array}
$$

\medskip \judgementHead{Type Checking}{$\hastype{\Gamma}{e}{\sigma}$}

$$\inference
  {  \hastype{\Gamma}{e}{\sigma_2} && \isSubType{\Gamma}{\sigma_2}{\sigma_1} 
  && \isWellFormed{\Gamma}{\sigma_1}
  }
  {\hastype{\Gamma}{e}{\sigma_1}}
  [\tsub]
\quad
\inference
  {}
  {\hastype{\Gamma}{c}{\tc{c}}}
  [\tconst]
$$
$$
\inference
  {x: \tpref{b}{\areft}{\reft} \in \Gamma}
  {\hastype{\Gamma}{x}{\tpref{b}{\areft}{e \land \vref = x}}}
  [\tbase]
\quad
\inference
  {x:\tau \in \Gamma}
  {\hastype{\Gamma}{x}{\tau}} 
  [\tvariable]
$$
$$
\inference
   {\hastype{\Gamma, x:\tau_x}{e}{\tau} 
    && \isWellFormed{\Gamma}{\tau_x}
   }
   {\hastype{\Gamma}{\efunt{x}{\tau_x}{e}}{\tfun{x}{\tau_x}{\tau}}}
   [\tfunction]
\quad
\inference
   {\hastype{\Gamma}{e_1}{\tfun{x}{\tau_x}{\tau}} 
   &&  \hastype{\Gamma}{e_2}{\tau_x}
   }
   {\hastype{\Gamma}{\eapp{e_1}{e_2}}{\SUBST{\tau}{x}{e_2}}}
   [\tapp]
$$
$$
\inference
  {\hastype{\Gamma, \alpha}{e}{\sigma}}
  {\hastype{\Gamma}{\etabs{\alpha}{e}}{\ttabs{\alpha}{\sigma}}}
  [\tgen]
\quad
\inference
  {\hastype{\Gamma}{e}{\ttabs{\alpha}{\sigma}} && 
   \isWellFormed{\Gamma}{\tau}
  }
  {\hastype{\Gamma}{\etapp{e}{\tau}}{\SUBST{\sigma}{\alpha}{\tau}}}
  [\tinst]
$$
$$
\inference
    {\hastype{\Gamma, \rvar:\tau}{e}{\sigma} &&
     \isWellFormed{\Gamma}{\tau} 
     % \tau \mbox{ is non-refined } 
     %\isWellFormed{\Gamma}{\tpabs{p}{\tau}{\pi}} && 
     %p \notin \fv{e}
    }
    {\hastype{\Gamma}{\epabs{\rvar}{\tau}{e}}{\tpabs{\rvar}{\tau}{\sigma}}}
    [\tpgen]
\ \
\inference
    {\hastype{\Gamma}{e}{\tpabs{\rvar}{\tau}{\sigma}} && 
     \hastype{\Gamma}{\efunbar{x:\tau_x}{\reft'}}{\tau}
    }
    {\hastype{\Gamma}
             {\epapp{e}{\efunbar{x:\tau_x}{\reft'}}}
             {\rpinst{\sigma}{\rvar}{\efunbar{x:\tau_x}{\reft'}}}
     %        {\sigma\sub{\eapp{p}{\overline{e_p}}}{\eapp{\reft'}{\overline{e_p}}}}
    }
    [\tpinst]
$$
\caption[Type checking of \corelan.]{Well-formedness, Subtyping and Type Checking of \corelan.}
\label{fig:rules}
\end{figure}


%

We define the denotations of types by combining the denotations 
of stratified types:
\begin{definition}{[Type Denotations]}
\begin{align*}
\interp{\tref{x}{\tbase}{}{p}} \defeq & 
    \{e \mid  \hastypebase{\emptyset}{e}{\tbase},
              \mbox{ if } \evals{e}{v} 
              \mbox{ then } \evals{\SUBST{p}{x}{v}}{\etrue} \}\\
\interp{\tlref{v}{\tbase}{\trivial}{p}} \defeq & 
    \interp{\tlref{v}{\tbase}{}{p}} \cap \{ e \mid \evals{e}{v} \}\\
\interp{\tlref{v}{\tbase}{\finite}{p}} \defeq & 
    \interp{\tlref{v}{\tbase}{\trivial}{p}} \cap \{ e \mid \evals{e}{d} \} \\
\interp{\tfun{x}{\typ_x}{\typ}} \defeq & 
    \{e \mid  \hastypebase{\emptyset}{e}{\erase{\tfunbasic{\typ_x}{\typ}}}, 
              \forall e_x \in \interp{\typ_x}.\ \eapp{e}{e_x} \in \interp{\typ\sub{x}{e_x}}
    \}
\end{align*}
\end{definition}

Finally, we define the constraints that should be satisfied by constants:
%
\begin{definition}{[Constants]}\label{def:constants}
For every basic type $T$ there are exactly  $n = \arity{T}$ 
data contractors with result type $T$, namely 
$\{D_T^i | 0 < i \leq n \}$.

\CRASH is an untyped constant.
%
For each constant $c \neq \CRASH$
\newcommand\pcond[1]{\ensuremath{}}
\newcommand\const{\ensuremath{c}}
\begin{enumerate}
\item \hastype{\emptyset}{c}{\constty{\const}} and \iswellformed{}{\constty{c}}
%
\item If $\constty{c} = \tfun{x}{\tau_x}{\tau}$, then for each $v$, 
	$\ceval{\const}{v}$ is defined and 
	if \hastype{\emptyset}{v}{\tau_x} then
	\shastype{}{\interp{c}(v)}{\tau\sub{x}{v}},
	otherwise  $\interp{c}(v) = \CRASH$.
%	
\item If $\constty{c} = \tref{v}{B}{l}{e}$, 
	then 
	$c \in \interp{\constty{c}}$ and 
	$\forall c', c' \neq c. c' \not \in \interp{\constty{c}}$ 
%
\item If $\constty{D_T^i} = \tfun{x_1}{\tau_1}{\dots\tfun{x_n}{\tau_n}{\tau}}$, 
then $\tau_i$ are unrefined and for every $e_i$ with $0 < i \leq n$,
such that \hastype{\emptyset}{e_i}{\tau_i}, 
$D_T^i\ \overline{e_i}\in \interp{\tau\sub{x_i}{e_i}}$.
\end{enumerate}
\end{definition}


\subsection{Denotational Typing}
We define denotational typing as follows:
\begin{align*}
\shastype{\Gamma}{e}{\tau} & \doteq
	\forall \theta . \theta\in\interp{\Gamma}\Rightarrow \theta\ e \in \interp{\theta \ \tau}\\
\sissubtype{\Gamma}{\tau_1}{\tau_2} & \doteq 
	\forall \theta . \theta\in\interp{\Gamma}\Rightarrow \interp{\theta\ \tau_1} \subseteq \interp{\theta\ \tau_2}
\end{align*}

And prove that syntactic typing implies denotational typing, 
\ie a general version of Lemma~\ref{lem:denotation} of the paper.



\begin{lemma}{[Denotation Typing]}\label{lem:proofs:denotation}
\begin{enumerate}
\item If \issubtype{\Gamma}{\tau_1}{\tau_2} then \sissubtype{\Gamma}{\tau_1}{\tau_2}. 
\item If \hastype{\Gamma}{e}{\tau} then \shastype{\Gamma}{e}{\tau}.
\end{enumerate}
\end{lemma} 
\begin{proof}
Helping Lemma:
\begin{lemma}\label{lemma:closesem}
If \evals{e}{e'} then $e' \in \interp{\tau}$ \textit{iff} $e \in \interp{\tau}$.
\end{lemma}
\begin{proofsketch}
Since the validity of $e \in \interp{\tau}$ depends on the evaluated $e$, 
the if direction is evident.
The only if direction follows from the deterministic operational semantics.
\end{proofsketch}

%
\begin{enumerate}
\item \label{proof:ssub} Assume \issubtype{\Gamma}{\tau_1}{\tau_2}. 
We will prove it by induction on the derivation tree:

\begin{itemize}
\item\rsubbase. We have
$$\issubtype{\Gamma}{\tref{v}{B}{l}{e_1}}{\tref{v}{B}{l}{e_2}}$$
By inversion we get 
$$\issubref{\Gamma, v\colon B}{e_1}{e_2}$$
By inversion of \rimpl we have
$$	\forall \theta. \theta\in \interp{\Gamma}\Rightarrow
	\generalconditionImpl{\thetasub{\theta}{e_1}}
						{\thetasub{\theta}{e_2}}
\ (1)$$

We want to prove 
$$\sissubtype{\Gamma}{\tref{v}{B}{l}{e_1}}{\tref{v}{B}{l}{e_2}}$$
Equivalently
$$	
	\forall \theta . \iswellformedtheta{\Gamma}{\theta} \Rightarrow 
	\interp{\theta\ \tref{v}{B}{l}{e_1}} \subseteq \interp{\theta\ \tref{v}{B}{l}{e_2}}
$$

Since the labels are the same it suffices to prove that
\begin{align*}
	\forall \theta . \iswellformedtheta{\Gamma}{\theta} & \Rightarrow 
		\{e \mid \hastype{}{e}{B} 
 			\land 
			\generalconditionInterp{e}{\thetasub{\theta}{e_1\sub{v}{e}}} 
		\}	
	\\& \subseteq 
		\{e \mid \hastype{}{e}{B} 
			\land 
			\generalconditionInterp{e}{\thetasub{\theta}{e_2\sub{v}{e}}}
		 \}	
\end{align*}
Since $e \in \interp{B}$, we have \iswellformed{\Gamma,v\colon B}{\theta,\sub{v}{e}}.
So, from $(1)$ for $\theta := \theta,\sub{v}{e}$
we have 
$$	
	\generalconditionImpl
		{\thetasub{\theta}{e_1\sub{v}{e}}}
		{\thetasub{\theta}{e_2\sub{v}{e}}}
$$
\item\rsubfun Assume
$$
	\issubtype{\Gamma}{\tfunref{x}{\tau_x}{\tau}{v}{e_1}}{\tfunref{x}{\tau'_x}{\tau'}{v}{e_2}}
$$
By inversion we have
$$	
	\issubtype{\Gamma}{\tau'_x}{\tau_x} \qquad
	\issubtype{\Gamma, x \colon \tau'_x}{\tau}{\tau'} 
$$
By IH
$$	
	\sissubtype{\Gamma}{\tau'_x}{\tau_x} \ (1) \qquad
	\sissubtype{\Gamma, x \colon \tau'_x}{\tau}{\tau'} \ (2)
$$
We want to show that 
$$
	\sissubtype{\Gamma}
		{\tfunref{x}{\tau_x}{\tau}{v}{e_1}}
		{\tfunref{x}{\tau'_x}{\tau'}{v}{e_2}}
$$
Equivalently
$$	
	\forall \theta . \iswellformedtheta{\Gamma}{\theta} \Rightarrow 
	\interp{\thetasub{\theta}{\tfunref{x}{\tau_x}{\tau}{v}{e_1}}} 
	\subseteq 
	\interp{\thetasub{\theta}{\tfunref{x}{\tau'_x}{\tau'}{v}{e_2}}}
$$
Equivalently
\begin{align*}
	&\forall \theta. \iswellformedtheta{\Gamma}{\theta} \\&\Rightarrow 
	\{e \mid \hastype{}{e}{\erase{\tau_x} \rightarrow \erase{\tau}} 
	\land 
	\forall e_x \in \interp{\thetasub{\theta}{\tau_x}}. \
	 \eapp{e}{e_x} \in \interp{\thetasub{\theta}{\tau\sub{x}{e_x}}} 
	 \}\\ &
	\subseteq 
	\{e \mid \hastype{}{e}{\erase{\tau'_x} \rightarrow \erase{\tau'}} 
	\land 
	\forall e_x \in \interp{\thetasub{\theta}{\tau'_x}}. \
	 \eapp{e}{e_x} \in \interp{\thetasub{\theta}{\tau'\sub{x}{e_x}}} 
	 \}
\end{align*}
The above holds, as for any $e, e_x$
if $e_x \in \interp{\thetasub{\theta}{\tau_x'}}$
then by $(1)$
$e_x \in \interp{\thetasub{\theta}{\tau_x}}$.
Also, by $(2)$
if $\eapp{e}{e_x} \in \interp{\thetasub{\theta}{\tau\sub{x}{e_x}}}$
then
$\eapp{e}{e_x} \in \interp{\thetasub{\theta}{\tau'\sub{x}{e_x}}}$.
\end{itemize}


\item Assume \hastype{\Gamma}{e}{\tau}. 
We will prove it by induction on the derivation tree.
\begin{itemize}
\item\rtvar Assume \hastype{\Gamma}{e}{\tau}
	where $e \equiv x$.
	By inversion we have
	$$(x,\tau) \in \Gamma$$
	We need to show that 
	$$	\forall \theta . \iswellformedtheta{\Gamma}{\theta} 
		\Rightarrow \thetasub{\theta}{x} \in \interp{\thetasub{\theta}{\tau}}$$
	Which holds by the definition of well-formed substitutions.

\item\rtconst. Assume \hastype{\Gamma}{e}{\tau}
	where $e \equiv c$  and $\tau\equiv\constty{c}$.
	Then \shastype{\Gamma}{e}{\tau} holds by Definition \ref{def:constants}.

\item\rtsub Assume \hastype{\Gamma}{e}{\tau}.
	By inversion
	$$
	\hastype{\Gamma}{e}{\tau'}\ (1) \qquad
	\issubtype{\Gamma}{\tau'}{\tau}\ (2) \qquad
	\iswellformed{\Gamma}{\tau}\ (3)
	$$
%
	By IH on $(1)$ we have
	$$\shastype{\Gamma}{e}{\tau'}\ (4)$$
%
	By \ref{proof:ssub} on $(2)$
	$$\sissubtype{\Gamma}{\tau'}{\tau}\ (5)$$
%
	By $(4)$ and $(5)$ we get
	$$\shastype{\Gamma}{e}{\tau}$$

\item\rtfun Assume \hastype{\Gamma}{e}{\tau},
	where $e \equiv \efun{x}{}{e'}$ and 
	$\tau \equiv\tfun{x}{\tau'_x}{\tau'}$.
	By inversion we get
	$$
	\hastype{\Gamma, x\colon\tau'_x}{e'}{\tau'}\ (1) \qquad
	\iswellformed{\Gamma}{\tau'_x}\ (2)
	$$
	By IH on $(1)$ we have
	$$
	\shastype{\Gamma, x\colon\tau'_x}{e'}{\tau'}\ (3)
	$$
	Equivalently
	$$	
	\forall \theta . \iswellformedtheta{(\Gamma,x\colon\tau'_x)}{(\theta\sub{x}{e_x})} 
		\Rightarrow \thetasub{(\theta\sub{x}{e_x})}{e'} \in 
		\interp{\thetasub{(\theta\sub{x}{e_x})}{\tau'}}\\
	$$
	Or
	$$	
	\forall \theta . \iswellformedtheta{\Gamma}{\theta} \Rightarrow
	\forall e_x . e_x \in \interp{\tau'_x} \Rightarrow
		\thetasub{\theta}{\eapp{e}{e_x}} \in \interp{\thetasub{\theta}{\tau'\sub{x}{e_x}}}\\
	$$
%
	Moreover, $\hastypebase{}{e}{\erase{\tau'_x}\rightarrow{\erase{\tau}}}$.
%
	So,
	$$	
	\forall \theta . \iswellformedtheta{\Gamma}{\theta}. \thetasub{\theta}{e}\in \interp{\thetasub{\theta}{\tau}}
	$$
	Or, $$\shastype{\Gamma}{e}{\tau}$$

\item\rtapp. Assume \hastype{\Gamma}{e}{\tau},
	where $e\equiv\eapp{e_1}{e_2}$ and $\tau\equiv\tau'\sub{x}{e_2}$.
	By inversion:
	$$
	\hastype{\Gamma}{e_1}{(\tfunref{x}{\tau'_{x}}{\tau'}{v}{e_r})}\ (1)\qquad
	\hastype{\Gamma}{e_2}{\tau'_{x}}\ (2)
	$$
	By IH we get
	$$
	\shastype{\Gamma}{e_1}{(\tfunref{x}{\tau'_{x}}{\tau'}{v}{e_r})}\ (3)\qquad
	\shastype{\Gamma}{e_2}{\tau'_{x}}\ (4)
	$$
	So 
	$$\forall \theta. \iswellformedtheta{\Gamma}{\theta}\Rightarrow
	\forall e_x \in \interp{\thetasub{\theta}{\tau'_x}} \Rightarrow
		\eapp{(\thetasub{\theta}{e_1})}{e_x} \in 
		\interp{\thetasub{\theta}{\tau'\sub{x}{e_x}}}
	\ (5)$$
	and
	$$\forall \theta. \iswellformedtheta{\Gamma}{\theta}\Rightarrow
		\thetasub{\theta}{e_2} \in 
		\interp{\thetasub{\theta}{\tau'_x}}
	\ (6)$$
%
	From $(5)$ and $(6)$, we get
	$$\forall \theta. \iswellformedtheta{\Gamma}{\theta}\Rightarrow
		\theta\ e \in \interp{\thetasub{\theta}{\tau}}
	$$
	Or $$\shastype{\Gamma}{e}{\tau}$$

\item\rtlet. Assume \hastype{\Gamma}{e}{\tau}, 
	where $e \equiv\elet{x}{e_x}{e'}$.
	By inversion:
	$$
	\hastype{\Gamma}{e_x}{\tau_{x}}\ (1) \qquad
	\hastype{\Gamma,x\colon\tau_x}{e'}{\tau}\ (2)\qquad
	\iswellformed{\Gamma}{\tau}\ (3)
	$$
	By IH we get
	$$
	\shastype{\Gamma}{e_x}{\tau_{x}}\ (4) \qquad
	\shastype{\Gamma,x\colon\tau_x}{e'}{\tau}\ (5)
	$$
	By $(5)$
	$$\forall \theta'. \iswellformedtheta{\Gamma, x:\tau_x}{\theta'}\Rightarrow
		\thetasub{\theta'}{e'} \in \interp{\thetasub{\theta'}{\tau}}
		\ (6)
	$$
	By $(4)$, 
	$$
	 	\iswellformedtheta{\Gamma}{\theta}
		\Rightarrow 
		\iswellformedtheta{\Gamma, x:\tau_x}{\theta\sub{x}{e_x}}
	\ (7)$$
	From $(6)$, $(7)$ and $(3)$, we get
	$$\forall \theta. \iswellformedtheta{\Gamma}{\theta}\Rightarrow
		\thetasub{\theta}{e'\sub{x}{e_x}} \in \interp{\thetasub{\theta}{\tau}}
	$$
	By Lemma \ref{lemma:closesem}, we get
	$$\forall \theta. \iswellformedtheta{\Gamma}{\theta}\Rightarrow
		\thetasub{\theta}{e} \in \interp{\thetasub{\theta}{\tau}}
	$$
	So, $$\shastype{\Gamma}{e}{\tau}$$
\item\rtbot Assume \hastype{\Gamma}{e}{\tau}, 
	where $e \equiv\ebot$ and $\tau \equiv \tref{v}{B}{}{p}$.
	Since \ebot does not evaluate, 
	$$\forall \theta. \iswellformedtheta{\Gamma}{\theta}\Rightarrow
		\thetasub{\theta}{e} \in \interp{\thetasub{\theta}{\tau}}
	$$
	So, $$\shastype{\Gamma}{e}{\tau}$$

\item\rtcase Assume \hastype{\Gamma}{e}{\tau}, 
	where $e' \equiv \ecase{e}{D^i_T}{\overline{y}}{e_i}{x}$.
	By inversion
$$
	l \not \in \{\finite, \trivial\} \Rightarrow \tau \ \text{is}\ \Div\ (1)\qquad
	\hastype{\Gamma}{e}{\tref{v}{T}{l}{e_T}} \ (2)\qquad
	 \iswellformed{\Gamma}{\tau}\ (3)
$$
$$
\forall i. 0 < i \leq \arity{T}\{
$$
$$
	\constty{D^i_T} = \tfun{y_1}{\tau_1}{\dots\rightarrow\tfun{y_n}{\tau_n}{\tref{v}{T}{}{e_{T_i}}}}\ (4)
$$
$$
		\hastype{\Gamma,  
				\overline{y_i\colon \tau_i},
				x\colon\tlref{v}{T}{\restrictdecidable{\trivial}
				{\addtechnical{}{\ltrivial}}
				}{e_T \land e_{T_i}}}{e_i}{\tau}\ (5) \}
$$

By IH on $(2)$ we get 
$$
	\shastype{\Gamma}{e'}{\tref{v}{T}{l}{e_T}}\ (6)
$$

We fix a $\theta$ such that $\iswellformedtheta{\Gamma}{\theta}$
We split cases on whether \thetasub{\theta}{e'} evaluates to a WNF or not:
\begin{itemize}
\item If \evals{\thetasub{\theta}{e'}}{v}.
By $(6)$, for some $i$ such that $0 < i \leq \arity{T}$, 
%
$\evals{\thetasub{\theta}{e'}}{D^i_T\ \overline{e_j}}$.

By IH on $(4)$ and the Definition~\ref{def:constants}
$$
		\shastype{\Gamma}{e_i\sub{y_i}{e_j}\sub{x}{e'}}{\tau}
$$
Finally, by Lemma~\ref{lemma:closesem}
$$
		\shastype{\Gamma}{e}{\tau}
$$
\item If $\thetasub{\theta}{e'}$, then by $(6)$
$l \not \in \{\finite, \trivial \}$.
Moreover, $e$ diverges so it trivially belongs to the 
interpretation of any \Div type, or by $(1)$
$$
		\shastype{\Gamma}{e}{\tau}
$$
%%$$
%%\interp{\tref{v}{T}{}{p}} \doteq
%%\{
%%e \mid \hastypebase{}{e}{T}, 
%%\evals{e}{D^i_T\overline{e_j}} \Rightarrow
%%\constty{D^i_T} = \tfun{y_1}{\tau_1}{\dots\rightarrow\tfun{y_n}{\tau_n}{\tref{v}{T}{}{q}}} \Rightarrow
%%e_i \in \interp{\tau_i\sub{y_j}{e_j}}, 
%%\evals{p \land q\sub{y_j}{e_j}}{\etrue}
%%\}
%%$$
\end{itemize}
\end{itemize}
\end{enumerate}

\end{proof}

We define \iswellformed{}{\Gamma}
as \iswellformed{}{\emptyset} and if \iswellformed{\Gamma}{\tau} then \iswellformed{}{\Gamma, x:\tau}.
Now, using Lemma~\ref{lem:proofs:denotation} we prove substitution Lemma:
\begin{lemma}{[Substitution]}\label{lemma:substitution}
If \hastype{\Gamma}{e_x}{\tau_x} and \iswellformed{}{\Gamma, x\colon\tau_x ,\Gamma'}, then 
\begin{enumerate}
\item If 
	\issubtype{\Gamma, x\colon\tau_x, \Gamma'}{\tau_1}{\tau_2}
	then
	\issubtype{\Gamma, \sub{x}{e_x}\Gamma'}{\sub{x}{e_x}\tau_1}{\sub{x}{e_x}\tau_2}.
\item If 
	\hastype{\Gamma, x\colon\tau_x, \Gamma'}{e}{\tau}
	then
	\hastype{\Gamma, \sub{x}{e_x}\Gamma'}{\sub{x}{e_x}e}{\sub{x}{e_x}\tau}.
\item If 
	\iswellformed{\Gamma, x\colon\tau_x, \Gamma'}{\tau}
	then
	\iswellformed{\Gamma, \sub{x}{e_x}\Gamma'}{\sub{x}{e_x}\tau}.
\end{enumerate}
\end{lemma}
\begin{proof}
\newcommand\generalconditionImpol[2]{\ensuremath{\evals{#1}{\etrue}\Rightarrow \evals{#2}{\etrue}}}
If \hastype{\Gamma}{e_x}{\tau_x} and \iswellformed{\Gamma, x\colon\tau_x ,\Gamma'}, then 
\begin{enumerate}
\item\label{proof:sub:sub} Assume
	$$\issubtype{\Gamma, x\colon\tau_x, \Gamma'}{\tau_1}{\tau_2}$$
We will prove the lemma by induction on the derivation tree.
\begin{itemize}
\item \rsubbase
Assume \issubtype{\Gamma, x\colon\tau_x, \Gamma'}{\tau_1}{\tau_2}
where $\tau_1 \equiv \tref{v}{B}{l}{e_1}$
and   $\tau_2 \equiv \tref{v}{B}{l}{e_2}$.
By inversion
	$$
	\issubref{\Gamma, x\colon\tau_x, \Gamma',v:B}{e_1}{e_2}
	$$
By inversion
	\begin{align*}
	\forall &\theta, e'_x, \theta',e .
	\iswellformedtheta{\Gamma, x\colon\tau_x, \Gamma',v:B}
		{\theta\sub{x}{e'_x}\theta'\sub{v}{e}}\\& \Rightarrow
	\generalconditionImpl{\thetasub{\theta\sub{x}{e'_x}\theta'\sub{v}{e}}{e_1}\\&}
						 {\thetasub{\theta\sub{x}{e'_x}\theta'\sub{v}{e}}{e_2}}
	\end{align*}

Since \hastype{\Gamma}{e_x}{\tau_x}, so
	\begin{align*}
	\forall &\theta, \theta',e .
	\iswellformedtheta{\Gamma,x\colon\tau_x, \Gamma',v:B}{\theta \sub{x}{e_x}\theta'\sub{v}{e}}\\& \Rightarrow
	\generalconditionImpl{\thetasub{\theta\sub{x}{e_x}\theta'\sub{v}{e}}{e_1}\\&}
						 {\thetasub{\theta\sub{x}{e_x}\theta'\sub{v}{e}}{e_2}}
	\end{align*}
Since \hastype{\Gamma}{e_x}{\tau_x}, so
	\begin{align*}
	\forall &\theta, \theta',e .
	\iswellformedtheta{\Gamma,\sub{x}{e_x}\Gamma',v:B}{\theta\theta'\sub{v}{e}}\\& \Rightarrow
	\generalconditionImpl{\thetasub{\theta\theta'\sub{v}{e}}{e_1\sub{x}{e_x}}\\&}
						 {\thetasub{\theta\theta'\sub{v}{e}}{e_2\sub{x}{e_x}}}
	\end{align*}
So,
	$$
	\issubref{\Gamma, \sub{x}{e_x}\Gamma',v:B}{e_1\sub{x}{e_x}}{e_2\sub{x}{e_x}}
	$$
Or
	$$
	\issubtype{\Gamma, \sub{x}{e_x}\Gamma',v:B}{t_1\sub{x}{e_x}}{t_2\sub{x}{e_x}}
	$$
\item \rsubfun
Assume \issubtype{\Gamma, x\colon\tau_x, \Gamma'}{\tau_1}{\tau_2},
where $\tau_1 \equiv \tfun{y}{\tau_y}{\tau}$
and   $\tau_2 \equiv \tfun{y}{\tau'_y}{\tau'}$.
By inversion
	$$
	\issubtype{\Gamma, x\colon\tau_x, \Gamma'}{\tau'_y}{\tau_y}\ (1) \qquad
	\issubtype{\Gamma, x\colon\tau_x, \Gamma',y\colon\tau'_y}{\tau}{\tau'}\ (2)
	$$
By IH	
	$$
	\issubtype{\Gamma, \sub{x}{e_x}\Gamma'}{\tau'_y\sub{x}{e_x}}{\tau_y\sub{x}{e_x}} 
	$$
	$$
	\issubtype{\Gamma, \sub{x}{e_x}\Gamma',y\colon\tau'_y\sub{x}{e_x}}{\tau\sub{x}{e_x}}{\tau'\sub{x}{e_x}}
	$$
By rule \rsubfun	
	$$
	\issubtype{\Gamma, \sub{x}{e_x}\Gamma'}{\tau_1\sub{x}{e_x}}{\tau_2\sub{x}{e_x}}
	$$
\end{itemize}


\item \label{proof:sub:type} 
Assume 
	\hastype{\Gamma, x\colon\tau_x, \Gamma'}{e}{\tau}.
We will prove the lemma by induction on the derivation tree.
\begin{itemize}
\item \rtvar Assume \hastype{\Gamma, x\colon\tau'_x, \Gamma'}{e}{\tau},
where $e \equiv y$.
By inversion 
$$(y,\tau )\in \Gamma, x\colon\tau'_x, \Gamma'$$
Assume
$$(y,\tau)\in \Gamma$$
By rule \rtvar
$$\hastype{\Gamma,\sub{x}{e_x}\Gamma'}{e}{\tau}$$
Since \iswellformed{}{\Gamma}, $x$ cannot appear in $\tau$
so $\tau\sub{x}{e_x}\equiv\tau$.
Also, $x\neq y$, so $e\sub{x}{e_x}\equiv e$.
So,
$$\hastype{\Gamma,\sub{x}{e_x}\Gamma'}{e\sub{x}{e_x}}{\tau\sub{x}{e_x}}$$
%
Assume
$$y \equiv x$$
By lemma's assumption 
$$\hastype{\Gamma}{e_x}{\tau_x}$$
so
$$\hastype{\Gamma,\sub{x}{e_x}\Gamma'}{e_x}{\tau'_x}$$
Since $x = y$, $e\sub{x}{e_x} \equiv e_x$.
Also, since $x \notin Dom(\Gamma)$ 
it cannot appear in $\tau'_x$,so
$\tau\sub{x}{e_x} \equiv \tau \equiv \tau'_x$.
So,
$$\hastype{\Gamma,\sub{x}{e_x}\Gamma'}{e\sub{x}{e_x}}{\tau\sub{x}{e_x}}$$
%
Otherwise, assume
$$(y,\tau)\in \Gamma'$$
So,
$$(y,\sub{x}{e_x}\tau)\in \sub{x}{e_x}\Gamma'$$
Also, $e\sub{x}{e_x}\equiv e \equiv y$.
By which and rule \rtvar, we get
$$\hastype{\Gamma,\sub{x}{e_x}\Gamma'}{e\sub{x}{e_x}}{\tau\sub{x}{e_x}}$$

\item Case \rtconst.
Assume \hastype{\Gamma, x\colon\tau_x, \Gamma'}{e}{\tau},
where $e \equiv c$ and $\tau\equiv\constty{c}$.
Since $e\sub{x}{e_x} \equiv e$ and $\tau\sub{x}{e_x}\equiv\tau$
$$\hastype{\Gamma,\sub{x}{e_x}\Gamma'}{e\sub{x}{e_x}}{\tau\sub{x}{e_x}}$$

\item\rtsub
Assume \hastype{\Gamma, x\colon\tau_x, \Gamma'}{e}{\tau}.
By inversion
$$
\hastype{\Gamma, x\colon\tau_x, \Gamma'}{e}{\tau'}\ (1)\qquad
\issubtype{\Gamma, x\colon\tau_x, \Gamma'}{\tau'}{\tau} \ (2)
$$
$$
\iswellformed{\Gamma, x\colon\tau_x, \Gamma'}{\tau} \ (3)
$$
By IH, \ref{proof:sub:sub} and \ref{proof:sub:wf}
$$
\hastype{\Gamma, \sub{x}{e_x}\Gamma'}{\sub{x}{e_x}e}{\sub{x}{e_x}\tau'}
$$
$$
\issubtype{\Gamma, \sub{x}{e_x}\Gamma'}{\sub{x}{e_x}\tau'}{\sub{x}{e_x}\tau}
$$
$$
\iswellformed{\Gamma, \sub{x}{e_x}\Gamma'}{\sub{x}{e_x}\tau}
$$
By rule \rtsub
$$\hastype{\Gamma,\sub{x}{e_x}\Gamma'}{e\sub{x}{e_x}}{\tau\sub{x}{e_x}}$$

\item\rtfun Assume \hastype{\Gamma, x\colon\tau_x, \Gamma'}{e}{\tau},
where $e\equiv\efun{y}{e'}$ and $\tau\equiv\tfun{y}{\tau'_y}{\tau'}$.
By inversion
	$$
	\hastype{\Gamma, x\colon\tau_x, \Gamma', y\colon\tau'_y}{e'}{\tau'}\ (1)\qquad
	\iswellformed{\Gamma, x\colon\tau_x, \Gamma'}{\tau'_y}\ (2)
	$$
By IH and \ref{proof:sub:wf}
	$$
	\hastype{\Gamma,\sub{x}{e_x} \Gamma', y\colon\sub{x}{e_x}\tau'_y}{\sub{x}{e_x}e'}{\sub{x}{e_x}\tau'} 	$$
	$$
	\iswellformed{\Gamma, \sub{x}{e_x}\Gamma'}{\sub{x}{e_x}\tau'_y}
	$$
	By rule \rtfun
	$$
	\hastype{\Gamma,\sub{x}{e_x} \Gamma'}{\sub{x}{e_x}e}{\sub{x}{e_x}\tau}
	$$
	
\item\rtapp Assume \hastype{\Gamma, x\colon\tau_x, \Gamma'}{e}{\tau},
where $e\equiv\eapp{e_1}{e_2}$ and $\tau\equiv\tau'\sub{y}{e_2}$.
By inversion
	$$
	\hastype{\Gamma, x\colon\tau_x, \Gamma'}{e_1}{\tfun{y}{\tau'_y}{\tau'}}\ (1)\qquad
	\hastype{\Gamma, x\colon\tau_x, \Gamma'}{e_2}{{\tau'_y}}\ (2)
	$$
By IH 
	$$
	\hastype{\Gamma,\sub{x}{e_x} \Gamma'}{\sub{x}{e_x}e_1}{\sub{x}{e_x}\tfun{y}{\tau'_y}{\tau'}} \qquad
	$$
	$$
	\hastype{\Gamma,\sub{x}{e_x} \Gamma'}{\sub{x}{e_x}e_2}{\sub{x}{e_x}{\tau'_y}}
	$$
	By rule \rtapp
	$$
	\hastype{\Gamma,\sub{x}{e_x} \Gamma'}{\sub{x}{e_x}e}{\sub{x}{e_x}\tau}
	$$

\item\rtlet Assume \hastype{\Gamma, x\colon\tau_x, \Gamma'}{e}{\tau},
where $e\equiv\elet{y}{e_y}{e'}$.
By inversion
	$$
	\hastype{\Gamma, x\colon\tau_x, \Gamma'}{e_y}{\tau_y}\ (1) \qquad
	\hastype{\Gamma, x\colon\tau_x, \Gamma', y\colon\tau_y}{e}'{\tau}\ (2)
	$$
	$$
	\iswellformed{\Gamma, x\colon\tau_x, \Gamma'}{\tau}\ (3)
	$$
By IH and \ref{proof:sub:wf}	
	$$
	\hastype{\Gamma, \sub{x}{e_x}\Gamma'}{e_y}{\tau_y}\ (4) \qquad
	\hastype{\Gamma, \sub{x}{e_x}\Gamma', y\colon\tau_y}{e}'{\tau}\ (5)
	$$
	$$
	\iswellformed{\Gamma, \sub{x}{e_x}\Gamma'}{\tau}\ (6)
	$$
So, 	$$
	\hastype{\Gamma,\sub{x}{e_x} \Gamma'}{\sub{x}{e_x}e}{\sub{x}{e_x}\tau}
	$$

\item\rtcase This case is similar to \rtlet.
\item\rtbot Assume \hastype{\Gamma, x\colon\tau_x, \Gamma'}{e}{\tau},
where $e\equiv\ebot$.
By inversion
$$	\iswellformed{\Gamma, x\colon\tau_x, \Gamma'}{\tau}$$
By \ref{proof:sub:wf}
$$	\iswellformed{\sub{x}{e_x}\Gamma'}{\sub{x}{e_x}\tau}$$
By rule \rtbot
	$$
	\hastype{\Gamma,\sub{x}{e_x} \Gamma'}{\sub{x}{e_x}e}{\sub{x}{e_x}\tau}
	$$
\end{itemize}




\item \label{proof:sub:wf}
Assume \iswellformed{\Gamma, x\colon\tau_x, \Gamma'}{\tau}.
We will prove it by induction on the derivation.
\begin{itemize}
\item \rwbase
Assume \iswellformed{\Gamma, x\colon\tau_x, \Gamma'}{\tau},
where $\tau\equiv\tref{v}{B}{l}{e}$.
By inversion
$$\hastypebase{\erase{\Gamma, x\colon\tau_x, \Gamma'},v\colon B}{e}{\tbool}$$
So,
$$\hastypebase{\erase{\Gamma, \sub{x}{e_x}\Gamma'},v\colon B}{e\sub{x}{e_x}}{\tbool}$$
By rule \rwbase
$$\iswellformed{\Gamma, \sub{x}{e_x}\Gamma'}{\tref{v}{B}{l}{e\sub{x}{e_x}}}$$
Or 
$$\iswellformed{\Gamma, \sub{x}{e_x}\Gamma'}{\tau\sub{x}{e_x}}$$
\item \rwfun
Assume \iswellformed{\Gamma, x\colon\tau_x, \Gamma'}{\tau},
where $\tau\equiv \tfun{y}{\tau'_y}{\tau'}$.
By inversion, we get
$$
	\iswellformed{\Gamma, x\colon\tau_x, \Gamma'}{\tau_x} \qquad
	\iswellformed{\Gamma, x\colon\tau_x, \Gamma', y \colon \tau'_y}{\tau'}
$$
By IH
$$
	\iswellformed{\Gamma, \sub{x}{e_x} \Gamma'}{\tau_x\sub{x}{e_x}}\qquad
	\iswellformed{\Gamma, \sub{x}{e_x}(\Gamma', y \colon \tau'_y)}{\tau'\sub{x}{e_x}}
$$
Due to $\alpha$-renaming, $x \neq y$, so
$$
	\iswellformed{\Gamma, \sub{x}{e_x} \Gamma'}{\tau'_y\sub{x}{e_x}}\qquad
	\iswellformed{\Gamma, \sub{x}{e_x}\Gamma', y \colon \sub{x}{e_x}\tau'_y}{\tau'\sub{x}{e_x}}
$$
By \rwfun
$$
	\iswellformed{\Gamma, \sub{x}{e_x} \Gamma'}{\tfun{y}{\tau'_y\sub{x}{e_x}}{\tau'\sub{x}{e_x}}}
$$
Or
$$
	\iswellformed{\Gamma, \sub{x}{e_x} \Gamma'}{\tau\sub{x}{e_x}}
$$
\end{itemize}
\end{enumerate}
\end{proof}


\subsection{Soundness}
Figure~\ref{fig:proofs:botomless} defines a \botomless{\bullet} predicate on expressions:

%\def\figone{%
\begin{figure*}[t!]
$$
\botomless{c} \qquad\botomless{x} \qquad \lnot \botomless{\ebot}
$$
$$
\botomless{D\ \overline{e_i}} \Leftrightarrow \bigwedge\botomless{e_i} \qquad
\botomless {\efun{x}{}{e}} \Leftrightarrow \botomless{e}
$$
$$
\botomless {e_1 \ e_2} \Leftrightarrow \botomless{e_1} \land \botomless{e_2} 
$$
$$
\botomless {\elet{x}{e_1}{e_2}} \Leftrightarrow \botomless{e_1} \land \botomless{e_2}
$$
$$
\botomless {\ecase{e}{D_i}{\overline{x}}{e_i}{x}} \Leftrightarrow \botomless{e} \land \bigwedge\botomless{e_i}
$$
\caption{\botomless{e}}
\label{fig:proofs:botomless}
\end{figure*}
%\global\let\figone\relax}
We prove Preservation and Progress only on expressions that do not contain \ebot:
%
\begin{lemma}[Preservation]\label{lemma:preservation}
If \hastype{\emptyset}{e}{\tau}, \botomless{e} and \eval{e}{e'} then \hastype{\emptyset}{e'}{\tau}.
\end{lemma}
\begin{proof}
Helping Lemmata:
\begin{lemma}\label{lemma:wftypes}
If \hastype{\Gamma}{e}{\tau} and \iswellformed{}{\Gamma} then \iswellformed{\Gamma}{\tau}.
\end{lemma}
\begin{proofsketch}
By case split on the derivation \iswellformed{\Gamma}{\tau}
\end{proofsketch}
\begin{lemma}\label{lemma:eval}
If \eval{e}{e'} then
	\issubtype{\emptyset}{\tau\sub{x}{e'}}{\tau\sub{x}{e}}
\end{lemma}
\begin{proofsketch}
By case split on the derivation \issubtype{\Gamma}{\tau\sub{x}{e'}}{\tau\sub{x}{e}}
\end{proofsketch}

Assume \botomless{e} and \hastype{\emptyset}{e}{\tau} and \eval{e}{e'}. 
We will prove the lemma by induction on the derivation tree. 
\begin{itemize}
\item Cases \rtvar, \rtconst, \rtfun and \rtbot trivially hold
       as there is no $e'$ for which \eval{e}{e'}. 

\item Case \rtsub. Assume \hastype{\emptyset}{e}{\tau}.
By inversion
$$	\hastype{\emptyset}{e}{\tau'} \ (1) \qquad
	\issubtype{\emptyset}{\tau'}{\tau}\ (2) \qquad
	\iswellformed{\emptyset}{\tau}\ (3)
$$

By IH on $(1)$
$$	\hastype{\emptyset}{e'}{\tau'} $$
By which, $(2), (3)$ and \rtsub
$$	\hastype{\emptyset}{e'}{\tau}$$

\item Case \rtlet. Assume \hastype{\emptyset}{e}{\tau}, 
where $e \equiv \elet{x}{e_x}{e_0}$. Then 
 $e' \equiv e_0\sub{x}{e_x}$.
By inversion
$$
	\hastype{\emptyset}{e_x}{\tau_{x}} \ (1) \qquad
	\hastype{x\colon\tau_x}{e_0}{\tau} \ (2) \qquad
	\iswellformed{\emptyset}{\tau} \ (3)
$$

By $(1)$, $(2)$ and Lemma \ref{lemma:substitution}, 
$$\hastype{\emptyset}{e'}{\tau\sub{x}{e_x}} \ (4)$$
By $(3)$ $x$ does not appear free in $\tau$, so, $\tau\sub{x}{e_x} \equiv \tau$ and
$$\hastype{\emptyset}{e'}{\tau}$$

\item Case \rtapp. Assume
$$	\hastype{\emptyset}{e}{\tau}\ (1)$$
where $e \equiv \eapp{e_1}{e_2}$, and
	  $\tau\equiv\tau'\sub{x}{e_2}$

By inversion
$$	
	\hastype{\emptyset}{e_1}{(\tfun{x}{\tau_{x}}{\tau'})}\ (2) \qquad
	\hastype{\emptyset}{e_2}{\tau_{x}}\ (3)
$$

We split cases on the structure of $e$.
\begin{itemize}
\item $e\equiv \eapp{c}{v_2}$.
Then, $e'\equiv\interp{c}(v_2)$.
By Definition \ref{def:constants},
$$\hastype{\emptyset}{e'}{\tau}$$

\item $e\equiv \eapp{c}{e_2}$ where $e_2$ is botomless and not a value, 
Then, by (3) and Lemma~\ref{lemma:progress},
\eval{e_2}{e_2'}, and $e' \equiv \eapp{e_1}{e_2'}$.
%
By IH on $(3)$
$$	\hastype{\emptyset}{e_2'}{\tau_{x}}$$
By which, $(2)$ and rule \rtapp we get
$$\hastype{\emptyset}{e'}{\tau'\sub{x}{e_2'}}\ (4)$$
By Lemma \ref{lemma:eval}
$$
	\issubtype{\emptyset}{\tau'\sub{x}{e_2'}}{\tau'\sub{x}{e_2}}\ (5)
$$
By $(1)$ and Lemma \ref{lemma:wftypes}, since \iswellformed{}{\emptyset}
$$
	\iswellformed{\emptyset}{\tau'\sub{x}{e_2}}\ (6)
$$
By $(4), (5), (6)$ and rule \rtsub
$$	\hastype{\emptyset}{e'}{\tau}$$

\item $e \equiv \eapp{\efun{x}{e_x}}{e_2}$.
Then, $e' \equiv e_x\sub{x}{e_2}$.

By inversion on $(2)$
$$
	\hastype{x\colon\tau_x}{e_x}{\tau'}
$$
By which, $(3)$ and Lemma \ref{lemma:substitution} (since \iswellformed{}{x\colon\tau_x})
$$\hastype{\emptyset}{e'}{\tau'}$$

\item $e \equiv \eapp{e_1}{e_2}$, where $e_1$ is botomless and not a value.
Then, by $(2)$ and Lemma \ref{lemma:progress}, \eval{e_1}{e_1'} and 
$e'\equiv\eapp{e_1'}{e_2}$
By IH on $(2)$
$$	\hastype{\emptyset}{e_1'}{(\tfun{x}{\tau_{x}}{\tau'})}
$$
By which, $(3)$ and rule \rtapp we get
$$	\hastype{\emptyset}{e'}{\tau}$$
\end{itemize}
\item Case \rtcase, assume $\hastype{\emptyset}{e}{\tau}$, 
where $e \equiv \ecase{e_0}{D_i}{\overline{y}}{e}{x}$.
By inversion
$$	\hastype{\emptyset}{e_T}{\tref{v}{T}{l}{e_T}}\ (1) $$
$$	 \iswellformed{\emptyset}{\tau}\ (2)$$
$$	\forall i, 0 < i \leq \arity{T}. (
		\constty{D^i_T} = \tfun{x_1}{\tau_1}{\dots\tfun{x_n}{\tau_n}{\tref{v}{T}{l}{e_{T_i}}}}\ (3)
$$
$$		\theta = \overline{\sub{x_i}{y_i}}\ (4) \qquad
		\hastype{x\colon\tlref{v}{T}{}{e_t \land e_{T_i}}, 
						\overline{y_i\colon \theta\ \tau_i}}{e_i}{\tau}\ (5)	
	)
$$
We split cases on the structure of $e_T$.
\begin{itemize}
\item Assume that $e_T \equiv D^i_T\ \overline{e}$,
then $e' \equiv e_i \sub{x}{e} \overline{\sub{y_i}{e_i}}$.

By $(5)$
$$
		\hastype{\overline{y_i\colon \theta\ \tau_i}, 
		x\colon\tlref{v}{T}{l}{e_t \land \theta\ e_{T_i}}}{e_i}{\tau}\	
$$

By inversion on $(1)$ 
\hastype{\emptyset}{e_j}{\tau_j\overline{\sub{x}{e}}} 
and
\hastype{\emptyset}{D^i_T\ \overline{e}}{\tref{v}{T}{}{e_{T_i}}\overline{\sub{x}{e}}}. 
So,
\hastype{\emptyset}{e_j}{\tau_j\overline{\sub{x}{y}\sub{y}{e}}} 
and
\hastype{\emptyset}{D^i_T\ \overline{e}}{\tref{v}{T}{}{e_{T_i}}\overline{\sub{x}{y}\sub{y}{e}}}. 
And,
\shastype{\emptyset}{e_j}{\tau_j\overline{\sub{x}{y}\sub{y}{e}}} 
and
\shastype{\emptyset}{D^i_T\ \overline{e}}{\tref{v}{T}{}{e_{T_i}}\overline{\sub{x}{y}\sub{y}{e}}}. 

Finally, by Definition~\ref{def:constants}
\shastype{\emptyset}{D^i_T\ \overline{e}}{\tref{v}{T}{}{e_{T_i} \land e_t}\overline{\sub{x}{y}\sub{y}{e}}}. 

Then, by Lemma \ref{lemma:substitution}

\hastype{\emptyset}{e'}{\tau \overline{\sub{y_i}{e_i}}\sub{x}{e}}.

Finally, by $(2)$, $\tau \overline{\sub{y_i}{e_i}}\sub{x}{e} \equiv \tau$, so
$$
\hastype{\emptyset}{e'}{\tau}.
$$

\item Otherwise, by $(1)$ and Lemma \ref{lemma:progress} \eval{e_0}{e'_0}.
So $e' \equiv \ecase{e'_0}{D_i}{\overline{y}}{e}{x}$.
By IH \hastype{\emptyset}{e'_0}{\tref{v}{T}{}{e_T}}, 
by which and $(1) - (6)$ $$\hastype{\emptyset}{e'}{\tau}$$
\end{itemize}

\end{itemize}
\end{proof}
\begin{lemma}[Progress]\label{lemma:progress}
If \hastype{\emptyset}{e}{\tau}, \botomless{e} and $e \neq v$ 
then there exists an $e'$ such that \botomless{e'} and \eval{e}{e'}.
\end{lemma}
\begin{proof}
Assume \hastype{\emptyset}{e}{\tau}.
We will prove the Lemma by induction on the derivation tree.
\begin{itemize}
\item Case \rtvar cannot occur, as $\Gamma = \emptyset$
\item Case \rtbot is trivial, 
		as $\lnot \botomless{e}$.
\item Cases \rtconst and \rtfun are trivial, 
		as $e = v$.
\item Case \rtsub. Assume \hastype{\emptyset}{e}{\tau}.
By inversion
$$	\hastype{\emptyset}{e}{\tau'}$$
By IH 
either $e \equiv v$ or there exists an botomless $e'$ such that \eval{e}{e'}.
\item Case \rtapp. Assume $$\hastype{\emptyset}{e}{\tau}\ (1)$$
where $e\equiv\eapp{e_1}{e_2}$ and $\tau\equiv\tau'\sub{x}{e_2}$.
By inversion
$$
	\hastype{\emptyset}{e_1}{(\tfun{x}{\tau_{x}}{\tau})}\ (2)\qquad
	\hastype{\emptyset}{e_2}{\tau_{x}}\ (3)
$$

We split cases on the structure of $e$.
\begin{itemize}
\item $e\equiv \eapp{c}{v_2}$.
Then, $e'\equiv\interp{c}(v_2)$ which is botomless by Definition of constants.

\item $e\equiv \eapp{c}{e_2}$ where $e_2$ is not a value, 
By IH on $(3)$ \eval{e_2}{e_2'} and  $e' \equiv \eapp{e_1}{e_2'}$

\item $e \equiv \eapp{\efun{x}{e_x}}{e_2}$.
Then, $e' \equiv e_x\sub{x}{e_2}$, which does not contain bottom.

\item $e \equiv \eapp{e_1}{e_2}$, where $e_1 \neq v$.
Then, by IH on $(2)$ \eval{e_1}{e_1'} and 
$e'\equiv\eapp{e_1'}{e_2}$.
\end{itemize}

\item Case \rtlet. Assume \hastype{\emptyset}{e}{\tau}, where 
$e \equiv \elet{x}{e_x}{e_0}$, then $e'\equiv e_0\sub{x}{e_x}$ which is botomless.

\item Case \rtcase. Assume \hastype{\emptyset}{e}{\tau}, where
$e \equiv \ecase{e_T}{D_{T_i}}{\overline{y}}{e_i}{x}$.
By inversion, 
$$
	\hastype{\Gamma}{e}{\tref{v}{T}{e_T}}\ (1)
$$
We split cases on the structure of $e_T$
\begin{itemize}
\item If $e_T$ is a value, then by $(1)$ it is of the form $e_T \equiv D_{T_i} \overline{e}$,
so $e' \equiv e_i \sub{x}{e_T}\overline{\sub{y}{e}} $
\item Otherwise, by IH there exists $e'_T$ such that \evals{e_T}{e'_T}, 
so $e' \equiv \ecase{e'_T}{D_{T_i}}{\overline{y}}{e_i}{x}$.
\end{itemize}
\end{itemize}
\end{proof}

We combine the above to prove 
\textit{Soundness of \undeclang}, \ie Theorem~\ref{thm:safety} in the paper:
%
\begin{theorem}{[Soundness of \undeclang]}\label{thm:proofs:safety}
If \hastype{\emptyset}{e}{\tau} and \botomless{e}, then 
\begin{itemize}
\item\textbf{Type-Preservation:} If 
       $\evals{e}{v}$ then $\hastypet{\emptyset}{v}{\typ}$.
\item\textbf{Crash-Freedom:} $\evals{e\not}{\ecrash}$.
\end{itemize}
\end{theorem}
\begin{proof}
1. 
Since \botomless{e} there exists by Lemma \ref{lemma:progress} a bottomless evaluation sequence 
$$
e \equiv e_0 \eval{}{} e_1 \eval{}{} \dots \eval{}{} \dots e_n \equiv v
$$
The Theorem is proven by $n$ applications of Preservation Lemma.

2. If $\evals{e}{\CRASH}$, then by Preservation \hastype{\emptyset}{\CRASH}{\tau}
which cannot happen, as \CRASH by definition is an untyped constant.
\end{proof}

\section{Tracking Substitutions}

Then we define the notion of tracking substitutions.
In Figure~\ref{fig:proofs:tracking} we extend the operational
semantics with a state $\sigma$, \ie a mapping from variables to 
expressions that tracks evaluation of its expressions

%\newcommand\teval[4]{\evalt{#1}{#2}{#3}{#4}}
\renewcommand\evalt[4]{{\teval{#1}{#3}{#2}{#4}}}

\begin{figure*}
\hfill\mbox{\evalt{\sigma}{\sigma}{e}{e}}
$$
\begin{array}{ll}
	%% CONTEXT
	\evalt{\sigma}{\sigma'}{\eapp{e_1}{e_2}}{\eapp{e_1'}{e_2}} 
	& \text{if}\ 
	\evalt{\sigma}{\sigma'}{e_1}{e_1'}\\
	%
	\evalt{\sigma}{\sigma'}{\eapp{c}{e}}{\eapp{c}{e'}} 
	& \text{if}\ 
	\evalt{\sigma}{\sigma'}{e}{e'}\\
	%
	\evalt{\sigma}{\sigma'}{\ecase{e}{D_i}{\overline{y_i}}{e_i}{x}}{\ecase{e'}{D_i}{\overline{y_i}}{e_i}{x}}
	&\text{if}\ \evalt{\sigma}{\sigma'}{e}{e'} \\
	%
	\evalt{\sigma}{\sigma'}
		{D\ \overline{e_i}\ e\ \overline{e_j}}
		{D\ \overline{e_i}\ e'\ \overline{e_j}}
	&\text{if}\ \evalt{\sigma}{\sigma'}{e}{e'} \\
	%
	%% EVALUATION
	\evalt{\sigma}{\sigma}{\eapp{c}{v}}{\ceval{c}{v}} &\\
	\evalt{\sigma}{\sigma}{\eapp{\efun{x}{}{e}}{e_x}}{e\sub{x}{e_x}} &\\
	\evalt{\sigma}{\sigma}{\elet{x}{e_x}{e}}{e\sub{x}{e_x}}& \\
	\evalt{\sigma}{\sigma}{\ecase{D_j\ \overline{e}}{D_i}{\overline{y_i}}{e_i}{x}}{e_j\sub{x}{D_j\ 			\overline{e}}\sub{\overline{y_j}}{\overline{e}}} \\
	%% TRACKING
\teval{(x,e_x)\sigma}{x}{(x,e'_x)\sigma'}{x} &\text{if}\ \teval{\sigma}{e_x}{\sigma'}{e'_x}\\
	\teval{(y,e_y)\sigma}{x}{(y,e_y)\sigma'}{e_x}
  &\text{if}\
	\teval{\sigma}{x}{\sigma'}{e_x}\\
	\teval{(x,v)\sigma}{x}{(x,v)\sigma}{v}
  &\text{if}\
  v \not = D\ \overline{e}\\
	\teval{(x,D\ \overline{y})\sigma}{x}{(x,D\ \overline{y})\sigma}{D\ \overline{y}}
  &\\
	\teval{(x,D\ \overline{e})\sigma}{x}{(x,D\ \overline{y})\overline{(y_i, e_i)}\sigma}{D\ \overline{y}}
  &\text{if}\
	\text{fresh}\ \overline{y_i}
  \\
\end{array}
$$
\caption{Tracking Substitutions}
\label{fig:proofs:tracking}
\end{figure*}

First we prove that evaluation to a constant exists \textit{iff} 
tracking evaluation to the same constant exists.
\begin{lemma}\label{lemma:teval}
$\forall\theta, e, c, \exists\theta'. \evals{\thetasub{\theta}{e}}{c} \Leftrightarrow 
	\tevals{\theta}{e}{\theta'}{c}$.
\end{lemma}
\showproofsketch{
\begin{proofsketch}
\begin{itemize} We prove each direction:
\item $\Rightarrow$.
Given the derivation $\evals{\thetasub{\theta}{e}}{v}$, we can track the appearances 
of each expressions $\theta(x_i)$ and its derivatives and replace them with $x_i$.
Thus, given the initial derivation we can transverse it
(left-to-right and post-order); 
for every tracked appearance we use the appropriate rules 
that update the stack every time a tracked expressions evaluates, ie., 
appears in the left hand side of a rule; 
and remove the multiple evaluations of expressions in the stack
and construct the evaluation 
\tevals{\theta}{e}{\theta'}{c}.

Note that if $\theta(x_i)$ goes to a value, then 
$\theta(x_i) \equiv e_0 \hookrightarrow \dots e_i \dots e_n \equiv v$.
By the way we transverse the tree, 
after the stack is updated to $e_k$ and before it is updated to $e_{k+1}$
all tracked computations for $x_i$ are $e_j, j \leq k$.

If $\theta(x_i)$ does not go to a value, it cannot appear in the left hand side of 
a rule, because evaluation would diverge, thus the stack is not updated for $x_i$.

When a tracked expression reaches a value, we use the appropriate value to 
substitute (and untrack) the value.
Since the result of the initial evaluation is a constant, 
then the result of the tracked computation is the same constant. 

\item $\Leftarrow$.
Given $\tevals{\theta}{e}{\theta'}{c}$
we can construct the derivation \evals{\thetasub{\theta}{e}}{c} replacing each query to the 
stack with the initial computation of the expression.
\end{itemize}
\end{proofsketch}
}

Then we define a \textit{bottomize} function \mkbot{\bullet}
that replaces non-evaluated expressions with \ebot:
\begin{definition}{[Bottomize]}
$$
\mkbot{\theta}(x) = 
\left\{
	\begin{array}{ll}
		D\ \overline{\mkbot{\theta}(y)}  & \mbox{if } \theta(x) = D\ \overline{y}\\
		v  & \mbox{if } \theta(x) = v \not = D\ \overline{y}\\
		\ebot & \mbox{otherwise}
	\end{array}
\right.
$$
\end{definition}

Using the bottomize function we show that evaluation does not depend 
on non-evaluated expressions:
\begin{lemma}\label{lemma:mkbot}
If \tevals{\theta}{e}{\theta'}{c}, 
then \evals{\mkbot{\theta'}\ e}{c}.
\end{lemma}
\showproofsketch{
\begin{proofsketch}
Since \tevals{\theta}{e}{\theta'}{c}$(1)$, 
then \tevals{\theta'}{e}{\theta'}{c}$(2)$:
From the evaluation tree $(1)$ we can construct the evaluation tree $(2)$.
The trees differ on store related rules.

Say that in $(1)$ the store in $x$ is updated, for an arbitrary $x$: 
\teval{(x,e_x)\theta_x}{x}{(x,e'_x)\theta_x}{x}
Since $(1)$ is finite, it should be that
$\tevals{\theta_x}{e_x}{\theta_x}{v} (3)$.
Call $v_x = D\ \overline{y}$ if $v = D\ \overline{e}$, $v$ otherwise.
Then in $(1)$ there should be a ``subtree'' with $(3)$
after which the value of $x$ cannot change in the store.
Or $\theta'(x) = v_x$.
We construct $(2)$ by removing the ``subtree'' with $(3)$.
After that all rules that relate store with $x$ will be the same
on $(1)$ and $(2)$.

If $x$ is not updated in $(1)$ then 
$x$ does not appear in the left hand side of a rule; 
thus $\theta'(x) = \theta(x)$.

We construct $\theta''(x)= \left\{
	\begin{array}{ll}
		v  & \mbox{if}\ \theta'(x)= v\\
		\ebot & \mbox{otherwise}
	\end{array}
\right.$

Then \tevals{\theta''}{e}{\theta''}{c}.
If $\theta'(x)$ is not a value, then it does not appear in the left hand side 
of any rule in $(2)$, thus evaluation of $e$ cannot depend on $x$.

Then by Lemma \ref{lemma:teval}, \evals{\thetasub{\theta''}{e}}{c}.
But $\mkbot{\theta'} e = \thetasub{\theta''}{e}$, so \evals{\thetasub{\mkbot{\theta'}}{e}}{c}.
\end{proofsketch}
}

Also, replacing \ebot with any expression yields the same evaluation:
\begin{lemma}\label{lemma:rmbot}
If \evals{\thetasub{\mkbot{\theta}}{e}}{c}, 
then \evals{\thetasub{\theta}{e}}{c}.
\end{lemma}
\showproofsketch{
\begin{proof}
Since \evals{\thetasub{\mkbot{\theta}}{e}}{c}$(1)$, then
\tevals{\mkbot{\theta}}{e}{\theta'}{c}$(2)$.
\ebot expressions in \mkbot{\theta} are not evaluated, 
otherwise $(2)$ would get stuck.
Thus they can be instantiated with any expression.
$\theta$ provides such an instantiation, thus
\tevals{\theta}{e}{\theta''}{c}$(3)$.
By Lemma \ref{lemma:teval}, \evals{\thetasub{\theta}{e}}{c}.
\end{proof}
}

Finally, we define lifting substitutions
%
\begin{definition}{[Lifting Substitutions]}
$
\trackevals{\sto}{\botsto} \doteq 
\exists e, e', \theta' \tevals{\theta}{e}{\theta'}{e'} \land \botsto = \mkbot{\theta'}
$
\end{definition}

and prove the Lifting Lemma
\begin{lemma}{[Lifting]}\label{lemma:proofs:lifting}
$\evals{\thetasub{\sto}{e}}{c}$ iff $\exists \trackevals{\sto}{\botsto}$ s.t. 
$\evals{\thetasub{\botsto}{e}}{c}$.
\end{lemma}
\showproofsketch{
\begin{proofsketch}
The $\Rightarrow$ direction follows immediately from Lemmata~\ref{lemma:teval} and~\ref{lemma:mkbot}.
The $\Leftarrow$ direction follows immediately from Lemmata~\ref{lemma:teval} and~\ref{lemma:rmbot}.
\end{proofsketch}
}



\section{Constants}

We can prove that all the above constants belong to the 
interpretations of their types.
%
\begin{theorem}{[Constants]}\label{thm:constant}
$c \in \interp{\constty{c}}$.
\end{theorem} 
%
The Theorem trivially holds for more of the constants.
For example, 
\newcommand\eqtype{\ensuremath{
	\tfun{x}{b^\lfinite}{
	\tfun{y}{b^\lfinite}{
	\tlref{v}{\tbool}{\lfinite}{v \Leftrightarrow x = y}
	}}
}}
$$= \in \interp{\eqtype}$$
as $\forall e_1, e_2, \evals{e_1}{d_1}, \evals{e_2}{d_2}\Rightarrow 
			\evals{(e_1=e_2 \Leftrightarrow e_1= e_2)}{\etrue}
			\land \exists d. \evals{(e_1 = e_2)}{d}$

Here we prove that for any type $\tau$, 
\efix{\tau} and \etfix{\tau} satisfy Theorem~\ref{thm:constant}.

Given the families of constants:
\begin{align*}
\ceval{\etfix{\tau}}{f} & \doteq \efun{n}{}{\efun{f}{}{\etfixn{\tau}{}{n}}}\\ 
%\ceval{\etfixf{\tau}{f}{??}}{n} &\doteq
%f\ n\ (\etfixn{\tau}{f}{n}\ f) \\
\ceval{\etfixn{\tau}{}{n}}{m} &\doteq
\efun{f}{}{
f\ m\ (\etfixn{\tau}{f}{m}\ f)} \\
\end{align*}
%
and their types
%
\begin{align*}
%\decr{\tau}{n} & \doteq \decrtypefull{\tau}{n}\\
\constty{\etfix{\tau}} &\doteq 
	(\decrty{\tau})
	 \rightarrow
	\tfun{m}{\tnat^\lfinite}{\tau\sub{x}{m}}\\ 
%\constty{\etfixf{\tau}{f}{n}} &\doteq 
%	 \etfixfty{\tau}{f}{n}\\ 
\constty{\etfixn{\tau}{f}{n}} &\doteq  
	(\decrty{\tau})
	\rightarrow\adecrty{\tau}{n}\\ 
%\constty{\etfixfn{\tau}{f}{n}} &\doteq 
%	 \etfixfty{\tau}{f}{n}\\ 
\end{align*}
we prove that the constants belong to the 
meanings of their types:
%
\begin{theorem}{[Terminating Fixpoint]}\label{thm:fixpoint}
\begin{enumerate}
\item\label{nfix}$\forall n. \etfixn{\tau}{f}{n} \in \constty{\etfixn{\tau}{f}{n}}$
\item\label{tfix}$\etfix{\tau} \in \constty{\etfix{\tau}}$
\item\label{fix}$\efix{\tau} \in \constty{\efix{\tau}}$, if the result of $\tau$ is a \Div type.
\end{enumerate}
\end{theorem}
\begin{proof}
\begin{itemize}
\item \ref{nfix}.
We prove that for all
%$f \in \interp{\decrty{\tau}}$
appropriate $f$
and $ m \in \interp{\tref{v}{\tnat}{\lfinite}{v < n}}$,
$e \equiv \etfixn{\tau}{f}{n}\ f \ m \in \interp{\tau\sub{x}{m}}$
% 
by induction on $n$.

For $n=0$, 
it is trivial, as 
there is no $m$ such that
$m \in \interp{\tref{v}{\tnat}{\lfinite}{v < 0}}$.

For the inductive step, $e$ reduces to 
$$
\etfixn{\tau}{f}{n}\ f\ m 
\hookrightarrow
\etfixfn{\tau}{f}{n}\ m 
\hookrightarrow
f\ m\ (\etfixn{\tau}{f}{m}\ f)\\
$$
By IH, since $m < n$,
$\etfixn{\tau}{f}{m} \in \constty{\etfixn{\tau}{f}{m}}$, 
so $f$ receives the appropriate arguments, 
and returns the appropriate result that proves the theorem.
%
\item \ref{tfix}.
We prove that 
for all appropriate $f$
% \\$f \in \interp{\decrty{\tau}}$
and     $ m \in \interp{\tnat^\lfinite}$,
$\etfix{\tau}\ f \ m \in \interp{\tau\sub{x}{m}}$.
%
Since $m \in \interp{\tref{v}{\tnat}{\lfinite}{v < m+1}}$
$$\etfixn{\tau}{f}{m+1}\ f \ m \in \interp{\tau\sub{x}{m}}$$
%
But operationally, 
$\etfixn{\tau}{f}{m+1}\ f \ m$
and
$\etfix{\tau}\ f \ m$
behave equivalently, which proves the theorem.
\item \ref{fix}. The prove for 
$\efix{\tau} \in \constty{\efix{\tau}}$.
is similar.
%
The only difference is that for the base case
\efixn{\tau}{0} should be defined to belong 
into the interpretation of any type.
%
Thus, it is defined as a diverging expression
and the type of \efix{\tau} is constrainted
to $\tau$'s with potentially diverging result. 
%
With refinement types we prove that the basic
\etfixn{\tau}{f}{0} operator
cannot be called, so we omit 
the definition of this basic case.
\end{itemize}
\end{proof}
%

\section{Algorithmic Verification}\label{sec:algorithmic}

Next, we describe \smtlan, a conservative approximation
of \corelan where the undecidable type subsumption rule
is replaced with a decidable one, yielding an SMT-based
algorithmic type system that enjoys the same soundness
guarantees.

\subsection{The SMT logic \smtlan}

\begin{figure}[t!]
\vspace{-5mm}
\centering
$$
\begin{array}{rrcl}
\emphbf{Predicates} 
  & \pred & ::= &
    \pred \binop \pred \spmid
    \unop \pred \\
  && \spmid & n \spmid b \spmid x \spmid \dc \spmid  x\ \overline{\pred}\\
%%  && \spmid & \forall \overline{\tbind{x}{\sort}}. \pred
  && \spmid & \eif{\pred}{\pred}{\pred}
\\[0.03in]

\emphbf{Integers} 
  & n
  & ::= & 0, -1, 1, \dots
\\[0.03in]

\emphbf{Booleans} 
  & b
  & ::= & \etrue \spmid \efalse
\\[0.03in]

\emphbf{Bin Operators} 
  & \binop
  & ::= & = \spmid < \spmid \land \spmid + \spmid - \spmid \dots
\\[0.03in]

\emphbf{Un Operators} 
  & \unop
  & ::= & \lnot \spmid \dots 
\\[0.03in]

\emphbf{Sort Args} 
  & \sort_a
  & ::= & \tint \spmid \tbool \spmid \tuniv 
         \spmid \tsmtfun{\sort_a}{\sort_a}
\\[0.03in]
\emphbf{Sorts} 
  & \sort
  & ::=  & \sort_a \rightarrow \sort
\end{array}
$$
\caption{\textbf{Syntax of \smtlan}.}
\label{fig:smtsyntax}
\vspace{-2mm}
\end{figure}


\mypara{Syntax: Terms \& Sorts}
%
Figure~\ref{fig:smtsyntax} summarizes the syntax
of \smtlan, the \emph{sorted} (SMT-)
decidable logic of quantifier-free equality,
uninterpreted functions and linear
arithmetic (QF-EUFLIA) ~\citep{Nelson81,SMTLIB2}.
%
The \emph{terms} of \smtlan include
integers $n$,
booleans $b$,
variables $x$,
data constructors $\dc$ (encoded as constants),
fully applied unary \unop and binary \binop operators,
and application $x\ \overline{\pred}$ of an uninterpreted function $x$.
%
The \emph{sorts} of \smtlan include built-in
integer \tint and \tbool for representing
integers and booleans.
%
%% NV reflected functions and measures are first order
%% NV because
%% NV 1. they can be partially applied
%% NV 2. they can be passed as arguments
The interpreted functions of \smtlan, \eg
the logical constants $=$ and $<$,
%% NV and the uninterpreted functions app and lam
%% NV but we have not introduced these yet
have the function sort $\sort \rightarrow \sort$.
%
Other functional values in \corelan, \eg
reflected \corelan functions and
$\lambda$-expressions, are represented as
first-order values with
uninterpreted sort \tsmtfun{\sort}{\sort}.
%
%%The uninterpreted functions of \smtlan, which
%%correspond to reflected \corelan functions,
%%have the function sort $\sort \rightarrow \sort$.
%%%
%%Other functional values in \corelan, \eg
%%$\lambda$-expressions, are represented as
%%first-order values in \smtlan with
%%uninterpreted sort \tsmtfun{\sort}{\sort}.
%%%
The universal sort \tuniv represents all other values.

\mypara{Semantics: Satisfaction \& Validity}
%
An assignment $\sigma$ is a mapping from
variables to terms
%
${\sigma \defeq \{ \assignto{x_1}{\pred_1}, \ldots, \assignto{x_n}{\pred_n} \}}$.
%
We write
%
${\sigma \models \pred}$
%
if the assignment $\sigma$ is a
\emph{model of} $\pred$, intuitively
if $\sigma\ \pred$ ``is true''~\cite{Nelson81}.
%
A predicate $\pred$ \emph{is satisfiable} if
there exists ${\sigma\models\pred}$.
%
A predicate $\pred$ \emph{is valid} if
for all assignments ${\sigma\models\pred}$.


\subsection{Transforming \corelan into \smtlan}
%
\label{subsec:embedding}

\newcommand\emptyaxioms{\ensuremath{\emptyset}\xspace}
\newcommand\andaxioms[2]{\ensuremath{{#1}\cup {#2}}\xspace}

\begin{figure}
\emphbf{Transformation}\hfill{\fbox{\tologicshort{\Gamma}{e}{\typ}{\pred}{\sort}{\smtenv}{\axioms}}}
$$
\inference{
}{
	\tologicshort{\env}{b}{\tbool}{b}{\tbool}{\emptyset}{\emptyaxioms}
}[\lgbool]
\qquad
\inference{
}{
	\tologicshort{\env}{n}{\tint}{n}{\tint}{\emptyset}{\emptyaxioms}
}[\lgint]
$$

$$
\inference{
    \tologicshort{\env}{e_1}{\typ}{\pred_1}{\embed{\typ}}{\smtenv}{\axioms_1} &
    \tologicshort{\env}{e_2}{\typ}{\pred_2}{\embed{\typ}}{\smtenv}{\axioms_2}
}{
	\tologicshort{\env}{e_1\binop e_2}{\tbool}{\pred_1 \binop\pred_2}{\tbool}{\smtenv}{\andaxioms{\axioms_1}{\axioms_2}}
}[\lgbinGEN]
$$

$$
\inference{
	\tologicshort{\env}{e}{\tbool}{\pred}{\tbool}{\smtenv}{\axioms}
}{
	\tologicshort{\env}{\unop e}{\tbool}{\unop\pred}{\tbool}{\smtenv}{\axioms}
}[\lgun]
\qquad
\inference{
}{
	\tologicshort{\env}{x}{\env(x)}{x}{\embed{\env(x)}}{\emptyset}{\emptyaxioms}
}[\lgvar]
$$

$$
\inference{
}{
	\tologicshort{\env}{c}{\constty{\odot}}{\smtvar{c}}{\embed{\constty{\odot}}}{\emptyset}{\emptyaxioms}
}[\lgpop]
\qquad
\inference{
}{
	\tologicshort{\env}{\dc}{\constty{\dc}}{\smtvar{\dc}}{\embed{\constty{\dc}}}{\emptyset}{\emptyaxioms}
}[\lgdc]
$$


%%$$
%%\inference{
%%  	\axioms_{f_1} = \forall \tbind{x}{\sort_x}.\smtappname{\sort_x}{\sort}\ f\ x = \pred \\
%%  	\axioms_{f_2} = \forall \tbind{g}{\sort'},\tbind{x}{\sort_x}.
%%  	\smtappname{\sort_x}{\sort}\ f\ x = \smtappname{\sort_x}{\sort}\ g\ x \Rightarrow f = g \\
%% 	f\ \text{fresh} &
%% 	\sort' = \embed{\tfun{x}{\typ_x}{\typ}} &
%% 	\sort  = \embed{\typ} &
%% 	\sort_x = \embed{\typ_x} \\
%% 	\tologicshort{\env,\tbind{x}{\typ_x}}{e}{\typ}{\pred}{\sort}{\smtenv, \tbind{x}{\sort_x}}{\axioms} &
%% 	\hastype{\env}{(\efun{x}{}{e})}{(\tfun{x}{\typ_x}{\typ})}\\
%%}{
%%	\tologicshort{\env}{\efun{x}{}{e}}{(\tfun{x}{\typ_x}{\typ})}
%%	        {f}{\sort'}{\smtenv, \tbind{f}{\sort'}}{\andaxioms{\{\axioms_{f_1}, \axioms_{f_2}\}}{\axioms}}
%%}[\lgfun]
%%$$

$$
\inference{
    \tologicshort{\env, \tbind{x}{\typ_x}}{e}{}{\pred}{}{}{} &
  	\hastype{\env}{(\efun{x}{}{e})}{(\tfun{x}{\typ_x}{\typ})}\\
}{
	\tologicshort{\env}{\efun{x}{}{e}}{(\tfun{x}{\typ_x}{\typ})}
	        {\smtlamname{\embed{\typ_x}}{\embed{\typ}}\ {x}\ {\pred}}
	        {\sort'}{\smtenv, \tbind{f}{\sort'}}{\andaxioms{\{\axioms_{f_1}, \axioms_{f_2}\}}{\axioms}}
}[\lgfun]
$$

$$
\inference{
	\tologicshort{\env}{e'}{\typ_x}{\pred'}{\embed{\typ_x}}{\smtenv}{\axioms'}
	&
	\tologicshort{\env}{e}{\tfun{x}{\typ_x}{\typ}}{\pred}{\tsmtfun{\embed{\typ_x}}{\embed{\typ}}}{\smtenv}{\axioms}
	& 
	\hastype{\env}{e}{{\typ_x}\rightarrow{\typ}}
}{
	\tologicshort{\env}{e\ e'}{\typ}{\smtappname{\embed{\typ_x}}{\embed{\typ}}\ {\pred}\ {\pred'}}{\embed{\typ}}{\smtenv}{\andaxioms{\axioms}{\axioms'}}
}[\lgapp]
$$


$$
\inference{
	\tologicshort{\env}{e}{\tbool}{\pred}{\tbool}{\smtenv}{\axioms} & 
	\tologicshort{\env}{e_i\subst{x}{e}}{\typ}{\pred_i}{\embed{\typ}}{\smtenv}{\axioms_i}
}{
	\tologicshorttwolines{\env}{\ecaseexp{x}{e}{\etrue \rightarrow e_1; \efalse \rightarrow e_2}}{\typ}
	 {\eif{\pred}{\pred_1}{\pred_2}}{\embed{\typ}}{\smtenv}{\andaxioms{\axioms}{\axioms_i}}
}[\lgcaseBool]
$$

$$
\inference{
	\tologicshort{\env}{e}{\typ_e}{\pred}{\embed{\typ_e}}{\smtenv}{\axioms}\\
	\tologicshort{\env}{e_i\subst{\overline{y_i}}{\overline{\selector{\dc_i}{}\ x}}\subst{x}{e}}{\typ}{\pred_i}{\embed{\typ}}{\smtenv}{\axioms_i}
}{
	\tologicshorttwolines{\env}{\ecase{x}{e}{\dc_i}{\overline{y_i}}{e_i}}{\typ}
	 {\eif{\smtappname{}{}\ \checkdc{\dc_1}\ \pred}{\pred_1}{\ldots} \ \mathtt{else}\ \pred_n}{\embed{\typ}}{\smtenv}
	 {\andaxioms{\axioms}{\axioms_i}}
}[\lgcase]
$$
\caption{\textbf{Transforming \corelan terms into \smtlan.}}
\label{fig:defunc}
\end{figure}

%
The judgment
\tologicshort{\env}{e}{\typ}{\pred}{\sort}{\smtenv}{\axioms}
states that a $\corelan$ term $e$ is transformed,
under an environment $\env$, into a
$\smtlan$ term $\pred$.
%
The transformation rules are summarized in Figure~\ref{fig:defunc}.

\mypara{Embedding Types}
%
We embed \corelan types into \smtlan sorts as:
%
$$
\begin{array}{rclcrcl}
\embed{\tint}                       & \defeq &  \tint &  &
\embed{T}                           & \defeq &  \tuniv \\
\embed{\tbool}                      & \defeq &  \tbool & &
\embed{\tfun{x}{\typ_x}{\typ}} & \defeq & \tsmtfun{\embed{\typ_x}}{\embed{\typ}}
\end{array}
$$
%%%The embedding extends to typing environments:
%%%% by embedding the types of the environment
%%%$$
%%%\embedsort{\{\tbind{x_1}{\typ_1}, \dots, \tbind{x_n}{\typ_n}\}}
%%%  \defeq
%%%  \{\tbind{x_1}{\embed{\typ_1}}, \dots, \tbind{x_n}{\embed{\typ_n}}
%%%  \}
%%%$$

\mypara{Embedding Constants}
%
Elements shared on both \corelan and \smtlan
translate to themselves.
%
These elements include
booleans (\lgbool),
integers (\lgint),
variables (\lgvar),
binary (\lgbinGEN)
and unary (\lgun)
operators.
%
SMT solvers do not support currying,
and so in \smtlan, all function symbols
must be fully applied.
%
Thus, we assume that all applications
to primitive constants and data
constructors are \emph{saturated},
%% NV eta converted
\ie fully applied, \eg by converting
source level terms like @(+ 1)@ to
@(\z -> z + 1)@.
%

%%% Thus, to translate \corelan's partially applied operators,
%%% we define an uninterpreted function
%%% $$
%%% \tbind{\smtvar{c}}{\embed{\constty{c}}}
%%% $$
%%% for every functional constant $c$ in \corelan.
%%% %
%%% For example, $+ 1$ will be translated to application of $\smtvar{+}$ to $1$, while
%%% $1+2$ will be translated to the identical $1+2$.

%%\spara{Lambda Lifting}
%%%
%%Since \smtlan does not support $\lambda$-functions.
%%the translation lifts function to axiomatized variables.
%%%
%%Rule~\lgfun
%%translates the term $\efun{x}{\typ}{e}$ to
%%a fresh variable $f$ that satisfies two axioms:
%%(1). $\beta$-reduction,
%%that is $f$ applied to $x$ is equal to $e$, and
%%(2). extentionality,
%%that is for every other function $g$ and argument $x$,
%%if $f$ applied to $x$ is equal to $g$ applied to $x$,
%%then $f = g$.

\mypara{Embedding Functions}
%
As \smtlan is a first-order logic, we
embed $\lambda$-abstraction and application
using the uninterpreted functions
\smtlamname{}{} and \smtappname{}{}.
%
We embed $\lambda$-abstractions
using $\smtlamname{}{}$ as shown in rule~\lgfun.
%
The term $\efun{x}{}{e}$ of type
${\typ_x \rightarrow \typ}$ is transformed
to
${\smtlamname{\sort_x}{\sort}\ x\ \pred}$
of sort
${\tsmtfun{\sort_x}{\sort}}$, where
%
$\sort_x$ and $\sort$ are respectively
$\embed{\typ_x}$ and $\embed{\typ}$,
%
${\smtlamname{\sort_x}{\sort}}$
is a special uninterpreted function
of sort
${\sort_x \rightarrow \sort\rightarrow\tsmtfun{\sort_x}{\sort}}$,
and
$x$ of sort $\sort_x$ and $r$ of sort $\sort$ are
the embedding of the binder and body, respectively.
%
As $\smtlamname{}{}$ is just an SMT-function,
it \emph{does not} create a binding for $x$.
%
Instead, the binder $x$ is renamed to
a \emph{fresh} name pre-declared in
the SMT environment.


\mypara{Embedding Applications}
%
Dually, we embed applications via
defunctionalization~\citep{Reynolds72}
using an uninterpreted \emph{apply}
function
$\smtappname{}{}$ as shown in rule~\lgapp.
%
The term ${e\ e'}$, where $e$ and $e'$ have
types ${\typ_x \rightarrow \typ}$ and $\typ_x$,
is transformed to
${\tbind{\smtappname{\sort_x}{\sort}\ \pred\ \pred'}{\sort}}$
where
%
$\sort$ and $\sort_x$ are respectively $\embed{\typ}$ and $\embed{\typ_x}$,
the
${\smtappname{\sort_x}{\sort}}$
is a special uninterpreted function of sort
${\tsmtfun{\sort_x}{\sort} \rightarrow \sort_x \rightarrow \sort}$,
and
$\pred$ and $\pred'$ are the respective translations of $e$ and $e'$.


\mypara{Embedding Data Types}
%
Rule~\lgdc translates each data constructor to a
predefined \smtlan constant ${\smtvar{\dc}}$ of
sort ${\embed{\constty{\dc}}}$.
%
Let $\dc_i$ be a non-boolean data constructor such that
$$
\constty{\dc_i} \defeq \typ_{i,1} \rightarrow \dots \rightarrow \typ_{i,n} \rightarrow \typ
$$
Then the \emph{check function}
${\checkdc{{\dc_i}}}$ has the sort
$\tsmtfun{\embed{\typ}}{\tbool}$,
and the \emph{select function}
${\selector{\dc}{i,j}}$ has the sort
$\tsmtfun{\embed{\typ}}{\embed{\typ_{i,j}}}$.
%
Rule~\lgcase translates case-expressions
of \corelan into nested $\mathtt{if}$
terms in \smtlan, by using the check
functions in the guards, and the
select functions for the binders
of each case.
%
%\mypara{Reflecting DataTypes}
%
% The above approach  makes it straightforward
% to reflect functions over datatypes into \smtlan.
%
For example, following the above, the body of the list append function
%
%%% reflect (++) :: xs:[Int] -> ys:[Int] -> [Int]
\begin{code}
  []     ++ ys = ys
  (x:xs) ++ ys = x : (xs ++ ys)
\end{code}
%
is reflected into the \smtlan refinement:
%
$$
\ite{\mathtt{isNil}\ \mathit{xs}}
    {\mathit{ys}}
    {\mathtt{sel1}\ \mathit{xs}\
       \dcons\
       (\mathtt{sel2}\ \mathit{xs} \ \mathtt{++}\  \mathit{ys})}
$$
%
We favor selectors to the axiomatic translation of
HALO~\citep{halo} and \fstar~\cite{fstar} to avoid
universally quantified formulas and the resulting
instantiation unpredictability.

%% $$
%% \tbind{\checkdc{\dc}}{\embed{\typ \rightarrow \tbool}}
%% \ \text{with}\ \constty{\dc} = \typ_1 \rightarrow \dots \rightarrow \typ_n\rightarrow\typ
%% $$
%% and the field selector is used to substitute the data constructor quantified variables $\overline{y_i}$:
%% eg. if \dc is [] then i == 0
%%     if \dc is (:) :: a -> [a] -> [a] then
%%         \dc_1 = head :: [a] -> a
%%         \dc_2 = tail :: [a] -> [a]
%% $$
%% \tbind
      %% {\embed{\typ \rightarrow \typ_i}}
%% \ \text{with}\ \constty{\dc} = \typ_1 \rightarrow \dots \rightarrow \typ_n\rightarrow\typ, i \leq n
%% $$
%% %
%% For example, the body of the @length@ function from~\S~\ref{sec:examples}
%% translates to the condition $\eif{\isN\ xs}{0}{1+\texttt{length} (\etail\ xs)}$,
%% as $\etail \defeq \selector{\dcons}{2}$.

\subsection{Correctness of Translation}

Informally, the translation relation $\tologicshort{\env}{e}{}{\pred}{}{}{}$
is correct in the sense that if $e$ is a terminating boolean expression
then $e$ reduces to \etrue \textit{iff} $\pred$ is SMT-satisfiable
by a model that respects $\beta$-equivalence.

%%\mypara{Type Preservation}
%%%
%%The \emph{initial environment} \smtenvinit
%%maps the uninterpreted symbols used
%%by the translation, namely
%%%
%%$\smtlamname{}{}$,
%%$\smtappname{}{}$,
%%$\smtvar{\dc}$,
%%$\checkdc{{\dc_i}}$,
%%$\selector{\dc}{{i,j}}$
%%and fresh binder names $x$ used  in $\smtlamname{}{}$
%%to their respective sorts.
%%%
%%The judgment $\smthastype{\smtenv}{\pred}{\sort}$ states
%%that the term $\pred$ has sort $\sort$ in environment
%%$\smtenv$. (We omit the standard derivation rules
%%for brevity.)
%%%
%%The translation is type (sort) preserving.
%%
%%\begin{lemma}
%%%  [Type Transformation]
%%If \tologicshort{\env}{e}{\typ}{p}{\sort}{\smtenv}{\axioms},
%%and \hastype{\env}{e}{\typ}, then
%%\smthastype{\smtenvinit, \embedsort{\env}}{p}{\embed{\typ}}.
%%\end{lemma}
%%
%%% are defined in the
%%% %
%%% Thus, \smtenvinit includes
%%% $$
%%% \begin{array}{rcll}
%%% \smtvar{c}  &\colon &\embed{\constty{c}}
  %%% &\forall c\in \corelan\\
%%% \smtlamname{\sort_x}{\sort}&\colon&\sort_x \rightarrow \sort\rightarrow\tsmtfun{\sort_x}{\sort}
  %%% &\forall \sort_x, \sort\in \smtlan\\
%%% \smtappname{\sort_x}{\sort}&\colon&\tsmtfun{\sort_x}{\sort} \rightarrow \sort_x \rightarrow \sort
  %%% &\forall \sort_x, \sort\in \smtlan\\
%%% \smtvar{\dc}&\colon&\embed{\constty{\dc}}
  %%% &\forall\dc\in\corelan\\
%%% \checkdc{\dc}&\colon&\embed{T \rightarrow \tbool}
  %%% &\forall \dc\in \corelan\ \text{of data type}\ T \\
%%% \selector{\dc}{i}&\colon&\embed{T \rightarrow \typ_i}
  %%% &\forall \dc\in \corelan\ \text{of data type}\ T \\
  %%% &&&\text{and}\ i\text{-th argument}\ \typ_i \\
%%% {x} & \colon&{\sort}&\text{for each lambda argument} \\
%%% \end{array}
%%% $$


% \mypara{Lifted Substitutions}


%
%% as defined
%% in~\citep{Vazou15} remove bottoms from expressions
%% in substitutions and translate via
%% \tologic{\emptyset}{\star}{}{\star}{}{}{}
%% to a set of models $\sigma \in \theta^\perp$
%% where each bottom maps to
%% %
%%
%% Such models $\sigma \in \theta^\perp$
%% map variables in $\theta$ to values
%% in the logic, without providing
%% interpretations for the
%% $\smtlamname{}{}$ and $\smtappname{}{}$.

\NV{below we use substitution in lambda s which is not formally defined}
%
\begin{definition}[$\beta$-Model]\label{def:beta-model}
A $\beta-$model $\bmodel$ is an extension of a model $\sigma$
where $\smtlamname{}{}$ and $\smtappname{}{}$
satisfy the axioms of $\beta$-equivalence:
$$
\begin{array}{rcl}
\forall x\ y\ e. \smtlamname{}{}\ x\ e
  & = & \smtlamname{}{}\ y\ (e\subst{x}{y}) \\
\forall x\ e_x\ e. (\smtappname{}{}\ (\smtlamname{}{}\ x\ e)\ e_x
  & = &  e\subst{x}{e_x}
\end{array}
$$
\end{definition}

\mypara{Semantics Preservation}
%
We define the translation of a \corelan term
into \smtlan under the empty environment as
${\embed{e} \defeq \pred}$
if ${\tologicshort{\emptyset}{\refa}{}{\pred}{}{}{}}$.
%
A \emph{lifted substitution}
$\theta^\perp$ is a set of models $\sigma$
where each ``bottom'' in the substitution
$\theta$ is mapped to an arbitrary logical
value of the respective sort~\citep{Vazou14}.
%
We connect the semantics of \corelan and translated
\smtlan via the following theorems:
% terms can connect evaluation of boolean
% \corelan expression to \smtlan predicates.

\begin{theorem}\label{thm:embedding-general}
If ${\tologicshort{\env}{\refa}{}{\pred}{}{}{}}$,
then for every ${\sub\in\interp{\env}}$
and every ${\sigma\in {\sub^\perp}}$,
if $\evalsto{\applysub{\sub^\perp}{\refa}}{v}$
then $\sigma^\beta \models \pred = \embed{v}$.
\end{theorem}

% For Boolean expressions we specialize the above to

\begin{corollary}\label{thm:embedding}
If ${\hastype{\env}{\refa}{\tbool}}$, $e$ reduces to a value and
${\tologicshort{\env}{\refa}{\tbool}{\pred}{\tbool}{\smtenv}{\axioms}}$,
then for every ${\sub\in\interp{\env}}$
and every ${\sigma\in {\sub^\perp}}$,
$\evalsto{\applysub{\sub^\perp}{\refa}}{\etrue}$ iff
$\sigma^\beta \models \pred$.
\end{corollary}



\subsection{Decidable Type Checking}
\begin{figure}[t!]
\centering
$$
\begin{array}{rrcl}
\emphbf{Refined Types} \quad
  & \typ
  & ::=   & \tref{v}{\btyp^{[\tlabel]}}{\reft} \spmid \tfun{x}{\typ}{\typ}
\\[0.10in]
\end{array}
$$
\emphbf{Well Formedness}\hfill{\fbox{\aiswellformed{\env}{\typ}}}\\
$$
\inference{
  \ahastype{\env,\tbind{v}{\btyp}}{\refa}{\tbool^{\tlabel}}
}{
  \aiswellformed{\env}{\tref{v}{\btyp}{\refa}}
}[\rwbase]
$$
\emphbf{Subtyping}\hfill{\fbox{\aissubtype{\env}{\typ}{\typ'}}}\\
$$
\inference{
\env' \defeq \env,\tbind{v}{\{\btyp^\tlabel | \refa\}} &
\tologicshort{\env'}{\refa'}{\tbool}{\pred'}{}{}{} &
\smtvalid{\vcond{\env'}{\pred'}}
%
}{
 \aissubtype{\env}{\tref{v}{\btyp}{\refa}}{\tref{v}{\btyp}{\refa'}}
}[\rsubbase]
$$
%%%% %\NV{REVERT TO OLD DEFINITIONS, what is e'?}
%%%% $$
%%%% \inference{
%%%% \tologicshort{\env'}{\refa_1}{\tbool}{\pred_1}{\tbool}{\smtenv_1}{\axioms_1} &
%%%% \tologicshort{\env'}{\refa_2}{\tbool}{\pred_2}{\tbool}{\smtenv_1}{\axioms_1} \\
%%%% % \isvalid{\env,\tbind{v}{\btyp}}{\refa_1}{\refa_2}
%%%% \env' \defeq \env,\tbind{v}{\btyp^\tlabel} &
%%%% % \tologicshort{\env'}{\refa'}{\tbool}{\pred'}{\tbool}{\smtenv'}{\axioms'} &
%%%% \smtvalid{\vcond{\env'}{\pred_1 \Rightarrow \pred_2}}
%%%% %
%%%% }{
  %%%% \aissubtype{\env}{\tref{v}{\btyp}{\refa_1}}{\tref{v}{\btyp}{\refa_2}}
%%%% }[\rsubbase]
%%%% $$
%%% \emphbf{Implication}\hfill{\isvalid{\env}{\refa_1}{\refa_2}}\\
%%% $$
%%% \inference{
  %%% \tologicshort{\env}{\refa_1}{\tbool}{\pred_1}{\tbool}{\smtenv_1}{\axioms_1} &
  %%% \tologicshort{\env}{\refa_2}{\tbool}{\pred_2}{\tbool}{\smtenv_2}{\axioms_i} \\
  %%% \text{is SMT-valid}\ (\embedexpr{\env} \Rightarrow \pred_1 \Rightarrow \pred_2)
%%% }{
  %%% \isvalid{\env}{\refa_1}{\refa_2}
%%% }
%%% $$
%%% \emphbf{Typing}\hfill{\ahastype{\env}{\prog}{\typ}}\\
\caption{\textbf{Algorithmic Typing (other rules in Figs~\ref{fig:syntax} and \ref{fig:typing}.)}}
\label{fig:modifications}
\end{figure}

Figure~\ref{fig:modifications} summarizes the modifications required
to obtain decidable type checking.
%
Namely, basic types are extended with labels that track termination
and subtyping is checked via an SMT solver.

\mypara{Termination}
%
Under arbitrary beta-reduction semantics
(which includes lazy evaluation), soundness
of refinement type checking requires checking
termination, for two reasons:
%
(1)~to ensure that refinements cannot diverge, and
(2)~to account for the environment during subtyping~\citep{Vazou14}.
%
We use \tlabel to mark provably terminating
computations, and extend the rules to use
refinements to ensure that if
${\ahastype{\env}{e}{\tref{v}{\btyp^\tlabel}{r}}}$,
then $e$ terminates~\citep{Vazou14}.
%
%% Here we assume termination is checked by an oracle,
%% but we can use refinement types themselves to prove
%% correctness of the termination labeling


\mypara{Verification Conditions}
The \emph{verification condition} (VC)
${\vcond{\env}{\pred}}$
is \emph{valid} only if the set of values
described by $\env$, is subsumed by
the set of values described by $\pred$.
%
$\env$ is embedded into logic by conjoining
(the embeddings of) the refinements of
provably terminating binders~\cite{Vazou14}:
%
%% We only trust refinements of terminating
%% expressions, as every diverging expression
%% can be unsoundly refined \efalse.
%% $$
%% \embed{\env} \defeq
  %% \bigwedge\{ p \mid \tbind{x}{\tref{v}{\btyp^{\tlabel}}{e}} \in \env
   %% \land \tologicshort{\env}{e\subst{v}{x}}{\btyp}{p}{\embed{\btyp}}{\smtenv}{\axioms}
   %% \}
%% $$
\begin{align*}
\embed{\env} \defeq & \bigwedge_{x \in \env} \embed{\env, x} \\
\intertext{where we embed each binder as}
\embed{\env, x} \defeq & \begin{cases}
                           \pred  & \text{if } \env(x)=\tref{v}{\btyp^{\tlabel}}{e},\
                                    \tologicshort{\env}{e\subst{v}{x}}{\btyp}{\pred}{\embed{\btyp}}{\smtenv}{\axioms} \\
                           \etrue & \text{otherwise}.
                         \end{cases}
\end{align*}

%We use the embedding of environment to decidably check subtyping.
%As defined in Figure~\ref{fig:modifications},
%\tref{v}{\btyp}{\refa_1} is subtype of \tref{v}{\btyp}{\refa_1}
%under the environment \env, when
%$\refa_i$ transforms to $\pred_i$ with axioms $\axioms_i$
%and assuming $\embedexpr{\env}$ and the axioms $\axioms_i$
%$\pred_i$ implies $\pred_2$.

\mypara{Subtyping via SMT Validity}
%
We make subtyping, and hence, typing decidable,
by replacing the denotational base subtyping
rule $\rsubbase$ with a conservative,
algorithmic version that uses an SMT
solver to check the validity of the subtyping VC.
%
We use Corollary~\ref{thm:embedding} to prove
soundness of subtyping. 
%
\begin{lemma}\label{lem:subtyping} %[Conservative Subtyping]
If {\aissubtype{\env}{\tref{v}{\btyp}{e_1}}{\tref{v}{\btyp}{e_2}}}
then {\issubtype{\env}{\tref{v}{\btyp}{e_1}}{\tref{v}{\btyp}{e_2}}}.
\end{lemma}

%
\mypara{Soundness of \smtlan}
%
Lemma~\ref{lem:subtyping} directly implies the soundness of \smtlan.
%
\begin{theorem}[Soundness of \smtlan]\label{thm:soundness-smt}
If \ahastype{\env}{e}{\typ} then \hastype{\env}{e}{\typ}.
\end{theorem}


\begin{comment}
\begin{proof}
By rule \rsubbase, we need to show that
$\forall \sub\in\interp{\env}.
  \interp{\applysub{\sub}{\tref{v}{\btyp}{\refa_1}}}
  \subseteq
  \interp{\applysub{\sub}{\tref{v}{\btyp}{\refa_2}}}$.
%
We fix a $\sub\in\interp{\env}$.
and get that forall bindings
$(\tbind{x_i}{\tref{v}{\btyp^{\downarrow}}{\refa_i}}) \in \env$,
$\evalsto{\applysub{\sub}{e_i\subst{v}{x_i}}}{\etrue}$.

Then need to show that for each $e$,
if $e \in \interp{\applysub{\sub}{\tref{v}{\btyp}{\refa_1}}}$,
then $e \in \interp{\applysub{\sub}{\tref{v}{\btyp}{\refa_2}}}$.

If $e$ diverges then the statement trivially holds.
Assume $\evalsto{e}{w}$.
We need to show that
if $\evalsto{\applysub{\sub}{e_1\subst{v}{w}}}{\etrue}$
then $\evalsto{\applysub{\sub}{e_2\subst{v}{w}}}{\etrue}$.

Let \vsub the lifted substitution that satisfies the above.
Then  by Lemma~\ref{thm:embedding}
for each model $\bmodel \in \interp{\vsub}$,
$\bmodel\models\pred_i$, and $\bmodel\models q_1$
for
$\tologicshort{\env}{e_i\subst{v}{x_i}}{\btyp}{\pred_i}{\embed{\btyp}}{\smtenv_i}{\axioms_i}$
$\tologicshort{\env}{e_i\subst{v}{w}}{\btyp}{q_i}{\embed{\btyp}}{\smtenv_i}{\beta_i}$.
%
Since \aissubtype{\env}{\tref{v}{\btyp}{e_1}}{\tref{v}{\btyp}{e_2}} we get
$$
\bigwedge_i \pred_i
\Rightarrow q_1 \Rightarrow q_2
$$
thus $\bmodel\models q_2$.
%
By Theorem~\ref{thm:embedding} we get $\evalsto{\applysub{\sub}{\refa_2\subst{v}{w}}}{\etrue}$.
\end{proof}
\end{comment}

\section{Implementation: \toolname}
Here we give some more examples on how we can use \toolname.
We start by proving termination on mutual recursive functions, 
using lexicographical ordering. 
%
Then we describe how we proved functional correctness on 
two commonly used functions, namely \texttt{ByteString}
and \texttt{Text}.

\subsection{Proving Termination}

   Next, consider the Ackermann function.
   %
   \begin{code}
     ack m n 
       | m == 0    = n + 1
       | n == 0    = ack (m-1) 1 
       | otherwise = ack (m-1) (ack m (n-1))
   \end{code}
   %
   There exists no integer termination metric that decreases at each recursive call.
   %
   However @ack@ terminates because at each call \emph{either}
   @m@ decreases \emph{or} @m@ remains the same and @n@ decreases. 
   %
   In other words, the pair @(m,n)@ strictly decreases according to
   \emph{lexicographic} ordering. 
   %
   To capture this requirement we extend termination metric
   from an integer to a list of integers
   and at each recursive call we check that this list is
   lexicographically decreasing.
   %
   In the case of
   @ack@ this list will simply be the parameters @m@
   and @n@:
   %
   \begin{code}
     ack :: m:Nat -> n:Nat -> Nat / [m,n]
   \end{code}
   %
   Thus, \toolname uses lexicographic ordering on 
   a list of natural numbers to prove termination.
   %
   Termination metrics could be generalized to 
   any \emph{well-found} metric.
   
   \spara{Mutual Recursion}
   %
   Equipped with termination metrics
   \toolname instantiates a powerful
   termination checker that like~\citep{XiTerminationLICS01}
   proves termination even for mutual recursive functions.
   %
   Consider the mutual recursive functions @isEven@ and @isOdd@
   \begin{code}
   {-@ isEven :: n:Nat -> Bool / [n, 0] @-}
   {-@ isOdd  :: n:Nat -> Bool / [n, 1] @-}
   
   isEven 0 = True
   isEven n = isOdd $ n-1
   
   isOdd n  = not $ isEven n 
   \end{code}
   Each call terminates as either @isEven@
   calls @isOdd@ with a decreasing argument, 
   or the argument remains the same, and @isOdd@
   calls @isEven@ that should then decrease the argument.
   % 
   We capture this reasoning using two lexicographic pairs:
   each function has its own metric, 
   and when @isEven@ calls @isOdd@
   the metric of the caller $(n, 0)$
   should be greater that callee's metric
   $(n-1, 1)$.
   %
   Similarly, at @isEven@'s call-site 
   \toolname verifies that	
   $(n, 1) > (n, 0)$.
   %
   For example, the call @isEven m@
   will fire the decreasing metric sequence
   $(m, 0) > (m-1, 1) > (m-1, 0) > (m-2, 1) > \dots$
   that ultimate terminates for \textit{any}
   natural number $m$.


\subsection{Bytestring}\label{sec:bytestring}
The single most important aspect of the \bytestring 
library, %~\cite{bytestring}, 
our first case study, is its pervasive intermingling of
high level abstractions like higher-order loops,
folds, and fusion, with low-level pointer 
manipulations in order to achieve high-performance. 
%
%% From the package description, \bytestring is, 
%% ``A time and space-efficient implementation of byte vectors using packed
%% Word8 arrays, suitable for high performance use, both in terms of large
%% data quantities, or high speed requirements. Byte vectors are encoded as
%% strict Word8 arrays of bytes, held in a ForeignPtr, and can be passed
%% between C and Haskell with little effort."
%
\bytestring is an appealing target for evaluating
\toolname, as refinement types are an ideal way to 
statically ensure the correctness of the delicate 
pointer manipulations, errors in which lie below 
the scope of dynamic protection.

The library spans $8$ files (modules) totaling about 3,500 lines.
We used \toolname to verify the library by giving precise 
types describing the sizes of internal pointers and bytestrings. 
These types are used in a modular fashion to verify the 
implementation of functional correctness properties of 
higher-level API functions which are built using 
lower-level internal operations. 
Next, we show the key invariants and how
\toolname reasons precisely about pointer
arithmetic and higher-order codes.

\spara{Key Invariants}
A (strict) @ByteString@ is a triple of a @pay@load pointer, 
an @off@set into the memory buffer referred to by the pointer 
(at which the string actually ``begins") and a @len@gth 
corresponding to the number of bytes in the string, which is 
the size of the buffer \emph{after} the @off@set, that
corresponds to the string.
%
We define a measure for the \emph{size} of 
a @ForeignPtr@'s buffer, and use it to define 
the key invariants as a refined datatype 
%
\begin{code}
  measure fplen  :: ForeignPtr a -> Int
  data ByteString = PS 
     { pay :: ForeignPtr Word8
     , off :: {v:Nat | v       <= fplen pay }
     , len :: {v:Nat | off + v <= fplen pay } }
\end{code}
%
The definition states that 
the offset is a @Nat@ no bigger than the size of 
the @payload@'s buffer, and that
the sum of the @off@set and non-negative @len@gth
is no more than the size of the payload buffer.
Finally, we encode a @ByteString@'s size as a measure.
%
\begin{code}
  measure bLen   :: ByteString -> Int
  bLen (PS p o l) = l
\end{code}

\spara{Specifications}
We define a type alias for a @ByteString@ whose length is the same
as that of another, and use the alias to type the API 
function @copy@, which clones @ByteString@s.

\begin{code}
  type ByteStringEq B = {v:ByteString | (bLen v) = (bLen B)}
  
  copy :: b:ByteString -> ByteStringEq b 
  copy (PS fp off len) 
    = unsafeCreate len $ \p -> 
        withForeignPtr fp $ \f ->
          memcpy len p (f `plusPtr` off) 
\end{code}

\spara{Pointer Arithmetic}
The simple body of @copy@ abstracts a fair bit of internal work. 
@memcpy sz dst src@, implemented in \C and accessed via the FFI is a potentially
dangerous, low-level operation, that copies @sz@ bytes starting
\emph{from} an address @src@ \emph{into} an address @dst@. 
Crucially, for safety, the regions referred to be @src@ and @dst@ 
must be larger than @sz@. We capture this requirement by defining
a type alias @PtrN a N@ denoting GHC pointers that refer to a region
bigger than @N@ bytes, and then specifying that the destination
and source buffers for @memcpy@ are large enough. 

\begin{code}
  type PtrN a N = {v:Ptr a | N <= (plen v)}
  memcpy :: sz:CSize -> dst:PtrN a siz 
                     -> src:PtrN a siz 
                     -> IO () 
\end{code}


The actual output for @copy@ is created and filled in using the 
internal function @unsafeCreate@ which is a wrapper around. 
% -- | Create ByteString of size @l@ and use
% --   action @f@ to fill it's contents.
\begin{code}
  create :: l:Nat -> f:(PtrN Word8 l -> IO ())
         -> IO (ByteStringN l)
  create l f = do
      fp <- mallocByteString l
      withForeignPtr fp $ \p -> f p
      return $! PS fp 0 l
\end{code}

% We include the comment to illustrate how the 
% refinement type captures the natural language 
% requirement in a machine checkable manner.
%
The type of @f@ specifies that the action
will only be invoked on a pointer of length at least 
@l@, which is verified by propagating the types of
@mallocByteString@ and @withForeignPtr@. 
%
The fact that the action is only invoked on such pointers 
is used to ensure that the value @p@ in the body of @copy@ 
is of size @l@. This, and the @ByteString@ 
invariant that the size of the payload @fp@ 
exceeds the sum of @off@ and @len@, ensures 
safety of the @memcpy@ call.

\spara{Interfacing with the Real World}
The above illustrates how \toolname analyzes code that interfaces 
with the ``real world" via the \C FFI. We specify the behavior 
of the world via a refinement typed interface. These types are then assumed
to hold for the corresponding functions, \ie generate pre-condition checks
and post-condition guarantees at usage sites within the Haskell code.


\spara{Higher Order Loops} 
@mapAccumR@ combines a @map@ and a @foldr@ over a @ByteString@. 
The function uses non-trivial recursion, and demonstrates 
the utility of abstract-interpretation based inference. 
%
\begin{code}
  mapAccumR f z b = unSP $ loopDown (mapAccumEFL f) z b
\end{code}
%$
To enable fusion \cite{streamfusion} 
@loopDown@ uses a higher order @loopWrapper@ 
to iterate over the buffer with a @doDownLoop@ action:
%
%% DONE \ES{should we use a termination expression for ``loop'' even though it won't actually work atm in LH?}
\begin{code}
  doDownLoop f acc0 src dest len = loop (len-1) (len-1) acc0
    where
     loop :: s:_ -> _ -> _ -> _ / [s+1]
     loop s d acc 
       | s < 0 
       = return (acc :*: d+1 :*: len - (d+1))
       | otherwise       
       = do x <- peekByteOff src s
            case f acc x of
              (acc' :*: NothingS) -> 
                   loop (s-1) d acc'
              (acc' :*: JustS x') -> 
                   pokeByteOff dest d x'
                >> loop (s-1) (d-1) acc'
\end{code}

The above function iterates across the @src@ and @dst@ 
pointers from the right (by repeatedly decrementing the 
offsets @s@ and @d@ starting at the high @len@ down to @-1@). 
Low-level reads and writes are carried out using the 
potentially dangerous @peekByteOff@ and @pokeByteOff@ 
respectively. To ensure safety, we type these low level 
operations with refinements stating that they are only 
invoked with valid offsets @VO@ into the input buffer @p@.

\begin{code}
  type VO P    = {v:Nat | v < plen P}
  peekByteOff :: p:Ptr b -> VO p -> IO a
  pokeByteOff :: p:Ptr b -> VO p -> a -> IO ()
\end{code}

The function @doDownLoop@ is an internal function.
Via abstract interpretation~\cite{LiquidPLDI08}, 
\toolname infers that
%
(1)~@len@ is less than the sizes of @src@ and @dest@,
(2)~@f@ (here, @mapAccumEFL@) always returns a @JustS@, so
(3)~source and destination offsets satisfy $\mathtt{0 \leq s, d < {len}}$,
(4)~the generated @IO@ action returns a triple @(acc :*: 0 :*: len)@,
%
thereby proving the safety of the accesses in @loop@ \emph{and}
verifying that @loopDown@ and the API function @mapAccumR@ 
return a \bytestring whose size equals its input's.
 
To prove \emph{termination}, we add a \emph{termination expression} 
@s+1@ which is always non-negative and decreases at each call.

\spara{Nested Data}
@group@ splits a string like @"aart"@ into the list
@["aa","r","t"]@, \ie a list of
(a)~non-empty @ByteString@s whose 
(b)~total length equals that of the input. 
To specify these requirements, we define a measure for 
the total length of strings in a list and use it to
define the list of \emph{non-empty} strings
whose total length equals that of another string:

\begin{code}
  measure bLens :: [ByteString] -> Int 
  bLens ([])     = 0
  bLens (x:xs)   = bLen x + bLens xs
  
  type ByteStringNE    = {v:ByteString | bLen v > 0}
  type ByteStringsEq B = {v:[ByteStringNE] | bLens v = bLen b}
\end{code}
%
\toolname uses the above to verify that
%
\begin{code}
  group :: b:ByteString -> ByteStringsEq b
  group xs
   | null xs   = []
   | otherwise = let x        = unsafeHead xs
                     xs'      = unsafeTail xs
                     (ys, zs) = spanByte x xs' 
                 in (y `cons` ys) : group zs
\end{code}
%
The example illustrates why refinements are critical for
proving termination. \toolname determines that @unsafeTail@ 
returns a \emph{smaller} @ByteString@ than its input and that
each element returned by @spanByte@ is no bigger than the 
input, concluding that @zs@ is smaller than @xs@, hence
checking the body under the termination-weakened environment.

To justify the output type, let's look at @spanByte@,
which splits strings into a pair:
%
\begin{code}
  spanByte c ps@(PS x s l) 
    = inlinePerformIO $ withForeignPtr x $
          \p -> go (p `plusPtr` s) 0
    where
      go :: _ -> i:_ -> _ / [l-i]
      go p i 
        | i >= l    = return (ps, empty)
        | otherwise = do
            c' <- peekByteOff p i
            if c /= c'
              then let b1 = unsafeTake i ps
                       b2 = unsafeDrop i ps
                   in  return (b1, b2)
              else go p (i+1)
\end{code}
%
Via inference, \toolname verifies the safety of 
the pointer accesses, and determines that the 
sum of the lengths of the output pair of 
@ByteString@s equals that of the input @ps@.
@go@ terminates as @l-i@ is a well-founded 
decreasing metric.

%%% Local Variables: 
%%% mode: latex
%%% TeX-master: "main"
%%% End: 


\subsection{Text}\label{sec:text}
Next %, to give a qualitative sense of the kinds of properties analyzed 
% during the course of our evaluation, 
we present a brief overview of the verification of \libtext, which 
is the standard library used for serious unicode text processing. 
\libtext uses byte arrays and stream fusion to guarantee 
performance while providing a high-level API.
In our evaluation of \toolname on \libtext,%~\cite{text},
we focused on two types of properties: 
(1) the safety of array index and write operations, and 
(2) the functional correctness of the top-level API.
%
These are both made more interesting by the fact that 
\libtext internally encodes characters using UTF-16, 
in which characters are stored in either two or four bytes.
%
\libtext is a vast library spanning 39 modules and 5,700 lines of
code, however we focus on the 17 modules that are relevant
to the above properties.
%
While we have verified exact functional correctness size properties
for the top-level API, we focus here on the low-level functions 
and interaction with unicode.

\spara{Arrays and Texts}
A @Text@ consists of an (immutable) @Array@ of 16-bit words,
an offset into the @Array@, and a length describing the
number of @Word16@s in the @Text@.  
The @Array@ is created and filled using a
\emph{mutable} @MArray@. 
All write operations in \libtext are performed on @MArray@s 
in the @ST@ monad, but they are \emph{frozen} into @Array@s
before being used by the @Text@ constructor.
%
We write a measure for the size of an @MArray@ and use
it to type the write and freeze operations.
%
\begin{code}
  measure malen       :: MArray s -> Int
  predicate EqLen A MA = alen A = malen MA
  predicate Ok I A     = 0 <= I < malen A
  type VO A            = {v:Int| Ok v A} 
  
  unsafeWrite  :: m:MArray s
               -> VO m -> Word16 -> ST s ()
  unsafeFreeze :: m:MArray s
               -> ST s {v:Array | EqLen v m}
\end{code}

\spara{Reasoning about Unicode}
The function @writeChar@ (abbreviating the function \texttt{unsafeWrite} from \texttt{UnsafeChar})
writes a @Char@ into an @MArray@.
\libtext uses UTF-16 to represent characters internally,
meaning that every @Char@ will be encoded using two or 
four bytes (one or two @Word16@s).
%
\begin{code}
  writeChar marr i c
      | n < 0x10000 = do
          unsafeWrite marr i (fromIntegral n)
          return 1
      | otherwise = do
          unsafeWrite marr i lo
          unsafeWrite marr (i+1) hi
          return 2
      where n = ord c
            m = n - 0x10000
            lo = fromIntegral
               $ (m `shiftR` 10) + 0xD800
            hi = fromIntegral
               $ (m .&. 0x3FF) + 0xDC00
\end{code}
%
The UTF-16 encoding complicates the specification of the function
as we cannot simply require @i@ to be less than the length of 
@marr@; if @i@ were @malen marr - 1@ and @c@ required two 
@Word16@s, we would perform an out-of-bounds write. 
%
We account for this subtlety with a predicate that states 
there is enough @Room@ to encode @c@.
%
% measure ord         :: Char -> Int
\begin{code}
  predicate OkN I A N  = Ok (I+N-1) A
  predicate Room I A C = if ord C < 0x10000
                         then OkN I A 1
                         else OkN I A 2
  
  type OkSiz I A = {v:Nat  | OkN  I A v}
  type OkChr I A = {v:Char | Room I A v}
\end{code}
%
@Room i marr c@ says 
``if @c@ is encoded using one @Word16@, 
  then @i@ must be less than @malen marr@,
  otherwise @i@ must be less than @malen marr - 1@.''
%
@OkSiz I A@ is an alias for a valid number of @Word16@s 
remaining after the index @I@ of array @A@. 
@OkChr@ specifies the @Char@s for which there is room (to write)
at index @I@ in array @A@.
%
The specification for @writeChar@ states that given an array \hbox{@marr@,}
an index @i@, and a valid @Char@ for which there is room at index \hbox{@i@,}
the output is a monadic action returning the number of @Word16@ occupied
by the @char@.
%
\begin{code}
  writeChar :: marr:MArray s
            -> i:Nat
            -> OkChr i marr
            -> ST s (OkSiz i marr)
\end{code}
%
\spara{Bug}
Thus, clients of @writeChar@ should only call it with suitable indices
and characters.
%
Using \toolname we found an error in one client, @mapAccumL@, 
which combines a map and a fold over a @Stream@, and stores 
the result of the map in a @Text@. Consider the inner loop of @mapAccumL@.
%
% \begin{code}
% mapAccumL f z0 (Stream next0 s0 len) =
%   (nz, Text na 0 nl)
%  where
%   mlen = upperBound 4 len
%   (na,(nz,nl)) = runST $ do
%        (marr,x) <- (new mlen >>= \arr ->
%                     outer arr mlen z0 s0 0)
%        arr      <- unsafeFreeze marr
%        return (arr,x)
%   outer arr top = loop
%    where
%     loop !z !s !i =
%       case next0 s of
%         Done          -> return (arr, (z,i))
%         Skip s'       -> loop z s' i
%         Yield x s'
%           | j >= top  -> do
%             let top' = (top + 1) `shiftL` 1
%             arr' <- new top'
%             copyM arr' 0 arr 0 top
%             outer arr' top' z s i
%           | otherwise -> do
%             let (z',c) = f z x
%             d <- writeChar arr i c
%             loop z' s' (i+d)
%           where j | ord x < 0x10000 = i
%                   | otherwise       = i + 1
% \end{code}
\begin{code}
  outer arr top = loop
   where
    loop !z !s !i =
      case next0 s of
        Done          -> return (arr, (z,i))
        Skip s'       -> loop z s' i
        Yield x s'
          | j >= top  -> do
            let top' = (top + 1) `shiftL` 1
            arr' <- new top'
            copyM arr' 0 arr 0 top
            outer arr' top' z s i
          | otherwise -> do
            let (z',c) = f z x
            d <- writeChar arr i c
            loop z' s' (i+d)
          where j | ord x < 0x10000 = i
                  | otherwise       = i + 1
\end{code}
%
Let's focus on the @Yield x s'@ case.
%
We first compute the maximum index @j@ to 
which we will write and determine the safety of a write. 
%
If it is safe to write to @j@ we call the provided 
function @f@ on the accumulator @z@ and the character 
@x@, and write the \emph{resulting} character @c@ into the array. 
%
However, we know nothing about @c@, in particular, 
whether @c@ will be stored as one or two @Word16@s! 
Thus, \toolname flags the call to @writeChar@ as \emph{unsafe}.
The error can be fixed by lifting @f z x@ into the @where@ clause and defining the
write index @j@ by comparing @ord c@ (not @ord x@). \toolname (and the authors)
readily accepted our fix.

%% INCLUDEPROOF To illustrate why the call is in fact buggy, 
%% INCLUDEPROOF consider a sample iteration of @loop@ 
%% INCLUDEPROOF where @i = malen arr - 1@ and
%% INCLUDEPROOF @ord x < 0x10000@. 
%% INCLUDEPROOF %
%% INCLUDEPROOF In this case @j@ will equal @i@ and we will enter
%% INCLUDEPROOF the @otherwise@ branch. 
%% INCLUDEPROOF %
%% INCLUDEPROOF Next, suppose @f z x@ returns a
%% INCLUDEPROOF @c@ such that  @ord c >= 0x10000@. 
%% INCLUDEPROOF %
%% INCLUDEPROOF The action @writeChar arr i c@ will write to
%% INCLUDEPROOF indices @i@ \emph{and} @i+1@ of @arr@, but 
%% INCLUDEPROOF @i+1 = malen arr@ and is not a valid index 
%% INCLUDEPROOF for writing! 
%% INCLUDEPROOF %
%% INCLUDEPROOF The error lies dormant till the next loop 
%% INCLUDEPROOF iteration, when @i = malen arr + 1@ and we 
%% INCLUDEPROOF trigger the @j >= top@ branch. 
%% INCLUDEPROOF %
%% INCLUDEPROOF Here, we allocate a larger array and copy 
%% INCLUDEPROOF the contents of the previous array into the 
%% INCLUDEPROOF new array. 
%% INCLUDEPROOF %
%% INCLUDEPROOF The @copyM arr' 0 arr 0 top@ call
%% INCLUDEPROOF only copies @top@ elements, \ie it 
%% INCLUDEPROOF \emph{does not}
%% INCLUDEPROOF copy the element \emph{at} \texttt{top},
%% INCLUDEPROOF \emph{losing} a @Word16@ and so 
%% INCLUDEPROOF yielding the wrong  output.
%% INCLUDEPROOF The fix is to replace...
%% INCLUDEPROOF \begin{code}
%% INCLUDEPROOF    | j >= top  -> do ...
%% INCLUDEPROOF    | otherwise -> do
%% INCLUDEPROOF      d <- writeChar arr i c
%% INCLUDEPROOF      loop z' s' (i+d)
%% INCLUDEPROOF    where (z',c) = f z x
%% INCLUDEPROOF          j | ord c < 0x10000 = i
%% INCLUDEPROOF            | otherwise       = i + 1
%% INCLUDEPROOF \end{code}

%%% Local Variables: 
%%% mode: latex
%%% TeX-master: "main"
%%% End: 


}
{}

\end{document}

%%% Local Variables: 
%%% mode: latex
%%% TeX-master: "main"
%%% End: 
