\section{Evaluation}\label{sec:evaluation}

Our goal is to build a practical and effective 
SMT \& refinement type-based verifier for Haskell. 
%
We have shown that lazy evaluation requires the 
verifier to reason about divergence; we have proposed 
an approach for implicitly reasoning about divergence 
by eagerly proving termination, thereby optimizing 
the precision of the verifier.
%
Next, we describe an experimental evaluation of our 
approach that uses \toolname to prove termination and 
functional correctness properties of a suite of widely 
used Haskell libraries totaling more than 10KLOC.
%
Our evaluation seeks to determine whether our approach is
%
\emph{suitable} for a lazy language (\ie do most Haskell functions terminate?),
\emph{precise}  enough to  capture the termination reasons (\ie is \toolname able to prove that most functions terminate?), 
\emph{usable}   without placing an unreasonably high burden on the user in the form of explicit termination annotations, and
\emph{effective}  enough to  enable the verification of functional correctness properties.
%
For brevity, we omit a description of the properties other
than termination, please see~\cite{realworldliquid14} for details.

\spara{Implementation} \toolname takes as input:
%
(1)~A Haskell \emph{source} file, 
%
(2)~Refinement type \emph{specifications}, 
    including refined datatype definitions,
    measures, predicate and type aliases,
    and function signatures, and
%
(3)~Predicate fragments called \emph{qualifiers}
    which are used to infer refinement types using 
    the abstract interpretation framework of Liquid
    Typing~\cite{LiquidPLDI08}.
%
The verifier returns as output, \textsc{Safe} or \textsc{Unsafe}, 
depending on whether the code meets the specifications or not, 
and, importantly for debugging the code (or specification!) 
the inferred types for all sub-expressions.

\spara{Benchmarks}
As benchmarks, we used the following libraries:
%
\texttt{GHC.List} and \texttt{Data.List}, which together implement many standard
list operations,
\texttt{Data.Set.Splay}, which implements an splay functional set,
\texttt{Data.Map.Base}, which implements a functional 
map,
\libvectoralgos, 
which includes a suite of ``imperative'' % (\ie monadic)
array-based sorting algorithms,
\bytestring, a library for manipulating byte arrays, and
\libtext, a library for high-performance Unicode text processing. 
%
These benchmarks represent a wide spectrum of idiomatic
Haskell codes: the first three are widely used libraries 
based on recursive data structures, the fourth and fifth 
perform subtle, low-level arithmetic manipulation of array
indices and pointers, and the last is a rich, high-level
library with sophisticated application-specific invariants,
well outside the scope of even Haskell's expressive type system.
Thus, this suite provides a diverse and challenging test-bed 
for evaluating \toolname.

%% Thus, proving termination of functions in these libraries should
%% require a variety of termination metrics, ranging from simple
%% structural recursion to sophisticated numerical invariants, and should
%% provide a rich test-bed for evaluating our approach.
%% These last three libraries are especially representative as they pervasively 
%% intermingle high level abstractions like higher-order loops, folds, and fusion, 
%% with low-level pointer manipulations in order to deliver high-performance. 
%% They are an appealing target for \toolname, as refinement types are an ideal 
%% way to statically enforce critical invariants that are outside the scope of
%% run-time checking as even Haskell's highly expressive type system.

\begin{table}
\begin{scriptsize}
\centering
\begin{tabular}{|l|r|rrrr|r|}
\hline
\textbf{Module} & \textbf{LOC} & \textbf{Fun} & \textbf{Rec} & \textbf{Div} & \textbf{Hint} & \textbf{Time}\\
\hline\hline
\texttt{GHC.List}       & 309   & 66   & 34  & 5  & 0   & 14 \\
\texttt{Data.List}      & 504   & 97   & 50  & 2  & 6   & 11 \\

\texttt{Data.Map.Base}  & 1396  & 180  & 94  & 0  & 12  & 175 \\
\texttt{Data.Set.Splay} & 149   & 35   & 17  & 0  & 7   & 26 \\

\bytestring             & 3505  & 569  & 154 & 8  & 73  & 285 \\

\libvectoralgos         & 1218  & 99   & 31  & 0  & 31  & 85 \\

\libtext                & 3128  & 493  & 124 & 5  & 44  & 481 \\

\hline
\textbf{Total}          & 10209 & 1539 & 504 & 20 & 173 & 1080 \\
\hline
\end{tabular}
\caption{\scriptsize A quantitative evaluation of our experiments.
  \textbf{LOC}  is the number of non-comment lines of source code as reported
                   by \texttt{sloccount}.
  \textbf{Fun}  is the total number of functions in the library.
  \textbf{Rec}  is the number of recursive functions.
  \textbf{Div}  is the number of functions marked as potentially non-terminating.
  \textbf{Hint} is the number of termination hints, in the form of
                   \emph{termination expressions}, given to \toolname.
  \textbf{Time} is the time, in seconds, required to run \toolname.}
\label{table:results}
\end{scriptsize}
\end{table}


\spara{Results}
Table~\ref{table:results} summarizes our experiments, which 
covered 39 modules totaling 10,209 non-comment lines of 
source code. % and 1,652 lines of specifications.
%
The results were collected on a machine with an Intel Xeon 
X5600 and 32GB of RAM~(no benchmark required more than 1GB).
%
Timing data was for runs that performed full
verification of safety and functional correctness 
properties in addition to termination. %; we only discuss the latter here.

\begin{itemize}
  \item{\emph{Suitable:}} Our approach of eagerly proving termination is in
    fact, \emph{highly} suitable: of the % a total of 1,539 functions, including
    504 recursive functions, only 12 functions were \emph{actually}
    non-terminating (\ie non-inductive).
    That is, 97.6\% of recursive functions 
    % (or 99.1\% of all functions)
    are inductively defined.

  \item{\emph{Precise:}} Our approach is extremely precise, as refinements
    provide auxiliary invariants and extensibility that is crucial for 
    proving termination. We successfully \emph{prove} that 96.0\% of 
    recursive functions % (or 98.6\% of all functions)
    terminate. 
    % (Of the 8 that we fail on, 1 terminates for reasons not fully clear to us!)

  \item{\emph{Usable:}} Our approach is highly usable and only places a 
    modest annotation burden on the user. The default metric, namely the first 
    parameter with an associated size measure, suffices to automatically 
    prove 65.7\% of recursive functions terminating. Thus, only 34.3\% require 
    explicit termination metric, totaling about 1.7 witnesses (about 1 line
    each) per 100 lines of code.

  \item{\emph{Effective:}} Our approach is extremely effective at improving the
    precision of the overall verifier (by allowing the VC to use facts
    about binders that provably reduce to values.) 
    % As shown in the \textbf{Err} column in Table~\ref{table:results}, 
    Without the termination optimization, \ie by only using information 
    for matched-binders (thus in WHNF), \toolname reports 1,395 
    unique functional correctness warnings -- about 1 per 7 lines.
    With termination information, this number goes to zero.
\end{itemize}

%%% Our default metric, namely the first parameter with an associated size
%%% measure, suffices to prove 67\% of (recursive) functions terminating.
%%% %
%%% 30\% require a hint (\ie the position of the decreasing argument) or a
%%% termination metric (3\% required both), and the remaining 3\% were marked as
%%% potentially diverging.
%%% %
%%% Of the 18 functions marked @lazy@ (\ie potentially diverging), 
%%% we suspect 6 actually terminate but were unable to prove so.
%%% 
%%% There are two important results to emphasize here:
%%% %
%%%     (1) \emph{most} Haskell functions terminate on finite inputs, and
%%% %
%%%     (2) the termination metrics are \emph{easily} specified in \toolname.


\begin{comment}


Next, we present a qualitative overview of how to prove termination
properties using \toolname.
\subsection{Proving Termination with Refinements}
\label{sec:proving-termination-with-refinements}
{

% 1. Nats
% 2. structural recursion on Lists
% 3. auxiliary invariants that need ghost variables or termination
% expressions, e.g. list merge
% 4. Bytestring
\ES{Assuming we have already outlined the approach with sized types
  encoded via refinements in the overview.}

Let us begin with one of the simplest recursive functions, the
fibonacci function.
%
\begin{code}
  fib  :: Nat -> Int
  fib 0 = 1
  fib 1 = 1
  fib n = fib (n-1) + fib (n-2)
\end{code}
%
Recall that we have simply defined @Nat@ as @{v:Int | v>=0}@. In
this case, the argument @n@ can itself be used as the termination
metric. The precondition ensures that @n@ will be non-negative, \ie
@n@ is \emph{well-founded}, and \toolname knows from the type of
@(-)@ that @n-1@ and @n-2@ are both strictly less than @n@, therefore
@fib@ \emph{must} terminate. 

Not all recursive functions on @Nat@s have such simple termination
metrics, one famous example is the Ackermann function.
%
\begin{code}
  ack m n 
    | m == 0    = n + 1
    | n == 0    = ack (m-1) 1 
    | otherwise = ack (m-1) (ack m (n-1))
\end{code}
%
Neither @m@ nor @n@ can be shown to decrease in each recursive call,
however @ack@ can still be proven terminating because \emph{either}
@m@ decreases \emph{or} @m@ remains the same and @n@ decreases. In
other words, the pair @(m,n)@ strictly decreases according to a
well-founded \emph{lexicographic} ordering. We express lexicographic
termination in \toolname using \emph{termination expressions}, which
take the form @x :: t / e@ where @e@ is a list of \toolname
expressions that should be used to prove termination. In the case of
@ack@ the expressions will simply be references to the parameters @m@
and @n@
%
\begin{code}
  ack :: m:Nat -> n:Nat -> Nat / [m,n]
\end{code}
%
encoding the requirement that @(m,n)@ is a lexicographically
decreasing pair.

\spara{Measuring the Size of Structures}
Many functions are structurally recursive and can be proven
terminating by showing that they always recur on a \emph{substructure}.
Consider the function @map@ defined over the standard list type.
%
\begin{code}
  map f []     = [] 
  map f (x:xs) = f x : map f xs
\end{code}
%
In @map@, the recursive call is made with a ``smaller'' input.
We formalize the notion of size with \emph{measures}.
%
\begin{code}
  measure len :: [a] -> Nat
  len []     = 0
  len (x:xs) = 1 + len xs
\end{code}
%
With the above definition, 
the @measure@ strengthens the type of the data constructors to:
%
\begin{code}
  []  :: {v: [a] | len v = 0}
  (:) :: x:a -> xs:[a]
      -> {v:[a] | len v = 1 + len xs}
\end{code}
%
where @len@ is simply an uninterpreted function in the SMT
logic~\cite{LiquidPLDI09}. With the strengthened data constructors,
\toolname happily proves that @map@ is always called recursively with
smaller inputs\footnote{\toolname defaults to the second parameter
  here for proving termination because it has no notion of the size of
  a function.}, and that the input size is bounded below by 0,
therefore it must terminate.
\spara{Expressing Termination} 
Sometimes, the decreasing metric cannot be associated with a single
parameter or lexicographically ordered sequence of parameters, but is
instead an auxiliary value that is a \emph{function} of the
parameters.
%
For example, here is the @union@ function from the splay-tree-based
set library
%
\begin{code}
  union Leaf         t = t
  union (Node x a b) t = Node x taa tbb
    where 
      taa          = union ta a
      tbb          = union tb b
      (ta, _, tb)  = split x t
\end{code}
%
which uses the following datatype:
%
\begin{code}
  data Splay a = Leaf 
               | Node a (Splay a) (Splay a)
\end{code}
%
Here, as the recursive call ``swaps'' the parameters, \ie the first
(resp. second) argument of the recursive call is computed from the
second (resp. first) parameter, neither parameter provably decreases,
nor do they form a lexicographically-ordered pair.
%
However, on closer inspection it turns out that the \emph{sum}
of the sizes of the trees strictly decreases, as @split@ has type
%
\begin{code}
  a -> t:Splay a -> (SplayL a t, SplayL a t)
\end{code}
%
where the output type uses the alias
%
\begin{code}
  type SplayL a T 
    = {v:Splay a | size v <= size T}
\end{code}
%
and @size@ is a \emph{measure} akin to @len@.
%
\begin{code}
  measure size     :: Splay a -> Nat
  size Leaf         = 0
  size (Node v l r) = 1 + size l + size r
\end{code}
%
We already know how to express this type of termination metric using
our handy termination expressions, so we give @union@ the following type:
%
\begin{code}
  union :: Ord a => x:Splay a -> y:Splay a
        -> Splay a / [size x + size y]
\end{code}

% For example, here is the standard @merge@ function from the eponymous sorting
% procedure:
% %
% \begin{code}
%     merge xs@(x:xs') ys@(y:ys') 
%       | x < y     = x : merge xs' ys
%       | otherwise = y : merge xs  ys'
% \end{code}
% %
% neither parameter provably decreases in both calls, but the \emph{sum}
% of the sizes of the parameters strictly decreases. 

% In these cases, we can reuse our termination expression syntax from
% above, but with a more complex expression.
% %
% \begin{code}
%     merge :: xs:[a] -> ys:[a] -> [a] 
%           / [len xs + len ys]
% \end{code}
% %
% This is equivalent to rewriting @merge@ to take a \emph{ghost}
% parameter that acts as a termination witness, \eg
% %
% \begin{code}
%     merge :: xs:[a] -> ys:[a] 
%           -> {v:Nat | v = len xs + len ys} 
%           -> [a] 
% \end{code}
% %
% except that with the termination expression, one need not modify the
% actual code! Now, \toolname verifies that in each recursive call the
% expression @len xs + len ys@ is strictly smaller, thereby proving that
% @merge@ terminates.

\spara{Pointer-based Termination}
For our last example we will look at the \bytestring
library. \bytestring is an interesting target for our termination
experiments because proving termination of \bytestring functions often
requires reasoning about the length of the memory segment that a
pointer points to.

A (strict) @ByteString@ is a triple of a @pay@load pointer, 
an @off@set into the memory buffer referred to by the pointer 
(at which the string actually ``begins'') and a @len@gth 
corresponding to the number of bytes in the string, which is 
the size of the buffer \emph{after} the @off@set, that
corresponds to the string.
%
We define a measure for the \emph{size} of 
a @ForeignPtr@'s buffer, and use it to define 
the key invariants as a refined datatype 
%
\begin{code}
  measure fplen  :: ForeignPtr a -> Nat
  data ByteString = PS
   { pay :: ForeignPtr Word8
   , off :: {v:Nat | v       <= fplen pay}
   , len :: {v:Nat | off + v <= fplen pay} 
   }
\end{code}
%
\ES{can probably kill this para about invariants if we need space}
The definition states that 
the offset is a @Nat@ no bigger than the size of 
the @payload@'s buffer, and that
the sum of the @off@set and non-negative @len@gth
is no more than the size of the payload buffer.
Finally, we encode a @ByteString@'s size as a measure.
%
\begin{code}
  measure blen   :: ByteString -> Nat
  blen (PS p o l) = l
\end{code}

Consider the function @findIndex@, which searches through a
@ByteString@ for a @Word8@ that satisfies some predicate @p@.

\begin{code}
  findIndex p (PS x s l) = inlinePerformIO $ 
    withForeignPtr x $ \f -> 
      go l (f `plusPtr` s) 0
  where
    go :: ptr:_ -> n:_ -> _ / [l - n]
    go ptr n
      | n >= l    = return Nothing
      | otherwise = do 
          w <- peek ptr
          if p w
            then return (Just n)
            else go (ptr `plusPtr` 1) (n+1)
\end{code}

The bulk of the work is done by the recursive inner function @go@,
which repeatedly increments @ptr@ until it finds an appropriate byte
or it reaches the end of the valid memory region. We use @l - n@ as
the termination metric for @go@, which denotes the number of bytes
remaining in the valid memory region.

\spara{Nested Data}
For our final example, consider @group@, which
splits a string like @"aart"@ into the list
@["aa","r","t"]@.
% , \ie a list of
% (a)~non-empty @ByteString@s whose 
% (b)~total length equals that of the input. 
% To specify these requirements, we define a measure for 
% the total length of strings in a list and use it to
% write an alias for a list of \emph{non-empty} strings
% whose total length equals that of another string:

% \begin{code}
% measure blens :: [ByteString] -> Int 
% blens ([])     = 0
% blens (x:xs)   = blen x + blens xs

% type ByteStringNE 
%   = {v:ByteString | blen v > 0}
% type ByteStringsEq B
%   = {v:[ByteStringNE] | blens v = blen b}
% \end{code}
%
% \toolname uses the above to verify that 
%
\begin{code}
  group xs
    | null xs   = []
    | otherwise = (y `cons` ys) : group zs
    where
      x         = unsafeHead xs
      xs'       = unsafeTail xs
      (ys, zs)  = spanByte x xs' 
\end{code}
%
This example illustrates why refinements are critical for
proving termination. \toolname determines that @unsafeTail@ 
returns a \emph{shorter} @ByteString@ than its input, and that
each element returned by @spanByte@ is no longer than the 
input, concluding that @zs@ is smaller than @xs@, and hence
checking the body under the termination-weakened environment.

\ES{figure out why things break when we change spanByte to use term exprs}

Let's also look at @spanByte@, which splits strings into a pair, to
see how \toolname infers its crucial type:
%
\begin{code}
  spanByte c ps@(PS x s l) 
    = inlinePerformIO $ withForeignPtr x $
        \p -> go (p `plusPtr` s) 0
    where
      go :: p:_ -> i:_ -> _ / [l - i]
      go p i 
        | i >= l    = return (ps, empty)
        | otherwise = do
            c' <- peekByteOff p i
            if c /= c'
            then let b1 = unsafeTake i ps
                     b2 = unsafeDrop i ps
                 in  return (b1, b2)
            else go p (i+1)
\end{code}
%
Here again, the work is all done by the inner recursive function @go@,
which uses the same termination metric as above. \toolname can 
infer that both @ByteString@s in the pair are no longer than @ps@
through a case analysis. In the base case, @blen ps <= blen ps@ and 
@blen empty <= blen ps@ both hold trivially, and in the recursive case
@blen b1 <= blen ps@ and @blen b2 <= blen ps@ both hold because
\toolname infers that @unsafeTake@ and @unsafeDrop@ both
return shorter strings. Thus, both @ByteString@s returned by
@spanByte@ are at most as long as its input @xs'@, which is
\emph{strictly} shorter than the input to @group@; therefore @group@
is only called recursively on smaller inputs and it must terminate!

So the bright side is that, although sound and precise refinements in
a lazy language incurs the additional burden of termination proofs,
refinements make proving termination easy!

\end{comment}

%%% Local Variables: 
%%% mode: latex
%%% TeX-master: "main"
%%% End: 
