% \newcommand\rectyp{\FinTy{\tnat}}

\newcommand\factyp{\typ}

\section{Implementation: \toolname}\label{sec:haskell}

We have implemented \declang in \toolname (\Sref{sec:overview}). 
Next, we describe the key steps in the transition from \declang to \lhaskell.

\subsection{Termination}

%Let ${\factyp \defeq \tfunbasic{\FinTy{\tnat}}{\FinTy{\tnat}}}$.
Haskell's recursive functions of type ${\tfunbasic{\FinTy{\tnat}}{\typ}}$ 
are represented, in GHC's Core \cite{SulzmannCJD07} as
${\mathtt{let\ rec}\ f = \ \efun{n}{}{e}}$ which is operationally
equivalent to ${\mathtt{let}\ f = \ \etfix{\typ}\ (\efun{n}{}{\efun{f}{}{e}})}$.
Given the type of $\etfix{\typ}$, checking that $f$ has 
type $\tfunbasic{\FinTy{\tnat}}{\typ}$ reduces to checking
$e$ in a \emph{termination-weakened environment} where
$$f \ \colon \ \tfunbasic{\tref{v}{\tnat}{\finite}{v < n}}{\typ}$$
%
Thus, \toolname proves termination just as \declang 
does: by checking the body in the above environment, 
where the recursive binder is called with $\tnat$ 
inputs that are strictly smaller than $n$.

\mypara{Default Metric}
For example, \toolname proves that 
%
\begin{code}
  fac n = if n == 0 then 1 else n * fac (n-1)
\end{code}
%
has type $\tfunbasic{\FinTy{\tnat}}{\FinTy{\tnat}}$ 
by typechecking the body of @fac@ 
in a termination-weakened environment 
${\mathtt{fac}\ : \tfunbasic{\tref{\ttv}{\tnat}{\finite}{\ttv < \ttn}}{\FinTy{\tnat}}}$
The recursive call generates the subtyping query:
\begin{align*}
\tbind{\ttn}{\{0 \leq \ttn\}}, \lnot (\ttn = 0) \vdash_D &\  \subt{\ttref{v=n-1}}{\ttref{0 \leq v \wedge v < n}}\\ 
%\tlref{n}{\tint}{\finite}{0 \leq n}, \lnot (n = 0) \vdash_D &\  \subt{\ttref{v=n-1}}{\ttref{0 \leq v \wedge v < n}}\\ 
\intertext{Which reduces to the valid VC}
0 \leq \ttn \wedge \lnot (\ttn = 0) \Rightarrow &\   (\ttv = \ttn-1) \Rightarrow (0 \leq \ttv \wedge \ttv < \ttn)
\end{align*}
%
proving that $\mathtt{fac}$ terminates, in essence because the
\emph{first parameter} forms a \emph{well-founded decreasing metric}.

\mypara{Refinements Enable Termination} 
Consider Euclid's GCD:
%
\begin{code}
  gcd :: a:Nat -> {v:Nat | v < a} -> Nat 
  gcd a 0 = a
  gcd a b = gcd b (a `mod` b)
\end{code}
%
Here, the first parameter is decreasing, but this requires
the fact that the second parameter is smaller than the first 
and that @mod@ returns results smaller than its second 
parameter. Both facts are easily expressed as refinements, 
but elude non-extensible checkers~\cite{Giesl11}.

\mypara{Explicit Termination Metrics}
The indexed-fixpoint combinator technique is easily extended to 
cases where some parameter \emph{other} than the first is the 
well-founded metric. For example, consider: 
% As an example, consider the tail-recursive factorial:
%
\begin{code}
  tfac     :: Nat -> n:Nat -> Nat / [n] 
  tfac x n | n == 0    = x
           | otherwise = tfac (n*x) (n-1)
\end{code}
%
We specify that the \emph{last parameter} is decreasing by 
specifying an explicit termination metric @/ [n]@ in the 
type signature.
%
\toolname \emph{desugars} the 
termination metric into a new $\tnat$-valued \emph{ghost parameter} @d@ 
whose value is always equal to the termination metric @n@:
\begin{code}
  tfac :: d:Nat -> Nat -> {n:Nat | d = n} -> Nat 
  tfac d x n | n == 0    = x
             | otherwise = tfac (n-1) (n*x) (n-1)
\end{code}
%
Type checking, as before, checks the body in an environment where 
the first argument of @tfac@ is weakened, \ie, requires proving @d > n-1@.
%
So, the system needs to know that the ghost argument @d@ 
represents the decreasing metric.
%
We capture this information in the type signature of @tfac@ where the \emph{last} 
argument exactly specifies that @d@ is the termination metric @n@, \ie, @d = n@.
%
Note that since the termination metric can depend on any argument, 
it is important to refine the last argument,
so that all arguments are in scope, with the fact that @d@ is the termination metric.

To generalize, desugaring of termination metrics proceeds as follows.
Let $f$ be a recursive function with parameters $\overline{x}$, and 
termination metric $\mu(\overline{x})$. Then \toolname will
\begin{itemize}
\item add a $\tnat$-valued ghost first parameter $d$ in the definition of $f$, 
\item weaken the last argument of $f$ with the refinement $d = \mu(\overline{x})$, %and
\item at each recursive call of $f\ \overline{e}$, 
apply $\mu(\overline{e})$ as the first argument.
\end{itemize}  
%
%%As will shall see this technique can be used 
%%when the termination metric is any logical expression.

\mypara{Explicit Termination Expressions} 
Let us now apply the previous technique in a function where
none of the parameters themselves decrease across recursive calls,
but there is some \emph{expression} that forms the decreasing metric.
%
%Sometimes, none of the parameters themselves decrease across recursive calls,
%but there is some \emph{expression} that forms the decreasing metric.
%
Consider @range lo hi@, which returns the list of 
@Int@s from @lo@ to @hi@:
%
We generalize the explicit metric specification to 
\emph{expressions} like @hi-lo@. \toolname \emph{desugars} the 
expression into a new $\tnat$-valued \emph{ghost parameter} 
whose value is always equal to @hi-lo@, that is:
\begin{code}
  range :: lo:Nat -> {hi:Nat | hi >= lo} -> [Nat] 
         / [hi-lo]
  range lo hi 
    | lo < hi = lo : range (lo + 1) hi
    | _       = [] 
\end{code}
%
Here, neither parameter is decreasing (indeed, the first one
is \emph{increasing}) but @hi-lo@ decreases across each call. 
%
We generalize the explicit metric specification to 
\emph{expressions} like @hi-lo@. \toolname \emph{desugars} the 
expression into a new $\tnat$-valued \emph{ghost parameter} 
whose value is always equal to @hi-lo@, that is:
%
\begin{code}
  range lo hi = go (hi-lo) lo hi
    where 
      go :: d:Nat -> lo:Nat 
         -> {hi:Nat | d = hi - lo} -> [Nat]
      go d lo hi
       | lo < hi = l : go (hi-(lo+1)) (lo+1) hi 
       | _       = []
\end{code}
%
After which, it proves @go@ terminating, by showing 
that the first argument @d@ is a \tnat that decreases across each 
recursive call (as in @fac@ and @tfac@).

\mypara{Recursion over Data Types}
The above strategy generalizes easily to functions that recurse
over (finite) data structures like arrays, lists, and trees.
In these cases, we simply use \emph{measures} to project the 
structure onto \tnat, thereby reducing the verification to 
the previously seen cases. For each user defined type, \eg
%
\begin{code}
  data L [sz] a = N | C a (L a)
\end{code}
%
we can define a \emph{measure}
%
\begin{code}
  measure sz  :: L a -> Nat
    sz (C x xs) = 1 + (sz xs)
    sz N        = 0
\end{code}
%
We prove that @map@ terminates using the type:
%
\begin{code}
  map :: (a -> b) -> xs:L a -> L b / [sz xs]
  map f (C x xs) = C (f x) (map f xs)
  map f N        = N
\end{code}
%
That is, by simply using @(sz xs)@  as the 
decreasing metric.

\mypara{Generalized Metrics Over Datatypes}
Finally, in many functions there is no single argument 
whose (measure) provably decreases. For example, consider:
%
\begin{code}
  merge :: xs:_ -> ys:_ -> _ / [sz xs + sz ys]
  merge (C x xs) (C y ys)
    | x < y     = x `C` (merge xs  (y `C` ys))
    | otherwise = y `C` (merge (x `C` xs)  ys)
\end{code}
%
from the homonymous sorting routine. Here, neither parameter
decreases, but the \emph{sum} of their sizes does. 
%
As before \toolname desugars the decreasing expression into 
a ghost parameter and thereby proves termination (assuming, 
of course, that the inputs were finite lists, \ie 
$\FinTy{\mathtt{L}}\ a$.)

\mypara{Automation: Default Size Measures}
Structural recursion on the first argument is a common pattern 
in \lhaskell code.
%
\toolname automates termination proofs for this common case,
by allowing users to specify a \emph{size measure} 
for each data type, (\eg @sz@ for @L a@).
%
Now, if \emph{no} termination metric is given, by default 
\toolname assumes that the \emph{first} argument whose type
has an associated size measure decreases.
%
Thus, in the above, we need not specify metrics for @fac@ 
or @gcd@ or @map@ as the size measure is automatically 
used to prove termination. 
%
This simple heuristic allows us to {automatically}
prove 67\% of recursive functions terminating.

%%% \mypara{Summary}
%%% To sum up, 
%%% %
%%% \begin{itemize}
%%% \item No termination check for functions marked @lazy@, 
%%% \item If no explicit termination metrice, then first 
%%%       argument with size measure used by default,
%%% \item Otherwise, explicit termination metric desugared 
%%%       into ghost @nat@ parameter that is used to prove 
%%%       termination.
%%% \end{itemize}

\subsection{Non-termination}

By default, \toolname checks that every function is 
terminating. We show in \Sref{sec:evaluation} that 
this is in fact the overwhelmingly common case in practice.
%
However, annotating a function as @lazy@ deactivates 
\toolname's termination check (and marks the result as a 
\Div type).
%
This allows us to check functions that are 
non-terminating, and allows \toolname to prove safety 
properties of programs that manipulate \emph{infinite} 
data, such as streams, which arise idiomatically with 
Haskell's lazy semantics.
% 
For example, consider the classic @repeat@ function:
%
\begin{code}
  repeat x = x `C` repeat x
\end{code}
%
We cannot use the $\etfix{}$ combinators to 
represent this kind of recursion, and hence, 
use the non-terminating $\efix{}$ combinator 
instead. 

Let us see how we can use refinements to statically 
distinguish between finite and infinite streams. 
The direct, \emph{global} route of using a measure
%
\begin{code}
  measure inf    :: L a -> Prop 
    inf (C x xs) = inf xs
    inf N        = false 
\end{code}
%
to describe infinite lists is unavailable as such 
a measure, and hence, the corresponding refinement
would be non-terminating.
%
Instead, we describe infinite lists in \emph{local} 
fashion, by stating that each \emph{tail} is non-empty.

\mypara{Step 1: Abstract Refinements} 
We can parametrize a datatype with abstract 
refinements that relate sub-parts of the 
structure \cite{vazou13}. 
For example, we parameterize the list type as:
%
\begin{code}
  data L a <p :: L a -> Prop> 
    = N | C a {v: L<p> a | (p v)}
\end{code}
%
which parameterizes the list with a refinement 
@p@ which holds \emph{for each} tail of the list, 
\ie holds for each of the second arguments to 
the @C@ constructor in each sub-list.


\mypara{Step 2: Measuring Emptiness} Now, we can write a measure that 
states when a list is \emph{empty}
%
\begin{code}
  measure emp  :: L a -> Prop 
    emp N        = true
    emp (C x xs) = false
\end{code}
%
As described in \Sref{sec:typing}, \toolname translates the 
abstract refinements and measures into refined types for 
@N@ and @C@.

\mypara{Step 3: Specification \& Verification}
Finally, we can use the abstract refinements and measures to 
write a type alias describing a refined version of @L a@ 
representing infinite streams:
%
\begin{code}
  type Stream a = 
    {xs: L <{\v -> not(emp v)}> a | not(emp xs)}
\end{code}
%
We can now type @repeat@ as:
%
\begin{code}
  lazy repeat :: a -> Stream a
  repeat x    = x `C` repeat x 
\end{code}
%
The @lazy@ keyword \emph{deactivates} termination checking, and 
marks the output as a \Div type.
%
Even more interestingly, we can prove safety properties of 
infinite lists, for example:
%
\begin{code}
  take            :: Nat -> Stream a -> L a
  take 0 _        = N
  take n (C x xs) = x `C` take (n-1) xs
  take _ N        = error "never happens"
\end{code}
%
\toolname proves, similar to the @head@ example from
\Sref{sec:overview}, that we never match a @N@ when 
the input is a @Stream@.

\mypara{Finite vs. Infinite Lists}
%
Thus, the combination of refinements 
and labels allows our stratified type 
system to specify and verify whether 
a list is finite or infinite.
%
Note that:
%
$\FinTy{\mathtt{L}}\ a$ represents
\emph{finite} lists \ie those 
produced using the (inductive) 
terminating fixpoint combinators,
%
$\WnfTy{\mathtt{L}}\ a$ represents 
(potentially) infinite lists which 
are guaranteed to reduce to values, 
\ie non-diverging computations that
yield finite or infinite lists,
and
$\DivTy{\mathtt{L}}\ a$ represents 
computations that may diverge or 
produce a finite or infinite list.

\subsection{User Specifications and Type Inference}

In program verification it is common that the user provides functional
specification that the code should satisfy.
In \toolname these specifications can be provided as type signatures 
for @let@-bound variables.
%
Consider the typechecking rules of Figure~\ref{fig:typing}
that is used by \declang.
%
$$
\inference{
	\hastype{\Gamma}{e_x}{\tau_{x}} &&
	\hastype{\Gamma,x\colon\tau_x}{e}{\tau} &&
	\iswellformed{\Gamma}{\tau}
}{
	\hastype{\Gamma}{\elet{x}{e_x}{e}}{\tau}
}[\rtlet]
$$
%
Note that \rtlet \emph{guesses} an appropriate type $\tau_x$
for $e_x$ and binds this type to $x$ to typecheck $e$.

\toolname allows the user to specify the type $\tau_x$ for top level bindings.
%
For every binding \elet{x}{e_x}{\dots}, if the user provides a type specification $\tau_x$,
\toolname checks using the appropriate environment 
(1)~that the specified type is well-formed, and 
(2)~that the expression $e_x$ typechecks under the specification $\tau_x$.
%
For the other top level bindings, \ie those without user-provided specifications, 
as well as all local bindings, \toolname uses the Liquid Types~\citep{LiquidPLDI08} 
framework to infer refinement types, thus greatly reducing the number of annotations 
required from the user.

%%
%%\mypara{Liquid Types} 
%%The Liquid Types method infers refinements in three steps. 
%%%
%%First, we create refinement \emph{templates} for the unknown, 
%%to-be-inferred refinement types. The \emph{shape} of the template 
%%is determined by the Haskell's unrefined type it corresponds to. 
%%The template is just the shape refined with fresh refinement variables
%%$\kappa$ denoting the unknown refinements at each type position. 
%%For example, from a type ${\tfun{x}{t_x}{t}}$ we create 
%%the template ${\tfun{x}{\tref{v}{t_x}{}{\kappa_x}}{\tref{v}{t}{}{\kappa}}}$.
%%%
%%Second, we perform type checking using the templates (in place of the
%%unknown types.) Each wellformedness check becomes a wellformedness
%%constraint over the templates, and hence over the individual $\kappa$,
%%constraining which variables can appear in $\kappa$.
%%Each subsumption check becomes a subtyping constraint
%%between the templates, which can be further simplified, via syntactic
%%subtyping rules, to a logical implication query between the variables
%%$\kappa$.
%%%
%%Third, we solve the resulting system of logical implication constraints
%%(which can be cyclic) via abstract interpretation --- in particular,
%%monomial predicate abstraction over a set of logical qualifiers
%%\cite{Houdini,LiquidPLDI08}. The solution is a map from $\kappa$ to
%%conjunctions of qualifiers, which, when plugged back into the templates,
%%yields the inferred refinement types.

%%%%%%%%%%%%%%%%%%%%%%%%%%%%%%%%%%%%%%%%%%%%%%%%%%%%%%%%%%%%%%%%
%%%%%%%%%%%%%%%%%%%%%%%%%%%%%%%%%%%%%%%%%%%%%%%%%%%%%%%%%%%%%%%%
%%%%%%%%%%%%%%%%%%%%%%%%%%%%%%%%%%%%%%%%%%%%%%%%%%%%%%%%%%%%%%%%

%%%%%%%%   \section{Implementation: \toolname}\label{sec:haskell}
%%%%%%%%   We implemented \undeclang in \toolname (\S~\ref{sec:overview}), 
%%%%%%%%   a type based verifier for \lhaskell.
%%%%%%%%   %
%%%%%%%%   Before~\citep{vazou13} \toolname assumed 
%%%%%%%%   eager semantics. 
%%%%%%%%   %
%%%%%%%%   We modified the tool as suggested in 
%%%%%%%%   \S~\ref{sec:language}: 
%%%%%%%%   the environment of each VC 
%%%%%%%%   contains refinements of
%%%%%%%%   only provably terminating bindings.
%%%%%%%%   %
%%%%%%%%   This modification enables us to 
%%%%%%%%   reason about coinductive data, like @Streams@
%%%%%%%%   (\S~\ref{subsec:infinite})
%%%%%%%%   and forces us to prove termination (\S~\ref{subsec:termination})
%%%%%%%%   
%%%%%%%%   \subsection{Reasoning about Streams}\label{subsec:infinite}
%%%%%%%%   We use Abstract Refinements~\citep{vazou13} to 
%%%%%%%%   define a list data type parameterized with
%%%%%%%%    a type @a@ and a predicate @p@:
%%%%%%%%   \begin{code}
%%%%%%%%     data L [size] a <p :: L a -> Prop> 
%%%%%%%%     	= N
%%%%%%%%     	| C (x:a) ({xs: L <p> a | p xs})  
%%%%%%%%   \end{code}
%%%%%%%%   Ignoring the @[size]@ annotation,
%%%%%%%%   the above defines a list @L@
%%%%%%%%   such that for any type @a@
%%%%%%%%   and any predicate @p@,
%%%%%%%%   if a value has type @L <p> a@
%%%%%%%%   it is either @N@ or
%%%%%%%%   @C x xs@ where
%%%%%%%%   the tail @xs@ satisfies the predicate @p@.
%%%%%%%%   %
%%%%%%%%   The predicate should recursively hold, 
%%%%%%%%   thus, \emph{all} the tails of a @List <p> a@
%%%%%%%%   satisfy @p@.
%%%%%%%%   
%%%%%%%%   But what is a property that all tails satisfy?
%%%%%%%%   By definition, a Stream
%%%%%%%%   is a list in which @N@ does not appear, 
%%%%%%%%   \ie a list in which \emph{all} the tails 
%%%%%%%%   are constructed by @C@.
%%%%%%%%   %
%%%%%%%%   We use an @isCons@ measure that checks if a list is 
%%%%%%%%   constructed by @C@ to define 
%%%%%%%%   a @Stream@ type alias:
%%%%%%%%   \begin{code}
%%%%%%%%   type Stream a = {v: L <isCons> a | isCons v}
%%%%%%%%   
%%%%%%%%   measure isCons :: L a -> Prop
%%%%%%%%    isCons N        = false
%%%%%%%%    isCons (C x xs) = true
%%%%%%%%   \end{code}
%%%%%%%%   
%%%%%%%%   In \toolname, 
%%%%%%%%   the above definitions, 
%%%%%%%%   translate into the appropriate t
%%%%%%%%   types for lists data constructors:
%%%%%%%%   \begin{code}
%%%%%%%%   N :: forall a, <p :: L a -> Prop>. 
%%%%%%%%          {v: L <p> a | not (isCons v)}
%%%%%%%%   C :: forall a, <p :: L a -> Prop>.
%%%%%%%%           a -> {xs : L <p> a | p xs}
%%%%%%%%        -> {v: L <p> a | isCons v}
%%%%%%%%   \end{code}
%%%%%%%%   %
%%%%%%%%   Thus @N@ returns a list that falsifies @isCons@
%%%%%%%%   and has any abstract
%%%%%%%%   predicate @<p>@,
%%%%%%%%   while @C x xs@ returns a list that satisfies
%%%%%%%%   @isCons@ and
%%%%%%%%   has any abstract refinements
%%%%%%%%   that is satisfied by @xs@, \ie @p xs@ holds.
%%%%%%%%   %
%%%%%%%%   When in the type of @C@ the abstract refinement is instantiated 
%%%%%%%%   with @isCons@ we get that
%%%%%%%%   ``a list is a @Stream@ \emph{iff} so is its tail'':
%%%%%%%%   \vbox{@C :: a -> Stream a -> Stream a@}
%%%%%%%%   %
%%%%%%%%   Given so, we prove that @repeat x@ returns a @Stream@:
%%%%%%%%   \begin{code}
%%%%%%%%   repeat :: a -> Stream a
%%%%%%%%   repeat x = x `C` xs where xs = repeat x 
%%%%%%%%   \end{code}
%%%%%%%%   Assuming the type of @repeat@ we get that
%%%%%%%%   @xs :: Stream a@.
%%%%%%%%   %
%%%%%%%%   Thus @C x xs@ is also a @Stream@.
%%%%%%%%   
%%%%%%%%   Finally, we prove that one can safely @take@ @n@ elements from a @Stream@:
%%%%%%%%   \begin{code}
%%%%%%%%   take :: Nat -> Stream a -> a
%%%%%%%%   take 0 _           = N
%%%%%%%%   take n ys@(C x xs) = x `C` take (n-1) xs
%%%%%%%%   \end{code}
%%%%%%%%   To prove @take@ safe we need to prove that 
%%%%%%%%   at the recursive call @xs :: Stream a@; but 
%%%%%%%%   this is provable as @ys :: Stream@.
%%%%%%%%   %%Even though this information exists, it is generally guarded 
%%%%%%%%   %%by the requirement of @ys@ begin a value.
%%%%%%%%   %%But the recursive call happens after @ys@ is matched
%%%%%%%%   %%thus @ys@'s is indeed a value, which unguards its type
%%%%%%%%   %%and concludes our proof.
%%%%%%%%   
%%%%%%%%   With these types for @take@ and @repeat@, 
%%%%%%%%   \toolname soundly proves 
%%%%%%%%   @take n $ repeat x@
%%%%%%%%   safe for any @n@ and @x@.
%%%%%%%%   
%%%%%%%%   \subsection{Proving Termination}\label{subsec:termination} %$
%%%%%%%%   Typing @repeat@ did not require 
%%%%%%%%   information from the environment, but (\S~\ref{sec;overview})
%%%%%%%%   this is not the general case.
%%%%%%%%   In \undeclang, to use information from the environment, 
%%%%%%%%   you need to prove termination.
%%%%%%%%   This is not really a problem, as \toolname 
%%%%%%%%   implements a rather practical termination prover.
%%%%%%%%   
%%%%%%%%   Assume the definition of a recursive \lhaskell function 
%%%%%%%%   @f = \n -> e@ with type specification 
%%%%%%%%   @f :: n:Nat -> t[n/x]@.
%%%%%%%%   %
%%%%%%%%   To map @f@ in \declang we need to use the 
%%%%%%%%   $\mathtt{fix}$ constant.
%%%%%%%%   By the definition of \etfix{\tau}, 
%%%%%%%%   @f@ behaves exactly as its fixed point version, 
%%%%%%%%   or typing @f@ is equivalent to typing:
%%%%%%%%   $\etfix{\tau}\ (\efun{n}{}{\efun{f}{}{e}})
%%%%%%%%      \colon\colon
%%%%%%%%      \tfun{n}{\tnat^\lfinite}{\tau\sub{x}{n}}
%%%%%%%%   $
%%%%%%%%   %
%%%%%%%%   Which, by the \rtapp rule reduces to 
%%%%%%%%   \begin{align*}
%%%%%%%%   \efun{n}{}{\efun{f}{}{e}}
%%%%%%%%      &\colon\colon
%%%%%%%%   \tfun{n}{\tnat^\lfinite}{
%%%%%%%%   \tfunbasic{(
%%%%%%%%   	\tfunbasic{\tref{n'}{\tnat}{\lfinite}{n' < n}}{\tau\sub{x}{n'}}
%%%%%%%%   )}{\tau\sub{x}{n}}}
%%%%%%%%   \end{align*}
%%%%%%%%   %
%%%%%%%%   Interestingly, the above allows 
%%%%%%%%   the result type of @f@ to be finite, 
%%%%%%%%   thus its success implies that
%%%%%%%%   @f@ terminates!
%%%%%%%%   %
%%%%%%%%   So, \toolname proves @f@ terminates, 
%%%%%%%%   if its body typechecks 
%%%%%%%%   assuming a second argument
%%%%%%%%   \begin{code}
%%%%%%%%     f :: {n' : Nat | n' < n} -> t[n'/x]
%%%%%%%%   \end{code}
%%%%%%%%   
%%%%%%%%   As a concrete example, 
%%%%%%%%   \toolname proves that
%%%%%%%%   fibonacci function terminates exactly
%%%%%%%%   as \undeclang would prove it.
%%%%%%%%   %
%%%%%%%%   \begin{code}
%%%%%%%%     fib  :: Nat -> Int
%%%%%%%%     fib 0 = 1
%%%%%%%%     fib 1 = 1
%%%%%%%%     fib n = fib (n-1) + fib (n-2)
%%%%%%%%   \end{code}
%%%%%%%%   %
%%%%%%%%   To typecheck the body of @fib@,
%%%%%%%%   \toolname assumes a second argument
%%%%%%%%    @fib :: {n':Nat | n' < n} -> Int@
%%%%%%%%   and checks that each call to @fib@, 
%%%%%%%%   \ie each recursive call, is performed with 
%%%%%%%%   decreasing arguments, proving that @fib@
%%%%%%%%   \emph{must} terminate.
%%%%%%%%   
%%%%%%%%   \newcommand{\tmetric}{metric\xspace}
%%%%%%%%   \newcommand{\smetric}{\ensuremath{\mu}\xspace}
%%%%%%%%   Now consider a @range l h@ function
%%%%%%%%   that returns a list ranging from @l@ to @h@:
%%%%%%%%   
%%%%%%%%   \begin{code}
%%%%%%%%   range :: l:Nat -> h:Nat -> [Nat]
%%%%%%%%   range l h | l < h     = l : range (l+1) h
%%%%%%%%             | otherwise = [] 
%%%%%%%%   \end{code}
%%%%%%%%   We cannot apply the previous reasoning 
%%%%%%%%   to prove @range@ terminating.
%%%%%%%%   Yet, @range@ does terminate as at each recursive 
%%%%%%%%   call the \tmetric $\smetric(l, h) = h - l$ decreases.
%%%%%%%%   %
%%%%%%%%   One option would be to use a ghost first argument 
%%%%%%%%   that captures exactly this \tmetric.
%%%%%%%%   %
%%%%%%%%   Then, at a recursive call with actual parameters 
%%%%%%%%   $e_l$ and $e_h$ we should prove that the 
%%%%%%%%   termination condition 
%%%%%%%%   $TC \doteq 0 \leq \smetric(e_l, e_h) < \smetric(l, h)$ holds.
%%%%%%%%   
%%%%%%%%   \toolname captures exactly this behaviour.
%%%%%%%%   We annotate the type of @range@ with the
%%%%%%%%   \emph{termination \tmetric} $\smetric(l, h)$, 
%%%%%%%%   \ie an integer expression that refers to @range@'s arguments
%%%%%%%%   \begin{code}
%%%%%%%%   range :: l:Nat -> h:Nat -> [Nat] / [h - l]
%%%%%%%%   \end{code}
%%%%%%%%   at each function call \toolname checks whether
%%%%%%%%   $TC$ holds; 
%%%%%%%%   and either succeeds and proves termination
%%%%%%%%   or fails with a ``Termination Check Error''.
%%%%%%%%   
%%%%%%%%   \begin{comment}
%%%%%%%%   Next, consider the Ackermann function.
%%%%%%%%   %
%%%%%%%%   \begin{code}
%%%%%%%%     ack m n 
%%%%%%%%       | m == 0    = n + 1
%%%%%%%%       | n == 0    = ack (m-1) 1 
%%%%%%%%       | otherwise = ack (m-1) (ack m (n-1))
%%%%%%%%   \end{code}
%%%%%%%%   %
%%%%%%%%   There exists no integer termination metric that decreases at each recursive call.
%%%%%%%%   %
%%%%%%%%   However @ack@ terminates because at each call \emph{either}
%%%%%%%%   @m@ decreases \emph{or} @m@ remains the same and @n@ decreases. 
%%%%%%%%   %
%%%%%%%%   In other words, the pair @(m,n)@ strictly decreases according to
%%%%%%%%   \emph{lexicographic} ordering. 
%%%%%%%%   %
%%%%%%%%   To capture this requirement we extend termination metric
%%%%%%%%   from an integer to a list of integers
%%%%%%%%   and at each recursive call we check that this list is
%%%%%%%%   lexicographically decreasing.
%%%%%%%%   %
%%%%%%%%   In the case of
%%%%%%%%   @ack@ this list will simply be the parameters @m@
%%%%%%%%   and @n@:
%%%%%%%%   %
%%%%%%%%   \begin{code}
%%%%%%%%     ack :: m:Nat -> n:Nat -> Nat / [m,n]
%%%%%%%%   \end{code}
%%%%%%%%   %
%%%%%%%%   Thus, \toolname uses lexicographic ordering on 
%%%%%%%%   a list of natural numbers to prove termination.
%%%%%%%%   %
%%%%%%%%   Termination metrics could be generalized to 
%%%%%%%%   any \emph{well-found} metric.
%%%%%%%%   
%%%%%%%%   \spara{Mutual Recursion}
%%%%%%%%   %
%%%%%%%%   Equipped with termination metrics
%%%%%%%%   \toolname instantiates a powerful
%%%%%%%%   termination checker that like~\citep{XiTerminationLICS01}
%%%%%%%%   proves termination even for mutual recursive functions.
%%%%%%%%   %
%%%%%%%%   Consider the mutual recursive functions @isEven@ and @isOdd@
%%%%%%%%   \begin{code}
%%%%%%%%   {-@ isEven :: n:Nat -> Bool / [n, 0] @-}
%%%%%%%%   {-@ isOdd  :: n:Nat -> Bool / [n, 1] @-}
%%%%%%%%   
%%%%%%%%   isEven 0 = True
%%%%%%%%   isEven n = isOdd $ n-1
%%%%%%%%   
%%%%%%%%   isOdd n  = not $ isEven n 
%%%%%%%%   \end{code}
%%%%%%%%   Each call terminates as either @isEven@
%%%%%%%%   calls @isOdd@ with a decreasing argument, 
%%%%%%%%   or the argument remains the same, and @isOdd@
%%%%%%%%   calls @isEven@ that should then decrease the argument.
%%%%%%%%   % 
%%%%%%%%   We capture this reasoning using two lexicographic pairs:
%%%%%%%%   each function has its own metric, 
%%%%%%%%   and when @isEven@ calls @isOdd@
%%%%%%%%   the metric of the caller $(n, 0)$
%%%%%%%%   should be greater that callee's metric
%%%%%%%%   $(n-1, 1)$.
%%%%%%%%   %
%%%%%%%%   Similarly, at @isEven@'s call-site 
%%%%%%%%   \toolname verifies that	
%%%%%%%%   $(n, 1) > (n, 0)$.
%%%%%%%%   %
%%%%%%%%   For example, the call @isEven m@
%%%%%%%%   will fire the decreasing metric sequence
%%%%%%%%   $(m, 0) > (m-1, 1) > (m-1, 0) > (m-2, 1) > \dots$
%%%%%%%%   that ultimate terminates for \textit{any}
%%%%%%%%   natural number $m$.
%%%%%%%%   \end{comment}
%%%%%%%%   \NV{ack and isEven used to be here}
%%%%%%%%   
%%%%%%%%   \spara{Measuring the Size of Structures}
%%%%%%%%   Consider again the list \emph{user-defined} data type 
%%%%%%%%   and let us define define its @size@ by a measure:
%%%%%%%%   \begin{code}
%%%%%%%%     size (C x xs) = 1 + size xs
%%%%%%%%     size N        = 0
%%%%%%%%   \end{code}
%%%%%%%%   %
%%%%%%%%   From our discussion so far, 
%%%%%%%%   \toolname can prove that a @map@
%%%%%%%%   function terminates:
%%%%%%%%   \begin{code}
%%%%%%%%   map :: (a -> b) -> xs:L a -> L b / [size xs]
%%%%%%%%   map f (C x xs) = C (f x) (map f xs)
%%%%%%%%   map f N        = N
%%%%%%%%   \end{code}
%%%%%%%%   Turns out that structural recursion on the 
%%%%%%%%   first recursive argument is 
%%%%%%%%   a common pattern in \lhaskell code, so
%%%%%%%%   \toolname proves @map@ terminating 
%%%%%%%%   \emph{without} the need
%%%%%%%%   of the termination metric.
%%%%%%%%   %
%%%%%%%%   If no termination metric is provided, 
%%%%%%%%   \toolname assumes that the first
%%%%%%%%   argument with size information decreases.
%%%%%%%%   %
%%%%%%%%   Data type definitions%, like the one for list @L@
%%%%%%%%   can be annotated
%%%%%%%%   with a measure defining the size of the type;
%%%%%%%%   for @L a@ this measure is @size@.
%%%%%%%%   %
%%%%%%%%   \toolname comes with two build-in sizing information.
%%%%%%%%   For @Int@s it uses their value and for standard lists @[a]@
%%%%%%%%   uses @len@ as defined in \S~\ref{sec:overview}.
%%%%%%%%   %
%%%%%%%%   Even na\"{\i}ve, the above heuristic turns out to be 
%%%%%%%%   quite useful in practise\ref{sec:evaluation}. 
%%%%%%%%   
%%%%%%%%   \mypara{Using Measure Information}
%%%%%%%%   In many functions there is no argument that 
%%%%%%%%   provably decreases its size; 
%%%%%%%%   still the sizes of the arguments can be used to 
%%%%%%%%   express sophisticated termination metrics, as
%%%%%%%%   in the bellow @merge@ function that 
%%%%%%%%   merges two sorted lists.
%%%%%%%%   \begin{code}
%%%%%%%%   merge :: xs:L a -> ys:L a -> L a 
%%%%%%%%          / [size xs + size ys]
%%%%%%%%   merge (C x xs) (C y ys)
%%%%%%%%     | x < y     = x `C` merge xs     (y:ys)
%%%%%%%%     | otherwise = y `C` merge (x:xs) ys
%%%%%%%%   \end{code}
%%%%%%%%   Here the default metric, \ie @size xs@
%%%%%%%%   is not decreasing,
%%%%%%%%   instead we use @size@ to express the more complicated 
%%%%%%%%   termination metric @size xs + size ys@, which witch \toolname
%%%%%%%%   happily proves termination.
%%%%%%%%   
%%%%%%%%   Finally, annotating a function as @lazy@ will deactivate
%%%%%%%%   \toolname's termination check,
%%%%%%%%   %
%%%%%%%%   which concludes how \toolname proves termination:
%%%%%%%%   \begin{itemize}
%%%%%%%%   \item No termination check for @lazy@ functions
%%%%%%%%   \item Termination metrics are used to prove termination, if provided
%%%%%%%%   \item The \textit{first} argument 
%%%%%%%%   	with sizing information should decrease.
%%%%%%%%   \end{itemize}
