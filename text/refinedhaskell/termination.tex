\section{Termination Analysis}\label{sec:termination}
\subsection{Theory}

We refer again to \undeclang 
and extend it with stratified types.
%
We can prove correctness of type-checking, in that 
typing judgements respect denotational typing:
%
\begin{theorem}[Denotation Typing]\label{thm:typing}
If $\hastype{}{e}{\tau}$ then $e \in \interp{\tau}$
\end{theorem} 
%
Stratification of types preserves Theorem \ref{thm:typing}, 
as all primitives in \undeclang preserves the stratum, 
\ie there is no way to construct a diverging computation.
%
Moreover the above theorem implies that 
expressions with finite (resp. WHNF) types 
are indeed finite (resp. reduce to WHNF).

\mypara{Recursive Functions}
We use the \efix{} operator to allow recursive definitions,
%
$$\ceval{\efix{\tau}}{f} \doteq f\ (\efix{\tau}\ f) $$
%
Systems without stratified types, like \hlang,
type \efix{\tau} as
$\constty{\efix{\tau}}\doteq{\fixtype{\tau}}$,
and can prove~\citep{PLC} that it belongs
into the interpretation of its type:
$\efix{\tau} \in \interp{\fixtype{\tau}}$.

In our system, the usual type for \efix{\tau}
cannot be proven correct, 
as the result expression of a \efix{\tau}
application may diverge.
%
For example, the diverging computation
$\efix{{(\tfunbasic{\tint^\ltrivial}{\tint^\ltrivial})}}\ 
	(\efun{f}{}{\efun{x}{}{f\ x}})\  4$
would type as $\tint^\ltrivial$
which is unsound.
%
Taking account of such cases we constraint 
the type parameter $\tau$ to return a unlabelled type,
\ie the result of a \efix{\tau} application \textit{may}
diverge:
%
$$\constty{\efix{\tau}} \doteq {\fixtype{\tau}}, 
	\text{if the result of}\ \tau\ \text{is Div type}$$
%
and prove that this type of \efix{\tau}
is sound, or $\efix{\tau} \in \interp{\constty{\efix{\tau}}}$.

\mypara{Finite Recursive Functions}
%
Not all recursive functions diverge
and we would like to allow provably terminating 
recursive functions return finite types.
%
Consider the fibonnaci function:
$$
\efib \doteq
\efun{n}{}{\efun{f}{}{
	\ecaseinstance{\_}{n}{
		 \begin{array}{l}
			 \ealt{0}{1}\\ 
			 \ealt{\_}{f \ (n-1) + f\ (n-2)}	
		\end{array}
	}
}}$$
We know that the recursive function $f$ 
is called with arguments less that the actual parameter $n$.
We encode this information to \efib 's type: 
\begin{align*}
\tnat &\doteq \tref{v}{\tint}{}{0 \leq v}\\
\efib &\colon\colon n\colon\tnat^\lfinite
	\rightarrow
		\tfunbasic{(\tfunbasic{\tref{n'}{\tnat}{\lfinite}{v < n}}{\tint^\lfinite})}
			{\tint^\lfinite}
\end{align*}
%
Note that in order to express the fact $f$ accepts arguments less that 
$n$ we need an order of arguments, so that $n$ is in scope on $f$'s type.
%
This modification precludes the natural fixpoint definition, 
instead we use \etfix{\tau} and some intermediate constants
that behave as follows:
%
\begin{align*}
\etfix{\tau}\ \efib\ n 
& \hookrightarrow
	\etfixf{\tau}{\efib}{??}\ n \\
& \hookrightarrow
    \efib\ n\ (\etfixn{\tau}{\efib}{n}\ \efib) \\
& \hookrightarrow
    (\etfixn{\tau}{f}{n}\ \efib \ (n-1))  + \dots \\
%    + (\etfixn{\tau}{f}{n}\ \efib \ (n-2)) \\
& \hookrightarrow
    (\etfixfn{\tau}{\efib}{n}\ (n-1)) + \dots \\
%    + (\etfixfn{\tau}{\efib}{n} \ (n-2)) \\
& \hookrightarrow
    (\efib\ (n-1)\ (\etfixn{\tau}{\efib}{n-1}\ \efib)) 
    +\dots \\
%     (\etfixfn{\tau}{\efib}{n} \ (n-2)) \\
& \hookrightarrow \dots
\end{align*}
%
Intuitively, 
the parameter $n$
means that $\mathtt{fix}^n$ can call itself 
at most $n$ times, 
while $f$ tracks the function that should 
be called when applied the function's 
first argument $n$.
%
Formally,  we define the following families of constants:
\begin{align*}
\ceval{\etfix{\tau}}{f} & \doteq \etfixf{\tau}{f}{??}\\ 
\ceval{\etfixf{\tau}{f}{??}}{n} &\doteq
f\ n\ (\etfixn{\tau}{f}{n}\ f) \\
\ceval{\etfixn{\tau}{f}{n}}{f} & \doteq
    \etfixfn{\tau}{f}{n}\\
\ceval{\etfixfn{\tau}{f}{n}}{m} &\doteq
f\ m\ (\etfixn{\tau}{f}{m}\ f) \\
\end{align*}
%
We define the appropriate types for the above constants
%
\begin{align*}
\decr{\tau}{n} & \doteq \decrtypefull{\tau}{n}\\
\constty{\etfix{\tau}} &\doteq (\decrty{\tau})\\
	& \rightarrow
	\tfun{m}{\tnat^\lfinite}{\tau\sub{x}{m}}\\ 
\constty{\etfixf{\tau}{f}{n}} &\doteq 
	 \etfixfty{\tau}{f}{n}\\ 
\constty{\etfixn{\tau}{f}{n}} &\doteq  
	(\decrty{\tau})\\&
	\rightarrow\adecrty{\tau}{n}\\ 
\constty{\etfixfn{\tau}{f}{n}} &\doteq 
	 \etfixfty{\tau}{f}{n}\\ 
\end{align*}
and prove that the constants belong to the 
meanings of their types:
%
\begin{theorem}\textbf{Fixpoint.}\label{thm:fixpoint}
\begin{itemize}
\item$\forall n. \etfixn{\tau}{f}{n} \in \constty{\etfixn{\tau}{f}{n}}$
\item$\etfix{\tau} \in \constty{\etfix{\tau}}$
\end{itemize}
\end{theorem}
\begin{proof}
\begin{itemize}
\item 1.
We prove that for all
%$f \in \interp{\decrty{\tau}}$
appropriate $f$
and $ m \in \interp{\tref{v}{\tnat}{\lfinite}{v < n}}$,
$e \equiv \etfixn{\tau}{f}{n}\ f \ m \in \interp{\tau\sub{x}{m}}$
% 
by induction on $n$.

For $n=0$, 
it is trivial, as 
there is no $m$ such that
$m \in \interp{\tref{v}{\tnat}{\lfinite}{v < 0}}$.

For the inductive step, $e$ reduces to 
$$
\etfixn{\tau}{f}{n}\ f\ m 
\hookrightarrow
\etfixfn{\tau}{f}{n}\ m 
\hookrightarrow
f\ m\ (\etfixn{\tau}{f}{m}\ f)\\
$$
By IH, since $m < n$,
$\etfixn{\tau}{f}{m} \in \constty{\etfixn{\tau}{f}{m}}$, 
so $f$ receives the appropriate arguments, 
and returns the appropriate result that proves the theorem.
%
\item 2.
We prove that 
for all appropriate $f$
% \\$f \in \interp{\decrty{\tau}}$
and     $ m \in \interp{\tnat^\lfinite}$,
$\etfix{\tau}\ f \ m \in \interp{\tau\sub{x}{m}}$.
%
Since $m \in \interp{\tref{v}{\tnat}{\lfinite}{v < m+1}}$
$$\etfixn{\tau}{f}{m+1}\ f \ m \in \interp{\tau\sub{x}{m}}$$
%
But operationally, 
$\etfixn{\tau}{f}{m+1}\ f \ m$
and
$\etfix{\tau}\ f \ m$
behave equivalently, which proves the theorem.
\end{itemize}
\end{proof}
%
Note that we used a similar proof for
$\efix{\tau} \in \constty{\efix{\tau}}$.
%
The only difference is that for the base case
\efixn{\tau}{0} should be defined to belong 
into the interpretation of any type.
%
Thus, it is defined as a diverging expression.
%
With refinement types we prove that the basic
\etfixn{\tau}{f}{0} operator
cannot be called, so we omit 
the definition of this basic case.

\mypara{Downgrade Labels}
%
With \etfix{\tau} we can construct recursive expressions
and prove them terminating.
%
But how do we prove that an expression is in WHNF?
%
Every terminating expression is in WHNF, 
thus, we define \edown{l_1}{l_2}, an id function that 
\textbf{downgrades} the stratum of from $l_1$ to $l_2$:
%
\begin{align*}
\ceval{\edown{l_1}{l_2}}{e} &\doteq e\\
\constty{\edown{l_1}{l_2}} & \equiv 
\tfunbasic{\tref{v}{b}{l_1}{p}}{\tref{v}{b}{l_2}{p}}
\end{align*}
We can show that $\edown{\lfinite}{\ltrivial} 
\in \interp{\constty{\edown{\lfinite}{\ltrivial}}}$
and use this constant in \undeclang to downgrade stata.
