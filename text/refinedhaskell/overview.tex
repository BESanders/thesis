\section{Overview}\label{sec:refinedhaskell:overview}

We start with an informal overview of a sound refinement type 
system for Haskell. 
%
After recapitulating the basics of refinement types
we illustrate why the classical approach based on 
verification conditions (VCs) is unsound due to 
lazy evaluation.  
%
Next, we step back to understand precisely how the 
VCs arise from refinement subtyping and how subtyping
is different under eager and lazy evaluation. 
In particular, we demonstrate that under lazy, but 
\emph{not} eager, evaluation, the refinement type 
system, and hence the VCs, must account for divergence.
%
Consequently, we develop a type system that accounts 
for divergence in a modular and syntactic fashion 
and illustrate its use via several small examples.
%
Finally, we show how a refinement-based termination
analysis is used to improve precision, yielding 
a highly effective SMT-based verifier for Haskell.

\subsection{Standard Refinement Types: From Subtyping to VC}\label{subsec:vc}


\begin{figure}[!t]
\centering
\captionsetup{justification=centering}
$$
\begin{array}{rrcl}
\emphbf{Refinements} & \refi     & ::= & \dots \text{varies} \dots \\[0.05in] 
\emphbf{Basic Types} & \base  & ::= & \tref{v}{\tint}{}{\refi} \spmid \dots \\[0.05in] 
\emphbf{Types}       & \typ   & ::= & \base \spmid \tfunref{\x}{\typ}{\typ}{v}{\refi} \\[0.05in] 
\emphbf{Environment} & \env   & ::= & \emptyset \spmid \bind{x}{\typ}, \env \\[0.1in]

\multicolumn{3}{l}{\emphbf{Subtyping}}   & \issubtype{\env}{\typ_1}{\typ_2} \\[0.1in]

\multicolumn{4}{l}{\mbox{\emphbf{Abbreviations}}} \\[0.05in]
  \multicolumn{2}{r}{\bind{x}{\ttref{\refi}}} & \doteq & \bind{x}{\tref{x}{\tint}{}{\refi}} \\[0.05in]
  \multicolumn{2}{r}{\ttreftsimple{\x}{\refi}}   & \doteq & {\tref{x}{\tint}{}{\refi}} \\[0.05in]
  \multicolumn{2}{r}{\ttref{\refi}}             & \doteq & {\tref{v}{\tint}{}{\refi}} \\[0.05in]
  \multicolumn{2}{r}{\reft{\x}{\reft{\y}{\tint}{\refi_\y}}{\refi_\x}} & \doteq & 
  					{\reft{\x}{\tint}{\refi_\x \land \refi_\y\sub{\y}{\x}}} \\[0.05in]


\multicolumn{4}{l}{\mbox{\emphbf{Translation}}} \\[0.05in]
\multicolumn{2}{r}{\embed{\issubtype{\env}{\base_1}{\base_2}}} & \defeq& \isvalidvc{\embed{\env}}{\embed{\base_1}}{\embed{\base_2}}\\[0.05in]
  \multicolumn{2}{r}{\embed{\tttref{\x}{\tint}{}{\refi}}} & \defeq & \lref\\[0.05in]
  \multicolumn{2}{r}{\embed{\ttbind{\x}{\tttref{\vv}{\tint}{}{\refi}}}} &\defeq & \ \mbox{``$\mathtt{\x}$ is a value"}  \Rightarrow \SUBST{\lref}{\vv}{\x}\\[0.05in]
  \multicolumn{2}{r}{\embed{\bind{\x}{(\tfun{\y}{\typ_\y}{\typ})}}} &\defeq & \tttrue\\[0.05in]
  \multicolumn{2}{r}{\embed{\ttbind{\x_1}{\typ_1},\ldots,\ttbind{\x_\n}{\typ_\n}}} & \defeq & \embed{\ttbind{\x_1}{\typ_1}} \wedge \ldots \wedge \embed{\ttbind{\x_\n}{\typ_\n}} \\
\end{array}
$$
\caption[Summary of Informal Notation.]{Informal Notation: Types, Subtyping, and VCs}
\label{fig:overview:syntax}
\end{figure}


First, let us see how standard refinement type systems 
\cite{LiquidPLDI08, Knowles10} will 
use
the aforementioned refinement type aliases @Pos@ and @Nat@ 
and the specification for @div@
%
to \emph{accept} @good@ and
\emph{reject} @bad@.
We use the syntax of Figure~\ref{fig:overview:syntax}, where $r$ is a
\emph{refinement expression}, or just \emph{refinement} for short.
We will vary the expressiveness of the language of refinements in
different parts of this document.
%
\begin{code}
    good     :: Nat -> Nat -> Int
    good x y = let z = y + 1 in x `div` z

    bad      :: Nat -> Nat -> Int
    bad x y  = x `div` y
\end{code}

\spara{Refinement Subtyping} 
To analyze the body of @bad@, the refinement type system will 
check that the second parameter @y@ has type @Pos@ at the call 
to @div@; formally, that %. Formally, the system will check that the type of
the actual parameter @y@ is a \emph{subtype} of the type of 
@div@'s second input, via a subtyping query:
%
$$
 \ttbind{x}{\tref{x}{Int}{}{x \geq 0}}, 
 \ttbind{y}{\tref{y}{Int}{}{y \geq 0}}
 \vdash
 \subt{\tttref{y}{Int}{}{y \geq 0}}{\tttref{v}{Int}{}{v > 0}}$$
%
We use the Abbreviations of Figure~\ref{fig:overview:syntax} to simplify the 
syntax of the queries.
%
So the above query simplifies to:
$$\ttbind{x}{\ttref{x \geq 0}},\ \ttbind{y}{\ttref{y \geq 0}}\ \vdash\ \subtref{v \geq 0}{v > 0}$$

\spara{Verification Conditions}
To discharge the above subtyping query, a refinement type system
generates a \emph{verification condition} (VC), a logical formula 
that stipulates that under the assumptions corresponding to the 
environment bindings, the refinement in the sub-type \emph{implies} 
the refinement in the super-type.
%
We use the translation \embed{\cdot} shown in Figure~\ref{fig:overview:syntax}
to reduce a subtyping query to a verification condition.
%
The translation of a basic type into logic is the refinement of the type.
The translation of an environment is the conjunction of its bindings.
Finally, the translation of a binding \ttbind{\x}{\typ} is the embedding of \typ 
guarded by a predicate denoting that ``\x is a value''.
%
For now, let us ignore this guard and see how the subtyping query for 
@bad@ reduces to the \emph{classical} VC:
%%
$$\mathvc{\mathvc{(\ttx \geq 0)}} \wedge \mathvc{(\tty \geq 0)}\ \Rightarrow\ \mathvc{(\ttv \geq 0)} \Rightarrow \mathvc{(\ttv > 0)}$$
%%
Refinement type systems are carefully engineered (\Sref{sec:typing}) 
so that (\emph{unlike} with full dependent types) the logic of 
refinements \emph{precludes} arbitrary functions and only includes 
formulas from efficiently decidable logics, \eg the quantifier-free 
logic of linear arithmetic and uninterpreted functions (\logiclang).
Thus, VCs like the above can be efficiently validated by SMT 
solvers \cite{z3}. 
%
In this case, the solver will reject the above VC as \emph{invalid} meaning the 
implication, and hence, the relevant subtyping requirement does not hold.
So the refinement type system will \emph{reject} @bad@.

On the other hand, a refinement system \emph{accepts} @good@.
Here, @+@'s type exactly captures its behaviour into the logic:
\begin{code}
  (+) :: x:Int -> y:Int -> {v:Int | v = x + y} 
\end{code}
Thus, we can conclude that the divisor @z@ is a positive number. 
The subtyping query for the argument to @div@ is
%
\begin{align*}
\ttbind{x}{\ttref{x \geq 0}},
\ttbind{y}{\ttref{y \geq 0}}, 
\ttbind{z}{\ttref{z = y + 1}} 
\vdash\ & \subtref{v = y + 1}{v > 0}
%\label{sub:good}
\intertext{which reduces to the \emph{valid} VC}
	\mathvc{(\ttx \geq 0)} \wedge \mathvc{(\tty \geq 0)}\wedge 
	\mathvc{(\ttz = \tty+1)} 
   \Rightarrow\ & \mathvc{(\ttv = \tty+1)} 
   \Rightarrow \mathvc{(\ttv > 0)} \nonumber
\end{align*}

\subsection{Lazy Evaluation Makes VCs Unsound} \label{sec:overview:unsound}

To generate the classical VC, we ignored the ``\x is a value'' guard 
that appears in the embedding of a binding \embed{\ttbind{\x}{\typ}} (Figure~\ref{fig:overview:syntax}). 
%
Under lazy evaluation, ignoring this ``is a value'' guard can lead to unsoundness.
%
Consider 
%
\begin{code}
   diverge   :: Int -> {v:Int | false}
   diverge n = diverge n
\end{code}
%
The output type captures the \emph{post-condition} 
that the function returns an @Int@ satisfying @false@. 
This counter-intuitive specification states, in essence, 
that the function \emph{does not terminate}, \ie does not
return \emph{any} value. 
%
Any standard refinement type checker (or Floyd-Hoare
verifier like Dafny~\cite{dafny}) 
will verify the given signature for @diverge@ 
via the classical method of inductively \emph{assuming} 
the signature holds for @diverge@ and then 
\emph{guaranteeing} the signature~\cite{Hoare71,Nipkow02}.
%
Next, consider the call to @div@ in @explode@:
%
\begin{code}
   explode   :: Int -> Int
   explode x = let {n = diverge 1; y = 0}
               in  x `div` y
\end{code}
%
To analyze @explode@, the refinement type system will check 
that @y@ has type @Pos@ at the call to @div@, \ie will check 
that 
%
\begin{align}
   \ttbind{\ttn}{\ttref{\ttfalse}},\ \ttbind{\tty}{\ttref{\tty = 0}}\ \vdash\ & \subtref{\ttv = 0}{\ttv > 0} 
   \label{sub:explode}
\intertext{In the environment, \texttt{n} is 
  bound to the type corresponding to the \emph{output} 
  type of \texttt{diverge} and \texttt{y} is bound to 
  the type stating \texttt{y} equals \texttt{0}.
  In this environment, we must prove that actual parameter's 
  type -- \ie that of \texttt{y} -- is a subtype of \texttt{Pos}. 
  The subtyping, using the embedding of Figure~\ref{fig:overview:syntax} 
  and ignoring the ``is a value'' guard, 
  reduces to the VC:}
    \mathvc{\ttfalse} \wedge \mathvc{\tty = 0}\ \Rightarrow\ & \mathvc{(\ttv = 0)} \Rightarrow \mathvc{(\ttv > 0)}
    \label{vc:explode}
\end{align}
%
The SMT solver proves this VC valid using the 
contradiction in the antecedent, thereby unsoundly 
proving the call to @div@ safe!

\mypara{Eager vs. Lazy Verification Conditions}
%
We pause to emphasize that the problem lies 
in the fact that the classical technique for encoding subtyping 
(or generally, Hoare's ``rule of consequence" \cite{Hoare71}) 
with VCs is \emph{unsound under lazy evaluation}.
%
To see this, observe that the VC~(\ref{vc:explode}) is perfectly 
\emph{sound} under eager (strict, call-by-value) evaluation.
%
In the eager setting, the program is safe in that
@div@ is never called with the divisor @0@, as it 
is not called at all!
%
The inconsistent antecedent in the VC logically encodes the 
fact that, under eager evaluation, the call to @div@ is 
\emph{dead code}.
%
Of course, this conclusion is spurious under Haskell's lazy 
semantics. As @n@ is not required, the program will dive 
headlong into evaluating the @div@ and hence crash, 
rendering the VC meaningless.


\mypara{The Problem is Laziness} Readers familar 
with fully dependently typed languages like 
Cayenne~\cite{cayenne}, Agda~\cite{norell07}, 
Coq~\cite{coq-book}, or Idris~\cite{Brady13}, may be
tempted to attribute the unsoundness to the presence 
of arbitrary recursion and hence non-termination 
(\eg @diverge@).
While it \emph{is} possible to define a sound semantics 
for dependent types that mention potentially non-terminating
expressions~\citep{Knowles10}, it is not clear how to reconcile
such semantics with decidable type checking. 

Refinement type systems avoid this situation by carefully restricting 
types so that they do not contain arbitrary terms (even through 
substitution), but rather only terms from restricted logics that
preclude arbitrary user-defined functions~\cite{pfenningxi98,Dunfield07,fstar}.
Very much like previous work, for now, we enforce the same restriction
with a \emph{well-formedness condition} on 
refinements (\rwbased in Fig.~\ref{fig:declang:typing}).
%
In Chapter~\ref{refinementrflection} we present how 
our logic is extended with provably terminating arbitrary terms, 
while preserving both soundness and decidability.

However, we show that restricting the logic of refinements
\emph{is plainly not sufficient for soundness} 
when laziness is combined with non-termination, 
as binders can be bound to diverging expressions. 
Unsurprisingly, in a strongly
normalizing language the question of lazy or strict semantics is irrelevant for soundness, and hence 
an ``easy'' way to solve the problem would be to completely eliminate non-termination and rely on the soundness
of previous refinement or dependent type systems! Instead, we show here how to 
recover soundness for a lazy language \emph{without} imposing such a drastic requirement. 


\subsection{Semantics, Subtyping \& Verification Conditions} \label{sec:den-sem}

To understand the problem, let us take a step 
back to get a clear view of the relationship 
between the operational semantics, subtyping,
and verification conditions.
%
We use the formulation of evaluation-order 
independent refinement subtyping developed 
for \hlang~\cite{Knowles10} in which 
refinements $r$ are \emph{arbitrary} expressions $e$ from the source language.
We define a denotation for types and use it 
to define subtyping declaratively.

\mypara{Denotations of Types and Environments}
%
Recall the type @Pos@ defined as {@{v:Int | 0 < v}@}.
Intuitively, @Pos@ denotes the \emph{set of} @Int@ 
expressions which evaluate to values greater than @0@.
%
We formalize this intuition by defining the denotation
of a type as:
$$\interp{\tttref{\x}{\typ}{}{\refi}} \defeq \{e \mid \hastype{\emptyset}{e}{\typ}, \mbox{ if } \goesto{e}{\val} \mbox{ then } \goesto{\SUBST{\refi}{\x}{\val}}{\etrue} \}$$
%
That is, the type denotes the set of expressions \mie that have
the corresponding base type \typ which \emph{diverge or} reduce 
to values that make the refinement reduce to $\etrue$. 
%
The guard \goesto{\mie}{\val} is crucially required to prove soundness in the presence of recursion.
Thus, quoting~\cite{Knowles10}, ``refinement types specify partial and not total correctness''. 

An \emph{environment} $\env$ is a sequence of type bindings
and a \emph{closing substitution} \sto\xspace is a sequence of expression bindings:
$$\env \defeq \ttbind{\x_1}{\typ_1},\ldots\ttbind{\x_\n}{\typ_\n} \qquad \qquad
  \sto \defeq \tgbind{\x_1}{\mie_1}, \ldots, \tgbind{\x_\n}{\mie_\n}$$ %%
%% 
Thus, we define the denotation of $\env$ as the set of substitutions:
%%
$$\interp{\env} \defeq \{\sto \mid \forall \ttbind{\x}{\typ} \in \env. \sto(\x) \in \interp{\thetasub{\sto}{\typ}} \}$$

\mypara{Declarative Subtyping}
Equipped with interpretations for types and environments, 
we define the \emph{declarative subtyping} \rtdsub 
(over basic types $\base$, shown in Figure~\ref{fig:overview:syntax}) 
to be containment between the types' denotations:
$$
\inference
  {\forall \sto\in \interp{\env}.\  
  		 \interp{\thetasub{\sto}{\tttref{\vv}{\Base}{}{\refi_1}}} 
  		\subseteq   \interp{\thetasub{\sto}{\tttref{\vv}{\Base}{}{\refi_2}}}}
  {\issubtype{\env}{\tttref{\vv}{\Base}{}{\refi_1}}{\tttref{\vv}{\Base}{}{\refi_2}}}
  \rtdsub
$$
%
Let us revisit the @explode@ example from \Sref{sec:overview:unsound}; 
recall that the function is safe under eager evaluation but unsafe under 
lazy evaluation. Let us see how the declarative subtyping allows us to 
reject in the one case and accept in the other.

\mypara{Declarative Subtyping with Lazy Evaluation}
Let us revisit the query (\ref{sub:explode}) to see whether it
holds under the declarative subtyping rule \rtdsub. The denotation
containment
%
\begin{align}
\label{con:explode}
   \forall \sto \in &\interp{
   		\ttbind{n}{\ttref{\ttfalse}},\ \ttbind{y}{\ttref{y = 0}}
   		}. 
     \interp{{{\sto}\ {\ttref{v = 0}}}} \subseteq \interp{{\sto}\ {\ttref{v > 0}}} 
\end{align}
%
\emph{does not} hold. To see why, consider a $\sto$ that maps 
$n$ to any diverging expression of type $\ttInt$ and $y$ 
to the value $0$.
%
Then, $0 \in \interp{\sto\ \ttref{v = 0}}$ but $0 \not \in \interp{\sto\ \ttref{v > 0}}$, 
thereby showing that the denotation containment does not hold.


\mypara{Declarative Subtyping with Eager Evaluation}
Since denotational containment (\ref{con:explode}) does not hold,
\hlang cannot verify @explode@ under eager evaluation.
%
However, Belo \etal~\cite{Greenberg11} note that under eager (call-by-value) 
evaluation, each binder in the environment is only added \emph{after} 
the previous binders have been reduced to \emph{values}. 
%
Hence, under eager evaluation we can \emph{restrict the range} of 
the closing substitutions to values (as opposed to expressions).
%
Let us reconsider (\ref{con:explode}) in this new light: 
%
there \emph{is no value} that we can map \ttn to, so the set of
denotations of the environment is empty. Hence, the 
containment~(\ref{con:explode}) vacuously holds under
eager evaluation, which proves the program safe.
%
Belo's observation 
is implicitly used by refinement types for eager languages
to prove that the standard (\ie call-by-value)
reduction from subtyping to VC is sound.

\mypara{Algorithmic Subtyping via Verification Conditions}
The above subtyping (\rtdsub) 
rule allows us to prove preservation and progress~\cite{Knowles10} 
but quantifies over evaluation of arbitrary expressions, 
and so is undecidable.
%
To make checking \emph{algorithmic} we approximate the 
denotational containment using \emph{verification conditions} (VCs), 
formulas drawn from a decidable logic, that are valid 
only if the undecidable containment holds.
%
As we have seen, the classical VC is sound only under 
eager evaluation. Next, let us use the distinctions 
between lazy and eager declarative subtyping, 
to obtain both sound and decidable VCs for the lazy setting. 

\mypara{Step 1: Restricting Refinements To Decidable Logics}
%
Given that \hlang refinements can be \emph{arbitrary} expressions,
the first step towards obtaining a VC, regardless of evaluation order,
is to restrict the refinements to a \emph{decidable} logic. 
%
We choose the quantifier free logic of equality, uninterpreted functions 
and linear arithmetic (\logiclang). 
%
Our typing rules ensure that for any valid derivation, all
refinements belong in this restricted language.

\mypara{Step 2: Translating Containment into VCs}
Our goal is to encode the denotation containment antecedent of \rtdsub
%
\begin{align}\label{denotcont}
\forall \sto\in \interp{\env}.\  
  		 \interp{\thetasub{\sto}{\tttref{\vv}{\Base}{}{\refi_1}}} 
  		\subseteq   \interp{\thetasub{\sto}{\tttref{\vv}{\Base}{}{\refi_2}}}
\end{align}
as a logical formula, that is valid \emph{only when} the above holds.
Intuitively, we can think of the closing substitutions $\sto$ as 
corresponding to \emph{assignments} or \emph{interpretations} 
$\embed{\sto}$ 
of variables $X$ of the VC.
%
We use the variable \x to approximate denotational containment 
by stating that  
if \x belongs to the type $\tttref{\vv}{\Base}{}{\refi_1}$
then \x belongs to the type $\tttref{\vv}{\Base}{}{\refi_2}$:
$$
\forall X \in \mathit{dom}(\Gamma), \x. 
    \embed{\Env}\Rightarrow 
  		 {\embed{\ttbind{\x}{\tttref{\vv}{\Base}{}{\refi_1}}}} 
  		\Rightarrow \embed{\ttbind{\x}{\tttref{\vv}{\Base}{}{\refi_2}}}
$$
where $\embed{\env}$ and $\embed{\ttbind{\x}{\typ}}$ are respectively the translation 
of the environment and bindings into logical formulas 
that are only satisfied by assignments $\embed{\sto}$
as shown in Figure~\ref{fig:overview:syntax}.
%
Using the translation of bindings, 
and by renaming \x to \vv,
we rewrite the condition as
$$
\forall X \in \mathit{dom}(\Gamma), \vv. 
    \embed{\env}
    \Rightarrow
     ({\text{``}\vv\ \text{is a value"}}  \Rightarrow \lref_1)
    \Rightarrow
     ({\text{``}\vv\ \text{is a value"}}  \Rightarrow \lref_2)
$$
Type refinements are carefully chosen to belong to the 
decidable logical sublanguage \logiclang, so we directly translate
type refinements into the logic.
%
Thus, what is left is to translate into logic the environment and the ``is a value'' guards.
%
We postpone translation of the guards as we approximate the above formula by 
a \emph{stronger}, \ie sound with respect to \ref{denotcont},
VC that just omits the guards:
$$
\forall X \in \mathit{dom}(\Gamma), \vv. 
    \embed{\env}
    \Rightarrow
      \lref_1 \Rightarrow \lref_2
$$
%
To translate environments, we conjoin their bindings' translations:
\begin{align*}
%
\embed{\ttbind{\x_1}{\typ_1},\ldots,\ttbind{\x_\n}{\typ_\n}} \defeq & 
\embed{\ttbind{\x_1}{\typ_1}} \wedge \ldots \wedge \embed{\ttbind{\x_\n}{\typ_\n}} \\
%
\intertext{However, since types denote \emph{partial correctness},
   the translations must also explicitly account for possible divergence:}
\embed{\ttbind{\x}{\tlref{\vv}{\tint}{}{\refi}}} \defeq & \ \mbox{``$\x$ is a value"}  \Rightarrow \SUBST{\lref}{\vv}{\x}
\end{align*}
%
That is, we \emph{cannot} assume that each \x satisfies its 
refinement \refi; we must \emph{guard} that assumption with a 
predicate stating that \x is bound to a value (not a diverging term).

The crucial question is: \emph{how} can one discharge these guards
to conclude that \x indeed satisfies \refi?
%
One natural route is to enrich the refinement logic with 
a predicate that states that ``\x is a value", and then 
use the SMT solver to \emph{explicitly} reason about 
this predicate and hence, divergence.
%
Unfortunately, we show in \Sref{sec:conclusion}, that 
such predicates lead to three-valued logics,
which fall outside the scope of the efficiently 
decidable theories supported by current solvers.
%
Hence, this route is problematic if we want to use 
existing SMT machinery to build automated verifiers 
for Haskell.

\subsection{Our Answer: Implicit Reasoning About Divergence}

One way forward is to \emph{implicitly} reason about divergence 
by \emph{eliminating} the ``\x is a value" guards (\ie \emph{value guards}) from the VCs.

\mypara{Implicit Reasoning: Eager Evaluation}
Under eager evaluation the domain of the closing substitutions 
can be restricted to values~\cite{Greenberg11}.
%
Thus, we can trivially eliminate the value guards, 
as they are guaranteed to hold by virtue of the 
evaluation order.
%
Returning to @explode@, we see that after eliminating 
the value guards, we get the VC (\ref{vc:explode})
which is, therefore, sound under eager evaluation.

\mypara{Implicit Reasoning: Lazy Evaluation}
However, with lazy evaluation, we cannot just 
eliminate the value guards, as the closing 
substitutions are not restricted to just values.
%
Our solution is to take this reasoning out of the 
hands of the SMT logic and place it in the hands of 
a \emph{stratified type system}.
%
We use a non-deterministic $\beta$-reduction 
(formally defined in \Sref{sec:language})
to label each type as:
%
A \Div-type, written $\DivTy{\typ}$,
which are the default types given to binders 
that \emph{may diverge}, or,
%
a \Wnf-type, written $\WnfTy{\typ}$, 
which are given to binders that are guaranteed to 
reduce, in a finite number of steps, to \emph{Haskell values}
in Weak Head Normal Form (WHNF).
%
Up to now we only discussed \tint basic types, but 
our theory supports user-defined algebraic data types.
%
An expression like @0 : repeat 0@ is an infinite Haskell value.
%
As we shall discuss, such infinite values cannot be represented in the logic.
%
To distinguish infinite from finite values, we use a \Fin-type, written $\FinTy{\typ}$,
to label binders of expressions that are guaranteed to reduce to \emph{finite values} with no redexes.
%%%
This stratification lets us generate VCs that are sound 
for lazy evaluation. 
%
Let \Base be a basic labelled type. 
%
The key piece is the translation of environment bindings:
$$
\embed{\ttbind{\x}{\tttref{\vv}{\Base}{}{\refi}}} \defeq 
  \begin{cases}
    \tttrue,         & \mbox{if $\Base$ is a \Div type} \\ 
    \SUBST{\lref}{\vv}{\x}, & \mbox{otherwise}
  \end{cases}
$$
That is, if the binder may diverge, we simply \emph{omit} 
any constraints for it in the VC, and otherwise the translation 
directly states (\ie without the value guard) that the 
refinement holds.
%
Returning to @explode@, the subtyping query (\ref{sub:explode}) 
yields the \emph{invalid} VC
$$
\tttrue \Rightarrow \ttv = 0 \Rightarrow \ttv > 0
$$
and so @explode@ is soundly rejected under lazy evaluation.

As binders appear in refinements and binders may refer 
to potentially infinite computations (\eg @[0..]@), we must 
ensure that refinements are well defined (\ie do not diverge).
%
We achieve this via stratification itself, \ie by ensuring that 
all refinements have type $\FinTy{\tbool}$. By
Corollary~\ref{cor:stratification}, this suffices to ensure that 
all the refinements are indeed well-defined and converge.



\subsection{Verification With Stratified Types}\label{sec:overview:examples}
While it is reassuring that the lazy VC soundly \emph{rejects} 
unsafe programs like @explode@, we demonstrate by example
that it usefully \emph{accepts} safe programs.
% 
First, we show how the basic system -- all terms
have \Div types -- allows us to prove ``partial correctness''
properties without requiring termination.
%
Second, we extend the basic system by 
using Haskell's pattern matching semantics to assign 
the pattern match scrutinees \Wnf types, thereby 
increasing the expressiveness of the verifier.
%
Third, we further improve the precision 
and usability of the system by using a termination 
checker to assign various terms \Fin types.
%
Fourth, we close the loop, by illustrating how the 
termination checker can itself be realized using 
refinement types.
%
Finally, we use the termination checker to ensure that
all refinements are well-defined (\ie do converge).

\mypara{Example: VCs and Partial Correctness}
The first example illustrates how, unlike Curry-Howard based
systems, refinement types \emph{do not require} termination. 
%
That is, we retain the Floyd-Hoare notion of ``partial correctness''
and can verify programs where \emph{all} terms have \Div-types.
%
Consider @ex1@ which uses the result of @collatz@ as a divisor.
%
\begin{code}
  ex1   :: Int -> Int 
  ex1 n = let x = collatz n in 10 `div` x 

  collatz :: Int -> {v:Int | v = 1}
  collatz n 
    | n == 1    = 1 
    | even n    = collatz (n / 2)
    | otherwise = collatz (3*n + 1)
\end{code}
%
The jury is still out on \emph{whether} the @collatz@ 
function terminates~\cite{collatzWiki}, but it is easy
to verify that its output is a \Div @Int@ equal to @1@.
%
At the call to @div@ the parameter @x@ has the output type 
of @collatz@, yielding the subtyping query:
%
\begin{align*}
   \tbind{\ttx}{\tref{\ttv}{\ttInt}{}{\ttv=1}} \vdash\ & \subtref{v = 1}{v > 0} 
\end{align*}
% 
where the sub-type is just the type of $\ttx$. 
As $\ttInt$ is a \Div type, the above reduces to the VC 
${(\etrue \Rightarrow\  \ttv = 1 \Rightarrow\ \ttv > 0)}$
which the SMT solver proves valid, verifying @ex1@.

\mypara{Example: Improving Precision By Forcing Evaluation}
\label{sec:overview:pattern-match}
If all binders in the environment have \Div-types then, effectively, 
the verifier can make \emph{no} assumptions about the context in 
which a term evaluates, which leads to a drastic loss of precision. 
Consider:
\begin{code}
  ex2 = let {x = 1; y = inc x} in 10 `div` y

  inc :: z:Int -> {v:Int | v > z }
  inc = \z -> z + 1
\end{code}
%
The call to @div@ in @ex2@ is obviously safe, but the system 
would reject it, as the call yields the subtyping query:
%
$$
   \tbind{\ttx}{\tref{\ttx}{\ttInt}{}{\ttx = 1}},\
   \tbind{\tty}{\tref{\tty}{\ttInt}{}{\tty > \ttx}}\ 
   \vdash\ \subtref{v > \ttx}{v > 0} 
$$
Which, as $\ttx$ is a \Div type, reduces to the invalid VC:
$$
   \tttrue \Rightarrow\ \ttv > \ttx \Rightarrow \ttv > 0
$$
%
We could solve the problem by forcing evaluation of @x@.
%
In Haskell the @seq@ operator or a bang-pattern can be used to force evaluation.
%
In our system the same effect is achieved by the @case-of@ primitive:
inside each case the matched 
binder is guaranteed to be a Haskell value in WHNF.  
%
This intuition is formalized by the typing rule (\rtcased), which checks each 
case after assuming the scrutinee and the match binder have \Wnf types. 
%

If we force @x@'s evaluation, using the case primitive, 
the call to @div@ yields the subtyping query:
\begin{align}
   \ttbind{\ttx}{\ttreft{\ttx}{\WnfTy{\ttInt}}{\ttx = 1}},\
   \ttbind{\tty}{\ttreft{\tty}{\ttInt}{\tty > \ttx}}      	
	& \vdash\ \subtref{v > \ttx}{v > 0}\label{sub:pm:ex3:good}
\intertext{As $\ttx$ is \Wnf, we accept \texttt{ex2} by proving the validity of the VC:} 
   \ttx = 1 \Rightarrow\  \ttv > \ttx & \Rightarrow \ttv > 0
\label{vc:pm:ex3:good}
\end{align}

\mypara{Example: Improving Precision By Termination}
While forcing evaluation allows us to ensure that certain 
environment binders have non-\Div types, 
it requires program rewriting using case-splitting 
or the @seq@ operator which leads to non-idiomatic code.

Instead, our next key optimization is based on 
the observation that in practice, \emph{most terms don't diverge}.
%
Thus, we can use a termination analysis to 
aggressively assign terminating expressions 
\Fin types, which lets us strengthen the 
environment assumptions needed to prove 
the VCs.
%
For example, in the @ex2@ example the term @1@ obviously terminates.
Hence, we type @x@ as $\FinTy{\ttInt}$, yielding
the subtyping query for @div@ application:
%
\begin{align}
   \ttbind{\ttx}{\ttreft{\ttx}{\FinTy{\ttInt}}{\ttx = 1}},\
   \ttbind{\tty}{\ttreft{\tty}{\ttInt}{\tty > \ttx}}      	
	& \vdash\ \subtref{v > \ttx}{v > 0}\label{sub:ex3:good}
\intertext{As $\ttx$ is \Fin, we accept \texttt{ex2} by proving the validity of the VC:} 
   \ttx = 1 \Rightarrow\ \ttv > \ttx & \Rightarrow \ttv > 0
\label{vc:ex3:good}
\end{align}

\mypara{Example: Verifying Termination With Refinements}
%
While it is straightforward to conclude that the term @1@ 
does not diverge, how do we do so in general?
%
For example:
%
\begin{code}
  ex4 = let {x = f 9; y = inc x} in 10 `div` y
  
  f   :: Nat -> {v:Int | v = 1}
  f n = if n == 0 then 1 else f (n-1)
\end{code}
%
We check the call to @div@ via subtyping query (\ref{sub:ex3:good}) and 
VC (\ref{vc:ex3:good}), which requires us to prove that @f@ terminates on
\emph{all} $\FinTy{\mathtt{Nat}}$ inputs.

We solve this problem by showing how
refinement types may themselves be 
used to prove termination, by following
the classical recipe of proving termination 
via decreasing metrics~\cite{Turing36}
as embodied in sized types~\cite{HughesParetoSabry96,XiTerminationLICS01}.
%
The key idea is to show that each 
recursive call is made with arguments 
of a \emph{strictly smaller} size, 
where the size is itself a well 
founded metric, \eg a natural number.

We formalize this intuition by type checking
recursive procedures in a {termination-weakened environment}
where the procedure itself may only be called with arguments 
that are strictly smaller than the current parameter 
(using terminating fixpoints of \Sref{sec:typing:stratify}).
%
For example, to prove @f@ terminates, we check its body in an environment 
%
$${\ttn} : {\FinTy{\ttNat}}, \qquad {\ttf} : {\ttreft{\ttnp}{\FinTy{\ttNat}}{\ttnp < \ttn} \rightarrow \ttref{v=1}}$$
%
where we have weakened the type of $\ttf$ to stipulate that it 
\emph{only} be (recursively) called with $\ttNat$ values $\ttnp$ that are 
\emph{strictly less than} the (current) parameter $\ttn$. 
The argument of \ttf exactly captures these constraints, 
as using the Abbreviations of Figure~\ref{fig:overview:syntax}
the argument of \ttf is expanded to 
$\ttreft{\ttnp}{\FinTy{\tint}}{\ttnp < \ttn \land \ttnp >= 0}$.
The body
type-checks as the recursive call generates the valid VC:
%
$$0 \leq \ttn \wedge \lnot (0 = \ttn)  \Rightarrow \ttv = \ttn - 1 \Rightarrow (0 \leq \ttv < \ttn)$$

\mypara{Example: Diverging Refinements}
Finally, we discuss why refinements should always converge
and how we statically ensure convergence.
%
Consider the invalid specification
\begin{code}
  diverge 0 :: {v:Int | v = 12}
\end{code}
that states that the value of a diverging integer is @12@.
%
The above specification should be rejected, 
as the refinement $\mathtt{v = 12}$ does not evaluate to \etrue 
(\goesto{\mathtt{diverge\ 0 = 12} \not}{\etrue}),
instead it diverges.

We want to check the validity of the formula $\mathtt{v = 12}$
under a model that maps \ttv to the diverging integer $\mathtt{diverge \ 0}$.
%
Any system that decides this formula to be true
will be unsound, \ie the VCs will not soundly approximate subtyping.
%
For similar reasons, the system should not decide that this formula is false.
%
To reason about diverging refinements one needs three valued logic, 
where logical formulas can be solved to true, false, or diverging.
%
Since we want to discharge VC using SMT solvers that currently do not support 
three valued reasoning, we exclude diverging refinements from types.
%
To do so, we restrict $\mathtt{=}$ to finite integers
$$ \mathtt{=} :: {\tint^{\finite}}\rightarrow{\tint^\finite}\rightarrow{\tbool^\finite}$$
and we say that \ttreft{\vv}{\Base}{\refi} 
is well-formed \emph{iff} \refi has a $\tbool^\finite$ type (Corollary~\ref{cor:stratification}).
%
Thus the initial invalid specification will be rejected as non well-formed.
%%\DV{This paragraph is confusing. I thought
%%$len :: [a]^\Downarrow -> Int^\Downarrow$ no? But then we are not able to apply len to anything else? Is len (repeat 35) ill typed? This does not look good. 
%%Or is this typing only affecting refinements and not actual programs?}
%%\NV{
%%It can be either 
%%$len :: [a]^\Downarrow -> Int^\Downarrow$
%%or 
%%$len :: [a]^\Downarrow -> Int$.
%%Either ways len (repeat 35) should be ill typed, when len is required to terminate, 
%%for example when the result of len is used inside the refinements.
%%Though, in actual programs we can have diverging expressions, so 
%%length (repeat 35) can be well-typed, with 
%%$length :: [a] -> Int$.
%%}

%
%%% In \Sref{sec:typing} we formalize our system for proving termination 
%%% with refinements, and then show, in \S~\ref{sec:haskell}, how to generalize
%%% it to handle complex data types.
%%% The proof of the pudding is in \S~\ref{sec:evaluation} where we demonstrate 
%%% empirically the effectiveness of \toolname's approach of proving termination
%%% \emph{for} and \emph{by} refinements, on a large corpus of 10KLOC of widely 
%%% used, complex, real-world Haskell libraries.


%% JUNK
%%         take 0 _  = []
%%         take n xs = case xs of
%%                       (x:xs') -> x : take (n-1) xs'
%%                       []      -> liquidError 
%%
%%
%%
%%         findPos (x:xs) 
%%           | x > 0       = Just x 
%%           | otherwise   = find p xs
%%         findPos []      = Nothing
%% 
%%         ex0 ys = 24 `div` (findPos ys)
%% 
%%         x:Nat -> {y | x < y} -> Int
%%         ex1 = 24 `div` y

%% \subsection{An Optimization: Termination}
%%   x:Nat -> {y | x < y} -> Int
%%   ex1 = 24 `div` y
%%
%% fail      :: n:Nat -> {m:Nat | n < m} -> Int
%% fail n m  = 10000 `div` m
%% 
%% fail'     :: n:Nat -> {m:Nat | n < m} -> Int
%% fail' n m = case n of _ -> 10000 `div` m
%% 
%% bar     :: Int -> Int -> Int
%% bar     = foo x y 
%%   where 
%%     x   = fac 10
%%     y   = x + 1


%%% THE FOLLOWING FAILS IN APROVE BUT COMPLETES AUTOMATICALLY IN LH
%%% 
%%% foo   :: Int -> Int
%%% foo 0 = 0
%%% foo n = 0 + foo (n-1)
%%% 
%%% top   :: Int -> Int
%%% top n = if n < 0 then 1 else foo (foo n)


%%% Local Variables: 
%%% mode: latex
%%% TeX-master: "main"
%%% End: 
