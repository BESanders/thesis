\section{Related Work}\label{sec:related}

Next we situate our work with closely related lines of research.

\spara{Dependent Types} are the basis of many verifiers, 
or more generally, proof assistants.
%
In this setting arbitrary terms may appear inside types,
so to prevent logical inconsistencies, and enable
the checking of type equivalence, all terms must
terminate.
%
``Full'' dependently typed systems like Coq~\cite{coq-book}, 
Agda~\cite{norell07}, and Idris~\cite{Brady13} typically use 
\emph{structural} checks where recursion is allowed on 
sub-terms of ADTs to ensure that \emph{all} terms terminate.
%
We differ in that, since the refinement logic is
restricted, we do not require that all functions terminate,
and hence, we can prove properties of possibly diverging 
functions like @collatz@ as well as lazy functions like @repeat@.
%
Recent languages like Aura~\citep{AURA} and Zombie~\citep{Zombie}
allow general recursion, but constrain the logic to a terminating 
sublanguage, as we do, to avoid reasoning 
about divergence in the logic.
%
In contrast to us, the above systems crucially assume 
\emph{call-by-value} semantics to ensure that binders are bound
to values, \ie cannot diverge.




%% MiniAgda~\cite{miniagda} is a recent 
%% variant that uses sized types to enable 
%% non-structural termination proofs.
%% %
%% ``Light" dependently typed languages like 
%% Omega~\cite{Sheard06}.
%
   Haskell itself can be used to \emph{fake} ``lightweight'' dependent 
   types~\citep{ChakravartyKJ05,JonesVWW06,Weirich12}.
   In this style, the invariants are expressed in 
   a restricted~\citep{Jia10} total 
   index language and relationships (\eg $x<y$ and $y<z$) 
   are combined (\eg $x<z$) by explicitly constructing
   a term denoting the consequent from terms 
   denoting the antecedents.
   %
   On the plus side this ``constructive'' approach
   ensures soundness. 
   It is impossible to witness inconsistencies, 
   as doing so triggers diverging computations.
   %
   However, it is not easy to use restricted indices
   with explicitly constructed relations to verify 
   complex properties~\citep{hasochism}.

%% like @Map@, @Text@ and @ByteString@. \RJ{CHECK}
%% NEW In this style, relationships are expressed by
%% NEW explicit constructed GADT values.
%% NEW which does not admit inconsistencies of the kind that
%% NEW show up with more expressive refinement 
%% NEW logics of the kind we are working with.

%%% % more GADTS ~\cite{Cheney02, GRDC03, SchrijversJSV09}, 
%%% allow non-terminating computations but do 
%%% not allow terms inside types. 
%%% %
%%% Instead, they use a separate, structurally
%%% terminating index language, and use singleton 
%%% types to connect indices to computation.
%%% %
%%% \RJ{CHECK: Does index/logic prevent inconsistency?}

\spara{Refinement Types} are a form of dependent types where 
invariants are encoded via a combination of types and predicates
from a restricted \emph{SMT-decidable} 
logic~\cite{Rushby98,pfenningxi98,Dunfield07,GordonTOPLAS2011}. 
%
The restriction makes it safe to support arbitrary recursion, 
which has hitherto never been a problem for refinement types.
%
However, we show that this is because all the above systems 
implicitly assume that all free variables are bound to values, 
which is only guaranteed under CBV and, as we have seen, leads 
to unsoundness under lazy evaluation.

%% Refinement types offer a highly automated means of
%% verification and have been applied to check
%% a variety of program properties, including
%% %
%% functional correctness of data structures~\cite{Dunfield07,LiquidPLDI09}, 
%% security protocols~\cite{GordonTOPLAS2011,Bhargavan13} 
%% and compilers~\cite{SwamyPOPL13}.
%
%% The language of refinements is restricted to ensure consistency.
%% however, program variables (binders), or their
%% singleton representatives~\cite{pfenningxi98}, 
%% \emph{are} crucially allowed in the refinements. 
%% As discussed in \S~\ref{sec:overview} this 
%% is sound under \emph{call-by-value} evaluation, 
%% but under Haskell's semantics any innocent
%% binder can be potentially diverging, causing 
%% unsoundness.
%
%% Finally, our implementation is based on 
%% \toolname~\cite{vazou13}, which was unsound 
%% as it assumed CBV evaluation.

%%%% HEREHEREHEREHEREHERE

\spara{Tracking Divergent Computations}
The notion of type stratification to track potentially 
diverging computations dates to at least~\citep{ConstableS87} 
which uses $\bar{\typ}$ to encode diverging terms, and types 
$\efix{}$ as $(\bar{\typ}\rightarrow\bar{\typ}) \rightarrow \bar{\typ}$).
%
More recently, \cite{Capretta05} tracks diverging 
computations within a \emph{partiality monad}.
%
Unlike the above, we use refinements to 
obtain terminating fixpoints (\etfix{}), which let us prove 
the vast majority (of sub-expressions) in real world libraries 
as non-diverging, avoiding the restructuring that would
be required by the partiality monad.

\spara{Termination Analyses}
Various authors have proposed techniques to verify termination 
of recursive functions, either using the ``size-change principle'' 
\cite{JonesB04,Sereni05}, or by annotating types with size indices 
and verifying that the arguments of recursive calls have smaller 
indices~\cite{HughesParetoSabry96,BartheTermination}.
%
Our use of refinements to encode terminating fixpoints is most 
closely related to~\cite{XiTerminationLICS01}, but this work 
also crucially assumes CBV semantics for soundness.

AProVE~\cite{Giesl11} implements a powerful, fully-automatic
termination analysis for Haskell based on term-rewriting.
%
While we could use an external analysis like AProVE,
we have found that encoding the termination proof via 
refinements provided advantages that are crucial in 
large, real-world code bases. Specifically, refinements
let us
%
(1) prove termination over a subset 
    (not all) of inputs; many functions (\eg @fac@) 
    terminate only on @Nat@ inputs and not all @Int@s,
%
(2) encode pre-conditions, 
    post-conditions, and auxiliary invariants that 
    are essential for proving termination, (\eg @gcd@),
%
(3) easily specify non-standard 
    decreasing metrics and prove termination, (\eg @range@).
%
In each case, the code could be (significantly) 
\emph{rewritten} to be amenable to AProVE but this defeats
the purpose of an automatic checker.
%
Finally, none of the above analyses have been empirically
evaluated on large and complex real-world libraries.


% Could use APROVE, but didn't
% 1. refinements let us reason about /partial functions/
% 2. refinements let us reason about /auxiliary invariants/
% 3. refinements let us specify witness expressions for termination

% Given the existence of such standalone termination provers, one might
% ask why we chose to prove termination ourselves. While certainly
%% We could use one such existing termination analysis; however, we found that
%% encoding the termination proof via refinements provided a synergy that
%% would have been impossible with an external oracle. 
%% %
%% Specifically, we
%% found that 
%% (1) refinements let us prove termination of \emph{partial functions},
%% (2) refinements let us reason about \emph{auxiliary invariants}, and 
%% (3) refinements let us express termination metrics that require 
%%     a \emph{witness} without editing the actual code.

% Given our the structure of our proof, with the Termination Oracle
% Hypothesis, one might ask why we chose not to use an existing
% termination prover for Haskell to discharge the termination
% requirement. 
% While certainly possible, we found that encoding the
% termination proof via refinements provided a synergy that would have
% been impossible with an external tool like AProVe. Consider the
% following program:
% %
% \begin{code}
%   -- countDown :: Nat -> Int
%   countDown 0 = 0
%   countDown n = countDown (n - 1)
% \end{code}
% %
% We include the type as a comment to show the intended usage. A
% TRS-based oracle will (correctly) say that @countDown@ may not
% terminate, it could be called with a \emph{negative} input! Suppose,
% however, that @countDown@ is only called with @Nat@s. \toolname will
% then \emph{infer} the commented-out type, which will allow it to prove
% that @countDown@ does, in fact, terminate on all \emph{provided}
% inputs. The distinction between terminating on all possible vs.\ all
% actual inputs is crucial for us since we are only concerned with
% termination insofar as it allows us to prove safety properties.

\spara{Static Contract Checkers} 
like ESCJava~\cite{ESCJava} are a classical way of verifying 
correctness through assertions and pre- and post-conditions. 
%
Side-effects like modifications of global variables are a 
well known issue for static checkers for imperative languages;
the standard approach is to use an effect analysis to determine
the ``modifies clause'' \ie the set of globals modified by a procedure.
%
Similarly, one can view our approach as implicitly computing 
the non-termination effects.
%
%% One can view Refinement Types as a type-based 
%% generalization of this approach. 
%
%% Classical contract checkers check ``partial''
%% (as opposed to ``total'') correctness (\ie safety) 
%% for \emph{eager}, typically first-order, languages
%% and need not worry about termination.
%% %
%% We have shown that in the lazy setting, even 
%% ``partial'' correctness requires proving ``total''
%% correctness!
%
\cite{XuPOPL09} describes a static contract checker for 
Haskell that uses symbolic execution to unroll procedures
upto some fixed depth, yielding weaker ``bounded'' soundness
guarantees.
% 
%% The (checker's) termination requires that recursive 
%% procedures only be unrolled up to some fixed depth.
%% While this approach removes inconsistencies, it yields 
%% weaker, ``bounded'' soundness guarantees.
%
Similarly, Zeno~\cite{ZENO} is an automatic Haskell 
prover that combines unrolling with heuristics for rewriting
and proof-search. 
Based on rewriting, it is sound but 
``Zeno might loop forever'' when faced with 
non-termination.
%
Finally, the Halo~\cite{halo} contract checker encodes 
Haskell programs into first-order logic by directly 
modeling the code's denotational semantics,
again, requiring heuristics for instantiating axioms 
describing functions' behavior. Halo's translation of Haskell
programs directly encodes constructors as uninterpreted functions,
axiomatized to be injective (as the denotational semantics requires).
This heavyweight encoding is more precise than predicate abstraction 
but leads to model-theoretic problems (outlined in the Halo paper) and 
affects the efficiency of the encoding when scaling to larger programs 
(see also \ref{sec:conclusion}, paragraph B) in the lack of specialized 
decisions procedures.
%
Unlike any of the above, our type-based approach does 
not rely on heuristics for unrolling recursive procedures, 
or instantiating axioms. 
%
Instead we are based on decidable SMT validity 
checking and abstract interpretation~\cite{LiquidPLDI08} 
which makes the tool predictable and the overall workflow
scale to the verification of large, real-world
code bases.




%%% Our notion of refinement types comes from 
%%% Liquid Types~\cite{LiquidPLDI08} that provide not only 
%%% decidable type checking, but also type inference,
%%% to verify partial correctness of programs.
%%% %
%%% In the original work, Liquid Types were used 
%%% for \textsc{Ocaml}, thus assume eager evaluation, 
%%% \ie, if an argument to an expression or a let binder
%%% diverges, the whole expression diverges and the program
%%% trivially satisfies any property.


%% RJ %% Conversely, as stated in~\cite{Jia10}, 
%% RJ %%   ``others, such as DML\cite{XiPfenning99}, ATS\cite{Chen05}"
%% RJ %%     Omega\cite{Sheard06}, or Haskell, 
%% RJ %%   allow nonterminating computation, but do not allow those terms 
%% RJ %% to appear in types. 
%% RJ %% Instead, they identify a terminating index 
%% RJ %% language and use singleton types to connect indices to computation.''

%% RJ %% Dependent types have been used under a ``lazy'' environment.
%% RJ %% Haskell-like, dependently-typed languages, such as
%% RJ %% Omega~\cite{Sheard06}, Idris~\cite{Brady13}, and
%% RJ %% {$\lambda_\rightarrow$}~\cite{LohMS10}
%% RJ %% support lazy arguments but
%% RJ %% are compiled with eager evaluation.
%% RJ %% %
%% RJ %% Haskell itself can be considered a dependently-typed language,
%% RJ %% as type level computation is allowed via 
%% RJ %% Type Families~\cite{McBride01},
%% RJ %% Singleton Types\cite{Weirich12}, 
%% RJ %% Generalized Algebraic  Datatypes (GADTs)~\cite{JonesVWW06, SchrijversJSV09}, 
%% RJ %% % more GADS ~\cite{Cheney02, GRDC03},
%% RJ %% and type-level functions~\cite{ChakravartyKJ05}.
%% RJ %% %
%% RJ %% All of these features provide support for type-level computation
%% RJ %% while prohibiting diverging terms to appear inside types.

%\emph{Lightweight Dependent-type Programming}
%	``the programmer
%	has to do as little work as possible to demonstrate that a program satisfies the
%	required invariants''.
% http://okmij.org/ftp/Computation/lightweight-dependent-typing.html



%%% Local Variables: 
%%% mode: latex
%%% TeX-master: "main"
%%% End: 
