\newcommand\hnull{\ensuremath{\text{[]}}\xspace}

\subsection{Measures: From Integers to Data Types}\label{sec:measures}

\begin{figure}
\hrule width 0.48\textwidth \vspace{0.05in}
$$
\begin{array}{rcl}
\multicolumn{3}{l}{\emphbf{Definition}}               \\
  \mathit{def} & ::=  &  \mathtt{measure} \ f :: \tau \\
               &      &  \quad eq_1 \ldots eq_n       \\[0.05in]

\multicolumn{3}{l}{\emphbf{Equation}}                 \\ 
  \mathit{eq}  & ::=  &   f\ (D\ \overline{x}) = r    \\[0.15in] 

\multicolumn{3}{l}{\emphbf{Equation to Type}}         \\[0.05in]
\quad \embed{f\ (D\ \overline{x}) = r} & \defeq & D :: \overline{\tbind{x}{\tau}} \rightarrow \tref{\mathtt{v}}{\tau}{}{f\ \mathtt{v} = r}
\end{array}
$$
\caption{Syntax of Measures}
\label{fig:measures}
\end{figure}


%% \begin{figure}
%% \hrule width 0.48\textwidth \vspace{0.05in}
%% $$
%% \begin{array}{rrcl}
%% \emphbf{Definition} \quad 
%%   & def & ::=&  \mathtt{measure} \ f :: \tau \\
%%   &     &    &  \quad eq_1 \ldots eq_n
%%   \\[0.05in]
%% 
%% \emphbf{Equation} \quad 
%%   & eq & ::=&   f\ (D\ \overline{x}) = r 
%%   \\[0.05in] 
%% 
%% \emphbf{Equation-Type} &&& \\[0.05in]
%% \multicolumn{2}{r}{\embed{f\ (D\ \overline{x}) = r}} & \defeq & 
%%     D :: \overline{\tbind{x}{\tau}} \rightarrow \tref{\mathtt{v}}{\tau}{}{f\ \mathtt{v} = r}
%% 
%% %% \emphbf{Equation to Type} &&&
%% %%   \\[0.05in]
%% %% \multicolumn{4}{c}{ \embed{f\ (D\ \overline{x}) = r} =
%% %%  D :: \overline{\tbind{x}{\tau}} \rightarrow \tref{\mathtt{v}}{\tau}{}{f\ \mathtt{v} = r}
%% %% }
%% \end{array}
%% $$
%% \caption{Syntax of measures}
%% \label{fig:measures}
%% \end{figure}

So far, all our examples have used only integer and boolean expressions in refinements.
To describe properties of algebraic data types, we use \emph{measures},
introduced in prior work on Liquid Types~\cite{LiquidPLDI09}.
%% Next, we define \emph{measures}~\cite{LiquidPLDI09} (also called logical functions~\cite{fstar})
%% that capture properties of data expressions.
Measures are inductively defined functions that can be used in refinements, and
provide an efficient way to axiomatize properties of data types.
%
For example, @emp@ determines whether a list is empty:
%
\begin{code}
  measure emp  :: [Int] -> Bool
    emp []     = true
    emp (x:xs) = false
\end{code}
The syntax for measures deliberately looks like Haskell, but it is \emph{far} more
restricted, and should really be considered as a separate language.
A measure has exactly one argument, and is defined by a list of equations,
each of which has a simple pattern on the left hand side (see Figure~\ref{fig:measures}).
The right-hand side of the equation is a refinement expression $r$.
Measure definitions are typechecked in the usual way; we omit the typing rules which are standard.
(Our metatheory does not support type polymorphism,
so in this paper we simply reason about lists of integers;
however, our implementation supports polymorphism.)

\paragraph{Denotational semantics}
The denotational semantics of types in \hlang in \Sref{sec:den-sem} is readily extended to
support measures.  In \hlang a refinement $r$ is an arbitrary expression, and
calls to a measure are evaluated in the usual way by pattern matching.
For example, with the above definition of @emp@ it is straightforward to show that
\begin{align}
  \mathtt{[1, 2, 3]} \dcolon \tref{\mathtt{v}}{[\tint]}{}{\mathtt{not}\ (\mathtt{emp}\ \mathtt{v})} \label{type:len}
\end{align}
as the refinement @not (emp ([1, 2, 3]))@ evaluates to $\tttrue$.

\mypara{Measures as Axioms}
How can we reason about invocations of measures in the decidable logic of VCs?
A natural approach is to treat a measure like @emp@ as an uninterpreted function,
and add logical axioms that capture its behaviour. This looks easy: each equation 
of the measure definition corresponds to an axiom, thus:
%
\begin{align*}
\ttemp\ \hnull &= \tttrue\\
\forall \ttx, \ttxs.\, \ttemp\ (\ttx:\ttxs) &= \ttfalse
\end{align*}
%
Under these axioms the judgement~\ref{type:len} is indeed valid. 
% % Measures as data constructor refinements

\mypara{Measures as Refinements in Types of Data Constructors}
Axiomatizing measures is \emph{precise}; that is, 
the axioms exactly capture the meaning of measures.
Alas, axioms render SMT solvers \emph{inefficient}, and render the VC mechanism \emph{unpredictable}, 
as one must rely on various brittle syntactic matching and instantiation heuristics~\cite{simplifyj}.

Instead, we use a different approach that is \emph{both} precise \emph{and} efficient.
The key idea is this: \emph{instead of translating each measure equation into an axiom, 
we translate each equation into a refined type for the corresponding data constructor}~\citep{LiquidPLDI09}.
This translation is given in Figure~\ref{fig:measures}.
For example, the definition of the measure @emp@ yields the following refined types for the list data constructors:
$$
\begin{array}{lcl}
\hnull  & :: & \ttreft{v}{[\tint]}{emp\ v = true}\\
{:}  & :: & \tfun{\ttx}{\tint}{\tfun{\ttxs}{[\tint]}{\ttreft{v}{[\tint]}{emp\ v = false}}}
\end{array}
$$
These types ensure that:
%
~(1) each time a list value is \emph{constructed}, 
its type carries the appropriate emptiness information. 
Thus our system is able to statically decide that 
(\ref{type:len}) is valid, and,
~(2) each time a list value is \emph{matched}, 
the appropriate emptiness information is used to 
improve precision of pattern matching, as we see next.

\mypara{Using Measures}
\label{sec:pattern-match}
As an example, we use the measure @emp@ to 
provide an appropriate type for the @head@ function:
%
\begin{code}
  head    :: {v:[Int] | not (emp v)} -> Int 
  head xs = case xs of
              (x:_) -> x
              []    -> error "yikes"  

  error   :: {v:String | false} -> a
  error   = undefined
\end{code}
%
@head@ is safe as its input type stipulates that it will only 
be called with lists that are \emph{not} @[]@, and so
@error "..."@ is dead code.
%
The call to @error@ generates the subtyping query
%
\begin{align*}
	\begin{array}{l}
   \tbind{\ttxs}{\tref{\ttxs}{[\tint]}{\trivial}{\lnot (\ttemp\ \ttxs)}}\\
   \tbind{\ttb}{\tttref{\ttb}{[\tint]}{\trivial}{(\ttemp\ \ttxs)= true}} 	
	\end{array} & \vdash \subtref{\tttrue}{\ttfalse} 
\end{align*}
%
The match-binder $\ttb$ holds the result of the 
match~\cite{SulzmannCJD07}. In the \texttt{[]} case,
we assign it the refinement of the type of \texttt{[]} 
which is $(\ttemp\ \ttxs) = \tttrue$. %~\cite{LiquidPLDI09}.
%
Since the call is done inside a @case-of@ expressions 
both @xs@ and @b@ are guaranteed to be in WHNF,
thus they have \Wnf types. 
  
The verifier \emph{accepts} the program as the above subtyping reduces to the valid VC
\begin{align*}
\lnot (\ttemp\ \ttxs) \wedge ((\ttemp\ \ttxs)= \tttrue) \Rightarrow\ & \tttrue \Rightarrow\ \ttfalse
\end{align*}
%
%%The above method also applies to terms that have been @seq@-ed or have strictness 
%%annotations.
%
Consequently, our system can naturally support idiomatic 
Haskell, \eg taking the @head@ of an infinite list:
%
\begin{code}
  ex x     = head (repeat x)
  
  repeat   :: Int -> {v:[Int] | not (emp v)}
  repeat y = y : repeat y
\end{code}
%
%% We verify the signature for @repeat@ as the only returned value 
%% is a cons-ed list for which @emp@ is $\ttfalse$.
%% %
%% Note that as the return has a \Div-type, \toolname would soundly 
%% reject the signature 
%% %
%% \begin{code}
%%   repeat   :: a -> {v:[a] | false}
%% \end{code}


\mypara{Multiple Measures}
If a type has multiple measures, we simply refine each data constructor's type
with the \emph{conjunction} of the refinements from each measure.
%
For example, consider a measure that computes the length of a list:
\begin{code}
  measure len  :: [Int] -> Int
    len ([])   = 0
    len (x:xs) = 1 + len xs
\end{code}
%
Using the translation of Figure~\ref{fig:measures},
we extract the following types for list's data constructors.
$$
\begin{array}{lcl}
\text{[]}  & :: & \ttreft{v}{[\tint]}{len\ v = 0}\\
{:}  & :: & \tfun{\ttx}{\tint}{\tfun{\ttxs}{[\tint]}{\ttreft{v}{[\tint]}{len\ v = 1 + (len\ xs)}}}\\[0.05in]
\multicolumn{3}{l}{
\text{The final types for list data constructors will be the 
conjunction of}}\\
\multicolumn{3}{l}{
\text{the refinements from $\mathtt{len}$ and $\mathtt{emp}$:}
}\\[0.05in]
\text{[]}  & :: & \ttreft{v}{[\tint]}{emp\ v = true \land len\ v = 0}\\
{:}  & :: & \tfun{\ttx}{\tint}{\tfun{\ttxs}{[\tint]}\\ &&
{\ttreft{v}{[\tint]}{emp\ v = false \land len\ v = 1 + (len\ xs)}}}
\end{array}
$$




%
\begin{comment}
* uninterpreted functions approximated haskell functions, 
for soundness we need to evaluate them

* measure definitions => types of data contructors
Measure definitions are not used in any other way
Otherwise they are simply uninterpretd functions

* Many measures => conjunction.

* State that you can use measures that lack definitions; they are uninterpreted functions

* give measure syntax definition

* measure evaluation is haskell evaluation

* eqivalently use axioms, but what we do is more efficient, see HALO
see related work discussion about why this is not an option and HALO was Bad
\end{comment}
