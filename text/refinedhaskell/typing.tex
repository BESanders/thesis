\renewcommand\rimpl{\rsubbase}
\renewcommand{\rtdimp}{\rsubbased}
%
\section{Algorithmic Typing: \declang}\label{sec:typing}

% Here is the syntax and typing rules for \lambda-D and QF-EUFLIAD

\renewcommand\restrictdecidable[2]{#1}
\newcommand\samerule[1]{}


\begin{figure}[t!]
\centering
\captionsetup{justification=centering}
$$
\begin{array}{rrcl}
\multicolumn{4}{l}{\text{\emphbf{Expressions}, \emphbf{Values}, \emphbf{Constants}, \emphbf{Basic types}:
see Figure~\ref{fig:undeclang}}} \\[0.05in]

\emphbf{Types} \quad 
  & \tau
  & ::=  
  & 	 \tref{v}{\ttbase}{}{\refi}
  \spmid \tref{v}{\ttbase}{l}{\refi}\\
  &&
  \spmid & \tfunref{x}{\tau}{\tau}{v}{\refi}   \\[0.05in]

\emphbf{Labels} \quad 
  & l
  & ::= 
  & \trivial \spmid \finite 
  \\[0.05in] 

\emphbf{Refinements} \quad 
  & \refi
  & ::=
  & p
  \\[0.05in] 

\emphbf{Predicates} \quad 
  & p
  & ::= 
  &   	 \lterm = \lterm
  \spmid \lterm < \lterm
  \spmid       p \land p
  \spmid \lnot p\\
  &&
  \spmid &   	 n 
  \spmid x 
  \spmid f \ \overline{\lterm}
  % \spmid D \ \overline{\lterm}
  \spmid \lterm \oplus \lterm
  \\ && \spmid &
  \etrue
  \spmid \efalse 
  \\[0.05in] 

\emphbf{Measures} \quad 
  & \multicolumn{3}{l}{f,g,h}
  \\[0.05in] 
  
\emphbf{Operators} \quad 
  & \oplus
  & ::= 
  &   	 + 
  \spmid -  
  \spmid \dots
  \\[0.05in] 
  
\emphbf{Integers} \quad 
  & n
  & ::= 
  &   	 0 
  \spmid 1
  \spmid -1
  \spmid \dots
  \\[0.05in] 

\emphbf{Domain} \quad 
  & d
  & ::= 
  & n 
  \spmid c_w 
  \spmid D\ \overline{d}  
  \spmid \etrue
  \spmid \efalse
  \\[0.05in] 
  
\emphbf{Model} \quad 
  & \sigma & ::= & \gbind{x_1}{d_1},\ldots,\gbind{x_n}{d_n}
  \\[0.05in] 

\emphbf{Lifted Values} \quad 
  & \botv
  & ::= 
  &   	 c 
  \spmid \efun{x}{}{e} 
  \spmid D \ \overline{\botv}
  \spmid \ebot
\end{array}
$$
\caption{Syntax of \declang}
\label{fig:declang:syntax}
\end{figure}


\begin{figure}[t!]
\centering
\captionsetup{justification=centering}

All rules as in Figure~\ref{fig:typing} except as follows: \\

\judgementHead{Well-Formedness}{\deciswellformed{\Gamma}{\tau}}
$$
\inference{
	\dechastype{\Gamma, \tbind{v}{\Base}}{\restrictdecidable{p}{r}}{\FinTy{\tbool}}
}{
	\deciswellformed{\Gamma}{\tref{v}{\Base}{}{\restrictdecidable{p}{r}}}
}[\rwbased]
%%%% \samerule{\qquad
%%%% \inference{
%%%% 	\deciswellformed{\Gamma}{\tau_x} &&
%%%% 	\deciswellformed{\Gamma, x \colon \tau_x}{\tau}
%%%% }{
%%%% 	\deciswellformed{\Gamma}{\tfunref{x}{\tau_x}{\tau}{v}{e}}
%%%% }[\rwfun]
%%%% }
$$

%%\judgementHead{Implication}{\issubref{\Gamma}{\restrictdecidable{p}{e}}{\restrictdecidable{p}{e}}}
%%\restrictdecidable{
%%$$
%%\inference{
%%  \VCOND{\Env}{p_1}{p_2}\ \mbox{is u-valid}
%%}{
%%	\issubref{\Env}{p_1}{p_2}
%%}[\rtdimp]
%%$$}{$$
%%\inference{
%%	\forall \theta. \deciswellformed{\Gamma}{\theta} \land
%%				\evals{\theta\ e_1}{\etrue} 
%%	\Rightarrow \evals{\theta\ e_2}{\etrue}
%%}{
%%	\issubref{\Gamma}{e_1}{e_2}
%%}[\rimpl]
%%$$}
\judgementHead{Subtyping}{\decissubtype{\Gamma}{\tau_1}{\tau_2}}

$$
\inference{
  \VCOND{\Env, v:B}{p_1}{p_2}\ \mbox{is valid}
}{
	\decissubtype{\Gamma}
		{\tref{v}{B}{}{\restrictdecidable{p_1}{e_1}}}
		{\tref{v}{B}{}{\restrictdecidable{p_2}{e_2}}}
}[\rsubbased]
\samerule{\qquad
\inference{
	\decissubtype{\Gamma}{\tau'_x}{\tau_x} &&
	\decissubtype{\Gamma, x \colon \tau'_x}{\tau}{\tau'}
}{
	\decissubtype{\Gamma}{\tfunref{x}{\tau_x}{\tau}{v}{e_1}}{\tfunref{x}{\tau'_x}{\tau'}{v}{e_2}}
}[\rsubfun]
}$$

\judgementHead{Typing}{\dechastype{\Gamma}{e}{\tau}}
\samerule{$$
\inference{
	(x,\tau) \in \Gamma 
}{
	\dechastype{\Gamma}{e}{\tau}
}[\rtvar]
$$

$$
\inference{
}{
	\dechastype{\Gamma}{c}{\constty{c}}
}[\rtconst]
\qquad
\inference{
	\dechastype{\Gamma}{e}{\tau'} &&
	\decissubtype{\Gamma}{\tau'}{\tau} &&
	\deciswellformed{\Gamma}{\tau} &&
}{
	\dechastype{\Gamma}{e}{\tau}
}[\rtsub]
$$}
$$
\samerule{
\inference{
	\dechastype{\Gamma, x\colon\tau_x}{e}{\tau} &&
	\deciswellformed{\Gamma}{\tau_x}
}{
	\dechastype{\Gamma}{\efun{x}{e}}{(\tfun{x}{\tau_x}{\tau})}
}[\rtfun]
\qquad}
\inference{
	\dechastype{\Gamma}{e_1}{(\tfunref{x}{\tau_{x}}{\tau}{v}{e_v})} &&
	\dechastype{\Gamma}{\restrictdecidable{y}{e_2}}{\tau_{x}}
}{
	\dechastype{\Gamma}{\eapp{e_1}{\restrictdecidable{y}{e_2}}}{\tau\sub{x}{\restrictdecidable{y}{e_2}}}
}[\rtapp-D]
$$
\samerule{
$$
\inference{
	\dechastype{\Gamma}{e_x}{\tau_{x}} &&
 in Figure~\ref{fig:}	\dechastype{\Gamma,x\colon\tau_x}{e}{\tau} &&
	\deciswellformed{\Gamma}{\tau}
}{
	\dechastype{\Gamma}{\elet{x}{e_x}{e}}{\tau}
}[\rtlet]
$$
}
$$\inference{
	l \not \in \{\finite, \trivial\} \Rightarrow \tau \ \text{is}\ \Div &&
	\dechastype{\Gamma}{e}{\tref{v}{T}{l}{r}} &&
	 \deciswellformed{\Gamma}{\tau}\\
	\forall i. \constty{D_i} = \overline{y_j}{\tau_j} \rightarrow \tref{v}{T}{}{r_i} &&
	\dechastype{\Gamma,  \overline{\tbind{y_j}{\tau_j}},
				\tbind{x}{\tref{v}{T}{\restrictdecidable{\trivial}
				{\ltrivial}
				}{r \land r_i}}}{e_i}{\tau}	
}{
	\dechastype{\Gamma}{\ecase{e}{D_i}{\overline{y_j}}{e_i}{x}}{\tau}
}[\rtcased]$$
\caption{Typechecking for \declang}
\label{fig:declang:typing}
\end{figure}



While \undeclang is sound, it cannot be \emph{implemented} 
thanks to the undecidable denotational containment rule \rsubbase
(Figure~\ref{fig:refinedhaskell:typing}).
%
Next, we go from \undeclang to \declang, a core calculus 
with sound, SMT-based algorithmic type-checking in four 
steps.
%
First, we show how to restrict the language of 
refinements to an SMT-decidable sub-language 
\logiclang~(\Sref{sec:typing:logic}).
%
Second, we \emph{stratify} the types to specify 
whether their inhabitants may diverge, must reduce
to values, or must reduce to finite values~(\Sref{sec:typing:stratify}).
%
Third, we show how to \emph{enforce} the stratification
by encoding recursion using special fixpoint combinator 
constants (\Sref{sec:typing:termination}).
%
Finally, we show how to use \logiclang and the 
stratification to approximate the undecidable \rsubbase
with a decidable verification condition \rsubbased, thereby 
obtaining the algorithmic system \declang~(\Sref{sec:typing:vc}).


\subsection{Refinement Logic: \logiclang}\label{sec:typing:logic}

Figure~\ref{fig:declang:syntax} summarizes the syntax of \declang.
Refinements $r$ are now predicates $p$, drawn from 
\logiclang, the decidable logic of equality, 
uninterpreted functions and linear arithmetic
~\cite{Nelson81}.
%
Predicates $p$ include linear arithmetic constraints,
function application where function symbols correspond
to measures (as described in \Sref{sec:measures}), and 
boolean combinations of sub-predicates.

\mypara{Well-Formedness} 
For a predicate to be well-formed it should be boolean and
arithmetic operators 
should be applied to integer terms, measures should be applied 
to appropriate arguments 
(\ie \eisNull\ is applied to \tintlist),
and equality or inequality to basic 
(integer or boolean) terms.
%
Furthermore, we require that refinements, and thus measures,
always evaluate to a value.
%
We capture these requirements by assigning appropriate types 
to operators and measure functions, after which we require that
each refinement $r$ has type $\FinTy{\tbool}$ (rule \rwbased in
Figure~\ref{fig:declang:typing}).

\mypara{Assignments}
Figure~\ref{fig:declang:syntax} defines the elements $d$
of the domain \dom 
of integers, booleans, and data constructors that wrap
elements from \dom.
%
The domain \dom also contains a constant $c_w$
for each value $w$ of \undeclang that does 
not otherwise belong in \dom (\eg functions or other primitives).
%
An \emph{assignment} $\sigma$ is a map from variables 
to $\dom$.

%% \NV{DONE:Can we just drop the measure + axioms step?}
\mypara{Satisfiability \& Validity}
We interpret boolean predicates in the logic over the domain 
\dom.
%
We write \lmodels{\sigma}{p} %(resp. \umodels{\sigma}{p}) 
if $\sigma$ is a model of $p$.
%% where the interpretations of 
%%measures respect the measure axioms (resp. measures are 
%%uninterpreted).
%
We omit the formal definition for space.
%
A predicate $p$ is \emph{satisfiable}  %(resp. \emph{u-satisfiable}) 
if there \emph{exists} $\lmodels{\sigma}{p}$. %(resp. $\umodels{\sigma}{p}$).
%
A predicate $p$ is \emph{valid} %(resp. \emph{u-valid}) 
if \emph{for all} assignments $\lmodels{\sigma}{p}$. % (resp. \umodels{\sigma}{p}).
%
%Note that if $p$ is u-valid then $p$ is trivially valid.


%\begin{lemma}\label{ref:valid} If $p$ is u-valid then $p$ is valid.
%\end{lemma}

\mypara{Connecting Evaluation and Logic}
To prove soundness, we need to formally connect the
notion of logical models with the evaluation of a 
refinement to \etrue.
%
We do this in several steps, briefly outlined for brevity
(the detailed proof is in~\cite{vazou14techrep}).
%
First, we introduce a primitive \emph{bottom expression} 
\ebot that can have \emph{any} \Div type, but does not evaluate.
%%(\NV{only values can be applied to primitive constants, 
%%so $c\ \ebot$ will just not evaluate.}
%%\ie yields \ebot when applied to any primitive constant).
%
Second, we define \emph{lifted values} \botv 
(Figure~\ref{fig:declang:syntax}), which are values that
contain \ebot.
%
Third, we define \emph{lifted substitutions} 
$\botsto$, which are mappings from variables to 
lifted values.
%
Finally, we show how to \emph{embed} a lifted substitution 
\botsto into a \emph{set of} assignments \embed{\botsto} 
where, intuitively speaking, each \ebot is replaced by
some arbitrarily chosen element of \dom.
%
%Equipped with the above notions, we prove the following which
%connects evaluation and logical satisfaction.
Now, we can connect evaluation and logical satisfaction:
%
\begin{theorem}\label{thm:equiv}
If \dechastype{\emptyset}{\botsto(p)}{\FinTy{\tbool}}, then
$\evals{\botsto(p)}{\etrue}\ 
\mbox{iff}\ \ 
\forall \sigma \in \embed{\botsto}. \lmodels{\sigma}{p}$.
\end{theorem}

\mypara{Restricting Refinements to Predicates}
Our goal is to restrict \rimpl so that only predicates 
from the decidable logic \logiclang (not arbitrary expressions)
appear in implications \isimplied{\Env}{\p_1}{\p_2}.
%
Towards this goal, as shown in Figures~\ref{fig:declang:syntax}
and~\ref{fig:declang:typing}, 
we restrict the syntax and well-formedness of types to contain
only predicates
%
and we convert the program to ANF after which we can 
restrict the application rule \rtappd to applications 
to variables, which ensures that refinements remain 
within the logic after substitution~\cite{LiquidPLDI08}.
%
Recall, that this is not enough to ensure that refinements do converge, 
as under lazy evaluation,
even binders can refer to potentially divergent values.

\subsection{Stratified Types}\label{sec:typing:stratify}

The typing rules for \declang are given in Figure~\ref{fig:declang:typing}.
Instead of \emph{explicitly} reasoning about divergence or 
strictness in the refinement logic, which leads to significant
theoretical and practical problems, as discussed in \Sref{sec:refinedhaskell:conclusion}, 
we choose to reason \emph{implicitly} about divergence within the type system.
%
Thus, the second critical step in our path to \declang is the 
stratification of types into those inhabited by potentially
diverging terms, terms that only reduce to values, and 
terms which reduce to finite values.
%
Furthermore, the stratification crucially allows us to prove 
Theorem~\ref{thm:equiv}, which requires that refinements do 
not diverge (\eg by computing the length of an infinite list)
by ensuring that inductively defined measures are only applied 
to finite values.
%
Next, we describe how we stratify types with labels and 
then type the various constants, in particular the fixpoint 
combinators, to enforce stratification.

\mypara{Labels}
We specify stratification using two \emph{labels} for types.
%
The label \trivial\ (resp. \finite) is assigned to types given 
to expressions that reduce (using $\beta$-reduction from Figure~\ref{fig:operational})  
to a value $w$ (resp. \emph{finite} value,
\ie an element of the inductively defined \dom).
%
Formally,
%
\begin{align}
  \mbox{\emphbf{\Wnf types}} \quad 
\interp{\tref{\vv}{\Base}{\trivial}{r}} \defeq & 
    \interp{\tref{\vv}{\Base}{}{r}} \cap \{ e \mid \evals{e}{w} \}
    \label{eq:trivial} \\
  \mbox{\emphbf{\Fin types}} \quad 
\interp{\tref{\vv}{\Base}{\finite}{r}} \defeq & 
    \interp{\tref{\vv}{\Base}{}{r}} \cap \{ e \mid \evals{e}{d} \} 
    \label{eq:finite} 
\end{align}
%
Unlabelled types are assigned to expressions that may diverge.
%
Note that for any $\Base$ and refinement $r$ we have
$$
\interp{\tref{\vv}{\Base}{\finite}{r}} \subseteq
\interp{\tref{\vv}{\Base}{\trivial}{r}} \subseteq
\interp{\tref{\vv}{\Base}{}{r}}
$$ 
%
The first two sets are \emph{equal} for $\tint$ and $\tbool$, 
and \emph{unequal} for (lazily) constructed data types $T$. 
%
We need not stratify function types (\ie they are \Div types)
as binders with function types do not appear inside the VC, and 
are not applied to measures.

\mypara{Enforcing Stratification}\label{sec:typing:termination}
%
We enforce stratification in two steps.
%
First, the $\rtcased$ rule uses the operational semantics of case-of to 
type-check each case in an environment where the scrutinee
$x$ is assumed to have a \Wnf type. 
%
All the other rules, not mentioned in Figure~\ref{fig:declang:typing},
remain the same as in Figure~\ref{fig:refinedhaskell:typing}.
%
Second, we create stratified variants for the primitive constants 
and \emph{separate} fixpoint combinator constants for 
(arbitary, potentially non-terminating) recursion (\efix{}) 
and bounded recursion (\etfix{}).

\mypara{Stratified Primitives}
First, we restrict the primitive operators whose output 
types are refined with logical operators, so they are only 
invoked on finite arguments (so that the corresponding 
refinements are guaranteed to not diverge).
%
\begin{align*}
  \constty{\mathtt{\n}} \defeq & \tlref{\vv}{\tint}{\finite}{\vv = \n} \\
  \constty{\mathtt{=}} \defeq & \tfun{x}
                                     {\tbase^\finite}
                                     {\tfun{y}
                                           {\tbase^\finite}
                                           {\tlref{\vv}{\tbool}{\finite}{\vv \Leftrightarrow \x = \y}}} \\
  \constty{\mathtt{+}} \defeq & \tfun{x}
                                     {\tint^\finite}
                                     {\tfun{y}
                                           {\tint^\finite}
                                           {\tlref{\vv}{\tint}{\finite}{\vv = \x + \y}}}\\
  \constty{\mathtt{\land}} \defeq & \tfun{x}
                                         {\tbool^\finite}
                                         {\tfun{y}
                                               {\tbool^\finite}
                                               {\tlref{\vv}{\tbool}{\finite}{\vv \Leftrightarrow \x \land \y}}}
\end{align*}
%
It is easy to prove that the above primitives respect 
their stratification labels, \ie belong in the denotations
of their types. 

Note that the above types are restricted in that they can only be applied to finite arguments.
%
In future work~\ref{chapter:conclusion}, we could address this issue with unrefined versions of primitive types
that soundly allow operation on arbitrary arguments.
%
For example, with the current type for $\mathtt{+}$, addition of potentially 
diverging expressions is rejected.
%
Thus, we could define an unrefined signature 
\begin{align*}
  \constty{\mathtt{+}} \defeq & \tfun{x}
                                     {\tint}
                                     {\tfun{y}
                                           {\tint}
                                           {\tint}}
\end{align*}
and allow the two types of $\mathtt{+}$ to co-exist (as an intersection type),
where the type checker would choose the precise refined type 
if and only if both of $\mathtt{+}$'s arguments are finite.

\mypara{Diverging Fixpoints ($\efix{\typ}$)}
Next, note that the only place where divergence enters the picture is 
through the fixpoint combinators used to encode recursion. 
%
For any function or basic type $\typ \defeq \typ_1 \rightarrow \ldots \rightarrow \typ_n$,
we define the \emph{result} to be the type $\typ_n$.

For each $\typ$ whose result is a \Div type, 
there is a \emph{diverging fixpoint} combinator
$\efix{\typ}$, such that
%
\begin{align*}
  \ceval{\efix{\typ}}{f} \defeq & f\ (\efix{\tau}\ f) \\
  \constty{\efix{\typ}}   \defeq & (\typ \rightarrow \typ) \rightarrow \typ
\end{align*}
%
\ie, $\efix{\typ}$ yields recursive functions of type $\typ$.
Of course, $\efix{\typ}$ belongs in the denotation of
its type~\cite{PLC} \emph{only if} the result type is 
a \Div type (and \emph{not} when the result is a 
\Wnf or \Fin type). 
%
Thus, we restrict diverging fixpoints to functions with \Div result types.

\mypara{Indexed Fixpoints ($\tfixn{\typ}{n}$)}
For each type $\typ$ whose result is a \Fin 
type, we have a family of \emph{indexed} 
fixpoints combinators $\tfixn{\typ}{n}$: 
%
\begin{align*}
  \ceval{\tfixn{\typ}{n}}{f} \defeq & {\efun{m}{}{f\ m\ (\tfixn{\typ}{m}\ f)}}\\
%% (n:nat → τ_n → τ) → τ_n
\constty{\tfixn{\typ}{n}} \defeq & 
  \tfunbasic{(\tfun{n}
                   {\FinTy{\tnat}}
                   {\tfunbasic{\decr{\typ}{n}}{\typ}})}
            {\decr{\typ}{n}} \\
\mbox{where,}\ \decr{\typ}{n} \defeq & \tfunbasic{\tref{v}{\tnat}{\finite}{v < n}}{\typ}
\end{align*}
%
$\decr{\typ}{n}$ is a \emph{weakened} version of $\typ$ 
that can only be invoked on inputs \emph{smaller} than $n$.
%
Thus, we enforce termination by requiring that $\tfixn{\typ}{n}$ is 
\emph{only} called with $m$ that are \emph{strictly smaller than} $n$. 
% 
As the indices are well-founded $\tnat$s, evaluation will terminate. 
  
\mypara{Terminating Fixpoints ($\etfix{\typ}$)}
Finally, we use the indexed combinators to define the 
\emph{terminating} fixpoint combinator $\etfix{\typ}$ as:
\begin{align*}
  \ceval{\etfix{\typ}}{f} \defeq & {\efun{n}{}{f\ n\ (\tfixn{\typ}{n}\ f)}}\\
%% (n:nat → τ_n → τ) → m:nat⇓ → τ
\constty{\etfix{\typ}} \defeq & 
  \tfunbasic{(\tfun{n}
                   {\FinTy{\tnat}}
                   {\tfunbasic{\decr{\typ}{n}}{\typ}})}
            {\tfunbasic{\FinTy{\tnat}}{\typ}}
\end{align*}
%
Thus, the top-level call to the recursive function
requires a $\FinTy{\tnat}$ parameter $n$ that acts
as a \emph{starting} index, after which, all ``recursive''
calls are to combinators with \emph{smaller} indices, 
ensuring termination.

\mypara{Example: Factorial} 
Consider the factorial function:
$$\efac \defeq 
\efun{n}{}{\efun{f}{}{
	\ecaseinstance{\_}{(n = 0)}{
		 \begin{array}{l}
			 \ealt{\etrue}{1}\\ 
			 \ealt{\_}{n \times f (n-1)}	
		\end{array}
	}
}}$$
Let $\typ \defeq {\FinTy{\tnat}}$. 
We prove termination by typing 
\begin{align*}
  \emptyset \vdash_D \etfix{\typ}\ \efac : &\ \tfunbasic{\FinTy{\tnat}}{\typ} \\
\intertext{To understand \emph{why}, note that $\tfixn{\typ}{n}$ is 
only called with arguments strictly smaller than $n$ 
}
% fix fac n 
\etfix{\typ}\ \efac\ n 
%   = fixn fac n
% \stepsym^* & \ \tfixn{\typ}{n}\ \efac\ n\\
%   = fac n (fixn fac)
\stepsym^* & \ \efac\ n\ (\tfixn{\typ}{n}\ \efac)\\
%   = n * (fixn fac (n-1))
\stepsym^* & \ n \times (\tfixn{\typ}{n}\ \efac\ (n-1))\\
%   = n * (fac (n-1) (fix_n-1 fac))
\stepsym^* & \ n \times (\efac\ (n-1)\ (\tfixn{\typ}{n-1}\ \efac))\\
%   = n * (n-1) * (fix_n-1 fac (n-2))
\stepsym^* & \ n \times n-1 \times (\tfixn{\typ}{n-1}\ \efac\ (n-2))\\
%   = n * (n-1) * (fac (n-2) (fix_n-2 fac))
%   = n * (n-1) * (n-2) * (fix_n-2 fac (n-3))
%   = n * (n-1) * ... * (fix_1 fac 0)
\stepsym^* & \ n \times n-1 \times \ldots \times (\tfixn{\typ}{1}\ \efac\ 0)\\
\stepsym^* & \ n \times n-1 \times \ldots \times (\efac\ 0\ (\tfixn{\typ}{0}\ \efac))\\
%   = n * (n-1) * ... * 1
\stepsym^* & \ n \times n-1 \times \ldots \times 1
\end{align*}

\mypara{Soundness of Stratification}
To formally \emph{prove} that stratification is soundly 
enforced, it suffices to prove that the Denotation 
Lemma~\ref{lem:denotation} holds for \declang.
%
This, in turn, boils down to proving that each 
(stratified) constant belongs in its type's denotation, 
\ie each $c \in \interp{\constty{c}}$
or that the Lemma~\ref{lemma:constants} holds for \declang.
%
The crucial part of the above is proving that 
the indexed and terminating fixpoints inhabit 
their types' denotations.
%

\begin{theorem}{[Fixpoint Typing]} 
\begin{itemize}
  \item $\efix{\typ} \in \interp{\constty{\efix{\typ}}}$, 
  \item $\forall n. \etfixn{\typ}{f}{n} \in \interp{\constty{\etfixn{\typ}{f}{n}}}$,
  \item $\etfix{\typ} \in \interp{\constty{\etfix{\typ}}}$.
\end{itemize}
\end{theorem}

With the above we can prove soundness of Stratification as a corollary 
Denotation Lemma~\ref{lem:denotation}, given the interpretations of 
the stratified types. 
%
\begin{corollary}{[Soundness of Stratification]}\label{cor:stratification} 
\begin{enumerate}
  \item If $\dechastype{\emptyset}{e}{\FinTy{\typ}}$, then  evaluation of $e$ is finite.
  \item If $\dechastype{\emptyset}{e}{\WnfTy{\typ}}$, then  $e$ reduces to WHNF.
  \item\label{col:ref} If $\dechastype{\emptyset}{e}{\ttreft{\vv}{\typ}{\p}}$, then \p cannot diverge.
\end{enumerate}
\end{corollary}

Finally, as a direct implication the well-formedness rule \rwbased 
we conclude \ref{col:ref}, \ie that refinements cannot diverge.

\subsection{Verification With Stratified Types}\label{sec:typing:vc}

We can put the pieces together to obtain an algorithmic implication 
rule \rtdimp instead of the undecidable \rimpl (from Figure~\ref{fig:refinedhaskell:typing}).
%
Intuitively, each closing substitution $\sto$ will correspond to 
a set of logical assignments $\embed{\sto}$. 
%
Thus, we will translate $\Env$ into logical
formula $\embed{\Env}$ and 
denotation inclusion into logical implication 
such that:
%
\begin{itemize}
\item $\sto \in \interp{\Env}$ \textit{iff} all $\sigma \in \embed{\sto}$ 
     satisfy $\embed{\Env}$, and 
%%\item $\evals{\sto(p_i)}{\etrue}$ iff each $\sigma \in \embed{\sto}$ 
%%     satisfies $p_i$, and
\item ${\sto}{\ttreft{\vv}{\mathit{B}}{\p_1}}\subseteq {\sto}{\ttreft{\vv}{\mathit{B}}{\p_2}}$ 
		\textit{iff} all $\sigma \in \embed{\sto}$ satisfy $p_1 \Rightarrow p_2$.
\end{itemize} 

%%Thus, we will translate $\Env$, $p_1$, and $p_2$ into logical
%%formulas $\embed{\Env}$, $\embed{p_1}$, and $\embed{p_2}$ such 
%%that:
%%%
%%\begin{itemize}
%%\item $\evals{\sto(p_i)}{\etrue}$ iff each $\sigma \in \embed{\sto}$ 
%%     satisfies $\embed{p_i}$, and
%%\item $\sto \in \interp{\Env}$ iff each $\sigma \in \embed{\sto}$ 
%%     satisfies $\embed{\Env}$.
%%\end{itemize}

\mypara{Translating Refinements \& Environments}
To translate environments into logical formulas, recall that $\sto \in \interp{\Env}$
iff for each $\tbind{x}{\typ} \in \Env$, we have 
$\sto(x) \in \interp{\thetasub{\sto}{\typ}}$. Thus, 
%
\begin{align*}
\embed{\tbind{x_1}{\typ_1},\ldots,\tbind{x_n}{\typ_n}} \defeq &
  \embed{\tbind{x_1}{\typ_1}} \wedge \ldots \wedge \embed{\tbind{x_n}{\typ_n}}\\
\intertext{How should we translate a single binding? Since a binding denotes}
\interp{\tref{x}{\ttbase}{}{p}} \defeq &
  \{ e \mid \mbox{if } \evals{e}{w}\ \mbox{then}\ \evals{\SUBST{p}{x}{w}}{\etrue} \}\\
\intertext{a direct translation would require a logical value 
  predicate $\isvalue{x}$, which we could use to obtain the logical 
  translation}
\embed{\tref{x}{\ttbase}{}{p}} \defeq & \lnot \isvalue{x} \vee p
\end{align*}
%
This translation poses several theoretical and 
practical problems that preclude the use of existing 
SMT solvers (as detailed in \Sref{sec:conclusion}). 
However, our stratification guarantees 
(cf. (\ref{eq:trivial}), (\ref{eq:finite}))
that labeled types reduce to values and 
so we can simply conservatively translate 
the \Div and labeled (\Wnf, \Fin) bindings as:
$$\embed{\tref{x}{\ttbase}{}{p}}\ \defeq\ \etrue \qquad  \embed{\tref{x}{\ttbase}{l}{p}}\ \defeq\ p$$


\mypara{Soundness} We prove soundness by showing that
the decidable implication \rtdimp approximates 
the undecidable \rimpl. 

\begin{theorem}{}\label{thm:approximation} If \decissubref{\Env}{p_1}{p_2} is
  valid then 
  $\undecissubtype{\Env}{\ttreft{\vv}{\Base}{\p_1}}{\ttreft{\vv}{\Base}{\p_2}}$.
\end{theorem}

%% \begin{theorem}{[Approximation]}\label{thm:approximation} Let $\VC \defeq \VCOND{\Env}{p_1}{p_2}$.
%%   \begin{itemize}
%%     \item If \VC is u-valid then \VC is valid, 
%%     %\item If \VC is valid then \isimplied{\Env}{p_1}{p_2}, 
%%     \item If \decissubref{\Env}{p_1}{p_2} then \isimplied{\Env}{p_1}{p_2}.
%%   \end{itemize}
%% \end{theorem}

To prove the above, let ${\VC \defeq \decissubref{\Env}{p_1}{p_2}}$. 
We prove that if the \VC is valid then 
\undecissubtype{\Env}{\ttreft{\vv}{\base}{\p_1}}{\ttreft{\vv}{\base}{\p_2}}.
%%
%%First, note that
%%if $\VC$ is u-valid then it is valid as the addition of axioms preserves
%%validity. Next, we prove that if the \VC is valid then 
%%\undechastype{\Env}{\ttreft{\vv}{\base}{\p_1}}{\ttreft{\vv}{\base}{\p_2}}.
%
This fact relies crucially on a notion of \emph{tracking evaluation} 
which allows us to reduce a closing substitution $\sto$ to a lifted substitution 
$\botsto$, written $\trackevals{\sto}{\botsto}$, after which we prove:

\begin{lemma}{[Lifting]} 
$\evals{\sto(e)}{c}$ iff $\exists \trackevals{\sto}{\botsto}$ s.t. $\evals{\botsto(e)}{c}$.
\end{lemma}

We combine the Lifting Lemma and the equivalence Theorem~\ref{thm:equiv} 
to prove that the validity of the \VC demonstrates 
the denotational containment
$\forall \sto\in \interp{\Env}. 
  		 \interp{\thetasub{\sto}{\ttreft{\vv}{\miBase}{\p_1}}} 
  		$ $\subseteq$ $\interp{\thetasub{\sto}{\ttreft{\vv}{\miBase}{\p_2}}}$.
%
The soundness of algorithmic typing follows from
Theorems~\ref{thm:approximation} and~\ref{thm:refinedhaskell:safety}:

\begin{theorem} {[Soundness of \declang]} 
\begin{itemize}
\item\textbf{Approximation:} If \dechastype{\emptyset}{e}{\typ} then
  \undechastype{\emptyset}{e}{\typ}.
\item\textbf{Crash-Freedom:} If \dechastype{\emptyset}{e}{\typ} 
        then $\evals{e\not}{\ecrash}$.
\end{itemize}
\end{theorem}

To prove approximation we need to prove that Lemma~\ref{lemma:constants} holds for
each constant, and thus it holds for data 
constructors.
In the metatheory we assume a stronger notion 
of validity that respects the measure axioms. 
%
%\RJ{CHECK}
However, since our implementation does not use axioms and instead, 
without loss of precision, treats measures as uninterpreted 
during SMT validity checking, we omit further discussion of axioms 
for clarity.


%% \NV{DONE: remove Uninterpreted Validity Checking}
%%\mypara{Uninterpreted Validity Checking}
%%We reduce the undecidable implication 
%%to checking u-validity of the \VC, \ie, validity 
%%where measures are uninterpreted.
%%%
%%This trivially implies validity of the \VC.
%%%
%%Recall that we use u-validity instead of validity to ensure 
%%predictable and efficient checking, as then the 
%%\VC belongs to \logiclang.
%%%
%%The elimination of the measure axioms is crucial 
%%for efficient and practical verification. 
%%%
%%% As demonstrated by our evaluation (\Sref{sec:evaluation}),
%%The absence of the axioms does not lead to loss 
%%of precision in practice; the semantics of the 
%%axioms are encoded in the refinement types of 
%%the data constructors, and hence already \emph{instantiated} 
%%inside (the environment and) the VC during type checking.
%%

%%% Local Variables: 
%%% mode: latex
%%% TeX-master: "main"
%%% End: 
