\section{Overview}\label{sec:overviews}

\def\sm{\texttt{sm}}
\def\mc{\texttt{mconcat}}

\subsection{Monoids and Parallelism}\label{subsec:monoids}

As noted in section~\ref{sec:intro}, associativity ultimately
allows monoidal operations to be run in parallel. However, for our
purposes, we would like a way of breaking an input into pieces and
running our computation, string matching, in parallel on all the
pieces. For this, we will focus on the standard Haskell function \mc,
\begin{code}
  mconcat :: Monoid m => [m] -> m
  mconcat [] = mempty
  mconcat (x:xs) = x <> mconcat xs
\end{code}
where @mappend@ and @mempty@ are the monoid operation and identity for
monoid \texttt{m} (usually written \texttt{mappend} and
\texttt{mempty})~\footnote{\mc\, is usually defined as \texttt{foldr mappend mempty}}.

\mc is a monoid morphism from lists of values of
a monoid to that monoid, i.e.
%
$$
\begin{array}{rcl}
\mc\;[] & \quad \equiv \quad & \mempty \\
\mc\;([x_0,\;\ldots x_m]\;\texttt{++}\;[y_0,\ldots y_n])
& \quad \equiv \quad &
\mc\;[x_0, \ldots x_m]\;\mappend\;\mc\;[y_0,\ldots y_n]
\end{array}
$$
%
where \texttt{++} is list append.

$$
\begin{array}{l}
\mc\;[\sm\;x_0,\;\ldots \sm\;x_i, \sm\;x_{i+1},\ldots \sm\;x_j, \sm\;x_{j+1},\ldots \sm\;x_n] \\
\;\;\equiv\;\; \quad
\mc\;([\sm\;x_0,\;\ldots \sm\;x_i]\;\texttt{++}\;[\sm\;x_{i+1},\ldots \sm\;x_j]\;\texttt{++}\;[\sm\;x_{j+1},\ldots \sm\;x_n]) \\
\;\;\equiv\;\; \quad
%\mc\;[\mc\;[\sm\;x_0,\;\ldots \sm\;x_i]\;,\;\mc\;[\sm\;x_{i+1},\ldots \sm\;x_n]] \\
%\;\;\equiv\;\; \quad
\mc\;[\sm\;x_0, \ldots \sm\;x_i]\;\mappend\;\mc\;[\sm\;x_{i+1},\ldots \sm\;x_j]\;\mappend\;\mc\;[\sm\;x_{j+1},\ldots \sm\;x_n]
\end{array}
$$


\subsection{String Matching}\label{subsec:string-matching}

String matching is determining all the indices in a source string
where a given target string begins; for example, for source string
\texttt{abababa} and target \texttt{aba} the results of string
matching would be \texttt{[0, 2, 4]}. We will use the following code
as a specification of string matching:
\begin{code}
makeIndices inp trg = go 0 where
  trgLen = length trg
  go i | i >= length inp = []
       | take trgLen (drop i inp) == trg = i : rest
       | otherwise = rest
  where rest = go (i + 1)

*Main> makeIndices "abababa" "aba" == [0, 2, 4]
\end{code}
Though this implementation is not efficient, it is simple and provides
a good starting point from which to create a verifiable function (see
sec~\ref{sec:monoid}).

\subsection{Monoidal String Matching}\label{subsec:monoidal-string-matching}
