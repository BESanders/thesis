\section{Introduction}\label{sec:intro}
In this paper, we prove correctness of parallelization of a na\"ive string matcher
using Haskell as a theorem prover.
%
We use refinement types to specify correctness properties,
Haskell terms to express proofs and Liquid Haskell to
check correctness of proofs.

Optimization of sequential functions via parallelization
is a well studied technique~\cite{jaja,blelloch}.
%
Paper and pencil proofs of program have been developed to support the
correctness of the transformation~\cite{Cole93parallelprogramming}.
%
However, these paper and pencil proofs
show correctness of the parallelization algorithm
and do not reason about the actual implementation
that may end up being buggy.

Dependent Type Systems (like Coq~\cite{coq-book} and Adga~\cite{agda} )
enable program equivalence proofs
for the actual implementation of the functions to be parallelized.
%
For example, SyDPaCC~\cite{SyDPaCC} is a Coq extension that
given a na\"ive Coq implementation of a function,
returns an Ocaml parallelized version with a proof of program equivalence.
%
The limitation of this approach is that the initial
function should be implemented in
the specific dependent type framework
and thus cannot use features and libraries from one's favorite
programming language.

Refinement Types~\cite{ConstableS87,FreemanPfenningDONTCITE91,Rushby98}
on the other hand, enable verification of existing general purpose languages
(including
ML~\cite{pfenningxi98,GordonRefinement09,LiquidPLDI08},
C~\cite{deputy,LiquidPOPL10},
Haskell~\cite{Vazou14},
Racket~\cite{RefinedRacket}
and Scala~\cite{refinedscala}).
%
Traditionally, refinement types are limited to
``shallow'' specifications,
that is, they are used to specify and verify properties
that only talk about behaviors of program functions
but not functions themselves.
%
This restriction critically limits the expressiveness
of the specifications
but allows for automatic SMT~\cite{SMTLIB2} based verification.
%
Yet, program equivalence proofs were out of the expressive
power of refinement types.

Recently, we extended refinement types
with Refinement Reflection~\cite{reflection},
a technique that reflects each function's implementation
into the function's type
and is implemented in Liquid Haskell~\cite{Vazou14}.
%
We claimed that Refinement Reflection can turn any programming
language into a proof assistant.
%
In this paper we check our claim and use Liquid Haskell
to prove program equivalence.
%
Specifically,
we \textit{define in Haskell}
a sequential string matching function, @toSM@, and its parallelization, @toSMPar@,
using existing Haskell libraries; then,
we \textit{prove in Haskell} that these two functions are equivalent,
and we check our proofs using Liquid Haskell.

\paragraph{Theorems as Refinement Types}
Refinement Types refine types with properties
drawn from decidable logics.
For example, the type @{v:Int | 0 < v}@
describes all integer values @v@ that are greater than @0@.
%
We refine the unit type to express theorems,
define unit value terms to express proofs, and use
Liquid Haskell to check that the proofs prove the theorems.
%
For example, Liquid Haskell accepts
the type assignment @() :: {v:()| 1+1=2}@,
as the underlying SMT can always prove the equality @1+1=2@.
%
We write @{1+1=2}@ to simplify the type @{v:()| 1+1=2}@
from the irrelevant binder @v:()@.

\paragraph{Program Properties as Types}
The theorems we express can refer to program functions.
As an example, the type of @assoc@ expresses that @mappend@
is associative.
%
\begin{code}
assoc :: x:m -> y:m -> z:m -> {x mappend (y mappend z) = (x mappend y) mappend z}
\end{code}
%
In \S~\ref{sec:haskell-proofs} we explain
how to write Haskell proof terms to prove theorems like @assoc@
by proving that list append @(++)@ is associative.
%
Moreover, we prove that the empty list @[]@ is the identity element of
list append, and conclude that the list
(with @[]@ and @(++)@, \ie the triple (@[a]@, @[]@, @(++)@))
is provably a monoid.

\paragraph{Corectness of Parallelization}
In \S~\ref{sec:parallelization}, we define the type @Morphism n m f@ that specifies
that @f@ is a \textit{morphism} between two monoids
(@n@, @$\eta$@, @<+>@) and (@m@, @$\epsilon$@, @<>@),
% \ie @f@ distributes among them or, equivalently, @f (x <+> y) = f x <> f y@.
\ie @f :: n -> m@ where @f $\eta$ = $\epsilon$@ and @f (x <+> y) = f x <> f y@.

A morphism @f@ on a ``chunkable'' input type can be parallelized by:
\begin{enumerate}
  \item chunking up the input in @j@ chunks (@chunk j@),
  \item applying the morphism in parallel to all chunks (@pmap f@), and
  \item recombining the mapped chunks via @mappend@, also in parallel (@pmconcat i@).
\end{enumerate}
%
We specify correctness of the above transformation as a refinement type.
%
\begin{code}
parallelismEquivalence
  :: f:(n -> m) -> Morphism n m f -> x:n -> i:Pos -> j:Pos
   -> {f x = pmconcat i (pmap f (chunk j x))}
\end{code}
%
\S~\ref{sec:parallelization} describes the parallelization transformation in details,
while Correctness of Parallelization Theorem~\ref{theorem:two-level} proves correctness
by providing a
term that satisfies the above type.

\paragraph{Case Study: Parallelization of String Matching}
We use the above theorem to parallelize string matching.
We define a string matching function @toSM :: RString -> toSM target@
from a \textit{refined string} to a \textit{string matcher}.
%
A refined string (\S~\ref{subsec:refinedstrings}) is a wrapper around
the efficient string manipulation library
@ByteString@ that moreover assumes
various string properties, including the monoid laws.
%
A string matcher @SM target@ (\S~\ref{subsec:stringmatcher}) is a data type that contains
a refined string and a list of all the indices
where the type level symbol @target@ appears in the input.
%
We prove that @SM target@ is a monoid and @toSM@ is a morphism,
thus by the aforementioned Correctness of Parallelization Theorem~\ref{theorem:two-level}
we can correctly parallelize string matching.




To sum up, we present the first realistic proof that 
uses Haskell as a theorem prover:
correctness of parallelization on string matching.
%
Our contributions are summarized as follows
\begin{itemize}
\item We explain how theorems and proofs are encoded and checked in Liquid Haskell
by formalizing monoids and proving that lists are monoids
(\S~\ref{sec:haskell-proofs}).
\item We formalize morphisms between monoids and
specify and prove correctness of parallelization of morphisms
(\S~\ref{sec:parallelization}).
\item We show how libraries can be imported as trusted components by wrapping
@ByteString@s as refined strings which satisfy the monoid laws (\S~\ref{subsec:refinedstrings}).
\item As an application, we prove that a string matcher is a morphism between the monoids of refined strings
and string matchers,  thus we get provably correct parallelization of string matching (\S~\ref{sec:stringmatching}).
\item Based on our 1839LoC proof we evaluate the approach of using Haskell as a theorem prover
(\S~\ref{sec:evaluation}).
\end{itemize}
