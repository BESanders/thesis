\section{Evaluation: Strengths \& Limitations}\label{sec:stringmatcher:evaluation}

Verification of Parallel String Matching is the first realistic
proof that uses (Liquid) Haskell
to prove properties \textit{about} program functions.
%
In this section we use the String Matching proof
to quantitatively and qualitatively evaluate theorem proving in Haskell.

\paragraph{Quantitative Evaluation.}
The Correctness of Parallel String Matching proof
can be found online~\cite{implementation}.
%
Verification time, that is the time Liquid Haskell needs to check the proof,
is 75 sec on a dual-core Intel Core i5-4278U processor.
%
The proof consists of \textit{1839} lines of code.
%
Out of those
\begin{itemize}
\item \textit{226} are Haskell ``runtime'' code,
\item \textit{112} are liquid comments on the ``runtime'' Haskell code,
\item \textit{1307} are Haskell proof terms, that is functions with @Proof@ result type, and
\item \textit{194}  are liquid comments to specify theorems.
\end{itemize}
Counting both liquid comments and Haskell proof terms as verification code,
we conclude that the proof requires 7x the lines of ``runtime'' code.
%
This ratio is high and takes us to 2006 Coq,
when Leroy~\cite{Leroy06formalcertification} verified
the initial CompCert C compiler with
the ratio of verification to compiler lines being 6x.
% 4,400 lines of compiler code and 28,000 lines devoted to verification.

\paragraph{Strengths.}
Though currently verbose,
deep verification using Liquid Haskell has many benefits.
%
First and foremost,
\textit{the target code is written in the general purpose Haskell}
and thus can use advanced Haskell features, including
type literals, deriving instances, inline annotations
and optimized library functions like @ByteString@.
Even diverging functions can coexist with the target code, as long
as they are not reflected into logic~\cite{Vazou14}.

Moreover, \textit{SMTs are used to automate the proofs}
over key theories like linear arithmetic and equality.
%
As an example, associativity of @(+)@
is assumed throughout the proofs while shifting indices.
%
Our proof could be further automated 
by mapping refined strings to SMT strings and  
using the automated SMT string theory.
%
We did not follow this approach because we want to show that
our techinique can be used to prove any (and not only domain specific)
program properties.

Finally, we get further automation via
\textit{Liquid Type Inference}~\cite{LiquidPLDI08}.
%
Properties about program functions,
expressed as type specifications with unit result,
often depend on program invariants,
expressed as vanilla refinement types, and vice versa.
%
For example, we need the invariant that all indices of
a string matcher are good indices
to prove associativity of @(mappend)@.
%%NV simplify the above.
%%%For example, we need the invariant that all indices of
%%%a string matcher are good indices
%%%to prove that
%%%when the target is smaller than the input
%%%then the indices field is an empty list,
%%%which is required to prove associativity of @(mappend)@
%
Even though Liquid Haskell cannot currently synthesize proof terms,
it performs really well at inferring and propagating program invariants (like good indices)
via the abstract interpretation framework of Liquid Types.

\paragraph{Limitations.}
There are severe limitations that should be addressed
to make theorem proving in Haskell a pleasant and usable technique.
%
As mentioned earlier \textit{the proofs are verbose}.
%
There are a few cases where the proofs require domain specific knowledge.
%
For example, to prove associativity of string matching
@x mappend (y mappend z) = (x mappend y) mappend z@
we need a theorem that performs case analysis on the relative length of
the input field of @y@ and the target string.
%
Unlike this case split though, most proofs
do not require domain specific knowledge and merely proceed
by term rewriting and structural inductuction
that should be automated
via Coq-like~\cite{coq-book} tactics or/and Dafny-like~\cite{dafny} heuristics.
%
For example, synquid~\cite{polikarpova16} could be used to automatically
synthesize proof terms.

Currently, we suffer from two engineering limitations.
%
First, all reflected function should exist in the same module,
as reflection needs access to the function implementation
that is unknown for imported functions.
%
This is the reason why we need to use a user defined,
instead of Haskell's built-in, list.
%
In our implementation we used @CPP@ as a current workaround
of the one module restriction.
%
Second, class methods
cannot be currently reflected.
%
Our current workaround is to define Haskell functions instead
of class instances.
For example (@append@, @nil@) and (@concatStr@, @emptyStr@)
define the monoid methods of List and Refined String respectively.

Overall, we believe that the strengths outweigh the limitations which
will be addressed in the near future,
rendering Haskell a powerful theorem prover.
