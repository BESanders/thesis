\section{Conclusion}\label{sec:conclusion}
We made the first non-trivial use of (Liquid) Haskell as a proof
assistant. 
We proved the parallelization of chunkable monoid
morphisms to be correct
and applied our parallelization technique to string matching,
resulting in a formally verified parallel string matcher.
%
Our proof uses refinement types to specify
equivalence theorems,
Haskell terms to express proofs,
and Liquid Haskell to check that the terms prove the theorems.
%
Based on our 1839LoC sophisticated proof we conclude that
Haskell can be successfully used as a theorem prover
to prove arbitrary theorems about real Haskell code
using SMT solvers to automate proofs
over key theories like linear arithmetic and equality.
%
However, Coq-like tactics or Dafny-like heurestics are required
to ease the user from manual proof term generation.


\begin{comment}
  - lines of code
  - interaction of proofs with code
        no interaction: the main proof
        invariant good indices are requied to prove that
           if target it bigger than input then indices is empty
           proof oof good indexing requires casting
   - proof reuse
   - trust library factions with assume annotations
\end{comment}
