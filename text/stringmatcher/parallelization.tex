\section{Verification of Parallel Concatenation on Monoids}\label{sec:parallelization}
In this section we define how to concatenate a list of monoids
into a monoid. 
%
Then, we parallelize list concatenation 
and use LiquidHaskell to prove that the 
parallel and sequential versions are equivalent. 
\NV{What does equivalence means here?}

\subsection{Concatenation of lists using monoids}
If @m@ is a monoid, then @mconcat@ reduces a list of monoids @xs@
to one monoid. 
\begin{code}
mconcat :: List m -> m 
mconcat N        = mempty
mconcat (C x xs) = x mappend (mconcat xs)
\end{code}
%
Concatenation folds the elements of the list using @(mappend)@ 
and returns @mempty@ when the list is empty. 

Next, we define a parallel version of list concatenation 
%
\begin{code}
pmconcat :: Int -> List m -> m
pmconcat i xs  | i <= 1 || llen xs <= i
  = mconcat xs 
pmconcat i xs 
  = pmconcat i $ withStrategy parStrategy $  
                 map mconcat (chunk i xs)
\end{code}%$
The function @pmconcat i xs@ calls @mconcat xs@ in the base case, 
otherwise it
(1) chunks the list @xs@ in lists of size @i@, 
(2) runs in parallel @mconcat@ to each chunk, 
(3) recursively runs itself in the result list.
%
Termination of @pmconcat@ holds, as the length of @chunk i xs@
is smaller than the length of @xs@, when @1 < i@. 
%
Concretely, we define @chunk@ using the standard definition for 
list @take@ and @drop@.
\begin{code}
chunk :: i:Pos -> xs:List a 
  -> {v:List (List a) | chunkRes i xs } 
chunk i xs 
  | length xs <= i 
  = C xs N 
  | otherwise
  = C (take i xs) (chunk i (drop i xs))

chunkRes i xs
  | length xs <= i = length v == 1 
  | i == 1         = length v == length xs
  | otherwise     = length v < length xs
\end{code}

Where @take@ and @drop@ type as follows
\begin{code}
drop :: i:Nat -> xs:{List a | i <= length xs } 
  -> {v:List a | length v == length xs - i }
take :: i:Nat -> xs:{List a | i <= length xs } 
  -> {v:List a | length v == i }
\end{code}


\paragraph{Runtime Parallelism.}
To parallelize concatenation of the chunks we use parallel strategies 
from the Haskell library @Control.Parallel.Strategies@.
%
For example, in our implementation @parStrategy@ is defined as
\begin{code}
parStrategy :: Strategy (List a)
parStrategy = parTraversable rdeepseq
\end{code}

Note that parallelization of concatenation leads to runtime speedups
only when @(mappend)@ runs in constant time with respect to the monoid size. 
%
Otherwise, at each recursive call of @pmconcat@,
runtime will be dominated by the size of the biggest list element. 

Even though runtime of the parallel version depends on the implementation details 
of the monoids, correctness holds for every monoid. 
%
In the rest of this section we prove equivalence of the sequential and parallel concatenation operations. 


\paragraph{Parallelism in the Logic.}
To prove equivalence we make the \textit{assumption}
that the Haskell function @withStrategy s x@ 
returns @x@ for every strategy.
\begin{code}
assume withStrategy :: Strategy a -> x:a -> {v:a | v == x}
\end{code}
%
Moreover, to prove equivalence, we need to represent 
the implementation of @pmconcat@ in logic.
%
Which, in turn, requires representation of @withStrategy@. 
%
LiquidHaskell represents @withStrategy@ in logic as a function that merely returns
its second argument @withStrategy _ x = x@
and does not reason about parallelism. 

\subsection{Equivalence of Parallel and Sequential Concatenation}

\begin{theorem}[Monoid Concatenation Equivalence]\label{theorem:equivalence:concat}
If @m@ is monoid, then the parallel and sequential concatenations are equivalent:
\begin{code}
pmconcatEquivalence 
  :: i:Int -> is:List m -> { pmconcat i is == mconcat is }
\end{code}
\end{theorem}

\begin{proof}
We prove the theorem by providing a Haskell implementation of @pmconcatEquivalence@
that is safe under LiquidHaskell. The details of the proof can be found in~\cite{implementation}, 
here we provide the sketch of the proof. 

First we prove that @mconcat@ distributes over list splitting
\begin{code}
mconcatSplit 
  :: i:Nat -> xs:{List m | i <= llen xs} 
  -> { mconcat xs == mconcat (take i xs) 
                 <> mconcat (drop i xs) }
\end{code}
%
The proofs proceeds by structural induction, using monoid left identity in the base case
and associavity in the inductive step.

We generalize the above lemma
to prove that @mconcat@ distributes over list chunking
\begin{code}
mconcatChunk 
  :: i:Pos -> xs:List (Monoid a) 
  -> { mconcat xs == mconcat (map mconcat (chunk i xs)) }
\end{code}
%
The proofs proceeds by structural induction, using monoid left identity in the base case
and lemma @mconcatSplit@ in the inductive step.

Lemma @mconcatChunk@ is sufficient to prove @pmconcatEquivalence@ by structural induction, 
using monoid left identity in the base case. 
\qed\end{proof}

Note that the proof uses only the monoid laws and not the implementations of the monoid
operators, thus the theorem holds \textit{for any} type that satisfies the monoid laws. 
%
Next we define a monoid structure for string matching and prove it satisfies the monoid laws. 
Thus we conclude that string matching concatenation can be parallelized. 
 
