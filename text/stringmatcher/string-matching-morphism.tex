Next, we define the function @toSM :: RString -> SM target@ which does
the actual string matching computation for a set
target~\footnote{\texttt{toSM} assumes the target is clear from the
  calling context; it is also possible to write a wrapper function
  taking an explicit target which gets existentially reflected into
  the type.}
%
\ignore{ The function @toSM input@ creates a string matcher @SM
  target@ structure that matches the type level symbol target to the
  string value argument @input@.  }
%
\begin{code}
toSM :: forall (target :: Symbol). (KnownSymbol target)
     => RString -> SM target
toSM input = SM input (makeSMIndices input tg) where
  tg = fromString (symbolVal (Proxy :: Proxy target))

makeSMIndices
  :: x:RString -> tg:RString -> [GoodIndex x tg]
makeSMIndices x tg
  = makeIndices x tg 0 (lenStr tg - 1)
\end{code}
%
The input field of the result is the input string;
the indices field is computed by calling @makeIndices@
within the range of the @input@, that is from
@0@ to @lenStr input - 1@.
%
\begin{code}
\end{code}

We now prove that @toSM@ is a monoid morphism.
%
\begin{theorem}[$\texttt{toSM}$ is a Morphism]\label{theorem:smmorphism}
@toSM :: RString -> SM t@ is a morphism between the monoids
(@RString@, @stringMempty@, @stringMappend@) and (@SM t@, @mempty@, @mappend@).
\end{theorem}
\begin{proof}
Based on definition~\ref{definition:morphism}, proving @toSM@
is a morphism requires constructing a valid inhabitant of the type
\begin{code}
Morphism RString (SM t) toSM
  = x:RString -> y:RString
  -> {toSM stringMempty = mempty && toSM (x <+> y) = toSM x <> toSM y}
\end{code}
%
We define the function @distributestoSM :: Morphism RString (SM t) toSM@
to be the required valid inhabitant.

The core of the proof starts from
exploring the string matcher @toSM x <> toSM y@.
%
This string matcher contains three sets of indices
as illustrated in Figure~\ref{fig:mappend:indices}:
(1) @xis@ from the input @x@,
(2) @xyis@ from appending the two strings, and
(3) @yis@ from the input @y@.
%
We prove that appending these three groups of indices together gives
exactly the good indices of @x stringMappend y@, which are also the
value of the indices field in the result of
%
@toSM (x stringMappend y)@.
\begin{code}
distributestoSM x y
  =   (toSM x :: SM target) <> (toSM y :: SM target)
  ==. (SM x is1) <> (SM y is2)
  ==. SM i (xis ++ xyis ++ yis)
  ==. SM i (makeIndices i tg 0 hi1 ++ yis)
      ? (mapCastId tg x y is1 && mergeNewIndices tg x y)
  ==. SM i (makeIndices i tg 0       hi1
         ++ makeIndices i tg (hi1+1) hi)
      ? shiftIndicesRight 0 hi2 x y tg
  ==. SM i is
      ? mergeIndices i tg 0 hi1 hi
  ==. toSM (x <+> y)
  *** QED
  where
    xis  = map (castGoodIndexRight tg x y) is1
    xyis = makeNewIndices x y tg
    yis  = map (shiftStringRight   tg x y) is2
    tg   = fromString (symbolVal (Proxy::Proxy target))
    is1  = makeSMIndices x tg
    is2  = makeSMIndices y tg
    is   = makeSMIndices i tg
    i    = x <+> y
    hi1  = lenStr x - 1
    hi2  = lenStr y - 1
    hi   = lenStr i - 1
\end{code}
%
The most interesting lemma we use is
@mergeIndices x tg lo mid hi@
that states that for the input @x@ and the target @tg@
if we append the indices in the range  from @to@ to @mid@
with the indices in the range from @mid+1@ to @hi@,
we get exactly the indices in the range from @lo@ to @hi@.
%
This property is formalized in the type of the lemma.
\begin{code}
mergeIndices
 :: x:RString -> tg:RString
  -> lo:Nat -> mid:{Int | lo <= mid} -> hi:{Int | mid <= hi}
  -> {   makeIndices x tg lo hi
     =  makeIndices x tg lo mid
     ++ makeIndices x tg (mid+1) hi}
\end{code}
%
The proof proceeds by induction on @mid@ and using three more lemmata:
\begin{itemize}
\item @mergeNewIndices@ states that appending the indices @xis@ and @xyis@
is equivalent to the good indices of @x stringMappend y@ from @0@ to @lenStr x - 1@.
%
The proof case splits on the relative sizes of @tg@ and @x@
and is using @mergeIndices@ on @mid = lenStr x1 - lenStr tg@
in the case where @tg@ is smaller than @x@.
%
\item @mapCastId@ states that casting a list of indices returns the same list.
\item @shiftIndicesRight@ states that shifting right @i@ units the indices from @lo@ to @hi@
is equivalent to computing the indices from @i + lo@
to @i + hi@ on the string @x stringMappend y@, with @lenStr x = i@.
\end{itemize}
%%\begin{code}
%%mergeNewIndices
%%  :: tg:RString -> x:RString -> y:RString
%%  -> { makeSMIndices x tg
%%    ++ makeNewIndices x y tg
%%    == makeIndices (x <+> y) tg 0 (lenStr x - 1)}
%%\end{code}
%%The theorem states that appending the indices that appear in the first string @x@,
%%that is the indices @xis@ that do not and the indices @xyis@ that do involve @y@,
%%is equivalent to the indices @is@ on the input @x stringMappend y@
%%in the range from @0@ to @lenStr x -1@.
%%%
%%The proof case splits on the sizes of @tg@ and @x@.
%%%
%%If the size of the target is less than two, then @xyis@ is empty,
%%but @xis == is@, as appending @y@ to the input cannot create more indices.
%%%
%%Otherwise, if the input @x@ is smaller than the target,
%%then @xis@ is empty and @xyis@ is equal to @is@.
%%%
%%Otherwise, we set
%%and use the previous @mergeIndices@ lemma to merge @xis@ and @yis@ into @is@.
%%\item Shifting is left concatenation.
%%\begin{code}
%%shiftIndicesRight
%%  :: lo:Nat -> hi:Int
%%  -> x:RString -> y:RString
%%  -> tg:RString
%%  -> { map (shiftStringRight tg x y)
%%           (makeIndices y tg lo hi)
%%  == makeIndices (x stringMappend y) tg
%%                 (lenStr x + lo) (lenStr x + hi) }
%%
%%\end{code}
%%The lemma states that shifting indices from @lo@ to @hi@ on @y@
%%is equivalent to computing the indices from @lenStr x + lo@
%%to @lenStr x + hi@ on @x stringMappend y@,
%%and it is proved by induction on the difference @hi - lo@.
%%\end{itemize}
\qed\end{proof}
