We define a new type @RString@, which is a chunkable monoid, to be the
domain of our string matching function. Our type simply wraps
Haskell's existing @ByteString@.
\begin{code}
  data RString = RS BS.ByteString
\end{code}
Similarly, we wrap the existing @ByteString@ functions we will need to
show @RString@ is a chunkable monoid.
\begin{code}
  stringMempty = RS (BS.empty)
  (RS x) stringMappend (RS y)= S (x `BS.append` y)

  lenStr    (RS x) = BS.length x
  takeStr i (RS x) = RS (BS.take i x)
  dropStr i (RS x) = RS (BS.take i x)
\end{code}
Although it is possible to explicitly prove that @ByteString@
implements a chunkable monoid~\cite{realworldliquid14}, it is time
consuming and orthogonal to our purpose. Instead, we
just \textit{assume} the chunkable monoid properties of @RString@--
thus demonstrating that refinement reflection is capable of doing
gradual verification.



For instance, we define a logical uninterpreted function
@stringMappend@ and relate it to the Haskell @stringMappend@ function
via an assumed (unchecked) type.
%
\begin{code}
  assume (stringMappend) :: x:RString -> y:RString -> {v:RString | v = x stringMappend y}
\end{code}
%
Then, we use the uninterpreted function @stringMappend@ in the logic
to assume monoid laws, like associativity.
%
\begin{code}
  assume assocStr 
    :: x:RString -> y:RString -> z:RString 
    -> {x <+> (y <+> z) = (x <+> y) <+> z}
  assocStr _ _     = trivial
\end{code}
%
Haskell applications of @stringMappend@ are interpreted in the logic
via the logical @stringMappend@ that satisfies associativity via theorem @assocStr@.

Similarly for the chunkable methods, we define the uninterpreted functions
@takeStr@, @dropStr@ and @lenStr@ in the logic,
and use them to strengthen the result types of the respective functions.
%
\ignore{
For example the type of @takeStr@ includes both the length specifications
from chunkable monoid and the uninterpreted function equality @v = takStr i x@.
\begin{code}
  assume takeStr
    :: i:Nat -> x:{RString | i <= lenStr x}
    ->  {v:RString | lenStr v = i && v = takeStr i x}
\end{code}
We use the uninterpreted function @takeStr@ and @dropStr@ to
specify and \textit{assume} the @take@-@drop@ property of chunkable monoids.
\begin{code}
  assume takeDropPropStr
   :: i:Nat -> x:RString -> {x = takeStr i x <+> dropStr i x}
takeDropPropStr _ _ = trivial
\end{code}
%
} With the above function definitions (in both Haskell and logic) and
assumed type specifications, Liquid Haskell will check (or rather
assume) that the specifications of chunkable monoid, as defined in the
Definitions~\ref{definition:monoid} and~\ref{definition:chunkable},
are satisfied.
%
We conclude with the assumption (rather that theorem)
that @RString@ is a chunkable monoid.
\begin{assumption}[RString is a Chunkable Monoid]\label{assumption:rstring}
(@RString@, @stringMempty@, @stringMappend@)
combined with the methods
@lenStr@, @takeStr@, @dropStr@ and @takeDropPropStr@
is a chunkable monoid.
\end{assumption}
