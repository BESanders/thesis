\documentclass[12pt]{ucsddissertation}
% mathptmx is a Times Roman look-alike (don't use the times package)
% It isn't clear if Times is required. The OGS manual lists several
% "standard fonts" but never says they need to be used.
\usepackage{mathptmx}
\usepackage[NoDate]{currvita}
\usepackage{array}
\usepackage{tabularx}
\usepackage{booktabs}
\usepackage{ragged2e}
\usepackage{microtype}
\usepackage[breaklinks=true,pdfborder={0 0 0}]{hyperref}
\usepackage{graphicx}


%% These are the packages required for the todo notes
\usepackage{ifthen}
\usepackage{xkeyval}
\usepackage{tikz}
\usepackage{calc}
\usepackage{xcolor}


\usepackage[textwidth=3.7cm]{todonotes}
\setlength{\marginparwidth}{4cm}

\newcounter{todocounter}
\newcommand{\todonum}[2][]
{\stepcounter{todocounter}\todo[#1]{\thetodocounter: #2}}
%$ end todo notes requirements



\AtBeginDocument{%
	\settowidth\cvlabelwidth{\cvlabelfont 0000--0000}%
}

% OGS recommends increasing the margins slightly.
\increasemargins{.1in}

% These are just for testing/examples, delete them
\usepackage{trace}
%\usepackage{showframe} % This package was just to see page margins
\usepackage[english]{babel}
\usepackage{blindtext}
\overfullrule5pt
% ---

% Required information
\title{Liquid Haskell: Haskell as a Theorem Prover}
\author{Niki Vazou}
\degree{Computer Science}{Doctor of Philosophy}
% Each member of the committee should be listed as Professor Foo Bar.
% If Professor is not the correct title for one, then titles should be
% omitted entirely.
\chair{Professor Ranjit Jhala}
% Your committee members (other than the chairs) must be in alphabetical order
\committee{Professor Samuel R. Buss}
\committee{Professor Cormac Flanagan}
\committee{Professor Sorin Lerner}
\committee{Professor Daniele Micciancio}
\degreeyear{2016}

% Start the document
\begin{document}
%% Note: Do not forget to remove this list
\listoftodos

% Begin with frontmatter and so forth
\frontmatter
\maketitle
\makecopyright
\makesignature
% Optional
\begin{dedication}
\setsinglespacing
\raggedright % It would be better to use \RaggedRight from ragged2e
\parindent0pt\parskip\baselineskip
\center{ For my mother, father and sister.}
\end{dedication}
% Optional
\begin{epigraph}
\vskip0pt plus.5fil
\setsinglespacing
{\flushright
Simplicity is the ultimate sophistication.
\vskip\baselineskip
\textit{Leonardo Da Vinci}\par}
\end{epigraph}

% Next comes the table of contents, list of figures, list of tables,
% etc. If you have code listings, you can use \listoflistings (or
% \lstlistoflistings) to have it be produced here as well. Same with
% \listofalgorithms.
\tableofcontents
\listoffigures
\listoftables

% Preface
%%\begin{preface}
%%optional and may be omitted
%%\end{preface}

% Your fancy acks here. Keep in mind you need to ack each paper you
% use. See the examples here. In addition, each chapter ack needs to
% be repeated at the end of the relevant chapter.
\begin{acknowledgements}
\todonum{acks}
I want to thank my supervisor Ranjit Jhala for all the inspiration. 

I want to thank my committee members 
Samuel R. Buss, 
Cormac Flanagan for inspiring my work,
Sorin Lerner for direction on the PL group, 
and Daniele Micciancio for his strong Haskell enthousiasm 
for being in my committee. 

Next thank the group and specifically my coauthors
Patric Rondon,
Alexander Bakst, 
Eric Seidel for being there. 
and the rest of the group for the supporting and friendly working environment.

I want to thank all these people who hosted me in my internships. 
%
The Opa group in Paris who foresaw that types can help industrial development.
%
Then, Dimitrios Vytiniotis and Simon Peyton-Jones
at Microsoft Research Cambridge
who were the first ghc people who believed in Liquid Haskell
and have been still supporting both the project and me personally.
%
I want to thank Jeff Polakow and all the people at Awake Networks 
for giving me the opportunity to spend an awesome summer in Mountain View 
and use Liquid Haskell in real world development code. 
%
Last but not least, 
I want to thank Dann Leijen 
for hosting me at Microsoft Research Redmond 
and mentoring me since then, giving me all his insightful
knowledge and  about research and life in general.

Finally, I want to thanks all the family and friends, around the world,
who supported me throughout this journey.
%
All the people from San Diego, 
who made me feel San Diego as home, with all the goods and bads. 
%
All the friends I met in the infinite travelling in these five years, 
that made me realize how knowledge unites people from different parts of the globe.
%
All my friends and family in Greece that never let me forget that
whenever I want I have a paradise waiting for me.  
%
Specifically, my parents Paraskeui and Nikos and my sister Marianna
who staid up all these Sunday nights to talk to my and keep me sane. 


It is great that the destination is reached, 
but after all, it is the journey that matters. 




\todonum{paper acks}
Chapter 2, in full, is a reprint of the material as it appears in
Numerical Grid Generational in Computational Fluid Mechanics~2009.
Smith, Laura; Smith, Jane~D., Pineridge Press,~2009. The dissertation
author was the primary investigator and author of this paper.

Chapter 3, in part, has been submitted for publication of the material
as it may appear in Education Mechanics,~2009, Smith, Laura; Smith,
Jane~D., Trailor Press,~2009. The dissertation author was the primary
investigator and author of this paper.

Chapter 5, in part is currently being prepared for submission for
publication of the material. Smith, Laura; Smith, Jane~D\@. The
dissertation author was the primary investigator and author of this
material.


\end{acknowledgements}

% Stupid vita goes next
\begin{vita}
\noindent
\begin{cv}{}
\begin{cvlist}{}
\item[1996] Bachelor of Arts, University of California, Berkeley
\item[1996--2000] U.S. Marines
\item[2000--2002] Teaching Assistant, Department of Mechanical
Engineering\\University of California, San Diego
\item[2002--2006] Research Assistant, University of California, San
Diego
\item[2010] Doctor of Philosophy, University of California, San Diego
\end{cvlist}
\end{cv}

% This puts in the PUBLICATIONS header. Note that it appears inside
% the vita environment. It is optional.
\publications
\noindent``Distributions of Control Points in a System for Analysis of Stress
Distribution'' IRE Transactions of the I.R.E\@. Professional Group on
Automatic Control, vol. AC-7, pp 272--289, September 2005

% This puts in the FIELDS OF STUDY. Also inside vita and also
% optional.
\fieldsofstudy
\noindent Major Field: Engineering (Specialization or Focused Studies)
\vskip\baselineskip
Studies in Applied Mathematics\par
Professors Alpha Beta and Gamma Delta
\vskip\baselineskip
Studies in Mechanices\par
Professors Epsilon Zeta and Eta Theta
\vskip\baselineskip
Studies in Electromagnetism\par
Professors Iota Kappa and Lambda Mu
\end{vita}

% Put your maximum 350 word abstract here.
\begin{dissertationabstract}
The Abstract begins here. The abstract is limited to 350 words for a
doctoral dissertation. It should consist of a short statement of the
problem, a brief explanation of the methods and procedures employed in
generating the data, and a condensed summary of the findings of the
study. The abstract may continute onto a second page if necessary. The
text of the abstract must be double spaced.
\end{dissertationabstract}

% This is where the main body of your dissertation goes!
\mainmatter

% Optional Introduction
\begin{dissertationintroduction}
This optional section is barely described in the OGS manual other than
saying it is optional and that it appears in the table of contents
between the Abstract and the first chapter.

No formatting guidelines appear so presumably, it should be formatted
like an ordinary chapter. It should appear after the
\verb!\mainmatter! macro because it should start on page~1.
\end{dissertationintroduction}

\chapter{An ordinary page}
The purpose of this page is to illustrate an ordinary page of text in
a doctoral dissertation or master's thesis. All pages of the doctoral
dissertation or master's thesis must be kept within the margins of
1.5'' on the left, 1'' on the right, 1'' on the top and 1.25'' on the
bottom. All text must be double spaced except as indicated below.

It is recommended that to increase the margins as paper can shift in a
printer and as some photocopiers tend to increase the image being
copied.

The first line of each paragraph must be indented at least one 0.5''
tab, as done here.

This text is intended to be a part of the dissertation, for a doctoral
student, or the thesis if you are receiving a master's degree, and now
a quote is included here:
\begin{quote}
All quotes of more than six lines, even though this one is not, are to
be indented 0.5'' from the left and 0.5'' from the right. These longer
quotes are to be single spaced. Don't forget to adjust for proper
spacing after the last line of the quoted material.
\end{quote}
The rest of the paragraph would continue as so.

\chapter{Figures and Such}
This demonstrates how OGS wants figures and tables formatted. For
figures, the caption goes below the figure and ``Figure'' is in bold.
See Figure~\ref{fig:zen}. Tables are formatted with the caption above
the table. See Table~\ref{tab:bad}.

Of course, Table~\ref{tab:bad} looks horrible. It should probably be
formatted like Table~\ref{tab:good} instead.

For facing caption pages, see Table~\ref{tab:facing}. Of course,
facing caption pages are vaguely ridiculous and my implementation of
them in the class file is by far the most brittle part of the
implementation. It's entirely possible that something has changed and
these don't work at all. I implemented it merely for the challenge.

\begin{figure}
\centering
\fbox{\parbox{.9\linewidth}{%
	\noindent
	{\Huge PHD ZEN}\par
	\vskip.5in
	\centerline{comic here}
	\vskip.5in
}}
\caption[``Ph.D. Zen'']{Comic entitled ``Ph.D. Zen'' by Jorge Cham, 2005. Copyright
has not been obtained and so it isn't displayed.}
\label{fig:zen}
\end{figure}

\begin{table}
\centering
\caption[Electronic Dissertation Submission Rates]{Electronic
Dissertation Submission Rates at UCSD, Fall 2005 and Winter 2006.
(First two quarters that the program was available to all Ph.D.
candidates not in a Joint Doctoral Program with SDSU.)}
\label{tab:bad}
\begin{tabular}{|*{5}{>{\centering\arraybackslash}m{.15\linewidth}|}}
\hline
&Ph.D.s awarded (Including Joint degrees) & Electronic submission of
Dissertation & Paper Submission of Dissertation & Percentage of
Electronic Submission\\
\hline
Fall\par 2005 & 84 & 37 & 47 & 44.05\%\\
\hline
Winter\par 2006 & 64 & 42 & 22 & 65.63\%\\
\hline
\end{tabular}
\end{table}

\begin{table}
\centering
\caption[Electronic Dissertation Submission Rates]{Electronic
Dissertation Submission Rates at UCSD, Fall 2005 and Winter 2006.
(First two quarters that the program was available to all Ph.D.
candidates not in a Joint Doctoral Program with SDSU.)}
\label{tab:good}
\renewcommand\tabularxcolumn[1]{>{\RaggedRight\arraybackslash}p{#1}}
\begin{tabularx}{.9\linewidth}{lcccc}
\toprule
&\multicolumn{1}{X}{Ph.D.s awarded (Including Joint degrees)}
&\multicolumn{1}{X}{Electronic submission of Dissertation}
&\multicolumn{1}{X}{Paper Submission of Dissertation}
&\multicolumn{1}{X}{Percentage of Electronic Submission}\\
\midrule
Fall 2005 & 84 & 37 & 47 & 44.05\%\\
Winter 2006 & 64 & 42 & 22 & 65.63\%\\
\bottomrule
\end{tabularx}
\end{table}

\begin{facingcaption}{table}
\caption[UCSD Gender Distribution]{University of
California, San Diego Gender Distribution for the Campus Population,
October~2005\\
(http://assp.ucsd.edu/analytical/Campus\%20Population.shtml)\\
\emph{(This is an example of a facing caption page, the next page is
the example of the table/figure/etc.\ that corresponds to this
caption. It is also an example of table/figure that is rotated 90
degrees to fit the page.)}}
\label{tab:facing}
\renewcommand\tabularxcolumn[1]{>{\RaggedLeft\arraybackslash}p{#1}}
\parindent=0pt
\setbox0=\vbox}
& \multicolumn{1}{c}{\textbf{N}} & \multicolumn{1}{c}{\textbf{\%}}
& \multicolumn{1}{c}{\textbf{N}} & \multicolumn{1}{c}{\textbf{\%}}\\
\midrule
Students & 12,987 & 51\% & 12,686 & 49\% & 25,673 & 100\%\\
Employees & 9,943 & 56\% &  7,671 & 44\% & 17,614 & 100\%\\
\addlinespace
\hfill\textbf{Total} & \textbf{22,930} & \textbf{53\%} &
\textbf{20,357} & \textbf{47\%} & \textbf{43,287} & \textbf{100\%}\\
\bottomrule
\end{tabularx}
\singlespacing

\emph{Notes}:
\begin{enumerate}
\item The counts shown below will differ from the official quarterly
Registrar's registration report because 1) data for residents in the
Schools of Medicine and Pharmacy and Pharmaceutical Science are
excluded, and 2) registered, non-matriculated, visiting students are
included.
\item Student workers are excluded from employees; however emeritus
faculty and others on recall status are included.
\end{enumerate}

Campus Planning. Analytical Studies and Space Planning\\
31 January 2006
}
\centerline{\rotatebox{90}{\box0}}
\end{facingcaption}

% This will give us some more text
\Blinddocument

% Skipping a bunch of chapters
\setcounter{chapter}{50}
\chapter{Another chapter}
\setcounter{figure}{73}
\setcounter{table}{88}
\begin{figure}
\centering
\fbox{\hbox to.8\linewidth{\hss Another figure\hss}}
\caption{Another figure caption}
\end{figure}
\begin{table}
\centering
\caption{Another table caption}
\begin{tabular}{ccc}
\toprule
X&Y&Z\\
\midrule
a&b&c\\
\bottomrule
\end{tabular}
\end{table}
\begin{figure}
\caption{ASDF fig}
\end{figure}
\begin{table}
\caption{ASDF tab}
\end{table}

\appendix
\Blinddocument

% Stuff at the end of the dissertation goes in the back matter
\backmatter
\bibliographystyle{plain} % Or whatever style you want like plainnat
\bibliography{references}

\end{document}
